\chapter{Analisi non distributive}
Le analisi distributive sono analisi statiche che si basano su ciò che viene 
calcolato, entrando quindi nel merito del valore che viene attribuito alle variabili 
cercando proprietà su tali valori.
\section{Propagazione delle costanti}
La propagazione delle costanti è un'analisi che ha come obiettivo quello di determinare 
se una variabile ha sempre lo stesso valore in un certo punto di programma.

Dato un punto di programma $p$, determina se una variabile nel punto di programma $p$ è sempre 
un valore costante. Tale analisi collegata alla \textbf{valutazione parziale}, che in un certo senso, 
che è collegata al concetto di \textbf{specializzazione}, discusso nell'ambito dei linguaggi 
di programmazione.

Cerchiamo quindi di capire se in un certo punto di programma una variabile ha sempre
lo stesso valore e in questo caso è possibile utilizzare tale valore per valutare parzialmente 
il programma preso in considerazione, vedendo quindi se alcuni dei calcoli possono essere
elaborati in funzione di un valore costante che il programma in quel punto di programma assume.

In un caso pratico, se in un certo
punto di programma una variabile che è utilizzata come condizione di un \textit{branching},
assume sempre lo stesso valore, talvolta è possibile eliminare il \textit{branching} e
eseguire solo una delle due parti del \textit{branching}.

\begin{algorithm}[H]
    $x \gets 1$\;
    \dots\;
    \If{$x > 0$}{
        $e$\;
    }
    \Else{
        $e'$\;
    }
\end{algorithm}
\begin{algorithm}[H]
    $x \gets 1$\;
    \dots\;
    $e$\;
\end{algorithm}

In questo senso l'informazione può essere utilizzata in vari ambiti permettendoci inoltre 
di capire quali valori possono assumere le variabili in un certo punto di programma.
Come \textit{side effect} di tale analisi, possiamo inoltre capire se un punto di 
programma è raggiungibile o meno.
\subsubsection{Esempio}
Supponiamo di avere il seguente programma:

\begin{algorithm}[H]
    $a:=1; b:=2; c:=3; d:=3; e:=0$\;
    \While{$B$}
    {
        $b := 2 \cdot a; d:=d+1; e:=e-a$\;
        $c :=e + d; a:= b - a$\;
    }
\end{algorithm}
Dove $B$ indica che la condizione sul ciclo non è nota staticamente.
Verifichiamo lo stato delle variabili dopo ogni punto di programma:
\begin{figure}[H]
    \centering
    \begin{tabular}{c|ccccc}
        & \texttt{a} &\texttt{b} & \texttt{c} & \texttt{d} & \texttt{e} \\
        \hline
        \textbf{1} & $1$ & $2$ & $3$ & $3$ & $0$ \\
        \textbf{2} & $1$ & $2$ & $3$ & \redtext{$3$} & \redtext{$0$} \\
        \textbf{3} & $1$ & $2$ & $3$ & $4$ & $-1$ \\
        \textbf{4} & $1$ & $2$ & $3$ & \redtext{$4$} & \redtext{$-1$} \\
        \hline
        \textbf{1} & $1$ & $2$ & $3$ & $3$ & $0$ \\
        \textbf{2} & $1$ & $3$ & $3$ & $?$ & $?$ \\
        \textbf{3} & $1$ & $3$ & $2$ & $?$ & $?$ \\
        \textbf{4} & $2$ & $3$ & $?$ & $?$ & $?$ \\
    \end{tabular}
\end{figure}
Dopo la prima iterazione, al punto di programma $2$, viene collezionato 
ciò che viene fatto al punto di programma $4$, visto che dal punto $4$ si ritorna al punto $2$
data la presenza del ciclo.

Collezionando valori vediamo che variano sono i valori di $d$ e $e$, quindi 
non possiamo dire nulla sui loro valori. Calcolando il valore di $d$, viene 
eseguita una somma con un valore non conosciuto, quindi $?$ non può essere
a sua volta conosciuto.

Concludiamo quindi che non possiamo dire nulla sul valore di $c$, $d$ e $e$, 
mentre possiamo dire che $a$ e $b$ sono sempre uguali a $1$ e $2$ rispettivamente.
A differenza dell'analisi astratta, nel caso concreto il valore di $c$ sarebbe 
sempre $3$, di fatto costante.
\subsection{Costruzione dell'analisi}
Il primo passo per costruire l'analisi è quello di definire il dominio delle 
informazioni astratte, ovvero delle proprietà che vogliamo osservare con precisione.

Nel dominio concreto $\mathcal{C}= \wp(\mathbb{Z})$, lavoriamo su valori interi, 
insiemi di interi. Tra questi insiemi di interi, l'obiettivo è osservare con 
precisione i singoletti, quindi:
\[
    \mathcal{A} = \{n \mid n \in \mathbb{Z}\} \cup \{\bot, \top\} = \mathbb{Z}^\top
\]

Dal momento in cui una variabile colleziona più di un valore possibile,
l'informazione non è più precisa, quindi non è più possibile dire nulla 
sul fatto che possa essere costante o meno.

L'inserzione di Galois tra i due domini $\mathcal{C}$ e $\mathcal{A}$ è 
data dalle funzioni: 
\[
    \alpha(x) = 
    \begin{cases}
        \top & \text{se } x = \emptyset \\
        n & \text{se } S = \{n\} \quad n\in \mathbb{Z} \\
        \top & \text{altrimenti}
    \end{cases}
\]
\[
    \gamma(a) = 
    \begin{cases}
        \emptyset & \text{se } a = \top \\
        \{n\} & \text{se } a = n \quad n\in \mathbb{Z} \\
        \mathbb{Z} & \text{altrimenti}
    \end{cases}
\]
dove $x \in \wp(\mathbb{Z})$ e $a \in \mathcal{A}$ e
con $\alpha$ che è la funzione di astrazione e $\gamma$ che è la funzione di concretizzazione, 
entrambe funzioni monotone; si può dimostrare che formano una \textit{inserzione di Galois}.
\begin{figure}[H]
    \centering
    \begin{tikzpicture}
        \node (top) at (0, 2) {$\top$};
        \node (f1) at (6, 0) {$\dots$};
        \node (z) at (4, 0) {$2$};
        \node (a) at (2, 0) {$1$};
        \node (c) at (0, 0) {$0$};
        \node (e) at (-2, 0) {$-1$};
        \node (f2) at (-4, 0) {$-2$};
        \node (f3) at (-6, 0) {$\dots$};
        \node (bot) at (0, -2) {$\bot$};

        % Linee verso \top
        \draw (z.north) -- (top.south);
        \draw (a.north) -- (top.south);
        \draw (c.north) -- (top.south);
        \draw (e.north) -- (top.south);
        \draw (f2.north) -- (top.south);

        % Linee verso \bot
        \draw (z.south) -- (bot.north);
        \draw (a.south) -- (bot.north);
        \draw (c.south) -- (bot.north);
        \draw (e.south) -- (bot.north);
        \draw (f2.south) -- (bot.north);
    \end{tikzpicture}
\end{figure}