\documentclass[oneside,a4paper,11pt]{book}
\usepackage[utf8]{inputenc}
\usepackage{svg}
\usepackage[italian]{babel}
\usepackage{float}
\usepackage{fancyvrb}
\usepackage{titling}
\usepackage[margin=1in,footskip=0.25in]{geometry}
\usepackage{listings}
\usepackage[DIV=12,BCOR=2mm,headinclude=true,footinclude=false]{typearea}
\usepackage{color, colortbl,xcolor}
\usepackage[hidelinks]{hyperref}
\usepackage{tcolorbox}
\usepackage{chngcntr}
\usepackage{diagbox}
\usepackage{calc}
\usepackage{amssymb}
\usepackage{subcaption}
\usepackage{amsthm}
\usepackage{amsfonts}
\usepackage{mathtools}
\usepackage{parskip}
\usepackage{cancel}
\usepackage{forest}
\usepackage{listings}
\usepackage{mathrsfs}
\usepackage{enumitem}
\usepackage{makecell}
\usepackage{tikz}
\usepackage{pgfplots}
\pgfplotsset{compat=1.18}
\usepackage{fancyhdr}
\fancypagestyle{plain}{\fancyhf{}\renewcommand{\headrulewidth}{0pt}}
\pagestyle{fancy}
\fancyhf{}% Clear header/footer
\fancyhead[L]{\nouppercase\leftmark}
\fancyhead[R]{\thepage}
\usetikzlibrary{positioning,shapes.geometric,arrows.meta,matrix,automata,decorations.pathmorphing,patterns,decorations.pathreplacing,shapes.multipart,calc,snakes}
\usetikzlibrary{arrows.meta, backgrounds, chains, positioning, shapes.geometric, shapes.multipart}
\usetikzlibrary{positioning,fit,arrows.meta,backgrounds}
\tcbuselibrary{skins}
\counterwithin{figure}{section}
\usetikzlibrary{arrows,shapes,positioning,arrows.meta}
\usepackage{algorithm2e} % Pacchetto per algoritmi
\tikzset{
    block/.style={
        rectangle,
        draw,
        text width=1.5cm,
        align=center,
        minimum height=0.75cm,
        font=\sffamily
    },
    arrow/.style={-Latex}
}
\usetikzlibrary{calc}
\newcommand\unit[4]   {\draw[fill=#2] (#4) rectangle +(#1);
                       \node[font=\sffamily\bfseries] at ($(#4)+0.5*(#1)$) {#3};
                      }
\newcommand\gpualu    {\unit{ 0.9,0.9 }{green!30}{}}
\newcommand\gpucontrol{\unit{ 0.9,0.4 }{yellow!30}{}}
\newcommand\gpucache  {\unit{ 0.9,0.4 }{red!30}{}}
\newcommand\dram      {\unit{16.9,1.9 }{red!30}{DRAM}}
\newcommand\cpualu    {\unit{ 4.1,2.05}{green!30}{ALU}}
\newcommand\cpucontrol{\unit{ 8.4,4.2 }{yellow!30}{Control}}
\newcommand\cpucache  {\unit{16.9,4.2 }{red!30}{Cache}}

\usetikzlibrary{fit, backgrounds, arrows, calc, decorations.pathmorphing, positioning}

\tikzset{
    snake arrow/.style={
        ->, thick, 
        decorate, 
        decoration={snake, amplitude=.4mm, segment length=2mm, post length=1mm}
    },
    thread/.pic={
        \node[fill=orange,
              label={[anchor=north, name=th]90:Thread (\the\pgfmatrixcurrentcolumn,\the\pgfmatrixcurrentrow)},
              minimum width=2cm,  % ridotta da 3cm
              minimum height=2cm, % ridotta da 2.5cm
              draw] (Th) {};
        \draw[thick] (th.south) edge[snake arrow] ++(-90:10mm);
    },
    block/.pic={
        \node[fill=yellow,
              label={[anchor=north, name=bl]90:Block (\the\pgfmatrixcurrentcolumn,\the\pgfmatrixcurrentrow)},
              minimum width=2cm,  % ridotta da 3cm
              minimum height=2cm, % ridotta da 2.5cm
              draw] (Bl) {};
        \foreach \i in {-4,0,4}
            \draw[thick] ([xshift=\i mm]bl.south) edge[snake arrow] ++(-90:10mm); % ridotta la lunghezza della freccia
    }
}
%Nuovi comandi
\newcommand\myeq{\stackrel{\mathclap{\normalfont\mbox{def}}}{=}}
\newcommand\prodG{\stackrel{\mathclap{\normalfont\mbox{\tiny{G}}}}{\Longrightarrow}}
%asmthm
\newlength{\marginlabelsep}\setlength{\marginlabelsep}{0.5em}
\newtheoremstyle{italicstyle} %% Name
  {} %% <- Space above (empty = default = \topsep = 8.0pt plus 2.0pt minus 4.0pt)
  {} %% <- Space below (empty = default = \topsep = 8.0pt plus 2.0pt minus 4.0pt)
  {\itshape} %% <- Body font
  {} %% <- Indent amount (empty = no indent, \parindent = just that)
  {\bfseries} %% <- Thm head font
  {} %% <- Punctuation after thm head
  {1pt} %% <- Space after thm head (or " " or \newline) (default: 5pt plus 1pt minus 1pt)
  {\vtop to 0pt{\llap{\thmname{#1}\hskip\marginlabelsep}
                \llap{\thmnumber{#2}\hskip\marginlabelsep}}\thmnote{#3\\}%
  }
\newtheoremstyle{normStyle} %% Name
  {} %% <- Space above (empty = default = \topsep = 8.0pt plus 2.0pt minus 4.0pt)
  {} %% <- Space below (empty = default = \topsep = 8.0pt plus 2.0pt minus 4.0pt)
  {\normalfont} %% <- Body font
  {} %% <- Indent amount (empty = no indent, \parindent = just that)
  {\bfseries} %% <- Thm head font
  {} %% <- Punctuation after thm head
  {1pt} %% <- Space after thm head (or " " or \newline) (default: 5pt plus 1pt minus 1pt)
  {\vtop to 0pt{\llap{\thmname{#1}\hskip\marginlabelsep}
                \llap{\thmnumber{#2}\hskip\marginlabelsep}}\thmnote{#3\\}%
  }
\theoremstyle{italicstyle}
\newtheorem{corollary}{Corollario}[section]
\newtheorem{notazione}{Notazione}[section]
\newtheorem{lemma}{Lemma}[section]
\newtheorem{definizione}{Definizione}[section]
\newtheorem{nota}{Nota}[section]
\newtheorem{exercise}{Esercizio}[section]
\theoremstyle{normStyle}
\newtheorem{exmp}{Esempio}[section]
\newtheorem{theorem}{Teorema}[section]
\newtheorem{proposizione}{Proposizione}[section]
\tcbuselibrary{listings,skins}
\newtcblisting{mylisting}[2][]{
    arc=0pt, outer arc=0pt,
    listing only, 
    title=#2,
    #1,
    listing options= {escapechar=|}
}
\newcommand{\myboxedtext}[2][rectangle,draw]{%
    \tikz[baseline=-0.6ex] \node [#1]{#2};}%

    \newcommand{\lstfont}[1]{\color{#1}\scriptsize\ttfamily}

    \lstset{
        language=[ANSI]C++,
        showstringspaces=false,
        backgroundcolor=\color{white!90},
        basicstyle=\lstfont{black},
        identifierstyle=\lstfont{purple},
        keywordstyle=\lstfont{magenta!40},
        numberstyle=\lstfont{white},
        stringstyle=\lstfont{cyan},
        commentstyle=\lstfont{yellow!30},
        emph={
            cudaMalloc, cudaFree,
            __global__, __shared__, __device__, __host__,
            __syncthreads,
        },
        emphstyle={\lstfont{green!60!white}},
        breaklines=true
    }
\usetikzlibrary{arrows.meta, 
    bending,
    calc, chains,
    positioning
    }
\newcommand{\cube}[7]% position (center), size, color, front edges (1/0), back edges (1/0),
{%                      label, y shift for the label
  \begin{scope}[shift={#1},scale=0.5]
    \draw[fill=#3] (-#2,-#2,-#2) -- (-#2,-#2, #2) -- ( #2,-#2, #2) --
                   ( #2, #2, #2) -- ( #2, #2,-#2) -- (-#2, #2,-#2)  -- cycle;
    \ifnum #4 = 1% if we need front edges
      \draw ( #2,-#2,-#2) -- (-#2,-#2,-#2);
      \draw ( #2,-#2,-#2) -- ( #2, #2,-#2);
      \draw ( #2,-#2,-#2) -- ( #2,-#2, #2);    
    \fi
    \ifnum #5 = 1% if we need back edges
      \draw (-#2, #2, #2) -- ( #2, #2, #2);
      \draw (-#2, #2, #2) -- (-#2,-#2, #2);
      \draw (-#2, #2, #2) -- (-#2, #2,-#2);
    \fi
    \node[blue] at (0,-2*#7,0) {#6};
  \end{scope}
}
%%======================================================================
\usepackage{xcolor}
\usepackage{listings}

\usepackage{color}
\definecolor{backgroundColour}{RGB}{245,245,245}
\definecolor{mygreen}{RGB}{28,172,0} % Green for comments
\definecolor{mGray}{RGB}{128,128,128}
\definecolor{mPurple}{RGB}{128,0,128}
\definecolor{babyblue}{RGB}{137,207,240}
\definecolor{green}{RGB}{0,128,0}

% Define a custom color
\definecolor{backcolour}{rgb}{0.95,0.95,0.92}
\definecolor{codegreen}{rgb}{0,0.6,0}

% Define a custom style
\lstdefinestyle{myStyle}{
    backgroundcolor=\color{backcolour},   
    commentstyle=\color{codegreen},
    basicstyle=\ttfamily\footnotesize,
    breakatwhitespace=false,         
    breaklines=true,                 
    keepspaces=true,                 
    numbers=left,       
    numbersep=5pt,                  
    showspaces=false,                
    showstringspaces=false,
    showtabs=false,                  
    tabsize=2,
}
\usetikzlibrary{matrix,arrows,decorations.pathmorphing}
% l' unite
\newcommand{\myunit}{1 cm}
\tikzset{
    node style sp/.style={draw,circle,minimum size=\myunit},
    node style ge/.style={circle,minimum size=\myunit},
    arrow style mul/.style={draw,sloped,midway,fill=white},
    arrow style plus/.style={midway,sloped,fill=white},
}

% Use \lstset to make myStyle the global default
\lstset{style=myStyle}
%%======================================================================
\title{Programmazione Parallela}
\author{\textit{Alessio Gjergji}}
\date{}
\begin{document}
\begin{titlingpage}
  \centering
  \vspace*{\stretch{1}}
  \huge
  \textbf{\thetitle}\\[0.5cm]
  \normalsize
  Corso tenuto dal Professor Nicola Bombieri\\[0.5cm]
  Università di Verona\\[1cm]
  \large
  \theauthor\\[0.5cm]
  \vspace{\stretch{2}}
\end{titlingpage}
\tableofcontents
\chapter{Introduzione}
\section{Cos'è la programmazione parallela?}
Tradizionalmente, il software è stato scritto per delle 
computazioni sequenziali. Questo significa che le istruzioni
sono eseguite una dopo l'altra, in un ordine ben definito.
Le applicazioni, quindi, venivano eseguite su un singolo
computer con una singola unità centrale di elaborazione (\texttt{CPU}).

Nel senso più elementare, il calcolo parallelo consiste nell'utilizzo
simultaneo di diverse risorse di elaborazione
per affrontare un problema computazionale. Questo processo prevede:
\begin{itemize}
    \item \textbf{L'impiego di più unità di elaborazione (\texttt{CPU})}: per distribuire
    l'esecuzione del compito
    su diversi processori, accelerando così il tempo di elaborazione.
    \item \textbf{La divisione del problema in parti
    discrete}: ogni problema viene scomposto in segmenti più piccoli che
    possono essere
    processati in parallelo, ovvero contemporaneamente, su diverse \texttt{CPU}.
    \item \textbf{La suddivisione di ogni parte in una serie di istruzioni}:
    ciascuna frazione del problema viene poi ulteriormente frammentata in istruzioni
    specifiche,
    che definiscono esattamente cosa deve essere fatto.
    \item \textbf{L'esecuzione simultanea delle istruzioni su differenti \texttt{CPU}}: le
    istruzioni appartenenti a segmenti differenti del problema vengono
    eseguite nello stesso momento ma su processori distinti,
    permettendo così una soluzione più rapida del problema complessivo.
\end{itemize}
Questo approccio sfrutta al massimo le capacità delle moderne architetture
informatiche, permettendo di risolvere problemi complessi in tempi
significativamente ridotti rispetto al calcolo sequenziale,
dove le istruzioni vengono eseguite una dopo l'altra su un'unica \texttt{CPU}.

\section{Perché la programmazione parallela?}

La \textbf{computazione parallela} sfrutta l'uso simultaneo di molteplici
risorse di calcolo per risolvere problemi computazionali. Questo approccio
offre diversi vantaggi significativi, tra cui:
\begin{itemize}
    \item \textbf{Risparmio di Tempo e Denaro:} La distribuzione di un compito su più CPU può ridurre il tempo di completamento e i costi.
    \item \textbf{Risolvere Problemi Più Grandi:} Alcuni problemi sono troppo grandi o complessi per essere gestiti da un singolo computer.
    \item \textbf{Concorrenza:} Diverse risorse di calcolo permettono di eseguire molteplici operazioni in parallelo.
    \item \textbf{Uso di Risorse Non Locali:} L'accesso a risorse di calcolo su reti geografiche estese o su Internet consente di superare le limitazioni delle risorse locali.
\end{itemize}

\subsection{Limiti del Calcolo Seriale}
Il calcolo seriale presenta limiti fisici e pratici, tra cui:
\begin{itemize}
    \item \textbf{Velocità di Trasmissione:} I limiti alla velocità di trasmissione dei dati impongono un tetto alle prestazioni dei computer seriali.
    \item \textbf{Limiti alla Miniaturizzazione:} Esiste un limite fisico a quanto possano essere piccoli i componenti di un processore.
    \item \textbf{Limitazioni Economiche:} Aumentare la velocità di un singolo processore è progressivamente più costoso.
    \item \textbf{Consumo Energetico:} I core paralleli tendono a consumare meno energia rispetto a un equivalente core sequenziale.
\end{itemize}

\subsection{Tendenze e Futuro del Calcolo Parallelo}
L'evoluzione delle architetture informatiche evidenzia un crescente affidamento sul parallelismo hardware, attraverso:
\begin{itemize}
    \item Unità di esecuzione multiple
    \item Istruzioni in pipeline
    \item Processori Multi-core e Many-core
\end{itemize}
Queste tendenze confermano che il futuro del calcolo è orientato verso
il parallelismo.
\section{Concetti di base e terminologia}
\subsection{Tipologie di Computer e Sistemi}
Ci sono diversi tipi di computer e sistemi che possono
essere utilizzati per eseguire applicazioni parallele, tra cui:
\subsubsection{Computer Desktop}
I computer desktop sono sistemi personali comunemente usati in ambienti domestici e uffici per svariate applicazioni, da quelle produttive a quelle di intrattenimento.

\subsubsection{Computer Embedded}
I computer embedded sono sistemi specializzati progettati per eseguire compiti specifici all'interno di dispositivi più grandi, come automobili, elettrodomestici e sistemi di controllo industriale.

\subsubsection{Internet of Things (\texttt{IoT})}
\begin{itemize}
    \item \textbf{Computer Embedded Collegati a Internet:} Dispositivi embedded che sono connessi a Internet per fornire funzionalità avanzate, come il monitoraggio remoto e il controllo.
    \item \textbf{Sistemi Smart:} L'\texttt{IoT} abilita la creazione di sistemi intelligenti che combinano sensori, attuatori e connettività per interagire con il mondo fisico in modi avanzati.
\end{itemize}

\subsubsection{Dispositivi Mobili Personali (\texttt{PMDs})}
Include smartphone, tablet e altri dispositivi portatili che forniscono una vasta gamma di funzionalità, dalla comunicazione all'accesso a Internet e applicazioni specializzate.

\subsubsection{Server}
Potenti computer progettati per gestire richieste di dati e servizi da parte di altri computer e dispositivi all'interno di reti aziendali e su Internet.

\subsubsection{Cluster e Computer su Scala di Magazzino}
\begin{itemize}
    \item \textbf{Cluster:} Insiemi di computer connessi che lavorano insieme come un'unica entità per fornire elevate prestazioni di calcolo e disponibilità.
    \item \textbf{Computer su Scala di Magazzino:} Grandi infrastrutture informatiche che supportano applicazioni di cloud computing e servizi Internet su larga scala.
    \item \textbf{Supercomputer vs. Cluster:} Mentre i supercomputer sono sistemi altamente specializzati per compiti di calcolo intensivo, i cluster rappresentano un approccio più scalabile e flessibile al calcolo ad alte prestazioni.
\end{itemize}

\subsection{La Struttura von Neumann}

La struttura von Neumann, che prende il nome dal matematico
ungherese John von Neumann, rappresenta il modello di base seguito
dalla maggior parte dei computer moderni. Questo modello fu descritto
per la prima volta nei documenti del $1945$, evidenziando i requisiti
generali per un computer elettronico. A differenza dei primi computer,
programmati attraverso un cablaggio fisso, la struttura von Neumann
introduce un design flessibile e potente.

La struttura è composta da quattro componenti principali:

\begin{enumerate}
    \item \textbf{Memoria:} Serve per memorizzare le istruzioni del programma
    e i dati. Le istruzioni sono codificate per dire al computer cosa fare,
    mentre i dati sono le informazioni elaborate dal programma.
    \item \textbf{Unità di Controllo:} Preleva le istruzioni e i dati dalla
    memoria, decodifica le istruzioni e coordina le operazioni per eseguire
    il compito programmato.
    \item \textbf{Unità Logica Aritmetica (\texttt{ALU}):} Esegue le operazioni
    aritmetiche
    e logiche di base.
    \item \textbf{Input/Output:} Funge da interfaccia tra il computer e l'utente,
    permettendo l'ingresso e l'uscita dei dati.
\end{enumerate}

La memoria a accesso casuale (\texttt{RAM}), che permette sia la lettura che la scrittura,
è fondamentale in questa architettura per la memorizzazione sia delle istruzioni
che dei dati necessari per l'esecuzione del programma.

\subsection{Tassonomia di Flynn}
La tassonomia di Flynn è un sistema di classificazione per le architetture dei computer multi-processore, basato sul numero di flussi di istruzioni e dati che possono gestire in parallelo. Utilizza due dimensioni: Istruzione e Dati, ognuna delle quali può essere Singola o Multipla. Questo porta a quattro possibili classificazioni nella tassonomia di Flynn, che forniscono un quadro di riferimento per comprendere le diverse modalità di calcolo parallelo.

\section{Tassonomia di Flynn}
La tassonomia di Flynn classifica le architetture di calcolo parallelo
basandosi su due dimensioni: il numero di flussi di istruzioni
e il numero di flussi di dati che il sistema può gestire.
Ogni dimensione può essere Singola o Multipla,
portando a quattro categorie principali:

\begin{itemize}
    \item \textbf{\texttt{SISD} (Single Instruction, Single Data)}: un processore esegue un flusso di istruzioni su un flusso di dati.
    \item \textbf{\texttt{SIMD} (Single Instruction, Multiple Data)}: un'istruzione controlla simultaneamente più operazioni su diversi flussi di dati.
    \item \textbf{\texttt{MISD} (Multiple Instruction, Single Data)}: più istruzioni operano su un singolo flusso di dati, utilizzato raramente.
    \item \textbf{\texttt{MIMD} (Multiple Instruction, Multiple Data)}: più processori eseguono istruzioni diverse su flussi di dati diversi, comunemente usato per applicazioni parallele general-purpose.
\end{itemize}

In un sistema di elaborazione, l'esecuzione di programmi e la gestione dei dati
sono basate su cinque componenti chiave, che insieme formano il cuore funzionale
di qualsiasi computer moderno:

\begin{itemize}
    \item \texttt{IS} (Instruction Stream): il flusso di istruzioni, ovvero le operazioni
    che il sistema deve eseguire, organizzate in sequenza.
    \item \texttt{DS} (Data Stream): il flusso di dati comprende gli operandi sui quali
    operano le istruzioni e i risultati di tali operazioni.
    \item \texttt{CU} (Control Unit): l'unità di controllo, che si occupa di prelevare le
    istruzioni dalla memoria, decodificarle e coordinare l'esecuzione.
    \item \texttt{PU} (Processing Unit): l'unità di elaborazione, costituita dall'\texttt{ALU}
    (\textit{Arithmetic Logic Unit}) e dai registri, esegue le istruzioni operative.
    \item \texttt{MM} (Main Memory): la memoria principale, dove vengono allocati i
    dati e le istruzioni necessari per l'esecuzione di un programma.
\end{itemize}

Questi componenti interagiscono tra loro per processare efficacemente
i dati e le istruzioni, permettendo al sistema di eseguire una vasta
gamma di compiti.
\subsection{Single instruction, single data - \texttt{SISD}}
Nella struttura di Von Neumann, l'Unità di Controllo (\texttt{CU}) ha il
compito di prelevare le istruzioni dalla Memoria Principale (\texttt{MM}), mentre
l'Unità di Elaborazione (\texttt{PU}) esegue tali istruzioni interagendo con la \texttt{MM}
per modificare i dati. Questo schema rappresenta il funzionamento base della
struttura di Von Neumann, in cui un singolo programma è in esecuzione e si basa
su un unico flusso di dati.

La \texttt{CU} coordina il processo di esecuzione leggendo sequenzialmente
le istruzioni dal programma memorizzato nella \texttt{MM}, decodificandole e trasferendole
alla \texttt{PU} per la loro esecuzione. La \texttt{PU}, a sua volta, esegue le operazioni aritmetiche
e logiche specificate dalle istruzioni, utilizzando i dati memorizzati nella \texttt{MM}.
Questo processo iterativo tra \texttt{PU}, \texttt{PU} e \texttt{MM} permette l'elaborazione dei programmi
secondo il modello di flusso di dati e di controllo definito dalla struttura di
Von Neumann.

Un computer seriale (\textit{non parallelo}) si caratterizza per il singolo flusso di
istruzioni e dati. Durante ogni ciclo di clock, la \texttt{CPU} elabora:

\begin{itemize}
    \item \textbf{Singola istruzione:} Viene processato solo un flusso di istruzioni.
    \item \textbf{Singolo dato:} Viene utilizzato come input un solo flusso di dati.
\end{itemize}

Questo comporta un'esecuzione deterministica, in cui il risultato del calcolo
è direttamente determinato dall'algoritmo e dai dati in ingresso. Esempi comuni
di computer seriali includono mainframe di vecchia generazione, minicomputer
e workstation.

\subsection{Single instruction, multiple data - \texttt{SIMD}}
Un tipo di computer parallelo conosciuto come \texttt{SIMD} (\textit{Single Instruction, 
Multiple Data}) possiede le seguenti caratteristiche:

\begin{itemize}
    \item \textbf{Singola istruzione:} Tutte le unità di elaborazione
    eseguono la stessa istruzione in qualsiasi ciclo di clock.
    \item \textbf{Dati multipli:} Ogni unità di elaborazione può operare
    su un elemento di dati diverso.
\end{itemize}

Questa architettura è particolarmente adatta per problemi specializzati
caratterizzati da un alto grado di regolarità, come l'elaborazione di grafica
e immagini. L'esecuzione è sincrona (\textit{lockstep}) e deterministica.

La maggior parte dei computer moderni, in particolare quelli dotati di unità
di elaborazione grafiche (\texttt{GPU}), impiega istruzioni \texttt{SIMD} e
unità di esecuzione.

\subsection{Multiple instruction, single data - \texttt{MISD}}
Un computer parallelo \texttt{MISD} (\textit{Multiple Instruction, Single Data})
è caratterizzato da:

\begin{itemize}
    \item Un singolo flusso di dati che viene elaborato da più unità di elaborazione.
    \item Ogni unità di elaborazione processa i dati in modo indipendente
    attraverso propri flussi di istruzioni.
\end{itemize}

Questa architettura è raramente utilizzata, poiché è difficile da implementare
e non offre vantaggi significativi rispetto ad altre architetture parallele.

\subsection{Multiple instruction, multiple data - \texttt{MIMD}}
Il computer parallelo di tipo \texttt{MIMD} (\textit{Multiple Instruction, Multiple Data})
è attualmente la forma più comune di calcolo parallelo e la maggior parte
dei computer moderni rientra in questa categoria. Le caratteristiche distintive
sono:

\begin{itemize}
    \item \textbf{Istruzioni Multiple:} ogni processore può eseguire un flusso
    di istruzioni diverso.
    \item \textbf{Dati Multipli:} ogni processore può lavorare con un
    proprio flusso di dati.
    \item L'esecuzione può essere sincrona o asincrona, deterministica o
    non deterministica.
\end{itemize}

Esempi di questa architettura includono la maggior parte dei supercomputer attuali,
i cluster di computer paralleli connessi in rete e le ``griglie'' di calcolo,
i computer \texttt{SMP} (\textit{Symmetric Multi-Processing}) con più processori
e i \texttt{PC} multi-core.
\begin{nota}
    Molte architetture \texttt{MIMD} includono anche sottocomponenti di esecuzione
    \texttt{SIMD}.
\end{nota}

\section{Concetti di Esecuzione}

\subsection{Task}
Un task è un'unità di lavoro computazionale che corrisponde a un programma o a una sequenza di istruzioni eseguite da un processore. Ogni task è progettato per completare una parte specifica del lavoro generale richiesto dal programma completo.

\subsection{Esecuzione Seriale}
L'esecuzione seriale implica il processamento di istruzioni una alla volta, in una sequenza ordinata. Questo metodo di esecuzione è tipico dei computer con un singolo processore e si basa su un modello computazionale che non prevede l'esecuzione contemporanea di più istruzioni o task.

\paragraph{Implicazioni dell'Esecuzione Seriale}
Questo tipo di esecuzione è caratterizzato da un flusso di lavoro prevedibile e da una facile individuazione e risoluzione degli errori. È particolarmente efficace in applicazioni dove le operazioni devono essere svolte in una sequenza specifica e dove le istruzioni successive dipendono dai risultati di quelle precedenti.

\subsection{Esecuzione Parallela}
Contrariamente all'esecuzione seriale, l'esecuzione parallela permette a più task di essere eseguiti simultaneamente. Questo approccio sfrutta l'architettura dei computer multi-processore per ridurre il tempo totale di elaborazione.

\paragraph{Benefici dell'Esecuzione Parallela}
L'abilità di eseguire più task contemporaneamente porta a una riduzione significativa del tempo di esecuzione per i problemi che possono essere suddivisi in parti indipendenti, ottimizzando l'uso delle risorse di elaborazione disponibili.

\subsection{Pipelining}
Il pipelining è un'efficace strategia di esecuzione parallela in cui un task è diviso in diverse fasi. Ogni fase è elaborata da una diversa unità di processamento, permettendo un flusso continuo di esecuzione simile a quello di una catena di montaggio industriale.

\paragraph{Efficienza del Pipelining}
Attraverso il pipelining, diverse fasi di un processo possono essere eseguite simultaneamente, migliorando l'efficienza e la velocità complessive del sistema di elaborazione.

\section{Memoria nei Sistemi Paralleli}

\subsection{Memoria Condivisa}
In architetture con memoria condivisa, tutti i processori accedono a una memoria fisica comune, consentendo una comunicazione e sincronizzazione efficienti tra i task. Tuttavia, questo modello può comportare dei collo di bottiglia dovuti alla competizione per l'accesso alla memoria.

\subsection{Memoria Distribuita}
Nei sistemi con memoria distribuita, ciascun processore accede a una propria memoria locale. Questo approccio migliora la scalabilità del sistema ma richiede meccanismi di comunicazione complessi per coordinare i task distribuiti su diversi processori.

\section{Comunicazione e Sincronizzazione}

\subsection{Comunicazione}
La comunicazione tra i task è fondamentale in un ambiente di calcolo parallelo. I meccanismi di comunicazione variano a seconda dell'architettura e possono includere l'uso di bus di memoria condivisa o reti di comunicazione.

\subsection{Sincronizzazione}
Per mantenere la coerenza e l'ordine nell'esecuzione parallela, i task devono sincronizzarsi periodicamente. Ciò è spesso realizzato attraverso punti di sincronizzazione nel programma, dove ogni task deve attendere gli altri prima di procedere.

\section{Prestazioni e Scalabilità}

\subsection{Granularità}
La granularità nel calcolo parallelo descrive il livello di suddivisione del lavoro computazionale e ha un impatto diretto sull'equilibrio tra calcolo e comunicazione. La granularità fine potrebbe richiedere una comunicazione più frequente, mentre quella grossolana meno frequente.

\subsection{Speedup e Overhead Parallelo}
Lo speedup misura l'efficacia dell'esecuzione parallela rispetto a quella seriale. L'overhead parallelo, che include il tempo di avvio dei task, le sincronizzazioni e le comunicazioni, può influenzare negativamente questo indicatore di prestazione.

\[
    \textit{Speedup} = \frac{\textit{Tempo di esecuzione seriale}}{\textit{Tempo di esecuzione parallelo}}
\]

\section{Architetture e Computazione Parallela}

\subsection{Processori Multi-core}
I processori multi-core contengono più core di elaborazione in un unico chip, permettendo l'esecuzione parallela di task su un singolo dispositivo fisico.

\subsection{Cluster Computing}
Il cluster computing utilizza un insieme di unità di calcolo, spesso commerciali, configurate per lavorare insieme come un unico sistema parallelo.

\subsection{Supercomputing}
Il supercomputing si basa sull'uso di computer ad alte prestazioni per affrontare problemi computazionali di grande scala, dove la velocità e la capacità di elaborazione sono essenziali.

\subsection{Edge Computing}
L'edge computing mira a portare la potenza di calcolo e la memorizzazione dei dati più vicino al punto di necessità, riducendo i tempi di risposta e il consumo di banda.

\section{Memoria Condivisa}

I computer paralleli a memoria condivisa presentano diverse caratteristiche,
ma in generale condividono la capacità per tutti i processori di accedere a
tutta la memoria come uno spazio di indirizzamento globale.

\begin{itemize}
    \item I processori multipli possono operare in modo indipendente, ma condividono
    le stesse risorse di memoria.
    \item Le modifiche in una posizione di memoria effettuate da un processore sono
    visibili a tutti gli altri processori.
    \item Le macchine a memoria condivisa possono essere suddivise in due classi
    principali in base ai tempi di accesso alla memoria: \texttt{UMA}
    (\textit{Uniform Memory Access})
    e \texttt{NUMA} (\textit{Non-Uniform Memory Access}).
\end{itemize}

\subsection{\texttt{UMA} (\textit{Uniform Memory Access})}
Le architetture \texttt{UMA} sono comunemente rappresentate oggi dalle
macchine Symmetric Multiprocessor (\texttt{SMP}), caratterizzate da processori
identici e accesso uniforme alla memoria con tempi di accesso uguali. Questo
modello è noto anche come \texttt{CC-UMA} (\textit{Cache Coherent \texttt{UMA}}),
dove la coerenza della
cache indica che se un processore aggiorna una posizione nella memoria condivisa,
tutti gli altri processori vengono informati dell'aggiornamento. La coerenza della
cache è ottenuta a livello hardware.

\paragraph{Vantaggi e Svantaggi}
\begin{itemize}
    \item \textbf{Vantaggi:} Lo spazio di indirizzamento globale offre una prospettiva
    di programmazione user-friendly per la memoria. La condivisione dei dati tra i
    task è sia rapida che uniforme grazie alla prossimità della memoria ai \texttt{CPU}.
    \item \textbf{Svantaggi:} Il principale svantaggio è la mancanza di scalabilità
    tra memoria e \texttt{CPU}. Aggiungere più \texttt{CPU} può aumentare
    geometricamente il traffico sul percorso memoria-\texttt{CPU} condiviso e, per i
    sistemi con coerenza della cache, aumentare geometricamente il traffico
    associato alla gestione della cache/memoria. È responsabilità del programmatore
    utilizzare costrutti di sincronizzazione che assicurino un accesso ``corretto''
    alla memoria globale. Inoltre, diventa sempre più difficile e costoso progettare
    e produrre macchine a memoria condivisa con un numero crescente di processori.
\end{itemize}

\subsection{\texttt{NUMA} (\textit{Non-Uniform Memory Access})}
Le architetture \texttt{NUMA} sono spesso realizzate collegando fisicamente due
o più \texttt{SMP}. Un \texttt{SMP} può accedere direttamente alla memoria di un altro \texttt{SMP},
ma non tutti i processori hanno tempi di accesso uguali a tutte le memorie.
L'accesso alla memoria attraverso il collegamento è più lento. Se la coerenza
della cache è mantenuta, queste architetture possono anche essere chiamate
\texttt{CC-NUMA}
(\textit{Cache Coherent \texttt{NUMA}}).

\paragraph{Vantaggi e Svantaggi}
\begin{itemize}
    \item \textbf{Vantaggi:} Similmente a \texttt{UMA}, \texttt{NUMA} offre
    uno spazio di indirizzamento globale che facilita la programmazione e la
    condivisione dei dati tra i task. La struttura di \texttt{NUMA} permette
    una migliore scalabilità rispetto a \texttt{UMA} quando si aggiungono
    processori, grazie alla distribuzione della memoria tra i vari \texttt{SMP}.
    \item \textbf{Svantaggi:} L'accesso non uniforme alla memoria può portare
    a prestazioni inconsistenti, specialmente in carichi di lavoro che richiedono
    un accesso frequente alla memoria attraverso i collegamenti \texttt{SMP}. La gestione
    della coerenza della cache, sebbene fornisca una visione coerente della memoria,
    può introdurre overhead significativo, specialmente in sistemi di grande
    dimensione.
\end{itemize}

\section{Memoria Distribuita}
Come i sistemi a memoria condivisa, anche quelli a memoria distribuita
variano notevolmente ma condividono una caratteristica comune: richiedono
una rete di comunicazione per connettere la memoria tra i vari processori.
In questi sistemi, ogni processore dispone di una propria memoria locale e
gli indirizzi di memoria in un processore non sono mappati su un altro processore,
eliminando così il concetto di spazio di indirizzamento globale.

\subsection{Funzionamento Indipendente e Coerenza della Cache}
Poiché ogni processore ha la propria memoria locale, opera indipendentemente.
Le modifiche che effettua nella sua memoria locale non influenzano la memoria
degli altri processori, rendendo inapplicabile il concetto di coerenza della cache.
Quando un processore necessita di accedere ai dati in un altro processore, spesso
è compito del programmatore definire esplicitamente come e quando i dati vengono
comunicati. Anche la sincronizzazione tra i task è responsabilità del programmatore.

\subsection{Tessuto di Rete}
Il ``tessuto'' di rete utilizzato per il trasferimento dei dati varia ampiamente,
sebbene possa essere semplice come Ethernet.
\subsubsection{Vantaggi e svantaggi}
\begin{itemize}
    \item \textbf{Vantaggi} 
    \begin{itemize}
        \item La memoria è scalabile con il numero di processori. Aumentando
        il numero di processori, la dimensione della memoria aumenta proporzionalmente.
        \item Ogni processore può accedere rapidamente alla propria memoria senza
        interferenze e senza l'overhead necessario per mantenere la coerenza della
        cache.
        \item Costo-efficacia: è possibile utilizzare processori e reti commerciali.
    \end{itemize}
    \item \textbf{Svantaggi}
    \begin{itemize}
        \item Il programmatore è responsabile di molti dettagli associati alla
        comunicazione dei dati tra i processori.
        \item Può essere difficile mappare le strutture dati esistenti, basate
        sulla memoria globale, a questa organizzazione della memoria.
        \item Tempi di accesso alla memoria non uniformi (\texttt{NUMA}).
    \end{itemize}
\end{itemize}
\section{Memorie ibride, distribuite e condivise}

I supercomputer più grandi e veloci al mondo oggi impiegano architetture ibride che combinano elementi di memoria condivisa e memoria distribuita.

\subsection{Componente di Memoria Condivisa}
La componente di memoria condivisa è solitamente costituita da una macchina
\texttt{SMP} (\textit{Symmetric Multiprocessing}) con coerenza della cache.
I processori all'interno di un dato \texttt{SMP} possono indirizzare la memoria della
macchina come se fosse globale, permettendo un accesso rapido e efficiente ai
dati condivisi.

\subsection{Componente di Memoria Distribuita}
La componente di memoria distribuita si realizza tramite il collegamento
in rete di più macchine \texttt{SMP}. Ogni \texttt{SMP} è a conoscenza soltanto della propria
memoria e non di quella presente su un altro \texttt{SMP}. Di conseguenza, sono necessarie
comunicazioni di rete per spostare i dati da un \texttt{SMP} all'altro.

\paragraph{Tendenze Attuali}
Le tendenze attuali sembrano indicare che questo tipo di architettura di
memoria continuerà a prevalere e ad espandersi nell'alta fascia del calcolo
per il futuro prevedibile. L'approccio ibrido offre il meglio di entrambi i
mondi: l'efficienza e la facilità di programmazione della memoria condivisa
e la scalabilità e flessibilità della memoria distribuita.

\subsubsection{Vantaggi e svantaggi}
\begin{itemize}
    \item \textbf{Vantaggi} 
    \begin{itemize}
        \item \textbf{Scalabilità:} L'architettura ibrida permette ai supercomputer
        di scalare efficacemente aggiungendo più \texttt{SMP}, aumentando la potenza di
        calcolo e la memoria disponibile.
        \item \textbf{Flessibilità:} Gli sviluppatori possono ottimizzare
        le prestazioni sfruttando la memoria locale nei nodi \texttt{SMP} per l'accesso
        ad alta velocità e utilizzare la memoria distribuita per il lavoro
        collaborativo tra \texttt{SMP}.
        \item \textbf{Efficienza:} La combinazione di memoria condivisa e
        distribuita può migliorare l'efficienza complessiva del sistema,
        bilanciando carico di lavoro e comunicazioni di rete.
    \end{itemize}
    \item \textbf{Svantaggi}
    \begin{itemize}
        \item \textbf{Complessità:} La programmazione e la gestione di architetture
        ibride sono più complesse a causa della necessità di bilanciare l'uso
        di memoria condivisa e distribuita.
        \item \textbf{Costo:} La costruzione e manutenzione di supercomputer con
        architetture ibride possono essere costose, data la complessità del
        hardware e del software.
        \item \textbf{Coerenza dei Dati:} Mantenere la coerenza dei dati tra
        la memoria condivisa e quella distribuita può richiedere meccanismi di
        sincronizzazione avanzati, aggiungendo un ulteriore livello di complessità.
    \end{itemize}
\end{itemize}
\chapter{Modelli di programmazione parallela}
\section{Introduzione}
Ci sono diversi modelli di programmazione parallela,
ognuno con i propri vantaggi e svantaggi.
\begin{itemize}
  \item \textbf{Modelli di memoria condivisa}: i processi
    condividono un unico spazio di indirizzamento.
  \item \textbf{Modelli di memoria distribuita}: i processi
    hanno spazi di indirizzamento separati.
\end{itemize}
Sebbene possa sembrare, i modelli non sono legati 
ad una specifica architettura hardware, ma possono 
essere implementati (\textit{teoricamente}) su qualsiasi
architettura.

È importante notare che non c'è un modello migliore
rispetto ad un altro, ma dipende dal problema che si
vuole risolvere e dalle caratteristiche dell'architettura
hardware a disposizione.

\section{Modelli di memoria condivisa}
Nel \textbf{modello di programmazione a memoria condivisa},
i task condividono uno spazio di indirizzi comune, che
leggono e scrivono in modo asincrono. Esistono vari
meccanismi, come i lock o i semafori, utilizzati per
controllare l'accesso alla memoria condivisa. Da un punto
di vista del programmatore, un vantaggio di questo modello
è che manca la nozione di ``proprietà'' dei dati. Ciò
implica che:
\begin{itemize}
    \item Non è necessario specificare esplicitamente
    la comunicazione dei dati tra i task.
    \item Lo sviluppo del programma può spesso essere
    semplificato.
\end{itemize}

Tuttavia, un importante svantaggio, in termini di
prestazioni, è che diventa più difficile comprendere
e gestire la \textbf{località dei dati}. Mantenere i
dati locali al processore che ci lavora su conserva gli
accessi alla memoria, i refresh della cache e il traffico
sul bus che si verifica quando più processori utilizzano
gli stessi dati. Sfortunatamente, controllare la località
dei dati è difficile da capire ed è al di fuori del
controllo dell'utente medio.

\section{Modello a Thread}
Nel modello di programmazione parallela basato sui
\textbf{thread}, un singolo processo può avere più
percorsi di esecuzione concorrenti. Questa modalità
permette di eseguire diverse parti di un programma
in parallelo, aumentando l'efficienza e riducendo
il tempo di esecuzione.

\subsection{Analogia e Funzionamento}
Un'analogia semplice per descrivere i thread è il concetto
di un singolo programma che include un numero di subroutine:
\begin{itemize}
    \item Il programma principale \texttt{a.out} viene
    schedulato per l'esecuzione dal sistema operativo
    nativo. \texttt{a.out} carica e acquisisce tutte le
    risorse di sistema e utente necessarie per l'esecuzione.
    \item \texttt{a.out} esegue del lavoro seriale e poi
    crea un numero di task (\textit{thread}) che possono
    essere
    schedulati ed eseguiti contemporaneamente dal sistema
    operativo.
    \item Ogni thread ha dati locali, ma condivide anche
    tutte le risorse di \texttt{a.out}, risparmiando così
    l'overhead associato alla replicazione delle risorse
    del programma per ogni thread. Ogni thread beneficia
    anche di una visione globale della memoria perché
    condivide lo spazio di memoria di \texttt{a.out}.
    \item Il lavoro di un thread può essere descritto
    come una subroutine all'interno del programma
    principale. Qualsiasi thread può eseguire qualsiasi
    subroutine allo stesso tempo degli altri thread.
    \item I thread comunicano tra loro tramite la memoria
    globale (\textit{aggiornando le posizioni degli indirizzi}).
    Ciò richiede costrutti di sincronizzazione per
    assicurare che più di un thread non stia aggiornando
    lo stesso indirizzo globale contemporaneamente.
    \item I thread possono essere creati e terminati,
    ma \texttt{a.out} rimane presente per fornire le
    risorse condivise necessarie fino al completamento
    dell'applicazione.
\end{itemize}

\subsection{Implementazioni e Standardizzazione}
I thread sono comunemente associati con architetture
di memoria condivisa e sistemi operativi. Dal punto di
vista della programmazione, le implementazioni dei thread
comprendono comunemente:
\begin{itemize}
    \item Una libreria di subroutine che vengono chiamate
    all'interno del codice sorgente parallelo.
    \item Un insieme di direttive del compilatore
    integrate nel codice sorgente, sia seriale che
    parallelo.
\end{itemize}
In entrambi i casi, il programmatore è responsabile della
determinazione di tutto il parallelismo.

Le implementazioni basate su thread non sono una novità
nel campo dell'informatica. Storicamente, i fornitori
di hardware hanno implementato le loro versioni
proprietarie di thread, le quali differivano
sostanzialmente l'una dall'altra, rendendo difficile
per i programmatori sviluppare applicazioni threaded
portabili. Sforzi di standardizzazione non correlati
hanno risultato in due implementazioni molto diverse di
thread:
\begin{itemize}
    \item \texttt{POSIX Threads}
    \item \texttt{OpenMP}
\end{itemize}

\section{Modello Message Passing}
Il modello di passaggio di messaggi dimostra le seguenti
caratteristiche principali:

\begin{itemize}
    \item Un insieme di task che utilizzano la propria
    memoria locale durante il calcolo. Più task possono
    risiedere sulla stessa macchina fisica così come su
    un numero arbitrario di macchine.
    \item I task scambiano dati attraverso la
    comunicazione inviando e ricevendo messaggi.
    \item Il trasferimento di dati richiede di solito
    operazioni cooperative da eseguire da ciascun processo.
    Ad esempio, un'operazione di invio deve avere
    un'operazione di ricezione corrispondente.
\end{itemize}

Dal punto di vista della programmazione, le implementazioni
del passaggio di messaggi comprendono comunemente una
libreria di subroutine che sono incorporate nel codice
sorgente.

Il programmatore è responsabile della determinazione di
tutto il parallelismo.
\section{Modello di Parallelismo dei Dati}
Il modello di parallelismo dei dati dimostra le seguenti
caratteristiche principali:

\begin{itemize}
    \item La maggior parte del lavoro parallelo si
    concentra sull'esecuzione di operazioni su un
    insieme di dati. L'insieme di dati è tipicamente
    organizzato in una struttura comune, come un array
    o un cubo.
    \item Un insieme di task lavora collettivamente sulla
    stessa struttura di dati, tuttavia, ogni task lavora
    su una partizione diversa della stessa struttura di
    dati.
    \item I task eseguono la stessa operazione sulla loro
    partizione di lavoro, per esempio, ``aggiungere 4 a
    ogni elemento dell'array''.
\end{itemize}

Sulle architetture a memoria condivisa, tutti i task
possono avere accesso alla struttura di dati tramite
la memoria globale. Sulle architetture a memoria
distribuita, la struttura di dati è suddivisa e risiede
come "chunk" nella memoria locale di ogni task.

\subsection{Programmazione con il Modello di Parallelismo
dei Dati}
La programmazione con il modello di parallelismo dei dati
si realizza solitamente scrivendo un programma con
costrutti di parallelismo dei dati. I costrutti possono
essere chiamate a una libreria di subroutine di
parallelismo dei dati o direttive del compilatore
riconosciute da un compilatore di parallelismo dei dati.

\subsubsection{Direttive del Compilatore}
Permettono al programmatore di specificare la
distribuzione e l'allineamento dei dati. Le
implementazioni Fortran sono disponibili per le
piattaforme parallele più comuni.

\subsubsection{Implementazioni a Memoria Distribuita}
Le implementazioni di questo modello su memoria
distribuita di solito hanno il compilatore che converte
il programma in codice standard con chiamate a una
libreria di passaggio di messaggi (\textit{solitamente \texttt{MPI}})
per distribuire i dati a tutti i processi. Tutto il
passaggio di messaggi è invisibile al programmatore.

\section{Altri modelli di programmazione parallela}
Oltre ai modelli di programmazione parallela
precedentemente menzionati, esistono certamente
altri modelli, che continueranno a evolversi insieme
al mondo sempre in cambiamento dell'hardware e del
software per computer. Qui ne vengono menzionati
solamente tre tra i più comuni: ibrido, \texttt{SPMD}
(\textit{Single Program Multiple Data}) e \texttt{MPMD}
(\textit{Multiple Program Multiple Data}).

\subsection{Modello Ibrido}
In questo modello, vengono combinati due o più modelli
di programmazione parallela. Esempi comuni di modello
ibrido includono:
\begin{itemize}
    \item La combinazione del modello di passaggio di
    messaggi (\texttt{MPI}) con il modello dei thread
    (\textit{\texttt{POSIX} threads}) o il modello di
    memoria condivisa (\texttt{OpenMP}). Questo modello
    ibrido si presta bene all'ambiente hardware sempre
    più comune di macchine SMP in rete.
    \item La combinazione del parallelismo dei dati con
    il passaggio di messaggi. Come menzionato nella
    sezione relativa al modello di parallelismo dei dati,
    le implementazioni di parallelismo dei dati su
    architetture a memoria distribuita utilizzano
    effettivamente il passaggio di messaggi per
    trasmettere dati tra i task, in modo trasparente
    per il programmatore.
\end{itemize}

\subsection{Single Program Multiple Data (\texttt{SPMD})}
\begin{itemize}
    \item \texttt{SPMD} è un modello di programmazione
    ``di alto livello'' che può essere costruito su
    qualsiasi combinazione dei modelli di programmazione
    parallela precedentemente menzionati.
    \item Un singolo programma viene eseguito
    simultaneamente da tutti i task.
    \item In un dato momento, i task possono eseguire
    le stesse o diverse istruzioni all'interno dello
    stesso programma.
    \item A differenza di \texttt{SIMD}, in \texttt{SPMD},
    processori autonomi eseguono simultaneamente lo stesso
    programma in punti indipendenti, anziché in modo
    sincronizzato come impone \texttt{SIMD} su dati
    diversi.
    \item I programmi \texttt{SPMD} di solito hanno la
    logica necessaria programmata per permettere ai
    diversi task di eseguire condizionalmente solo
    quelle parti del programma che sono progettati per
    eseguire.
\end{itemize}

\subsection{Multiple Program Multiple Data (\texttt{MPMD})}
\begin{itemize}
    \item Come \texttt{SPMD}, anche \texttt{MPMD} è un
    modello di programmazione ``di alto livello'' che può
    essere basato su qualsiasi combinazione dei modelli
    di programmazione parallela menzionati.
    \item Le applicazioni \texttt{MPMD} tipicamente hanno
    più file oggetto eseguibili (\textit{programmi}). Mentre
    l'applicazione viene eseguita in parallelo, ciascun
    task può eseguire lo stesso programma o programmi
    diversi rispetto agli altri task.
\end{itemize}

\chapter{Misurazione delle Performance}

Ci sono due punti di vista nella misurazione delle
performance:
\begin{itemize}
  \item \textbf{Utente}: tempo di risposta, throughput,
  tempo di completamento.
  \item \textbf{Sviluppatore}: tempo di esecuzione, tempo
  di CPU, tempo di I/O, tempo di comunicazione.
\end{itemize}

Il \textbf{tempo di risposta} è il tempo che intercorre
tra la partenza del codice e la fine dell'esecuzione. Il
\textbf{throughput} è la quantità di lavoro fatto dal
sistema in un determinato periodo di tempo (\textit{si considera
l'ammontare di lavoro}). La \textbf{latenza} di
un'istruzione, nel caso della pipeline, è il tempo che
intercorre tra l'arrivo di un'istruzione e la sua completa
esecuzione.

\section{Benchmark}

Quando misuriamo le performance di un sistema, dobbiamo
capire come misurarle in termini di architettura. Un
\textbf{benchmark} è un programma che misura le
performance di un sistema. I benchmark possono essere:
\begin{itemize}
  \item \textbf{Sintetici}: programmi che eseguono
  operazioni tipiche di un'applicazione e stimolano
  quindi certi comportamenti dell'architettura.
  \item \textbf{Kernel}: frammenti di codice che
  rappresentano operazioni fondamentali e critiche per
  la performance di sistemi computazionali.
  \item \textbf{Toy}: programmi molto semplici che
  eseguono operazioni elementari.
  \item \textbf{Suits}: insiemi di benchmark che
  permettono una valutazione complessiva delle performance
  di un sistema.
\end{itemize}

Sul piatto della bilancia ci sono le \textbf{performance}
e il \textbf{costo}. Generalmente, si esegue un benchmark
più volte e si calcola la media dei risultati ottenuti
per ottenere una misura affidabile delle performance.

\section{Principi Quantitativi}

La prima cosa da fare per migliorare le performance
di un sistema è capire come parallelizzare un codice
sequenziale. Innanzitutto, si inizia analizzando e
ottimizzando la parte di codice che viene eseguita più
frequentemente, noto come \textit{hot spot} del codice.
\subsection{Legge di Amdahl}

La legge di Amdahl ci dice che la velocità di un sistemap
arallelo è limitata dalla frazione sequenziale del codice.
Tale legge ci aiuta a comprendere quanto possiamo
migliorare le performance di un sistema. Non ci
interessano le righe di codice, ma le funzioni che
vengono eseguite più frequentemente. Per identificarle,
generalmente vengono utilizzati dei \textit{profiler}.

La formula della legge di Amdahl per il calcolo dello
speedup \(S(n)\) è:
\[
  S(n) = \frac{\texttt{ExecutionTime}_{\texttt{old}}}
  {\texttt{ExecutionTime}_{\texttt{new}}}
  = \frac{1}{(1-\texttt{Fraction}_{\texttt{improved}})
  + \frac{\texttt{Fraction}_{\texttt{improved}}}
  {\texttt{Speedup}_{\texttt{improved}}}}
\]

Di natura, la legge di Amdahl ci indica che non possiamo
migliorare le performance di un sistema in maniera
illimitata. Dovremo quindi cercare di parallelizzare
solamente le parti che presentano un potenziale
parallelismo.

Ne segue che lo speedup globale può essere calcolato come:
\[
  S(n) = \frac{1}{(1-\texttt{Fraction}_{\texttt{improved}})
  + \frac{\texttt{Fraction}_{\texttt{improved}}}{n}} \leq 
    \frac{1}{(1-\texttt{Fraction}_{\texttt{improved}})}
\]
Ovvero, lo speedup è limitato dalla frazione sequenziale
del codice. Con il \(50\%\) di codice parallelo,
lo speedup massimo è \(2\).

\subsubsection{Esempio}

Supponiamo che una parte di un programma possa essere
migliorata di 10 volte e che questa parte sia eseguita
nel \(40\%\) del tempo totale. Lo speedup massimo sarà:
\[
  S(n) = \frac{1}{(1-0.4) + \frac{0.4}{10}} =
  \frac{1}{0.6 + 0.04} = \frac{1}{0.64} = 1.56
\]

\section{Il tempo è un'unità di misura}
Il tempo di esecuzione di un programma è una misura
delle performance. 
Nel tempo di \texttt{CPU} non si considera il tempo di
I/O, mentre nel tempo di risposta si considera anche il 
tempo di latenza di accesso al disco.

Un altro parametro da considerare è il numero di istruzioni
del programma e il numero di cicli di clock per queste istruzioni,
ovvero il \texttt{CPI}.
\[
  \texttt{CPI} = \frac{\texttt{ClockCycles}}{\texttt{Instructions}}  
\]
\[
  \texttt{CPU time} = \texttt{Instructions} \cdot \texttt{CPI} \cdot \texttt{clock cycle time}
  = \frac{\texttt{Instructions} \cdot \texttt{CPI}}{\texttt{clock frequency}}
\]
\[
  \texttt{CPU time} = \texttt{CI} \cdot \texttt{CPI} \cdot \texttt{T}_\texttt{clock}
\]
Il tempo di \texttt{CPU} dipende da tre fattori:
\begin{itemize}
  \item \textbf{Cicli di clock (\textit{o frequenza})}: dipende 
  dall'architettura del processore.
  \item \texttt{CPI}: dipende dall'organizzazione dell'architettura e 
  dall'insieme di istruzioni.
  \item \textbf{Numero di istruzioni}: dipende dall'insieme di istruzioni
  e dalla tecnologia di compilazione.
\end{itemize}
\section{\texttt{MIPS} o \texttt{GIPS}}
Il \texttt{MIPS} (\textit{Million Instructions Per Second}) è una
misura delle performance di un processore, legata al \textit{throughput}.
Tante volte quando si usano queste unità di misura, in alcuni casi, 
sono controituitive, non viene fatta una distinzione tra istruzioni complesse 
e istruzioni semplici.

Nasce quindi il \texttt{GFLOPS} (\textit{Giga Floating Point Operations Per Second}),
che misura il numero di operazioni in virgola mobile eseguite in un secondo.

\[
  \texttt{GFLOPS} = \frac{\texttt{Numero di operazioni floating point nel programma}}{10^9}
\]
Il problema di queste unità di misura è che non tengono conto
della complessità delle operazioni.
Usiamo quindi il \texttt{GFLOPS} normalizzato, che tiene conto
della complessità delle operazioni.
\chapter{Prospettiva sulla programmazione parallela}
\section{4 passi nella creazione di un programma parallelo}
\begin{figure}[H]
    \centering 
    \includegraphics[scale=0.5]{img/4steps.png}
\end{figure}
\begin{enumerate}
    \item \textbf{Decomposizione}: si divide il
    problema in sotto-problemi
    più piccoli.
    \item \textbf{Assegnazione}: si assegna ogni
    sotto-problema ad un 
    processore.
    \item \textbf{Orchestrazione}: si sincronizzano i processori.
    \item \textbf{Mapping}: si ricombinano i risultati.
\end{enumerate}

\section{Capire il problema e il programma}
Indubbiamente, il primo passo nello sviluppo di software
parallelo è comprendere a fondo il problema che si desidera
risolvere in parallelo. Se si parte da un programma
seriale esistente, questo necessita anche di una profonda
comprensione del codice attuale. Questo aspetto è cruciale
poiché la natura del problema influisce sulla fattibilità
di un approccio parallelo e su come esso potrebbe essere
strutturato.

Prima di investire tempo nello sviluppo di una soluzione
parallela, è essenziale determinare se il problema
in questione è adatto alla parallelizzazione.
Non tutti i problemi possono beneficiare dell'esecuzione
parallela, e riconoscere questo aspetto in anticipo
può risparmiare tempo e risorse significative.

Un chiaro esempio di problema che può essere
parallelizzato è il calcolo dell'energia potenziale per
ciascuna di diverse migliaia di conformazioni indipendenti
di una molecola. Una volta completati tutti i calcoli,
rimane il compito di trovare la conformazione con l'energia
minima.

Un esempio tipico di problema non parallelizzabile è
il calcolo della serie di Fibonacci
utilizzando la formula:
\[ F(k + 2) = F(k + 1) + F(k) \]

Questo rappresenta un problema non parallelizzabile
perché il calcolo della sequenza di Fibonacci, come
mostrato, implica calcoli dipendenti piuttosto che
indipendenti.

\begin{itemize}
    \item Il calcolo del valore \( F(k + 2) \)
    utilizza i valori di \( F(k + 1) \) e \( F(k) \).
    Questi tre termini non possono essere calcolati
    indipendentemente e, quindi, non possono essere
    eseguiti in parallelo.
\end{itemize}

La dipendenza diretta tra i termini consecutivi della
serie impedisce qualsiasi decomposizione che permetta
un'elaborazione parallela efficace, dimostrando così
le limitazioni di alcuni tipi di problemi rispetto alla
parallelizzazione.


\begin{itemize}
    \item Questo problema si presta bene al processo parallelo perché ogni conformazione molecolare può essere determinata indipendentemente dalle altre.
    \item Il compito successivo di identificazione della conformazione di energia minima è anch'esso parallelizzabile, in quanto coinvolge il confronto di risultati che possono essere elaborati in parallelo.
\end{itemize}

Questo esempio illustra come i compiti possono essere decomposti ed eseguiti in parallelo, accelerando significativamente il processo di calcolo complessivo nelle applicazioni adatte.

\subsection{Identificazione dei Punti Critici in un Programma}

\textbf{Identificare i Hotspots del Programma:}
Identificare i \textit{hotspots}, ovvero le zone dove il programma
compie la maggior parte del lavoro, è essenziale. Molti
programmi scientifici e tecnici concentrano gran
parte delle loro operazioni in poche aree critiche.
L'uso di strumenti di profilazione e analisi delle
prestazioni è quindi cruciale per individuare queste aree.
Concentrarsi sulla parallelizzazione di questi hotspots
può aumentare notevolmente l'efficienza del programma,
mentre le sezioni che utilizzano meno \texttt{CPU}
possono essere
trascurate in questa fase iniziale.

\textbf{Identificare i Colli di Bottiglia nel Programma:}
È importante riconoscere le aree del programma che
sono sproporzionatamente lente o che causano interruzioni
o ritardi nel lavoro che potrebbe essere parallelizzato.
Spesso, operazioni come l'\texttt{I/O} sono responsabili di questi
rallentamenti. Modificare la struttura del programma
o adottare un diverso algoritmo può aiutare a ridurre
o eliminare queste inefficienze,
migliorando così le prestazioni complessive.

\textbf{Identificare gli Inibitori al Parallelismo:}
Un altro aspetto fondamentale è identificare gli inibitori
al parallelismo. La dipendenza dai dati è un esempio
comune di questi ostacoli, come dimostrato dalla sequenza
di Fibonacci. Queste dipendenze creano una situazione in
cui i calcoli devono essere eseguiti in un ordine
specifico, il che limita le opportunità di eseguire
il processo in parallelo.

\textbf{Esplorare Altri Algoritmi:} Infine,
l'esplorazione di altri algoritmi può rappresentare
la considerazione più importante nella progettazione
di un'applicazione parallela. Trovare un algoritmo più
adatto al parallelismo può spesso offrire soluzioni più
efficienti e performanti.

\section{Decomposizione}
Identificare la concorrenza in un programma
e decidere il livello al quale sfruttarla è
fondamentale per ottimizzare l'esecuzione parallela.
Questo processo inizia con la suddivisione del calcolo
in compiti che possono essere distribuiti tra diversi processi. È importante notare che i compiti possono diventare disponibili dinamicamente e che il numero di compiti disponibili può variare nel tempo.

L'obiettivo principale è avere abbastanza compiti per
mantenere i processi occupati, ma non troppi; infatti,
il numero di compiti disponibili in un dato momento
rappresenta il limite superiore della velocità di
esecuzione che può essere raggiunta. Troppi
compiti potrebbero sovraccaricare i processi
e diminuire l'efficienza complessiva, mentre
troppo pochi compiti potrebbero lasciare alcune
risorse inutilizzate, riducendo la performance.

Per quanto riguarda la suddivisione del lavoro
computazionale tra i task paralleli, esistono
due metodi fondamentali: la decomposizione per
dominio e la decomposizione funzionale. La
\textbf{decomposizione per dominio} implica
dividere i dati su cui opera il programma in parti
che possono essere processate in parallelo, mentre
la \textbf{decomposizione funzionale} comporta
la divisione delle funzioni del programma in sotto-funzioni
che possono essere eseguite contemporaneamente.

\subsection{Decomposizione per Dominio}
La decomposizione per dominio è una tecnica
comune per suddividere il lavoro in un programma
parallelo. Questo approccio prevede la divisione
dei dati in parti che possono essere processate
indipendentemente. Questo metodo è particolarmente
utile quando i dati sono indipendenti tra loro
e possono essere elaborati senza interazioni
tra i processi.

Un esempio di decomposizione per dominio è
l'analisi di un'immagine in cui ogni pixel
può essere elaborato indipendentemente dagli
altri. In questo caso, l'immagine può essere
divisa in sezioni che possono essere processate
in parallelo, accelerando notevolmente il
processo di analisi.

La decomposizione per dominio può essere
svolta in diversi modi, nel caso mono-dimensionale
si può dividere il lavoro in blocchi di dati
o eseguire il lavoro in modo ciclico.
Nel caso bidimensionale, si può dividere il
lavoro in righe o colonne, o in blocchi di
dimensioni maggiori. Anche nel caso bidimensionale
è possibile eseguire il lavoro in modo ciclico.

\subsection{Decomposizione Funzionale}
La decomposizione funzionale è un'altra tecnica
importante per la progettazione di programmi
paralleli. Questo approccio prevede la divisione
delle funzioni del programma in sotto-funzioni
che possono essere eseguite contemporaneamente.
Questo metodo è particolarmente utile quando
le funzioni del programma possono essere
eseguite indipendentemente l'una dall'altra.

Un esempio di decomposizione funzionale è
l'elaborazione di un documento di testo in cui
ogni paragrafo può essere analizzato
indipendentemente dagli altri. In questo caso,
il documento può essere suddiviso in paragrafi
che possono essere processati in parallelo,
accelerando notevolmente il processo di analisi.

\section{Assegnazione}
Nel contesto della programmazione parallela, è cruciale
considerare i compiti come ``cose da fare'' e i thread
come ``lavoratori''. Questa analogia aiuta a visualizzare
la distribuzione del lavoro tra i processi. Ad esempio,
potremmo decidere quale processo calcola le forze su quali
stelle, o quali raggi vengono calcolati da quale processo.
L'obiettivo principale è bilanciare il carico di lavoro,
riducendo i costi di comunicazione e gestione, noto anche
come \textit{load balancing}.

\subsection{Approcci Strutturati alla Divisione dei
Compiti}
Gli approcci strutturati tendono a funzionare bene in
questo contesto:
\begin{itemize}
    \item L'ispezione del codice (\textit{come i loop
    paralleli}) o la comprensione dell'applicazione
    possono guidare la divisione efficace dei compiti.
    \item L'uso di euristiche ben note può facilitare
    questo processo.
    \item Si considerano assegnazioni statiche versus
    dinamiche dei compiti, a seconda della natura e
    delle esigenze dell'applicazione.
\end{itemize}

Come programmatori, tendiamo a preoccuparci prima della partizione, che di solito è indipendente dall'architettura o dal modello di programmazione. Tuttavia, il costo e la complessità nell'uso delle primitive possono influenzare le decisioni.

\subsection{Bilanciamento del Carico}

Il bilanciamento del carico si riferisce alla pratica
di distribuire i compiti tra i processi in modo che tutti
i processi siano costantemente occupati, minimizzando
il tempo di inattività dei processi. Questo aspetto è
fondamentale per le prestazioni dei programmi paralleli.
Ad esempio, se tutti i processi sono soggetti a un punto
di sincronizzazione a barriera, il compito più lento
determinerà le prestazioni complessive.

\subsubsection{Come Raggiungere il Bilanciamento del Carico}
\begin{itemize}
    \item L'assegnazione dinamica del lavoro può
    essere utilizzata per gestire i compiti in modo
    flessibile.
    \item Alcune classi di problemi risultano in squilibri
    di carico anche se i dati sono distribuiti
    uniformemente tra i processi:
    \begin{itemize}
        \item Array sparsi - alcuni processi avranno dati
        effettivi su cui lavorare mentre altri hanno
        prevalentemente ``zeri".
        \item Metodi di griglia adattivi - alcuni processi
        potrebbero dover raffinare la loro mesh mentre
        altri no.
        \item Simulazioni N-body - dove alcune particelle
        possono migrare verso o lontano da un processo.
    \end{itemize}
    \item Quando il lavoro che ogni processo eseguirà
    è intenzionalmente variabile o non prevedibile, può
    essere utile utilizzare un approccio di scheduler -
    task pool. Man mano che ogni processo completa il suo
    lavoro, si mette in coda per ottenere un nuovo pezzo
    di lavoro.
    \item Potrebbe diventare necessario progettare un
    algoritmo che rilevi e gestisca gli squilibri di
    carico man mano che si verificano dinamicamente
    all'interno del codice.
\end{itemize}

\subsection{Granularità nel Calcolo Parallelo}

\subsubsection{Rapporto Computazione/Comunicazione}
Nel calcolo parallelo, la granularità è una misura
qualitativa del rapporto tra computazione e comunicazione.
I periodi di computazione sono tipicamente separati dai
periodi di comunicazione attraverso eventi di
sincronizzazione. Questa distinzione è fondamentale
per comprendere e ottimizzare le prestazioni dei sistemi
paralleli.

\subsubsection{Granularità del Parallelismo}
Il parallelismo può essere classificato in base alla
granularità delle operazioni che vengono eseguite:
\begin{description}
    \item[Fine grain parallelism]
    dove piccole quantità di lavoro computazionale
    sono seguite da eventi di comunicazione, con un
    basso rapporto tra computazione e comunicazione.
    Questo tipo di parallelismo può aiutare a ridurre
    gli overhead dovuti agli squilibri di carico, ma
    potrebbe incrementare gli overhead di comunicazione
    e sincronizzazione.
    \item[Coarse grain parallelism] caratterizzato da
    quantità relativamente grandi di lavoro computazionale
    che intervallano gli eventi di
    comunicazione/sincronizzazione. Questo comporta
    un alto rapporto computazione/comunicazione,
    suggerendo maggiori opportunità di incremento delle
    prestazioni, anche se può essere più difficile
    da bilanciare efficacemente il carico di lavoro.
\end{description}

\subsubsection{Parallelismo Fine o Grossolano?}
La granularità più efficiente dipende dall'algoritmo
specifico e dall'ambiente hardware in cui opera. In molti
casi, l'overhead associato alle comunicazioni e alla
sincronizzazione è elevato rispetto alla velocità di
esecuzione, rendendo vantaggiosa una granularità
grossolana. Tuttavia, il parallelismo fine può essere
utile per ridurre gli overhead dovuti agli squilibri di
carico. La scelta tra granularità fine o grossolana deve
quindi essere ponderata in base alle specifiche esigenze
dell'applicazione e alle caratteristiche del sistema di
calcolo utilizzato.

\section{Orchestrazione}

L'orchestrazione in calcolo parallelo si concentra sulla
strutturazione della comunicazione, sulla sincronizzazione
e sull'organizzazione delle strutture di dati e sulla
programmazione temporale dei compiti. Gli obiettivi
principali di questa fase includono:

\begin{itemize}
    \item Ridurre i costi di comunicazione e
    sincronizzazione.
    \item Preservare la località del riferimento ai dati.
    \item Programmare i compiti in modo da soddisfare
    le dipendenze il più presto possibile.
    \item Ridurre l'overhead della gestione del
    parallelismo.
\end{itemize}

Le scelte in questa fase dipendono dal modello di
programmazione adottato, dall'astrazione della
comunicazione e dall'efficienza delle primitive offerte
dai progettisti di sistemi.

\subsection{Comunicazioni nel Calcolo Parallelo}

\subsubsection{Chi necessita delle comunicazioni?}
La necessità di comunicazioni tra compiti dipende
dalla natura del problema:

\begin{description}
    \item[Non necessita comunicazioni:] Alcuni tipi
    di problemi possono essere decomposti ed eseguiti
    in parallelo senza quasi alcuna necessità di
    condivisione di dati tra i compiti. Ad esempio,
    un'operazione di elaborazione di immagini dove ogni
    pixel in un'immagine in bianco e nero deve avere
    il suo colore invertito. I dati dell'immagine possono
    essere facilmente distribuiti a più compiti che
    agiscono indipendentemente l'uno dall'altro per
    svolgere la loro parte di lavoro. Questi tipi di
    problemi sono spesso chiamati parallelamente
    imbarazzanti perché sono così diretti.
    \item[Necessita comunicazioni:] La maggior parte
    delle applicazioni parallele non è così semplice e
    richiede che i compiti condividano dati tra di loro.
    Ad esempio, un problema di diffusione del calore in
    \texttt{3D} richiede che un compito conosca le
    temperature calcolate dai compiti che hanno dati
    adiacenti. Le modifiche ai dati adiacenti hanno un
    effetto diretto sui dati del compito.
\end{description}

\subsubsection{Fattori importanti nella progettazione
delle comunicazioni inter-task}
\begin{itemize}
    \item \textbf{Costo delle comunicazioni:} La
    comunicazione inter-task implica quasi sempre
    un overhead. I cicli di macchina e le risorse che
    potrebbero essere utilizzati per la computazione
    vengono invece utilizzati per impacchettare e
    trasmettere dati.
    \item \textbf{Latency vs Bandwidth:}
    \begin{itemize}
        \item La \textit{latency} è il tempo necessario
        per inviare un messaggio minimo ($0 byte$) da un
        punto A a un punto B.
        \item La \textit{bandwidth} è la quantità di
        dati che può essere comunicata per unità di tempo,
        comunemente espressa in megabyte/sec o
        gigabyte/sec.
    \end{itemize}
    \item \textbf{Visibilità delle comunicazioni:} Con
    il modello di passaggio di messaggi, le comunicazioni
    sono esplicite e generalmente ben visibili e sotto
    il controllo del programmatore. Con il modello di
    parallelismo sui dati, le comunicazioni spesso
    avvengono trasparentemente per il programmatore,
    particolarmente su architetture a memoria distribuita.
    \item \textbf{Comunicazioni sincrone vs asincrone:}
    Le comunicazioni possono essere sincrone, bloccanti
    o non bloccanti, a seconda delle necessità
    dell'applicazione.
\end{itemize}

\subsubsection{Tipi e Complessità delle Comunicazioni}
Nella progettazione di codici paralleli, è essenziale
determinare quali compiti devono comunicare tra loro.
Queste comunicazioni possono essere implementate sia
in modo sincrono che asincrono e si dividono in due
categorie principali:

\textbf{Comunicazione Punto-a-Punto:} Coinvolge due
compiti specifici dove uno agisce come mittente/produttore
di dati e l'altro come ricevitore/consumatore. Questa
tipologia è ideale per il trasferimento diretto di dati
tra due entità.

\textbf{Comunicazione Collettiva:} Implica la
partecipazione di più di due compiti, generalmente
definiti come membri di un gruppo collettivo. Le varianti
comuni includono:
\begin{itemize}
    \item \textbf{Broadcast:} Dati inviati da un mittente
    a tutti i partecipanti.
    \item \textbf{Scatter:} Dati divisi e inviati a
    diversi ricevitori dallo stesso mittente.
    \item \textbf{Gather:} Dati raccolti da diversi
    mittenti in un singolo ricevitore.
    \item \textbf{Allgather, Reduce, e Allreduce:}
    Operazioni che combinano le funzionalità di raccolta
    e distribuzione dati, con tutti i partecipanti che
    inviano o ricevono dati.
\end{itemize}

\textbf{Sovraccarico e Complessità:} Le comunicazioni
introducono un sovraccarico derivante dalla necessità
di sincronizzare i compiti, impacchettare e trasmettere
dati, e gestire il traffico di rete che può saturare la
banda disponibile. La scelta del tipo di comunicazione
(\textit{sincrona o asincrona}) può influenzare
significativamente l'efficienza del sistema parallelo.
Le comunicazioni asincrone possono ridurre i tempi di
attesa e migliorare la fluidità dell'esecuzione, mentre
quelle sincrone possono semplificare il design ma a costo
di potenziali ritardi.

Queste considerazioni sono fondamentali per ottimizzare
le prestazioni e l'efficienza di un sistema di calcolo
parallelo, bilanciando le esigenze di velocità e coerenza
dei dati all'interno dell'applicazione.

\subsection{Sincronizzazione nel Calcolo Parallelo}
La sincronizzazione è un aspetto cruciale del calcolo
parallelo, necessario per coordinare l'operato dei
processi paralleli. Qui di seguito vengono descritti
i tipi principali di sincronizzazione utilizzati nei
programmi paralleli.

\subsubsection{Tipi di Sincronizzazione}
\begin{description}
    \item[Barriera:] Una barriera è un punto di
    sincronizzazione dove tutti i processi devono
    arrivare prima di poter procedere. È comunemente
    utilizzata per separare le fasi di computazione e
    garantire che tutti i processi raggiungano un certo
    stato prima di avanzare.
    \item[Lock e Semaphore:] Questi meccanismi controllano
    l'accesso a risorse condivise per prevenire conflitti
    e inconsistenze. Un lock è un meccanismo di mutua
    esclusione, mentre un semaforo può permettere un
    accesso limitato a più processi.
    \item[Comunicazioni Sincrone:] Le operazioni di
    comunicazione sincrone richiedono che entrambi i 
    processi, mittente e ricevente, siano pronti per 
    inviare o ricevere dati, fungendo da forma di 
    sincronizzazione.
\end{description}

\subsubsection{Sincronizzazione di Eventi Globali}
Una sincronizzazione globale avviene tramite l'uso
di una \texttt{BARRIER(nprocs)}, che richiede che tutti 
i processi coinvolti raggiungano questo punto prima di
continuare. Questo tipo di sincronizzazione è spesso 
utilizzato per assicurare che tutte le operazioni 
precedenti, come l'accumulo di somme globali, siano 
completate prima di procedere alla fase successiva del 
calcolo.

\subsubsection{Sincronizzazione di Eventi Punto-a-Punto}
In alcuni scenari, un processo potrebbe dover notificare
un altro evento per poter procedere, tipico del modello
produttore-consumatore. In programmi paralleli con spazio
di indirizzi condiviso, ciò può essere gestito con
semafori o variabili ordinari usate come bandiere.
Questa pratica, nota come \textit{busy-waiting} o
\textit{spinning}, comporta che un processo rimanga
in attesa attiva fino alla ricezione del segnale per 
procedere.

\subsubsection{Sincronizzazione di Eventi di Gruppo}
Questa forma di sincronizzazione coinvolge solo un 
sottoinsieme di processi, che possono usare barriere 
o bandiere per coordinare l'azione tra di loro. Gli 
scenari tipici includono:
\begin{itemize}
    \item Produttore singolo, consumatori multipli.
    \item Produttori multipli, consumatore singolo.
    \item Produttori e consumatori multipli.
\end{itemize}

Questi metodi permettono di sincronizzare efficacemente
sottoinsiemi di processi in base alle necessità specifiche
del problema e del design dell'applicazione.

\section{Mapping}
La mappatura di processi su una topologia di rete specifica
pone interrogativi importanti:
\begin{itemize}
    \item Saranno eseguiti processi multipli sullo stesso
    processore?
    \item Come si possono sfruttare al meglio le
    connessioni fisiche e logiche nella rete per
    ottimizzare la comunicazione?
\end{itemize}

\subsubsection{Condivisione dello Spazio}
La condivisione dello spazio implica la divisione
della macchina in sottoinsiemi, ognuno dei quali può
ospitare un'applicazione per volta. Questo può essere
gestito in due modi:
\begin{itemize}
    \item I processi possono essere vincolati a processori
    specifici.
    \item I processi possono essere lasciati alla gestione
    del sistema operativo, che decide dinamicamente
    l'allocazione.
\end{itemize}

\subsection*{Allocazione del Sistema}
Nel mondo reale, l'utente specifica alcuni desideri
riguardo all'allocazione dei processi e il sistema
gestisce alcuni aspetti automaticamente. La visione
comune è quella di associare un processo a un processore,
ma questa può variare in base alle esigenze e alla
configurazione del sistema.

\subsection*{Prospettiva dell'Architettura}
L'architettura del sistema deve decidere:
\begin{itemize}
    \item Cosa può essere migliorato con un design
    hardware migliore?
    \item Quali sono le questioni fondamentalmente
    legate alla programmazione?
\end{itemize}

\section{Obiettivi di Alto Livello e Processo di
Parallelizzazione}
Segue una tabella che riassume gli step nel processo
di parallelizzazione e i loro obiettivi principali:

\noindent\resizebox{\textwidth}{!}{
\begin{tabular}{|c|c|c|}
\hline
\textbf{Step} & \textbf{Dipendente dall'Architettura?} & \textbf{Obiettivi di Performance} \\
\hline
Decomposizione & No & Esporre sufficiente concorrenza ma non troppo \\
Assegnazione & No & Bilanciare il carico di lavoro \\
Orchestrazione & Sì & Ridurre i volumi di comunicazione \\
Mappatura & Sì & Sfruttare la località nella topologia di rete \\
\hline
\end{tabular}
}

Questi obiettivi mirano a ottimizzare le prestazioni,
come il miglioramento della velocità rispetto ai programmi
sequenziali, riducendo al contempo l'utilizzo delle
risorse e lo sforzo di sviluppo. Queste considerazioni
sono cruciali sia per i progettisti di algoritmi che
per gli architetti di sistema.

\section{Come appare un programma parallelo}

Consideriamo una versione semplificata che calcola
le simulazioni oceaniche. Utilizziamo il metodo di
Gauss-Seidel per aggiornare i punti interni di una
griglia. Ogni punto della griglia è aggiornato
in base alla media ponderata del suo valore corrente
e dei valori dei suoi vicini immediati.

La formula per il calcolo è la seguente:
\[
A_{i,j} = 0.2 \cdot (A_{i, j} + A_{i-1, j} + A_{i+1, j}
+ A_{i, j-1} + A_{i, j+1})  
\]

\begin{figure}[H]
    \centering
    \begin{tikzpicture}
        \foreach \x in {0, 1, ..., 9}
            \foreach \y in {0, 1, ..., 9}
                \draw[] (\x, \y) circle (4pt);
            
            \draw[thick,->] (2, 3) -- (2, 2.3);
            \draw[thick,->] (3, 2) -- (2.3, 2);
            \draw[thick,->] (2, 1) -- (2, 1.7);
            \draw[thick,->] (1, 2) -- (1.7, 2);

    \end{tikzpicture}  
\end{figure}

Una versione sequenziale di questo programma potrebbe
essere scritta come segue:

\begin{algorithm}[H]
    \caption{\texttt{Simulazioni Oceaniche - Versione Sequenziale}}
    \DontPrintSemicolon  
    
    \KwIn{Dimensione della griglia $n$}
    \KwOut{Griglia aggiornata $A$ dopo la convergenza}

    \BlankLine
    $n \gets \texttt{read\_input()}$ \;
    $A \gets \texttt{allocate\_grid}(n + 2)$ \;
    \texttt{initialize\_grid}$(A)$ \;
    \texttt{Solve}$(A)$ \;
    
    \SetKwFunction{FMain}{Solve}
    \SetKwProg{Fn}{Function}{:}{}
    \Fn{\FMain{$A$}}{
        $done \gets 0$ \;
        \While{\textup{not} $done$}{
            $diff \gets 0$ \;
            \For{$i \gets 1$ \KwTo $n$}{
                \For{$j \gets 1$ \KwTo $n$}{
                    $tmp \gets A[i][j]$ \;
                    $A[i][j] \gets 0.2 \times (A[i][j] + A[i-1][j] + A[i+1][j] + A[i][j-1] + A[i][j+1])$ \;
                    $diff \gets diff + \texttt{abs}(A[i][j] - tmp)$ \;
                }
            }
            \If{$diff / (n \times n) < \texttt{threshold}$}{
                $done \gets 1$ \;
            }
        }
    }
\end{algorithm}

Il primo passo per parallelizzare questo programma è quello 
di identificare le dipendenze tra i calcoli. In questo caso,
ogni punto della griglia dipende dai valori dei suoi vicini
immediati. Questo implica che i calcoli per ogni punto
della griglia devono essere eseguiti in un ordine specifico
per garantire che i valori dei vicini siano disponibili
prima di calcolare il valore del punto stesso.

\begin{figure}[H]
    \centering
    \begin{tikzpicture}
        % Creazione griglia
        \foreach \x in {0, 1, ..., 9}
            \foreach \y in {0, 1, ..., 9}
                \draw[] (\x, \y) circle (4pt);
        
        % Frecce verticali
        \foreach \x in {1, ..., 8}
            \foreach \y in {2, ..., 9} 
            {
                \draw[thick,->] (\x, \y) -- (\x, \y - 0.5); 
            }
            
        % Frecce orizzontali
        \foreach \x in {0, ..., 7} 
            \foreach \y in {1, ..., 8}
            {
                \draw[thick,->] (\x, \y) -- (\x + 0.5, \y); %
            }
        % Diagonali da sinistra a destra
        \draw[thick, blue] (8, 2) -- (7, 1);
        \draw[thick, blue] (8, 3) -- (6, 1);
        \draw[thick, blue] (8, 4) -- (5, 1);
        \draw[thick, blue] (8, 5) -- (4, 1);
        \draw[thick, blue] (8, 6) -- (3, 1);
        \draw[thick, blue] (8, 7) -- (2, 1);
        \draw[thick, blue] (8, 8) -- (1, 1);
        \draw[thick, blue] (7, 8) -- (1, 2);
        \draw[thick, blue] (6, 8) -- (1, 3);
        \draw[thick, blue] (5, 8) -- (1, 4);
        \draw[thick, blue] (4, 8) -- (1, 5);
        \draw[thick, blue] (3, 8) -- (1, 6);
        \draw[thick, blue] (2, 8) -- (1, 7);
    \end{tikzpicture}  
\end{figure}
Quello che possiamo notare è che i calcoli per ogni punto
della griglia possono essere eseguiti in parallelo, a
patto che i valori dei vicini siano disponibili. Questo
significa che possiamo dividere la griglia in sezioni
più piccole e assegnare ciascuna sezione a un thread
parallelo. Ogni thread può quindi calcolare i valori
dei punti nella sua sezione in parallelo, riducendo
notevolmente il tempo di calcolo complessivo.

L'idea per migliorare le performance del'algoritmo
è quella di cambiare l'ordine in cui le celle vengono 
aggiornate. 
Il nuovo algoritmo itera le stesse soluzioni, ma converge 
diversamente.

Uno dei metodi utili nella risoluzione di sistemi di equazioni su griglie di calcolo, soprattutto in contesti paralleli, è l'adozione di un ordinamento specifico per la traversata della griglia. L'ordinamento "red-black", o scacchiera, è un esempio di questo approccio che consente di migliorare la convergenza degli aggiornamenti iterativi.

\subsection{Ordinamento rosso-nero}

L'ordinamento ``rosso-nero" suddivide la griglia di calcolo in due
insiemi disgiunti, comunemente denominati come
\textit{red} e \textit{black}. Questi due insiemi sono aggiornati sequenzialmente:
\begin{itemize}
  \item \textbf{Red Sweep:} Durante la fase red, solo i nodi rossi sono aggiornati.
  \item \textbf{Black Sweep:} Successivamente, durante la fase black, solo i nodi neri sono aggiornati.
\end{itemize}
La caratteristica principale di questo metodo è che ciascun ``sweep"
(\textit{passata}) può essere eseguito in modo completamente parallelo,
poiché non ci sono dipendenze dirette tra i nodi dello stesso colore
durante un singolo sweep. Questo riduce significativamente il tempo
di attesa per le sincronizzazioni globali tra i thread o i processi.


Nonostante la parallelizzazione, è necessaria una sincronizzazione globale
tra le due fasi per garantire che tutti i nodi rossi siano stati completamente
aggiornati prima di iniziare gli aggiornamenti dei nodi neri e viceversa.
Questa sincronizzazione, sebbene conservativa, è conveniente per mantenere
la coerenza dei dati tra le fasi di aggiornamento.
\begin{figure}[H]
    \centering
    \begin{tikzpicture}
        % Disegna i cerchi con colorazione alternata red-black
        \foreach \x in {0, 1, ..., 9}
            \foreach \y in {0, 1, ..., 9}
                {
                % Calcola se la somma delle coordinate è pari (rosso) o dispari (nero)
                \pgfmathparse{int(mod(\x+\y,2))}
                \ifnum\pgfmathresult=0
                    \fill[red] (\x, \y) circle (4pt); % Nodi rossi
                \else
                    \fill[black] (\x, \y) circle (4pt); % Nodi neri
                \fi
                }

        % Aggiungi le frecce specificate
        \draw[thick,->] (2, 3) -- (2, 2.3); % Freccia verso il basso
        \draw[thick,->] (3, 2) -- (2.3, 2); % Freccia verso sinistra
        \draw[thick,->] (2, 1) -- (2, 1.7); % Freccia verso l'alto
        \draw[thick,->] (1, 2) -- (1.7, 2); % Freccia verso destra

    \end{tikzpicture}  
\end{figure}

L'assegnamento dei processi a ciascuna fase può seguire un modello
di assegnazione statica o dinamica, a seconda delle esigenze dell'applicazione
e delle caratteristiche dell'architettura sottostante. L'obiettivo principale
è garantire che i processi siano costantemente occupati e che i tempi di attesa
siano ridotti al minimo.

\subsubsection{Assegnamento Statico}
L'assegnamento statico decompone la griglia in righe o colonne e assegna
ciascuna sezione a un processo specifico. Questo approccio è semplice e prevedibile,
ma può portare a squilibri di carico se le sezioni non sono bilanciate in termini
di complessità computazionale.

\subsubsection{Assegnamento Dinamico}
L'assegnamento dinamico si basa su un modello di coda di lavoro in cui i processi
richiedono e ricevono nuovi compiti man mano che terminano quelli precedenti.
Questo metodo è particolarmente vantaggioso in contesti in cui il carico di
lavoro è variabile, poiché permette una migliore adattabilità e minimizza i
tempi di inattività dei processi.


\subsection{Considerazione delle Dipendenze nel Flusso di Dati}

Il processo di calcolo parallelo richiede una gestione accurata delle dipendenze
per massimizzare l'efficienza. La seguente sezione descrive il flusso di lavoro
diviso in fasi di calcolo e comunicazione:

\begin{enumerate}
    \item Esecuzione parallela dell'aggiornamento delle celle rosse.
    \item Attesa fino al completamento dell'aggiornamento da parte di tutti
    i processori.
    \item Comunicazione delle celle rosse aggiornate agli altri processori.
    \item Esecuzione parallela dell'aggiornamento delle celle nere.
    \item Attesa fino al completamento dell'aggiornamento da parte di tutti
    i processori.
    \item Comunicazione delle celle nere aggiornate agli altri processori.
    \item Ripetizione del processo.
\end{enumerate}

Questo metodo alterna fasi di computazione e comunicazione, essenziale per
mantenere un flusso di lavoro efficiente e coordinato tra i diversi processori.

\subsection{Come pensare alla parallelizzazione di un programma}
Ci sono diversi aspetti da considerare quando si pensa alla parallelizzazione
di un programma. Alcuni di questi includono:
\begin{itemize}
    \item \textbf{Pensare alla parallelizzazione dei dati:} Considerare come
    i dati possono essere suddivisi tra i processi e come le operazioni
    possono essere eseguite in parallelo.
    \item \textbf{\texttt{SPMD}/spazi di indirizzi condivisi:} Pensare
    a come i processi possono condividere dati e comunicare tra di loro
    in un ambiente di spazio di indirizzi condiviso.
    \item \textbf{Passaggio di messaggi:} Considerare come i processi possono
    comunicare tra di loro utilizzando il passaggio di messaggi e come
    possono coordinare le loro attività.
\end{itemize}
\subsubsection{Sincronizzazione in Ambiente di Memoria Condivisa}
La programmazione in un ambiente di memoria condivisa impone agli sviluppatori la responsabilità della sincronizzazione tra i thread. Le primitive comuni di sincronizzazione includono:
\begin{itemize}
    \item \textbf{Locks}: Forniscono l'esclusione mutua, permettendo a un solo thread per volta di entrare in una regione critica.
    \item \textbf{Barriers}: I thread devono attendere che tutti raggiungano questo punto di barriera prima di procedere.
\end{itemize}

\subsubsection{Modello di Programmazione \texttt{SPMD}}
Il modello di programmazione Single Program, Multiple Data (\texttt{SPMD}) non segue il lockstep, ovvero non implica necessariamente l'esecuzione delle stesse istruzioni contemporaneamente da parte di tutti i processi:
\begin{itemize}
    \item L'assegnazione è controllata dai valori delle variabili utilizzate come limiti dei loop.
    \item Le operazioni speciali più interessanti in questo modello sono quelle di sincronizzazione, necessarie per assicurare l'accesso esclusivo a risorse condivise e per le operazioni di riduzione globale.
    \item La necessità di barriere deriva dalla necessità di assicurare che tutte le operazioni precedenti siano state completate prima di procedere.
\end{itemize}

\subsection{Primitive di Sincronizzazione con Barriere}

Le barriere sono un metodo di sincronizzazione essenziale nei sistemi di
calcolo parallelo. Servono per garantire che tutti i thread o i processi
raggiungano un certo punto nel loro esecuzione prima di procedere. Questo
permette di gestire le dipendenze tra diverse fasi di calcolo.

Una barriera, tipicamente invocata come \texttt{barrier(num\_threads)},
assicura che tutti i thread abbiano completato le loro operazioni prima
di superare questo punto di sincronizzazione. Le caratteristiche principali
includono:

\begin{itemize}
    \item Le barriere dividono il calcolo in fasi ben distinte.
    \item Tutti i calcoli eseguiti da tutti i thread prima della barriera
    devono essere completati prima che qualsiasi thread inizi i calcoli
    previsti dopo la barriera.
    \item In sostanza, si presume che tutti i calcoli che seguono la
    barriera dipendano da tutti quelli che la precedono.
\end{itemize}

Le barriere sono quindi un modo conservativo per esprimere le dipendenze
all'interno di un'applicazione parallela, assicurando che non ci siano
race condition o inconsistenze nei dati condivisi.

\section{Message Passing Grid Solver}

TODO


\chapter{General Purpose \texttt{GPU} - \texttt{GP-GPU}}

Le \texttt{GPU} (Graphics Processing Units) e le \texttt{CPU} (Central Processing Units) presentano differenze significative nella loro architettura e nel design, riflettendo le loro finalità di utilizzo orientate rispettivamente al throughput e alla latenza.

\begin{tabular}[b]{c}
  \begin{tikzpicture}[scale=0.4]
    \dram      { 0  ,-2.5}
    \cpucache  { 0  , 0}
    \cpucontrol{ 0  ,4.4}
    \cpualu    { 8.6,4.4}
    \cpualu    {12.8,4.4}
    \cpualu    { 8.6,6.55}
    \cpualu    {12.8,6.55}
    \node[font=\sffamily\bfseries] at (8.45,-3.5) {CPU};
  \end{tikzpicture}\\
  Central Processing Unit (\texttt{CPU})
  \end{tabular}
  \qquad
  \begin{tabular}[b]{c}
  \begin{tikzpicture}[scale=0.4]
    \dram{0,-2.5}
    \foreach \i in {0,...,7}
      \gpucontrol{0,1.1*\i+0.5}
      \gpucache{0,1.1*\i}
      \foreach \j in {1,...,16}
        \gpualu{\j,1.1*\i};
    \node[font=\sffamily\bfseries] at (8.45,-3.5) {GPU};
  \end{tikzpicture}\\
  Graphical Processing Unit (\texttt{GPU})
\end{tabular}

\section{Architettura delle \texttt{CPU}}
Le \texttt{CPU} sono progettate con un'orientamento alla riduzione della latenza:
\begin{itemize}
    \item \textbf{Grandi cache}: Per convertire gli accessi alla memoria a lunga latenza in accessi alla cache a breve latenza.
    \item \textbf{Controllo sofisticato}: Includono la previsione dei branch per ridurre la latenza dei branch e il forwarding dei dati per ridurre la latenza dei dati.
    \item \textbf{\texttt{ALU} potenti}: Unità di elaborazione aritmetica progettate per ridurre la latenza delle operazioni.
\end{itemize}

\section{Architettura delle \texttt{GPU}}
Al contrario, le \texttt{GPU} sono progettate per massimizzare il throughput:
\begin{itemize}
    \item \textbf{Piccole cache}: Ottimizzate per aumentare il throughput della memoria.
    \item \textbf{Controllo semplice}: Non dispongono di previsione dei branch né di forwarding dei dati.
    \item \textbf{\texttt{ALU} efficienti dal punto di vista energetico}: Numerose \texttt{ALU}, caratterizzate da lunghe latenze ma fortemente pipeline per un alto throughput.
    \item \textbf{Necessità di un numero massivo di thread}: Per tollerare le latenze grazie all'alto numero di thread in esecuzione.
\end{itemize}

\section{Differenze nella Gestione della Memoria}
Entrambe le architetture utilizzano la \texttt{DRAM} come memoria principale,
ma si differenziano significativamente nelle loro strategie di gestione della
cache e nel controllo delle unità di elaborazione, riflettendo i loro obiettivi
di progettazione di bassa latenza per le \texttt{CPU} e alto throughput per
le \texttt{GPU}.

\subsection{Comparazione delle Prestazioni tra \texttt{CPU} e \texttt{GPU}}

\paragraph{\texttt{CPU} per Codice Sequenziale}
Le \texttt{CPU} (Central Processing Units) sono ottimizzate per l'esecuzione
di codice sequenziale dove la latenza è un fattore critico. Grazie alla loro
architettura complessa, che include una gestione avanzata della cache e unità
di controllo sofisticate, le \texttt{CPU} possono eseguire codice sequenziale
molto più rapidamente rispetto alle \texttt{GPU}. Questo le rende particolarmente
adatte per applicazioni che richiedono un elevato grado di interattività o tempi
di risposta rapidi, poiché possono essere fino a 10 volte o più veloci delle
\texttt{GPU} nell'esecuzione di codice sequenziale.

\paragraph{\texttt{GPU} per Codice Parallelo}
Al contrario, le \texttt{GPU} (Graphics Processing Units) sono progettate
per massimizzare il throughput, specialmente utile in applicazioni che
beneficiano dell'elaborazione parallela. Con migliaia di core più piccoli,
le \texttt{GPU} gestiscono efficacemente compiti paralleli, distribuendo
il lavoro su molti core per elaborare grandi quantità di dati simultaneamente.
Questo le rende ideali per computazione scientifica, elaborazione di immagini
e applicazioni di machine learning, dove possono superare le \texttt{CPU} di
un fattore di 10 volte o più in termini di velocità.

\paragraph{Scelta dell'Hardware Adatto}
La scelta tra \texttt{CPU} e \texttt{GPU} deve essere fatta in base alla
natura del carico di lavoro. Mentre le \texttt{CPU} sono preferibili per
operazioni che richiedono decisioni rapide e poca parallelizzazione, le
\texttt{GPU} sono la scelta migliore per lavori che possono essere facilmente
suddivisi in compiti più piccoli e gestiti in parallelo. La decisione dovrebbe
quindi basarsi sull'analisi del carico di lavoro specifico e delle esigenze di
prestazione richieste.

\section{Introduzione a \texttt{CUDA C}}
\texttt{CUDA C} è un'estensione del linguaggio di programmazione \texttt{C} che
consente di scrivere codice per le \texttt{GPU} di \texttt{NVIDIA}. Questo
linguaggio permette di sfruttare la potenza di calcolo delle \texttt{GPU} per
eseguire operazioni parallele su grandi quantità di dati. \texttt{CUDA C} è
progettato per essere simile a \texttt{C}, ma include alcune estensioni e
funzionalità specifiche per la programmazione parallela su \texttt{GPU}.



\subsection{Interazione tra \texttt{CPU} e \texttt{GPU}}

In un'applicazione \texttt{CUDA}, la \texttt{CPU} funge da \textbf{host} e
la \texttt{GPU} come \textbf{device}. Il programma inizia la sua esecuzione
sulla \texttt{CPU} con la funzione \texttt{main} e, al richiamo di una
funzione \textbf{kernel}, trasferisce l'esecuzione sulla \texttt{GPU}.
Questo permette di sfruttare l'elaborazione parallela per poi ritornare
alla \texttt{CPU} una volta completate le operazioni sulla \texttt{GPU}.

\subsection{Esecuzione del \texttt{Kernel CUDA}}

Un \textbf{Cuda kernel} è eseguito da una griglia (\textit{array}) di thread.
Tutti i thread in questa griglia eseguono lo stesso codice scritto nel
kernel. È fondamentale identificare ogni thread mediante un indice unico
per garantire che ciascuno esegua il codice in modo appropriatamente ``diverso".

\subsection{Struttura della Grid e Blocchi di Thread}

Le thread non sono aggregate in un unico gruppo all'interno della griglia,
ma sono organizzate in \textbf{blocchi}. Quindi, abbiamo una griglia
composta da blocchi di thread. Questa strutturazione è essenziale
per facilitare la comunicazione e la sincronizzazione delle thread
all'interno dello stesso blocco, grazie alla presenza di una memoria
condivisa. Le thread in blocchi diversi, tuttavia, hanno capacità
limitate di cooperazione diretta, rendendo la memoria condivisa
all'interno dei blocchi uno strumento cruciale per l'efficienza
del parallelismo.


\begin{figure}[H]
  \centering
  \begin{tikzpicture}[font=\small]

    \matrix[label={[anchor=south west, name=gl]north west:Grid}] (OneGrid) [column sep=1mm, row sep=1mm]
    {\pic{block}; & \pic{block}; & \pic{block}; \\
    \pic {block}; & \pic (Ref) {block}; & \pic{block}; \\
    };
    \begin{scope}[on background layer]
    \node[fit=(OneGrid) (gl), inner sep=0pt, fill=green, draw=gray] (Grid) {};
    \end{scope}
    
    
    \matrix[label={[name=ml]Block(1,1)}, below=2cm of Grid] (OneBlock) [column sep=-\pgflinewidth, row sep=\pgflinewidth]
    {\pic{thread}; & \pic{thread}; & \pic{thread}; & \pic{thread}; \\
    \pic{thread}; & \pic{thread}; & \pic{thread}; & \pic{thread}; \\
    \pic{thread}; & \pic{thread}; & \pic{thread}; & \pic{thread}; \\
    };
    \begin{scope}[on background layer]
    \node[fit=(OneBlock) (ml), inner sep=0pt, fill=yellow, draw=gray] (Block) {};
    \end{scope}
    
    \draw[dashed] (RefBl.north west) -- (Block.north west);
    \draw[dashed] (RefBl.north east) -- (Block.north east);
    
    \end{tikzpicture}
\end{figure}

\section{Identificare le Thread in un Kernel \texttt{CUDA}}
La programmazione in \texttt{CUDA} permette di identificare le thread
in uno spazio che può avere fino a tre dimensioni, rappresentate come
\((x, y, z)\). Questo consente agli sviluppatori di adattare l'assegnazione
delle thread alla struttura dei dati che devono essere elaborati.

\subsection{Array Monodimensionale}

Consideriamo il caso di un array di \(128\) elementi. In una struttura
monodimensionale, è naturale assegnare a ogni blocco \(128\) thread,
utilizzando solo la dimensione \(x\) per identificarle. In questo modo,
ogni thread può essere associato direttamente a un elemento dell'array,
e l'indice \texttt{x} del thread corrisponde all'indice dell'elemento
nell'array:

\begin{lstlisting}
    index = threadIdx.x; 
    array[index] = ...; // Operazioni su ogni elemento dell'array
\end{lstlisting}

\subsection{Matrice Bidimensionale}

Per una matrice di dimensioni \(20 \times 20\) \textit{$400$ elementi},
possiamo utilizzare una configurazione bidimensionale per i blocchi
e le thread. Assegnando \(400\) thread per blocco, ogni thread può
essere identificato tramite le coordinate \((x, y)\), permettendo
di mappare direttamente le coordinate della matrice:

\begin{lstlisting}
    row = threadIdx.y;
    col = threadIdx.x; 
    matrix[row][col] = ...; // Operazioni sulla matrice
\end{lstlisting}

\subsection{Strutture Tridimensionali}

Nel caso di strutture tridimensionali, come un cubo, possiamo estendere
ulteriormente questo concetto. I thread e i blocchi possono essere
configurati in tre dimensioni \((x, y, z)\), ognuna corrispondente
a una dimensione del cubo. Questo permette di indirizzare un elemento
specifico dentro un array tridimensionale in modo diretto:

\begin{lstlisting}
    row = threadIdx.y;
    col = threadIdx.x; 
    depth = threadIdx.z;
    cube[row][col][depth] = ...; // Operazioni sul cubo
\end{lstlisting}

\begin{figure}[H]
  \centering 
  \begin{tikzpicture}[line join=round, line cap=round,%
    x={(1 cm,0 cm)}, y={(0 cm,-1cm)}, z={(0.5 cm,0.5 cm)}]
    \foreach\i in {0,1}
    {%
    \cube{(5*\i,0,0)}{4}{yellow}{0}{1}{}{0}
    \foreach\c in {1,0} \foreach\b in {1,0} \foreach\a in {0,1,2} 
    {%
    \cube{(5*\i+1.2*\a-1.2,1.2*\b-0.6,1.2*\c-0.6)}{1}{orange}{1}{0}{$\a,\b,\c$}{0.5+3*\b}
    }
    \cube{(5*\i,0,0)}{4}{none}{1}{0}{Block $\i$}{2}
    }
  \end{tikzpicture}
\end{figure}
\section{Somma tra Vettori in \texttt{CUDA}}
Per esemplificare il concetto di programmazione parallela in \texttt{CUDA},
consideriamo il problema della somma tra due vettori. Questo problema
è particolarmente adatto alla programmazione parallela, poiché ogni
elemento del vettore può essere sommato indipendentemente dagli altri.

\begin{figure}[H]
  \centering
  \begin{tikzpicture}[
    node distance = 5mm and 0mm,
      start chain = A going right,
       box/.style = {rectangle, draw, inner sep=1mm, outer sep=0mm,
                     minimum height=5mm, minimum width=8mm,
                     on chain=A},
        LA/.style = {-{Straight Barb[flex=0]},
                     thick, shorten >=1mm, shorten <=1mm,
                     looseness=1.6}
                            ]
    \node [box, label=left:{$A=$}]  {$A_1$};        % A-1
    \node [box]                     {$A_2$};
    \node [box, densely dashed]     {};
    \node [box]                     {$A_N$};        % A-4
    %
    \node [box, label=left:{$B=$},
           below=of A-1]            {$B_1$};        % A-5
    \node [box]                     {$B_2$};
    \node [box, densely dashed]     {};
    \node [box]                     {$B_N$};        % A-8
    %
    \node [box,right=12mm of $(A-4.east)!0.5!(A-8.east)$]
                                    {$c_1$};  % A-9
    \node [box]                     {$c_2$};
    \node [box, densely dashed]     {};
    \node [box]                     {$c_N$};  % A-12
    %
    \coordinate[left=3mm of A-9.west] (aux);
    \draw[LA]   (A-4) to [out=0, in=180] (aux) -- (A-9);
    \draw[LA]   (A-8) to [out=0, in=180] (aux) -- (A-9);
  \end{tikzpicture}
\end{figure}
\subsection{Versione Sequenziale}
La versione sequenziale di questo algoritmo è molto semplice: basta
scorrere i due vettori e sommare gli elementi corrispondenti.

\begin{lstlisting}
  // Somma tra due vettori
  void vecAdd(float *A, float *B, float *C, int n) {
    for (int i = 0; i < n; i++) {
      C[i] = A[i] + B[i];
    }
  }

  int main()
  {
    // Allocazione e inizializzazione dei vettori
    float *A, *B, *C;
    int n = 1024;
    A = (float *)malloc(n * sizeof(float));
    B = (float *)malloc(n * sizeof(float));
    C = (float *)malloc(n * sizeof(float));
    for (int i = 0; i < n; i++) {
      A[i] = i;
      B[i] = i;
    }
    // Somma tra i vettori
    vecAdd(A, B, C, n);
    // Deallocazione della memoria
    free(A);
    free(B);
    free(C);
    return 0;
  }
\end{lstlisting}

\subsection{Processo di Parallelizzazione in \texttt{CUDA}}

La programmazione in \texttt{CUDA} per l'elaborazione parallela su dispositivi \texttt{GPU} si articola generalmente in tre fasi principali. Queste fasi sono fondamentali per il corretto trasferimento e elaborazione dei dati tra la \texttt{CPU} (host) e la \texttt{GPU} (device).

\subsubsection{Allocazione della Memoria su Device}

Il primo passo in un'applicazione \texttt{CUDA} consiste nell'allocazione della memoria sul device per le variabili necessarie. Per esempio, se abbiamo due array \(A\) e \(B\) che devono essere sommati per ottenere un array \(C\), è necessario allocare la memoria per \(A\), \(B\), e \(C\) sul device:

\begin{lstlisting}
    cudaMalloc(&A, size);
    cudaMalloc(&B, size);
    cudaMalloc(&C, size);
\end{lstlisting}

\subsubsection{Esecuzione del Kernel}

Una volta che la memoria è stata allocata e i dati iniziali sono stati trasferiti al device, il prossimo passo è il lancio del kernel. Il kernel è il codice eseguito parallelamente dai thread della \texttt{GPU}:

\begin{lstlisting}
    dim3 threadsPerBlock(256);
    dim3 numBlocks((N + threadsPerBlock.x - 1) / threadsPerBlock.x);
    addKernel<<<numBlocks, threadsPerBlock>>>(A, B, C);
\end{lstlisting}

Questo esempio mostra un kernel chiamato \texttt{addKernel} che somma due array.

\subsubsection{Copia dei Risultati alla Host Memory}

Dopo l'esecuzione del kernel, l'ultimo passo è copiare il risultato dall'array \(C\) dalla memoria del device alla memoria dell'host. Questo permette all'host di utilizzare i risultati calcolati dalla \texttt{GPU}:

\begin{lstlisting}
    cudaMemcpy(hostC, C, size, cudaMemcpyDeviceToHost);
\end{lstlisting}

\subsection{Versione Parallela in \texttt{CUDA}}

La versione parallela di questo algoritmo utilizza la programmazione parallela
per sfruttare la potenza di calcolo della \texttt{GPU}. Il kernel \texttt{vecAdd}
viene eseguito da un insieme di thread, ognuno dei quali somma un elemento
dei vettori \(A\) e \(B\) per produrre l'elemento corrispondente del vettore \(C\).

\begin{lstlisting}
  __global__ void vecAdd(float *A, float *B, float *C, int n) {
    int i = blockIdx.x * blockDim.x + threadIdx.x;
    if (i < n) {
      C[i] = A[i] + B[i];
    }
  }

  int main()
  {
    // Allocazione e inizializzazione dei vettori
    float *A, *B, *C;
    float *d_A, *d_B, *d_C;
    int n = 1024;
    A = (float *)malloc(n * sizeof(float));
    B = (float *)malloc(n * sizeof(float));
    C = (float *)malloc(n * sizeof(float));
    for (int i = 0; i < n; i++) {
      A[i] = i;
      B[i] = i;
    }
    // Allocazione della memoria sul device
    cudaMalloc(&d_A, n * sizeof(float));
    cudaMalloc(&d_B, n * sizeof(float));
    cudaMalloc(&d_C, n * sizeof(float));
    // Copia dei dati dalla host alla device
    cudaMemcpy(d_A, A, n * sizeof(float), cudaMemcpyHostToDevice);
    cudaMemcpy(d_B, B, n * sizeof(float), cudaMemcpyHostToDevice);
    // Esecuzione del kernel
    vecAdd<<<(n + 255) / 256, 256>>>(d_A, d_B, d_C, n);
    // Copia dei risultati dalla device alla host
    cudaMemcpy(C, d_C, n * sizeof(float), cudaMemcpyDeviceToHost);
    // Deallocazione della memoria
    free(A);
    free(B);
    free(C);
    cudaFree(d_A);
    cudaFree(d_B);
    cudaFree(d_C);
    return 0;
  }
\end{lstlisting}

\subsection{Dimensionamento dei Blocchi e delle Thread}
Quando si esegue un kernel \texttt{CUDA}, è necessario specificare
il numero di blocchi e il numero di thread per blocco. Questi valori
dipendono dalla dimensione del problema e dalla capacità della \texttt{GPU}.
In generale, è consigliabile utilizzare un numero di thread per blocco
che sia un multiplo di \(32\), poiché le \texttt{GPU} di \texttt{NVIDIA}
organizzano i thread in gruppi di \(32\) chiamati \textit{warps}.
\begin{lstlisting}
  dim3 DimGrid((N / 256), 1, 1);
  if (N % 256 != 0) DimGrid.x++;
  dim3 DimBlock(256, 1, 1);

  vecAdd<<<DimGrid, DimBlock>>>(d_A, d_B, d_C, N);
\end{lstlisting}
\subsection{Definizione di Funzioni in \texttt{CUDA}}

Le funzioni in \texttt{CUDA} sono classificate secondo il contesto in
cui possono essere eseguite e da cui possono essere chiamate:

\begin{itemize}
  \item \texttt{\_\_device\_\_ type fun()} - Definisce una funzione che
  può essere chiamata e eseguita solo sul dispositivo (device). Queste
  funzioni sono utilizzate per eseguire calcoli direttamente sulla \texttt{GPU}.
  \item \texttt{\_\_global\_\_ type fun()} - Specifica una funzione che, pur
  essendo definita per il device, viene lanciata dall'host. Questo tipo di
  funzione è comunemente noto come \textit{kernel} e rappresenta il cuore
  dell'esecuzione parallela in \texttt{CUDA}.
  \item \texttt{\_\_host\_\_ type fun()} - Una funzione destinata a essere
  chiamata ed eseguita sull'host, ossia la \texttt{CPU}.
\end{itemize}
In \texttt{CUDA}, è anche possibile definire funzioni che possono
essere eseguite sia sull'host che sul device utilizzando le direttive
\texttt{\_\_host\_\_ \_\_device\_\_}.

\section{Moltiplicazione di Matrici in \texttt{CUDA}}
La moltiplicazione delle matrici può essere espressa come $C = A \times B$,
dove $A$, $B$, e $C$ sono matrici. Ogni elemento della matrice $C$ è il
prodotto scalare di una riga di $A$ e una colonna di $B$. Questo problema
è particolarmente adatto alla programmazione parallela, poiché i calcoli
per ogni elemento di $C$ possono essere eseguiti indipendentemente dagli
altri.

Ogni thread calcola un elemento della matrice $C$ utilizzando la formula
$C_{ij} = \sum_{k=0}^{N-1} A_{ik} \times B_{kj}$, dove $i$ e $j$ sono
gli indici di riga e colonna di $C$, rispettivamente, e $k$ è l'indice
della somma.
\begin{figure}[H]
  \centering
  \begin{tikzpicture}[>=latex]
    \matrix (A) [matrix of math nodes,
                 nodes = {node style ge},
                 left delimiter  = (,
                 right delimiter = )] at (0,0)
    {
      a_{11} & a_{12} & \ldots & a_{1p}  \\
      |[node style sp]| a_{21}
             & |[node style sp]| a_{22}
                      & \ldots
                               & |[node style sp]| a_{2p} \\
      \vdots & \vdots & \ddots & \vdots  \\
      a_{n1} & a_{n2} & \ldots & a_{np}  \\
    };
    \node [draw,below=10pt] at (A.south) 
        { $A$ : \textcolor{red}{$n$ rows} $p$ columns};
    
    \matrix (B) [matrix of math nodes,
                 nodes = {node style ge},
                 left delimiter  = (,
                 right delimiter = )] at (6*\myunit,6*\myunit)
    {
      b_{11} & |[node style sp]| b_{12}
                      & \ldots & b_{1q}  \\
      b_{21} & |[node style sp]| b_{22}
                      & \ldots & b_{2q}  \\
      \vdots & \vdots & \ddots & \vdots  \\
      b_{p1} & |[node style sp]| b_{p2}
                      & \ldots & b_{pq}  \\
    };
    \node [draw,above=10pt] at (B.north) 
        { $B$ : $p$ rows \textcolor{red}{$q$ columns}};
    \matrix (C) [matrix of math nodes,
                 nodes = {node style ge},
                 left delimiter  = (,
                 right delimiter = )] at (6*\myunit,0)
    {
      c_{11} & c_{12} & \ldots & c_{1q} \\
      c_{21} & |[node style sp,red]| c_{22}
                      & \ldots & c_{2q} \\
      \vdots & \vdots & \ddots & \vdots \\
      c_{n1} & c_{n2} & \ldots & c_{nq} \\
    };
    \draw[blue] (A-2-1.north) -- (C-2-2.north);
    \draw[blue] (A-2-1.south) -- (C-2-2.south);
    \draw[blue] (B-1-2.west)  -- (C-2-2.west);
    \draw[blue] (B-1-2.east)  -- (C-2-2.east);
    \draw[<->,red](A-2-1) to[in=180,out=90]
      node[arrow style mul] (x) {$a_{21}\times b_{12}$} (B-1-2);
    \draw[<->,red](A-2-2) to[in=180,out=90]
      node[arrow style mul] (y) {$a_{22}\times b_{22}$} (B-2-2);
    \draw[<->,red](A-2-4) to[in=180,out=90]
      node[arrow style mul] (z) {$a_{2p}\times b_{p2}$} (B-4-2);
    \draw[red,->] (x) to node[arrow style plus] {$+$} (y)%
        to node[arrow style plus] {$+\raisebox{.5ex}{\ldots}+$} (z)
        to (C-2-2.north west);
    
    
    \node [draw,below=10pt] at (C.south) 
        {$ C=A\times B$ : \textcolor{red}{$n$ rows}
                          \textcolor{red}{$q$ columns}};
    
    \end{tikzpicture}
\end{figure}

\subsection{Versione Sequenziale}
La versione sequenziale di questo algoritmo è molto semplice: basta
scorrere le righe di $A$ e le colonne di $B$ per calcolare ogni elemento
di $C$.

\begin{lstlisting}
  // Moltiplicazione tra due matrici
  void matMul(float *A, float *B, float *C, int n, int p, int q) {
    for (int i = 0; i < n; i++) {
      for (int j = 0; j < q; j++) {
        C[i * q + j] = 0;
        for (int k = 0; k < p; k++) {
          C[i * q + j] += A[i * p + k] * B[k * q + j];
        }
      }
    }
  }

  int main()
  {
    // Allocazione e inizializzazione delle matrici
    float *A, *B, *C;
    int n = 1024, p = 512, q = 256;
    A = (float *)malloc(n * p * sizeof(float));
    B = (float *)malloc(p * q * sizeof(float));
    C = (float *)malloc(n * q * sizeof(float));
    for (int i = 0; i < n * p; i++) {
      A[i] = i;
    }
    for (int i = 0; i < p * q; i++) {
      B[i] = i;
    }
    // Moltiplicazione tra le matrici
    matMul(A, B, C, n, p, q);
    // Deallocazione della memoria
    free(A);
    free(B);
    free(C);
    return 0;
  }
\end{lstlisting}

\subsection{Versione Parallela in \texttt{CUDA}}
La programmazione parallela su \texttt{GPU} sfrutta la potenza di
calcolo elevata delle unità di elaborazione grafica per eseguire operazioni
matematiche complesse, come la moltiplicazione di matrici,
in modo significativamente più veloce rispetto ai processori tradizionali.

Per sfruttare al meglio le capacità delle \texttt{GPU}, è cruciale adottare
una rappresentazione efficiente delle matrici. Le matrici \(A\), \(B\) e \(C\)
vengono comunemente rappresentate come array monodimensionali per facilitare
l'accesso e migliorare le prestazioni. Questa rappresentazione, nota come
\textit{row-major order}, organizza le righe della matrice in modo contiguo
nella memoria.

\begin{figure}[H]
    \centering
    \begin{tikzpicture}[x={(1pt,0)},y={(0,1pt)},scale=0.7]
      \draw[line width=3pt,gray!50,-stealth] (-20,0) 
      \foreach \y in {0,...,3}
      { --  (250,-50*\y) arc(90:-90:12.5)
      --  (-20,-25-50*\y) arc(90:270:12.5)
      -- (-20,-50-50*\y) -- (250, -50-50*\y)
      } --(255,-200);
      \foreach \x in  {0,...,4}
      {\foreach \y in {0,...,4}
       { \node[draw,fill=white] at (12.5+50*\x,-50*\y) (X-\x-\y){(\x,\y)};}
      } 
      \end{tikzpicture}
      \caption{Rappresentazione di una matrice in \textit{row-major order}}
\end{figure}

Nell'ambito della programmazione \texttt{CUDA}, l'accesso agli elementi di
una matrice organizzata secondo il \textit{row-major order} è effettuato
attraverso la seguente formula:
\[
  \texttt{index} = \texttt{row} \times \texttt{width} + \texttt{col}
\]
Questa formula calcola l'indice lineare dell'elemento nella memoria
unidimensionale, dove \texttt{row} rappresenta l'indice della riga,
\texttt{width} la larghezza della matrice, e \texttt{col} l'indice
della colonna.

L'utilizzo del \textit{row-major order} permette un accesso alla memoria
più rapido e prevedibile, essenziale per il parallelismo su \texttt{GPU}.
Gli accessi alla memoria che seguono il pattern naturale della memoria cache
e dei banchi di memoria della \texttt{GPU} riducono i colli di bottiglia
e migliorano il throughput delle operazioni.


\subsection{Gestione dei grid e dei blocchi}
Nella moltiplicazione di matrici tramite \texttt{CUDA}, il dimensionamento
dei blocchi spesso si basa su una tecnica chiamata \textit{tiling}. Questo
approccio permette di ottimizzare l'uso della memoria e migliorare le prestazioni
globali del sistema. Ogni blocco è responsabile del calcolo di un sottoinsieme
della matrice risultante \(C\), e ogni thread all'interno di un blocco calcola
un singolo elemento di \(C\).

Il dimensionamento dei grid e dei blocchi si effettua in modo da dividere
la matrice in ``piastrelle" che possono essere elaborate efficacemente
dai blocchi di thread:

\begin{lstlisting}[language=C]
  // Dimensionamento dei blocchi e dei grid
  dim3 dimGrid(width/TILE_WIDTH, width/TILE_WIDTH, 1);
  dim3 dimBlock(TILE_WIDTH, TILE_WIDTH, 1);
  matMul<<<dimGrid, dimBlock>>>(d_A, d_B, d_C, width);
\end{lstlisting}

Una volta configurati i grid e i blocchi, il kernel \texttt{CUDA} viene
eseguito su ciascun blocco per calcolare i valori corrispondenti nella
matrice \(C\). Il kernel dettagliato è mostrato di seguito:

\begin{lstlisting}[language=C]
  // Definizione del kernel CUDA
  __global__ void matMul(float *A, float *B, float *C, int width) {
    int col = blockIdx.x * blockDim.x + threadIdx.x;
    int row = blockIdx.y * blockDim.y + threadIdx.y;
    float p_val = 0;
    if (row < width && col < width) {
        for (int k = 0; k < width; ++k) {
            p_val += A[row * width + k] * B[k * width + col];
        }
        C[row * width + col] = p_val;
    }
  }
\end{lstlisting}

Questo kernel esegue un ciclo attraverso ogni elemento corrispondente
nelle righe di \(A\) e nelle colonne di \(B\) per accumulare il prodotto
nel valore \(p\_val\), che viene poi assegnato all'elemento corrispondente in \(C\).

\section{Thread \texttt{CUDA}}
Le thread in \texttt{CUDA} sono organizzate in blocchi e griglie, che
forniscono una struttura gerarchica per l'esecuzione parallela. Ogni
thread è identificata da un indice unico all'interno del blocco e della
griglia, che può essere utilizzato per accedere ai dati e coordinare
le operazioni tra le thread.

\section{Modello di Programmazione \texttt{CUDA}}
Il modello di programmazione \texttt{CUDA} consente una scalabilità trasparente attraverso un'assegnazione dinamica dei blocchi di thread e una gestione flessibile delle esecuzioni, adattandosi automaticamente al numero di processori paralleli disponibili.

\subsection{Scalabilità Trasparente nel Modello \texttt{CUDA}}
Nel modello \texttt{CUDA}, l'hardware ha 
la completa libertà di assegnare i blocchi di thread a 
qualsiasi processore in qualsiasi momento, consentendo al kernel 
di scalare su un numero arbitrario di processori paralleli. Questa 
flessibilità è fondamentale per massimizzare l'efficienza di 
esecuzione e le prestazioni del sistema. 

\begin{figure}[H]
  \centering
  \includegraphics[width=0.7\textwidth]{img/transparent_scalability.png}
\end{figure}

\subsection{Dettagli Tecnici di Scheduling e Gestione dei \texttt{SM}}
I blocchi di thread sono organizzati in griglie di blocchi, e ogni blocco 
può essere eseguito in un ordine indipendente rispetto agli altri. 
La granularità dell'assegnazione dei thread ai multiprocessori 
streaming (\texttt{SM}) varia in base alla generazione del processore:
\begin{itemize}
    \item Gli \texttt{SM Fermi} possono gestire fino a $1536$ thread, 
    con configurazioni come $256$ thread per blocco per $6$ blocchi o $512$ thread per blocco per $3$ blocchi.
    \item Gli \texttt{SM Kepler} possono gestire fino a $2048$ thread, 
    con i thread che vengono eseguiti contemporaneamente.
\end{itemize}
Gli \texttt{SM} mantengono e gestiscono gli identificativi di thread 
e blocco, pianificando l'esecuzione dei thread in modo efficiente.

\subsection{Warps come Unità di Scheduling}
All'interno degli \texttt{SM}, ogni blocco viene eseguito come
warps di $32$ thread, che sono le unità di scheduling. Ad esempio,
se a un \texttt{SM} sono assegnati $3$ blocchi e ogni blocco ha
$256$ thread, il numero totale di warps per \texttt{SM} sarà di
$24$, calcolato come \(256/32 = 8\) warps per blocco moltiplicato 
per $3$ blocchi.

\section{Partizione dei Blocchi di Thread e Controllo del Flusso
in \texttt{CUDA}}
Comprendere come i blocchi di thread sono partizionati e gestiti
all'interno del modello di esecuzione di \texttt{CUDA} è fondamentale
per ottimizzare le prestazioni e garantire un comportamento corretto
dell'applicazione.

\subsection{Partizione dei Blocchi di Thread}
I blocchi di thread in \texttt{CUDA} sono partizionati in warp,
con \texttt{ID} di thread consecutivi e crescenti all'interno di
un warp. Questa partizione è consistente tra le esecuzioni, il
che permette ai programmatori di anticipare e pianificare il
comportamento dei thread all'interno dei warp:
\begin{itemize}
    \item Ogni warp inizia con l'\texttt{ID} di thread 0.
    \item I warp sono strutturati in modo che gli \texttt{ID} di
    thread siano sempre consecutivi, migliorando la prevedibilità
    nel controllo del flusso.
\end{itemize}
Tuttavia, i programmatori non dovrebbero fare affidamento su un
ordine specifico tra i warp a causa delle potenziali dipendenze
tra i thread, che necessitano di una sincronizzazione esplicita
utilizzando \texttt{\_\_syncthreads()} per ottenere risultati di esecuzione corretti.

\subsection{Controllo del Flusso in \texttt{CUDA}}
Il controllo del flusso all'interno di \texttt{CUDA} è soggetto a
problemi di divergenza, dove i thread all'interno di un singolo warp
possono seguire percorsi di esecuzione differenti:
\begin{itemize}
    \item La divergenza si verifica quando i thread in un warp seguono
    percorsi di controllo differenti, che vengono poi serializzati,
    potenzialmente degradando le prestazioni.
    \item Un esempio di divergenza: se \texttt{(threadIdx.x > 2)},
    questa condizione crea due percorsi all'interno del primo warp.
    \item Per minimizzare la divergenza, è consigliabile allineare
    le condizioni di ramificazione con i confini dei warp, ad esempio,
    \texttt{if (threadIdx.x / WARP\_SIZE > 2)}.
\end{itemize}

\subsection{Schedulazione e Esecuzione dei Warp}
La schedulazione dei warp è gestita dai Multiprocessori di Streaming
(\texttt{SM}) con zero sovraccarico:
\begin{itemize}
    \item I warp vengono eseguiti in base alla prontezza degli operandi
    e alle politiche di schedulazione prioritarie.
    \item Questo assicura che tutti i thread in un warp eseguano
    la stessa istruzione simultaneamente quando selezionati, ottimizzando il throughput.
\end{itemize}

\subsection{Considerazioni sulla Granularità dei Blocchi per una
Performance Ottimale}
Scegliere la dimensione giusta del blocco è cruciale per la performance,
specialmente in applicazioni come la moltiplicazione di matrici:
\begin{itemize}
    \item Per una configurazione di blocco $8 \times 8$
    (\textit{$64$ thread per blocco}), un \texttt{SM} può supportare
    fino a $24$ blocchi, ma a causa dei vincoli hardware, spesso solo
    $8$ blocchi possono essere attivi contemporaneamente.
    \item Una configurazione di blocco $16 \times 16$
    (\textit{$256$ thread per blocco}) permette a un \texttt{SM}
    di operare vicino alla piena capacità con fino a $6$ blocchi.
    \item Blocchi più grandi, come $32 \times 32$
    (\textit{$1024$ thread per blocco}), possono superare la
    capacità per \texttt{SM} di alcune versioni di \texttt{CUDA}
    e hardware, riducendo l'efficienza.
\end{itemize}

Ciascuna di queste considerazioni gioca un ruolo significativo nel
determinare quanto efficacemente un programma \texttt{CUDA} può operare,
e comprenderle è fondamentale per sfruttare appieno le capacità di \texttt{CUDA}.

\chapter{\texttt{GPU} e memoria}
\section{Panoramica della Memoria in \texttt{CUDA} e Dichiarazione delle Variabili}

\subsection{Accesso alla Memoria nei Thread \texttt{CUDA}}
In \texttt{CUDA}, ogni thread può accedere a diverse tipologie di memoria con tempi di accesso variabili, che influenzano le prestazioni dell'applicazione:
\begin{itemize}
    \item \textbf{Registri per thread:} Ogni thread può leggere e scrivere nei registri specifici per thread con un tempo di accesso molto basso, circa 1 ciclo.
    \item \textbf{Memoria condivisa per blocco:} La memoria condivisa accessibile da tutti i thread di un blocco ha un tempo di accesso di circa 5 cicli.
    \item \textbf{Memoria globale per griglia:} Tutti i thread possono accedere alla memoria globale con un tempo di accesso significativamente più elevato, circa 500 cicli.
    \item \textbf{Memoria costante per griglia:} Accessibile in sola lettura, la memoria costante ha un tempo di accesso di circa 5 cicli se c'è caching.
\end{itemize}

\subsection{Qualificatori delle Variabili in \texttt{CUDA}}
Le variabili in \texttt{CUDA} possono essere qualificate in modo specifico per
definire la loro visibilità e il luogo di memorizzazione. Di seguito è presentata
una tabella che riassume i principali qualificatori delle variabili, la loro
visibilità e i luoghi di memorizzazione.

\begin{table}[H]
\centering
\begin{tabular}{|l|l|l|l|}
\hline
\textbf{Dichiarazione} & \textbf{Memoria} & \textbf{Visibilità} & \textbf{tempo di vita} \\
\hline
\texttt{int LocalVar;} & Registri & Thread & Thread \\
\texttt{\_\_device\_\_ int GlobalVar;} & Globale & Grid & Applicazione \\
\texttt{\_\_device\_\_ \_\_shared\_\_ int SharedVar;} & Condivisa & Blocco & Blocco \\
\texttt{\_\_device\_\_ \_\_constant\_\_ int ConstVar;} & Costante & Grid & Applicazione \\
\hline
\end{tabular}
\caption{Qualificatori delle variabili in \texttt{CUDA} e loro proprietà}
\label{tab:cuda_variable_qualifiers}
\end{table}
\subsubsection{Dove Dichiarare le Variabili?}
\begin{itemize}
    \item Variabili \texttt{global} e \texttt{constant} vanno dichiarate
    all'esterno di qualsiasi funzione per essere accessibili dall'host.
    \item Variabili \texttt{register} (automatiche) e \texttt{shared}
    sono dichiarate all'interno dei kernel.
\end{itemize}

\subsubsection{Esempio}
\begin{lstlisting}
__global__ void MatrixMulKernel(float* d_M, float* d_N, float* d_P, int Width) {
    __shared__ float ds_M[TILE_WIDTH][TILE_WIDTH];
    __shared__ float ds_N[TILE_WIDTH][TILE_WIDTH];
\end{lstlisting}


\section{Strategia di Programmazione Comune in \texttt{CUDA}: Tiling e Blocking}

\subsection{Utilizzo della Memoria Globale e Condivisa}
La memoria globale, situata nella memoria del dispositivo (\texttt{DRAM}), presenta
tempi di accesso lenti. Un metodo efficace per eseguire calcoli sul dispositivo
è quindi quello di usare la tecnica di tiling, che sfrutta la memoria condivisa
più veloce:
\begin{itemize}
    \item \textbf{Partizionamento dei dati:} I dati vengono suddivisi in sottoinsiemi
    che si adattano alla memoria condivisa.
    \item \textbf{Caricamento dei dati:} Ogni blocco di thread carica un sottoinsieme 
    dalla memoria globale alla memoria condivisa, utilizzando più thread per sfruttare 
    il parallelismo a livello di memoria.
    \item \textbf{Computazione sui dati:} I thread eseguono calcoli sui dati nella 
    memoria condivisa. Ogni thread può passare efficientemente più volte su qualsiasi 
    elemento dei dati.
    \item \textbf{Copia dei risultati:} I risultati vengono copiati dalla memoria 
    condivisa alla memoria globale.
\end{itemize}

\subsection{Vantaggi del Tiling e del Blocking}
\begin{itemize}
    \item \textbf{Efficienza:} L'utilizzo di memoria condivisa riduce la latenza di 
    accesso ai dati e aumenta la velocità di esecuzione.
    \item \textbf{Parallelismo a livello di memoria:} Più thread possono caricare e 
    scrivere dati in modo simultaneo, migliorando ulteriormente le prestazioni.
\end{itemize}

\subsection{Considerazioni sul Timing di Accesso dei Thread}
\begin{itemize}
    \item \textbf{Accesso simile:} Il tiling è particolarmente vantaggioso quando i 
    thread hanno tempi di accesso simili, poiché questo permette di massimizzare 
    l'efficienza e minimizzare i conflitti di accesso alla memoria.
    \item \textbf{Accesso diverso:} Se i tempi di accesso dei thread sono molto 
    diversi, il tiling potrebbe non essere efficace e potrebbe portare a una 
    serializzazione indesiderata delle operazioni, riducendo le prestazioni.
\end{itemize}

Queste strategie di programmazione sono cruciali per ottimizzare l'utilizzo delle 
risorse del dispositivo e migliorare le prestazioni generali delle applicazioni 
\texttt{CUDA}.

\section{Moltiplicazione di Matrici in \texttt{CUDA}: Tiling e Blocking}

\subsection{Descrizione dell'Algoritmo}
La moltiplicazione di matrici in \texttt{CUDA} può essere notevolmente
ottimizzata attraverso l'uso del tiling e del blocking. Queste tecniche
sfruttano la memoria condivisa del dispositivo per ridurre il numero di
accessi alla memoria globale, che è più lenta. L'idea di base è suddividere
le matrici in blocchi o ``tiles" più piccoli che possono essere caricati nella
memoria condivisa, permettendo così ai thread di lavorare su porzioni di dati 
più gestibili contemporaneamente.

\subsection{Implementazione in \texttt{CUDA}}
Il kernel \texttt{CUDA} per la moltiplicazione di matrici utilizzando il
tiling segue questi passaggi principali:
\begin{enumerate}
    \item Partizionare le matrici di input in blocchi di dimensioni
    \texttt{TILE\_WIDTH} x \texttt{TILE\_WIDTH}.
    \item Ogni blocco di thread carica un tile di una matrice in memoria condivisa,
    sincronizzandosi con gli altri thread per assicurare che tutti i dati siano
    caricati prima di procedere al calcolo.
    \item Calcolare il prodotto del blocco moltiplicando i corrispondenti tile delle
    due matrici.
    \item Ogni thread accumula il risultato in una variabile locale e, una volta
    completato il calcolo per tutti i tile, scrive il risultato nella memoria globale.
\end{enumerate}

\subsection{Codice \texttt{CUDA} per il Kernel di Moltiplicazione}
Di seguito è riportato un esempio di implementazione di un kernel \texttt{CUDA}
per la moltiplicazione di matrici utilizzando il tiling:

\begin{lstlisting}
__global__ void MatrixMulKernel(float* d_M, float* d_N, float* d_P, int Width) {
    __shared__ float ds_M[TILE_WIDTH][TILE_WIDTH];
    __shared__ float ds_N[TILE_WIDTH][TILE_WIDTH];
    int bx = blockIdx.x; int by = blockIdx.y;
    int tx = threadIdx.x; int ty = threadIdx.y;
    int Row = by * TILE_WIDTH + ty;
    int Col = bx * TILE_WIDTH + tx;
    float Pvalue = 0;
    for (int m = 0; m < Width/TILE_WIDTH; ++m) {
        ds_M[ty][tx] = d_M[Row * Width + (m * TILE_WIDTH + tx)];
        ds_N[ty][tx] = d_N[Col + (m * TILE_WIDTH + ty) * Width];
        __syncthreads();
        for (int k = 0; k < TILE_WIDTH; ++k) {
            Pvalue += ds_M[ty][k] * ds_N[k][tx];
        }
        __syncthreads();
    }
    d_P[Row * Width + Col] = Pvalue;
}
\end{lstlisting}

Questo kernel \texttt{CUDA} esegue la moltiplicazione di matrici utilizzando il tiling
per ottimizzare l'accesso alla memoria e migliorare le prestazioni complessive.

\subsection{Spiegazione del Codice}
\begin{figure}[H]
    \centering
    \includegraphics[width=0.6\textwidth]{img/matrix_mul_tiles.png}
    \caption{Moltiplicazione di matrici con tiling}
\end{figure}
\begin{itemize}
    \item Il kernel carica i tile delle matrici \(M\) e \(N\) in memoria condivisa
    e calcola il prodotto del blocco.
    \item I thread sincronizzano l'accesso alla memoria condivisa per evitare
    conflitti e assicurare che tutti i dati siano disponibili per il calcolo.
    \item Il risultato viene accumulato in una variabile locale e scritto nella
    memoria globale.
\end{itemize}

Questo approccio sfrutta la memoria condivisa per ridurre i tempi di accesso
alla memoria globale e migliorare le prestazioni della moltiplicazione di matrici
su \texttt{CUDA}.

\subsection{Dimensionamento dei Blocchi di Thread e Utilizzo della Memoria Condivisa}
La scelta della dimensione dei blocchi di thread e della larghezza delle tessere (\texttt{TILE\_WIDTH}) ha un impatto significativo sulla performance dei programmi \texttt{CUDA}. 

\begin{itemize}
    \item Un \texttt{TILE\_WIDTH} di $16$ porta a blocchi di $256$
    thread ($16\cdot16$), mentre un \texttt{TILE\_WIDTH} di $32$ comporta
    blocchi di 1024 thread (32x32).
    \item Per una \texttt{TILE\_WIDTH} di $16$, ogni blocco esegue
    $512$ caricamenti di float dalla memoria globale e può eseguire
    $8,192$ operazioni di moltiplicazione/addizione.
    \item Con una \texttt{TILE\_WIDTH} di $32$, il numero di
    operazioni di moltiplicazione/addizione sale a $65536$, per
    un totale di $2048$ caricamenti di float.
\end{itemize}

Queste operazioni di caricamento influenzano direttamente l'utilizzo
della memoria condivisa e la capacità di avere più blocchi attivi
contemporaneamente su un singolo Multiprocessore di Streaming (\texttt{SM}):
\begin{itemize}
    \item Con \texttt{TILE\_WIDTH} = $16$, ogni blocco di thread utilizza
    $2KB$ di memoria condivisa ($2*256*4B$), consentendo potenzialmente
    fino a 8 blocchi di thread attivi contemporaneamente.
    \item Con \texttt{TILE\_WIDTH} = $32$, l'utilizzo di memoria condivisa
    aumenta a $8KB$ per blocco, limitando il numero di blocchi attivi a
    $2$ o $6$ a seconda della configurazione della memoria condivisa e
    della dimensione totale disponibile sull'\texttt{SM}.
\end{itemize}

Utilizzando un tiling di 16x16, si riducono gli accessi alla memoria
globale di un fattore 16, incrementando così efficacemente la larghezza
di banda utilizzabile e il throughput di calcolo.

\subsection{Interrogazione delle Proprietà del Dispositivo}
Per sfruttare al meglio le risorse hardware, è essenziale interrogare
le proprietà del dispositivo \texttt{CUDA} disponibile:
\begin{lstlisting}
int dev_count;
cudaGetDeviceCount(&dev_count);
cudaDeviceProp dev_prop;
for (int i = 0; i < dev_count; i++) {
    cudaGetDeviceProperties(&dev_prop, i);
    // Decisioni basate su dev_prop.maxThreadsPerBlock,dev_prop.sharedMemPerBlock, ...
}
\end{lstlisting}
Questo codice permette di determinare il numero di dispositivi e
le loro specifiche, come il numero massimo di thread per blocco e
la memoria condivisa per blocco, elementi cruciali per configurare
correttamente i kernel.

\subsection{Riepilogo - Struttura Tipica di un Programma \texttt{CUDA}}
Un programma \texttt{CUDA} tipico segue questa struttura:
\begin{enumerate}
    \item Definizione dei kernel e configurazione dei blocchi di thread.
    \item Allocazione e inizializzazione delle strutture di dati.
    \item Trasferimento dei dati dall'host al device.
    \item Esecuzione dei kernel.
    \item Copia dei risultati dal device all'host.
    \item Liberazione delle risorse.
\end{enumerate}
Ripetere questi passi secondo necessità per ottenere il comportamento
desiderato e massimizzare le prestazioni.

\chapter{\texttt{GPU} e considerazioni sulle performance}

\section{Coalescenza della Memoria in \texttt{CUDA}}
La coalescenza della memoria è un concetto chiave per ottimizzare le prestazioni
delle applicazioni \texttt{CUDA}. La coalescenza si riferisce all'accesso
sequenziale e allineato alla memoria da parte dei thread, che consente alla
\texttt{GPU} di combinare le richieste di accesso in un singolo ciclo di
memoria, migliorando così l'efficienza e riducendo i tempi di accesso.

Consideriamo una matrice, rappresentata come un array monodimensionale,
in condizioni normali, l'accesso ai dati da parte dei thread non è
coalescente, poiché i thread accedono a posizioni di memoria non contigue.
Tuttavia, organizzando la matrice in modo che i thread accedano a posizioni
contigue, è possibile ottenere un accesso coalescente, che consente alla
\texttt{GPU} di combinare le richieste di accesso in un singolo ciclo di
memoria.
\begin{figure}[H]
  \centering
  \includegraphics[width=0.8\textwidth]{img/coalescenza.png}
  \caption{Accesso coalescente alla memoria}
\end{figure}
\begin{figure}[H]
  \centering
  \includegraphics[width=0.8\textwidth]{img/non-coalescenza.png}
  \caption{Accesso non coalescente alla memoria}
\end{figure}
\subsection{Coalescenza della memoria}
La coalescenza della memoria si occupa degli accessi alla memoria globale.
A differenza della memoria condivisa, la memoria globale può subire penalità
di prestazioni a causa di accessi non coalescenti. Gli accessi alla memoria
coalescenti riguardano i thread dello stesso warp; l'hardware verifica se gli
accessi alla memoria globale sono eseguiti dai thread dello stesso mezzo warp
e, in tal caso, coalesce gli accessi dei thread in un unico accesso consolidato.

\section{Partizione dinamica delle risorse di esecuzione}
\subsection{Risorse di esecuzione in uno \texttt{SM}}
Le risorse di esecuzione in uno Streaming Multiprocessor (\texttt{SM}) includono
registri, memoria condivisa e slot per blocchi di thread. Analizziamo un
dispositivo con un massimo di $1536$ thread per \texttt{SM}:
\begin{itemize}
  \item Con blocchi da $512$ thread, è possibile ospitare 3 blocchi per \texttt{SM},
  una configurazione ottimale.
  \item Con blocchi da $128$ thread, si teorizzano $12$ blocchi per \texttt{SM}, ma questo
  supera le capacità e si limita a $8$ blocchi da $128$ thread ciascuno, totalizzando
  $1024$ thread, sotto-utilizzando così le capacità dello \texttt{SM}.
\end{itemize}

\subsection{Gestione dei registri per \texttt{SM}}
Considerando uno \texttt{SM} con $16384$ registri disponibili:
\begin{itemize}
  \item Se un kernel istanzia 10 variabili ($32$ bit) per thread:
  \begin{itemize}
    \item Con blocchi da $256$ thread, si necessitano 2560 registri per blocco.
    \item Con un rapporto di $1536$ thread per \texttt{SM} suddivisi in blocchi da
    $256$ thread, si possono ospitare $6$ blocchi, che utilizzano 15360 registri in
    totale, una configurazione accettabile.
    \item Aggiungere anche solo un blocco ulteriore per \texttt{SM} comporterebbe
    il superamento del limite di registri disponibili, richiedendo una riduzione
    nel numero di blocchi.
  \end{itemize}
  \item Aggiungendo due variabili per kernel ($12$ variabili in totale per thread):
  \begin{itemize}
    \item Ogni blocco da $256$ thread richiederebbe 3072 registri.
    \item Sei blocchi da $3072$ registri ciascuno comportano un totale di 18432
    registri, superando il limite. La soluzione comporta una riduzione del
    numero di blocchi a 5, rientrando così nei $15360$ registri disponibili,
    ma riducendo il numero di thread attivi per \texttt{SM} a $1280$ anziché $1536$, a
    causa dell'aumento del numero di variabili.
  \end{itemize}
\end{itemize}

\section{Parallelizzazione dei task per il trasferimento dati}
Quando si schedula un kernel, è possibile definire un grafo di dipendenza tra
le esecuzioni dei kernel e i trasferimenti di memoria. Per esempio, il kernel
A non può eseguire
fino a quando non sono completati i trasferimenti di dati A e B, ma può iniziare
mentre il trasferimento C è ancora in corso.

\texttt{CUDA} offre la possibilità di eseguire operazioni di trasferimento dati
e computazione in modo asincrono attraverso l'uso degli stream. Gli stream
permettono di eseguire trasferimenti di dati e kernel in parallelo, migliorando
l'utilizzo delle risorse e riducendo i tempi di attesa.
\section{Device Overlap}

Nel contesto della programmazione \texttt{CUDA}, il \texttt{deviceOverlap}
rappresenta una funzionalità importante per aumentare l'efficienza e
la velocità di esecuzione dei programmi. Questa funzione permette ai
dispositivi \texttt{CUDA} di eseguire simultaneamente un kernel
e una operazione di copia di memoria tra il dispositivo e l'host.

\section*{Verifica del Supporto a \texttt{deviceOverlap}}
Per determinare se un dispositivo \texttt{CUDA} supporta il
\texttt{deviceOverlap},
è possibile utilizzare il seguente codice:

\begin{lstlisting}
int dev_count; 
cudaDeviceProp prop;
cudaGetDeviceCount(&dev_count);
for (int i = 0; i < dev_count; i++) {
    cudaGetDeviceProperties(&prop, i);
    if (prop.deviceOverlap) {
        // Il dispositivo supporta deviceOverlap
    }
}
\end{lstlisting}

Questo frammento di codice prima interroga il numero totale di dispositivi
\texttt{CUDA} disponibili. Successivamente, per ogni dispositivo, acquisisce
e analizza le sue proprietà attraverso \texttt{cudaGetDeviceProperties} per
verificare la presenza della caratteristica \texttt{deviceOverlap}.

\section*{Utilizzo del Timing Sovrapposto (Pipelined)}
Una tecnica efficace per sfruttare al meglio il \texttt{deviceOverlap}
consiste nel dividere i vettori di grandi dimensioni in segmenti più piccoli
e gestire il trasferimento e il calcolo di questi segmenti in maniera
sovrapposta. Questo approccio, noto come \textit{pipelined timing},
permette di ridurre i tempi morti in cui il dispositivo potrebbe altrimenti
rimanere inattivo, ottimizzando così sia le operazioni di trasferimento che
quelle di calcolo.

\subsection{Stram di \texttt{CUDA}}
Il parallelismo di task in \texttt{CUDA} è realizzato mediante divers
 \texttt{streams}. Le operazioni inserite in \texttt{streams} differenti
 possono essere eseguite in parallelo. Questo è particolarmente utile per
 sovrapporre la copia di memoria e l'esecuzione di kernel, migliorando
 l'efficienza complessiva del programma.

Le richieste di dispositivo effettuate dal codice host sono inserite
in una coda, che viene letta ed elaborata in modo asincrono dal driver
e dal dispositivo. Il driver assicura che i comandi nella coda siano
processati in sequenza, garantendo che le copie di memoria siano
completate prima del lancio dei kernel, tra gli altri.

Per permettere la copia e l'esecuzione di kernel in modo concorrente, è
necessario utilizzare più code, denominate \texttt{streams}. Gli ``eventi"
\texttt{CUDA} permettono al thread host di interrogare e sincronizzarsi
con le code individuali.

\section*{Codice Host Multi-Stream Semplice}
Di seguito è presentato un esempio di codice host che utilizza due
\texttt{streams} per gestire le operazioni in modo parallelo:

\begin{lstlisting}[language=C]
cudaStream_t stream0, stream1;
cudaStreamCreate(&stream0);
cudaStreamCreate(&stream1);

float *d_A0, *d_B0, *d_C0; // memoria del dispositivo per stream 0
float *d_A1, *d_B1, *d_C1; // memoria del dispositivo per stream 1

// Allocazione di cudaMalloc per d_A0, d_B0, d_C0, d_A1, d_B1, d_C1

for (int i = 0; i < n; i += SegSize * 2) {
    cudaMemCpyAsync(d_A0, h_A + i, SegSize * sizeof(float), ..., stream0);
    cudaMemCpyAsync(d_B0, h_B + i, SegSize * sizeof(float), ..., stream0);
    vecAdd<<<SegSize / 256, 256, 0, stream0>>>(d_A0, d_B0, ...);
    cudaMemCpyAsync(d_C0, h_C + i, SegSize * sizeof(float), ..., stream0);

    cudaMemCpyAsync(d_A1, h_A + i + SegSize, SegSize * sizeof(float), ..., stream1);
    cudaMemCpyAsync(d_B1, h_B + i + SegSize, SegSize * sizeof(float), ..., stream1);
    vecAdd<<<SegSize / 256, 256, 0, stream1>>>(d_A1, d_B1, ...);
    cudaMemCpyAsync(d_C1, h_C + i + SegSize, SegSize * sizeof(float), ..., stream1);
}
\end{lstlisting}

Il problema di questo codice è che non avviene in trasferimento dei dati
nel momento in cui il kernel esegue la somma vettoriale. Questo problema
può essere risolto riorganizzando il codice in modo che i trasferimenti
di dati avvengano in modo corretto:
\begin{lstlisting}[language=C]
for (int i = 0; i < n; i += SegSize * 2) {
    cudaMemCpyAsync(d_A0, h_A + i, SegSize * sizeof(float), ..., stream0);
    cudaMemCpyAsync(d_A1, h_A + i + SegSize, SegSize * sizeof(float), ..., stream1);
    cudaMemCpyAsync(d_B0, h_B + i, SegSize * sizeof(float), ..., stream0);
    cudaMemCpyAsync(d_B1, h_B + i + SegSize, SegSize * sizeof(float), ..., stream1);

    vecAdd<<<SegSize / 256, 256, 0, stream0>>>(d_A0, d_B0, ...);
    vecAdd<<<SegSize / 256, 256, 0, stream1>>>(d_A1, d_B1, ...);

    cudaMemCpyAsync(d_C0, h_C + i, SegSize * sizeof(float), ..., stream0);
    cudaMemCpyAsync(d_C1, h_C + i + SegSize, SegSize * sizeof(float), ..., stream1);
}
\end{lstlisting}

\subsection{Concorrenza in Fermi e Precedenti}
Le architetture \texttt{GPU} di \texttt{NVIDIA} precedenti a Kepler, come Fermi,
supportavano una concorrenza a $16$ vie. Questo significava che fino a
$16$ griglie potevano essere eseguite contemporaneamente. Tuttavia, le
\texttt{CUDA streams} venivano multiplexate in una singola coda hardware
per l'esecuzione, limitando la concorrenza a livello di stream.

\subsection{Hyper Queue in Kepler}
Con l'introduzione di Kepler, \texttt{NVIDIA} ha implementato Hyper Queue,
che fornisce multiple code reali per ciascun motore, permettendo una maggiore
concorrenza. Questo approccio consente ad alcune stream di fare progressi
per un motore mentre altre possono essere bloccate, migliorando l'efficienza
e la scalabilità.

\subsection{Miglioramento della Concorrenza con Kepler}
Kepler ha significativamente migliorato la concorrenza rispetto a Fermi,
consentendo una concorrenza a $32$ vie. Ogni stream ha una propria coda
di lavoro dedicata, che elimina le dipendenze inter-stream e consente
la concorrenza a livello di intero stream, migliorando così le
prestazioni generali:

\begin{itemize}
    \item \textbf{Concorrenza $32$-vie:} Ogni \texttt{GPU} Kepler può gestire
    fino a $32$ stream contemporaneamente.
    \item \textbf{Code di lavoro multiple:} Una coda di lavoro per ogni stream.
    \item \textbf{Nessuna dipendenza inter-stream:} Ogni stream opera
    indipendentemente dagli altri, permettendo una migliore scalabilità
    e efficienza.
\end{itemize}
\section{Introduzione al Prefix Scan}
L'operazione di \textbf{Prefix Scan}, prende in input:
\begin{itemize}
  \item Un operatore binario associativo $\oplus$.
  \item Un array di $N$ elementi $[x_0, x_1, \dots, x_{N-1}]$.
\end{itemize}
Essa restituisce un array trasformato che può essere:
\begin{itemize}
  \item \textbf{Inclusivo:} $[x_0, (x_0 \oplus x_1), \dots,
  (x_0 \oplus x_1 \oplus \dots \oplus x_{N-1})]$.
  \item \textbf{Esclusivo:} $[0, x_0, (x_0 \oplus x_1), \dots,
  (x_0 \oplus x_1 \oplus \dots \oplus x_{N-2})]$.
\end{itemize}

\subsubsection{Esempio}
Per $\oplus$ come addizione (somma prefissa):
\begin{itemize}
  \item \textbf{Inclusivo:} Da $[3, 1, 7, 4, 6]$ a $[3, 4, 11, 15, 21]$.
  \item \textbf{Esclusivo:} Da $[3, 1, 7, 4, 6]$ a $[0, 3, 4, 11, 15]$.
\end{itemize}
\begin{figure}[H]
  \centering
  \includegraphics[width=0.7\textwidth]{img/scan_1.png}
  \caption{Esempio di Prefix Scan}
\end{figure}
\subsubsection{Conversione tra Scansioni Inclusive ed Esclusive}
\begin{itemize}
  \item Da esclusiva a inclusiva:
  \begin{itemize}
    \item Spostare l'array risultante di un elemento a sinistra.
    \item Inserire alla fine la somma dell'ultimo elemento della
    scansione e l'ultimo elemento dell'array di input.
  \end{itemize}
  \item Da inclusiva a esclusiva:
  \begin{itemize}
    \item Spostare l'array risultante di un elemento a destra.
    \item Inserire l'identità all'inizio.
  \end{itemize}
\end{itemize}

\subsection{Usi del Prefix Scan}
Il Prefix Scan è una primitiva chiave in molti algoritmi paralleli
per convertire calcoli seriali in paralleli, utilizzato in:
\begin{itemize}
  \item Ordinamento (counting sort/radix sort)
  \item Istogramma
  \item Inserimento in coda di massa
  \item Compattazione di flusso/Partizione
  \item Moltiplicazione di matrici sparse
  \item Costruzione di strutture dati in parallelo
  \item e altro...
\end{itemize}

\subsection{Implementazione Sequenziale}
\textbf{Inclusivo:}
\begin{lstlisting}
Out[0] = In[0];
for (int i = 1; i < N; ++i) {
  Out[i] = Out[i-1] + In[i];
}
\end{lstlisting}

\textbf{Esclusivo:}
\begin{lstlisting}
Out[0] = 0;
for (int i = 1; i < N; ++i) {
  Out[i] = Out[i-1] + In[i-1];
}
\end{lstlisting}
\textbf{Nota:} Sono necessarie $N$ addizioni per $N$ elementi,
con una complessità di $O(n)$.

\subsection{Prefix Scan Parallelo}

Di seguito è riportato l'algoritmo per un'implementazione parallela naive del
Prefix Scan, usando un approccio iterativo che aumenta progressivamente la
distanza tra gli elementi combinati ad ogni step.

\begin{algorithm}[H]
\caption{Naive Parallel Prefix Scan}
\label{alg:parallel_prefix_scan}
  \For{$int \texttt{ level} = 0$; $level < \log_2(N)$; $\texttt{++level}$}{
    \ForPar{$\forall i \in N$}{
      \texttt{offset} = $2^{\texttt{level}}$\\
    }
    \If{$i \geq \texttt{offset}$}{
      $X[i] = X[i - \texttt{offset}] + X[i]$\\
    }
    }
\end{algorithm}

La complessità parallela, \textit{span}, di questo algoritmo è $O(\log(N))$, 
mentre la complessità di lavoro è $O(n \cdot \log(n))$, dove $n$ è il numero di
elementi da scansionare, poiché ogni elemento deve essere combinato con tutti
gli altri.

\section{Prefix scan per l'efficienza di lavoro}
L'algoritmo consiste in due fasi principali:
\begin{itemize}
  \item \textit{up-sweep}: calcola la somma prefissa per ogni potenza di $2$.
  \item \textit{down-sweep}: calcola la somma prefissa finale.
\end{itemize}
\begin{figure}[H]
  \centering
  \includegraphics[width=0.7\textwidth]{img/scan_2_fase1.png}
  \caption{Fase 1: Up-Sweep}
\end{figure}
\begin{figure}[H]
  \centering
  \includegraphics[width=0.7\textwidth]{img/scan_2_fase2.png}
  \caption{Fase 2: Down-Sweep}
\end{figure}

Di seguito è riportato il codice per l'implementazione parallela del Prefix Scan
con efficienza di lavoro:

\begin{algorithm}[H]
\caption{Efficient Parallel Prefix Scan}
\label{alg:efficient_parallel_prefix_scan}
\For{$\texttt{level} = 0$; $\texttt{level} < \log_2(N)$; $\texttt{++level}$}{
  \texttt{step} = $2^{\texttt{level}}$\\
  \ForPar{$\forall i \mid i \mod (\texttt{step} \cdot 2) = 0$}{
    \texttt{valueRight} = $(i + 1) \cdot (step \cdot 2) - 1$\\
    \texttt{valueLeft} = \texttt{valueRight} - \texttt{step}\\
    $X[\texttt{valueRight}] = X[\texttt{valueRight}] + X[\texttt{valueLeft}]$\\
  }
}
\end{algorithm}

In questo modo si riduce la complessità di lavoro a $O(n)$, mantenendo
una complessità parallela di $O(\log(n))$.
\end{document}
