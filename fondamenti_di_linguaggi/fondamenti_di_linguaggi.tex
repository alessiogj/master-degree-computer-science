\documentclass[oneside,a4paper,11pt]{book}
\usepackage[utf8]{inputenc}
\usepackage{svg}
\usepackage[italian]{babel}
\usepackage{float}
\usepackage{fancyvrb}
\usepackage{titling}
\usepackage[margin=1in,footskip=0.25in]{geometry}
\usepackage{listings}
\usepackage[DIV=12,BCOR=2mm,headinclude=true,footinclude=false]{typearea}
\usepackage{color, colortbl,xcolor}
\usepackage[hidelinks]{hyperref}
\usepackage{tcolorbox}
\usepackage{chngcntr}
\usepackage{diagbox}
\usepackage{calc}
\usepackage{amssymb}
\usepackage{subcaption}
\usepackage{amsthm}
\usepackage{amsfonts}
\usepackage{mathtools}
\usepackage{parskip}
\usepackage{cancel}
\usepackage{forest}
\usepackage{listings}
\usepackage{mathrsfs}
\usepackage{enumitem}
\usepackage{makecell}
\usepackage{tikz}
\usepackage{pgfplots}
\pgfplotsset{compat=1.18}
\usepackage{fancyhdr}
\usepackage{bussproofs}
\usepackage{syntax}
\usepackage[linesnumbered,ruled,vlined]{algorithm2e}
% Definizione di uno stile personalizzato per la grammatica
\setlength{\grammarparsep}{20pt plus 1pt minus 1pt} % increase separation between rules
\setlength{\grammarindent}{12em} % increase separation between LHS/RHS 
% Rimuovi le parentesi angolari attorno al nome delle produzioni
\renewcommand{\grammarlabel}[2]{\textit{#1}\ \ #2\ \ }

\newcommand{\llbracket}{[\![}
\newcommand{\rrbracket}{]\!]}


\fancypagestyle{plain}{\fancyhf{}\renewcommand{\headrulewidth}{0pt}}
\pagestyle{fancy}
\fancyhf{}% Clear header/footer
\fancyhead[L]{\nouppercase\leftmark}
\fancyhead[R]{\thepage}
\usetikzlibrary{positioning,shapes.geometric,arrows.meta,matrix,automata,decorations.pathmorphing,patterns,decorations.pathreplacing,shapes.multipart,calc,snakes}
\usetikzlibrary{arrows.meta, backgrounds, chains, positioning, shapes.geometric, shapes.multipart}
\tcbuselibrary{skins}
\counterwithin{figure}{section}
%Nuovi comandi
\newcommand\myeq{\stackrel{\mathclap{\normalfont\mbox{def}}}{=}}
\newcommand\prodG{\stackrel{\mathclap{\normalfont\mbox{\tiny{G}}}}{\Longrightarrow}}
%asmthm
\newlength{\marginlabelsep}\setlength{\marginlabelsep}{0.5em}
\newtheoremstyle{italicstyle} %% Name
  {} %% <- Space above (empty = default = \topsep = 8.0pt plus 2.0pt minus 4.0pt)
  {} %% <- Space below (empty = default = \topsep = 8.0pt plus 2.0pt minus 4.0pt)
  {\itshape} %% <- Body font
  {} %% <- Indent amount (empty = no indent, \parindent = just that)
  {\bfseries} %% <- Thm head font
  {} %% <- Punctuation after thm head
  {1pt} %% <- Space after thm head (or " " or \newline) (default: 5pt plus 1pt minus 1pt)
  {\vtop to 0pt{\llap{\thmname{#1}\hskip\marginlabelsep}
                \llap{\thmnumber{#2}\hskip\marginlabelsep}}\thmnote{#3\\}%
  }
\newtheoremstyle{normStyle} %% Name
  {} %% <- Space above (empty = default = \topsep = 8.0pt plus 2.0pt minus 4.0pt)
  {} %% <- Space below (empty = default = \topsep = 8.0pt plus 2.0pt minus 4.0pt)
  {\normalfont} %% <- Body font
  {} %% <- Indent amount (empty = no indent, \parindent = just that)
  {\bfseries} %% <- Thm head font
  {} %% <- Punctuation after thm head
  {1pt} %% <- Space after thm head (or " " or \newline) (default: 5pt plus 1pt minus 1pt)
  {\vtop to 0pt{\llap{\thmname{#1}\hskip\marginlabelsep}
                \llap{\thmnumber{#2}\hskip\marginlabelsep}}\thmnote{#3\\}%
  }
\theoremstyle{italicstyle}
\newtheorem{corollary}{Corollario}[section]
\newtheorem{notazione}{Notazione}[section]
\newtheorem{lemma}{Lemma}[section]
\newtheorem{definizione}{Definizione}[section]
\newtheorem{nota}{Nota}[section]
\newtheorem{exercise}{Esercizio}[section]
\theoremstyle{normStyle}
\newtheorem{exmp}{Esempio}[section]
\newtheorem{theorem}{Teorema}[section]
\newtheorem{proposizione}{Proposizione}[section]
\tcbuselibrary{listings,skins}
\newtcblisting{mylisting}[2][]{
    arc=0pt, outer arc=0pt,
    listing only, 
    title=#2,
    #1,
    listing options= {escapechar=|}
}
\newcommand{\myboxedtext}[2][rectangle,draw]{%
    \tikz[baseline=-0.6ex] \node [#1]{#2};}%

\newcommand\defeq{\mathrel{\stackrel{\makebox[0pt]{\mbox{\normalfont\tiny def}}}{=}}}

%%======================================================================
\title{Fondamenti di Linguaggi di Programmazione e Specifica}
\author{\textit{Alessio Gjergji}}
\date{}
\begin{document}
\maketitle
\tableofcontents
\chapter{Introduzione}
\section{Linguaggi di programmazione}
Un linguaggio di programmazione è un linguaggio formale che specifica un
insieme di istruzioni che possono essere usate per produrre un insieme di
output.
Esso è definito da:
\begin{itemize}
    \item \textbf{Sintassi}: specifica la forma delle istruzioni. Ci permette di
    capire quali stringhe sono ammissibili e quali no mediante diversi strumenti come 
    grammatiche, analizzatori lessicali e sintattici, teoria degli automi.
    \item \textbf{Pragmatica}: specifica l'effetto delle istruzioni. Ci permette
    di capire le ragioni per introdurre un nuovo linguaggio e di programmazione 
    invece di utilizzarne uno già esistente.
    \item \textbf{Semantica}: specifica il significato dei programmi scritti nel linguaggio, ovvero il loro 
    comportamento a tempo di esecuzione. Ci permette di capire se due programmi 
    apparentemente diversi sono equivalenti.
\end{itemize}
\subsection{Benefici di una semantica formale}
I benefici dei linguaggi di programmazione diversi, tra cui:
\begin{itemize}
    \item \textbf{Implementazione}: Consente di fornire la specifica (\textit{del comportamento}) 
    dei programmi indipendentemente dalla macchina o dal compilatore utilizzato.
    \item \textbf{Verifica}: una semantica formale consente di ragionare 
    sui programmi e sulle loro proprietà di correttezza.
    \item \textbf{Progettazione di Linguaggio}: spesso una semantica formale consente di 
    scoprire ambiguità all'interno di linguaggi già esistenti. Questo aiuta a progettare 
    nuovi linguaggi in maniera più accurata.
\end{itemize}
\section{Un linguaggio per le espressioni aritmetiche: sintassi}
Definiamo il seguente linguaggio:
\[
    \mathcal{E}\quad ::= \quad n \quad | \quad \mathcal{E} + \mathcal{E} \quad | 
    \quad \mathcal{E} * \mathcal{E} \quad | \quad \dots
\]
dove:
\begin{itemize}
    \item $n$ è lo spazio del dominio dei numerali.
    \item $\mathcal{E}$ è il range del dominio delle espressioni aritmetiche.
    \item $+, x, \dots$ sono simboli del linguaggio.
\end{itemize}
I numerali sono parte della sintassi del nostro linguaggio e non vanno confusi con i numeri
che sono oggetti matematici.
Ciò potrebbe significare che nel nostro linguaggio al posto di $0, 1, \dots$ avremmo 
potuto usare $zero, uno, \dots$ e sarebbero potuti essere uguali.

Nel nostro caso assumiamo che esista una corrispondenza ovvia tra il simbolo ``numerale" (n)
e il numero naturale n. Questo è fatto solo per semplificare la spiegazione. In un
altro contesto, il simbolo ``numeral" 3 potrebbe essere associato al numero 42!

\section{Semantica Operazionale}

La semantica operazionale ha l'obiettivo di valutare un'espressione aritmetica
del linguaggio per ottenere il suo valore numerico associato. Questo può essere
fatto in due modi differenti:

\begin{itemize}
  \item \textbf{Semantica Small-Step (\textit{o strutturale})}: Fornisce un metodo per
  valutare un'espressione passo dopo passo, considerando le azioni intermedie.
  Questo approccio fornisce una valutazione dettagliata dell'espressione.

  \item \textbf{Semantica Big-Step (\textit{o naturale})}: Ignora i passaggi intermedi
  e fornisce direttamente il risultato finale della valutazione dell'espressione. Questo
  approccio semplifica la valutazione, concentrando l'attenzione sul risultato finale.

\end{itemize}
\subsection{Big-Step Semantics}

\begin{tcolorbox}[title = {Valutazione}]  
    $E \Downarrow n$
\end{tcolorbox}
\textbf{Significato}: La valutazione dell'espressione $\mathcal{E}$ produce il numerale $n$.

\begin{tcolorbox}[title = {Assiomi e regole di inferenza}]  
\begin{figure}[H]
    \begin{subfigure}{0.3\textwidth}
    \begin{prooftree}
        \AxiomC{$-$}
        \LeftLabel{(B-Num)}
        \UnaryInfC{$n \Downarrow n$}
    \end{prooftree}
    \end{subfigure}%
    \begin{subfigure}{0.7\textwidth}
    \begin{prooftree}
        \AxiomC{$\mathcal{E}_1 \Downarrow n_1$}
        \AxiomC{$\mathcal{E}_2 \Downarrow n_2$}
        \LeftLabel{(B-Add)}
        \RightLabel{$n_3 = add(n_1, n_2)$}
        \BinaryInfC{$\mathcal{E}_1 + \mathcal{E}_2 \Downarrow n_3$}
    \end{prooftree}
    \end{subfigure}
\end{figure}
\end{tcolorbox}
\textbf{Significato}: 
\begin{itemize}
\item (B-Num): Questo è un assioma che afferma che quando valutiamo un singolo
numero $n$, otteniamo lo stesso numero $n$ come risultato. Questo è il caso
base della valutazione.

\item (B-Add): Questa regola di inferenza afferma che date due espressioni
$\mathcal{E}_1$ e $\mathcal{E}_2$:
\begin{itemize}
  \item Se è il caso che $\mathcal{E}_1 \Downarrow n_1$ (cioè $\mathcal{E}_1$ si valuta a $n_1$) e
  \item È anche il caso che $\mathcal{E}_2 \Downarrow n_2$ (cioè $\mathcal{E}_2$ si valuta a $n_2$),
  allora segue che $\mathcal{E}_1 + \mathcal{E}_2 \Downarrow n_3$, dove $n_3$ è il numerale associato
  al numero $n_3$ tale che $n_3 = add(n_1, n_2)$.
  Si noti che in questa regola, $E1$, $E2$, $n1$, $n2$, $n3$ sono meta-variabili.
\end{itemize}
\end{itemize}
Questa regola (B-Add) ci dice come valutare un'addizione tra due espressioni
$\mathcal{E}_1$ e $\mathcal{E}_2$ nel contesto della semantica big-step. La
regola stabilisce che se possiamo valutare entrambe le espressioni operandi
($\mathcal{E}_1$ e $\mathcal{E}_2$) e otteniamo i numeri $n_1$ e $n_2$ rispettivamente,
allora possiamo calcolare la somma di $\mathcal{E}_1$ e $\mathcal{E}_2$ come $n_3$,
dove $n_3$ è il risultato della somma dei numeri $n_1$ e $n_2$.
Si noti che la funzione di addizione $add$ opera sui numeri, non sui numerali.
\subsection{Small-Step Semantics}

\begin{tcolorbox}[title = {Valutazione}]  
$\mathcal{E}_1 \rightarrow \mathcal{E}_2$

\end{tcolorbox}
\textbf{Significato:} 
Dopo aver eseguito un passo di valutazione su $\mathcal{E}_1$, l'espressione $\mathcal{E}_2$ rimane da valutare.
\begin{tcolorbox}[title = {Assiomi e regole di inferenza}]  
\begin{prooftree}
    \AxiomC{$\mathcal{E}_1 \rightarrow \mathcal{E}_1'$}
    \LeftLabel{(S-Left)}
    \UnaryInfC{$\mathcal{E}_1 + \mathcal{E}_2 \rightarrow \mathcal{E}_1' + \mathcal{E}_2$}
    \end{prooftree}
    
    \begin{prooftree}
    \AxiomC{$\mathcal{E}_2 \rightarrow \mathcal{E}_2'$}
    \LeftLabel{(S-N.Right)}
    \UnaryInfC{$n_1 + \mathcal{E}_2 \rightarrow n_1 + \mathcal{E}_2'$}
    \end{prooftree}
    
    \begin{prooftree}
    \AxiomC{-}
    \LeftLabel{(S-Add)}
    \RightLabel{$n_3 = add(n_1, n_2)$}
    \UnaryInfC{$n_1 + n_2 \rightarrow n_3$}
    \RightLabel{(S-Add)}
\end{prooftree}
\end{tcolorbox}
Fissiamo l'ordine di valutazione da sinistra a destra. Qualcosa di 
simile non è possibile nella big-step semantics, dove le espressioni sono 
valutate in un solo passo.
\subsubsection{La scelta dell'ordine di valutazione}
\begin{tcolorbox}[title = {Assiomi e regole di inferenza}]  
    \begin{prooftree}
        \AxiomC{$\mathcal{E}_1 \rightarrow_{ch} \mathcal{E}_1'$}
        \LeftLabel{(S-Left)}
        \UnaryInfC{$\mathcal{E}_1 + \mathcal{E}_2 \rightarrow_{ch} \mathcal{E}_1' + \mathcal{E}_2$}
        \end{prooftree}
        
        \begin{prooftree}
        \AxiomC{$\mathcal{E}_2 \rightarrow_{ch} \mathcal{E}_2'$}
        \LeftLabel{(S-Right)}
        \UnaryInfC{$\mathcal{E}_1 + \mathcal{E}_2 \rightarrow_{ch} \mathcal{E}_1 + \mathcal{E}_2'$}
        \end{prooftree}
        
        \begin{prooftree}
        \AxiomC{-}
        \LeftLabel{(S-Add)}
        \RightLabel{$n_3 = add(n_1, n_2)$}
        \UnaryInfC{$n_1 + n_2 \rightarrow_{ch} n_3$}
        \RightLabel{(S-Add)}
    \end{prooftree}
\end{tcolorbox}
In questo caso non abbiamo precedenza stabilita per la valutazione delle espressioni.
Regole simili possono essere applicate anche con gli altri operatori.
\subsubsection{Esecuzione della small-step semantics}
La relazione $\rightarrow^k$, per $k \in \mathbb{N}$ è definita per un numero di passi 
di valutazione definito da $k$.
Mentre la relazione $\rightarrow^*$ è definita per un numero non definito di passi di valutazione.
\chapter{Un semplice linguaggio imperativo}
La sintassi del nostro semplice linguaggio imperativo è definita utilizzando
la notazione \texttt{BNF} come segue:

\begin{itemize}
    \item \texttt{true} e \texttt{false} sono booleani.
    \item I numeri interi $n$ appartengono a $\mathbb{N}$.
    \item Le locazioni $l$ sono identificatori di variabili.
  \end{itemize}
  La sintassi del linguaggio può essere definita dalle seguenti produzioni grammaticali:
  
\begin{grammar}
    <Operations> ::= $+$ | $\geq$
    
    <Expressions>  ::= $n$ | $b$ | $e$ op $e$ | \texttt{if} $e$ \texttt{then} $e$ \texttt{else} $e$ \alt 
    $l$ := $e$ | !$l$ | \texttt{skip} | $e$ ; $e$ 
    \alt \texttt{while} $e$ \texttt{do} $e$
\end{grammar}
\section{Valutazione delle espressioni}
I valori delle espressioni dipendono dai valori correnti all'interno 
delle locazioni.
\[
  !l_1 + !l_2 - 1
\] 
In questo caso, il valore dell'espressione dipende dai valori correnti nelle 
locazioni $l_1$ e $l_2$.

Quindi, per valutare un'espressione, dobbiamo considerare questi cambiamenti:
\begin{itemize}
  \item Come valutiamo un'espressione $e$, in questo caso $!l_1$?
  \item Come valutiamo un'assegnamento $l := e$?
\end{itemize}
Abbiamo bisogno di più informazioni relative allo stato della memoria.
\subsection{Funzioni parziali}
Una funzione parziale $f$ è una funzione che può non essere definita per
tutti gli input. In questo caso, scriveremo $f(x) \downarrow$ se $f$ è definita
per $x$ e $f(x) \uparrow$ se $f$ non è definita per $x$.

In generale una funzione parziale può essere definita come segue:
\[
  f : A \rightharpoonup  B
\]
dove $A$ è il dominio di $f$ e $B$ è il codominio di $f$.
\paragraph{Convenzioni}
\begin{itemize}
  \item $dom(f)$ è l'insieme degli elementi nel dominio di $f$, formalmente:
  \[
    dom(f) = \{x \in A : \exists b \in B \,s.t.\, f(a) = b\}
  \]
  \item $ran(f)$ è l'insieme degli elementi nel codominio di $f$, formalmente:
  \[
    ran(f) = \{b \in B : \exists a \in A \,s.t.\, f(a) = b\}
  \]
\end{itemize}
Quindi $f$ è una funzione totale se $dom(f) = A$ e $f$ è una funzione parziale
se $dom(f) \subset A$.
\subsection{Memoria}
Nel nostro linguaggio, la memoria è una funzione parziale che mappa locazioni
in interi. 
\[
  s : \mathbb{L} \rightharpoonup \mathbb{N}
\]
Per esempio: $\{l_1 \mapsto 3, l_3 \mapsto 6, l_3 \mapsto 7 \}$.
\paragraph{Aggiornamento della memoria}
L'aggiornamento della memoria è una funzione che prende in input una memoria
$s$, una locazione $l$ e un valore $n$ e restituisce una nuova memoria $s'$.
\begin{align*}
  s' = s[l \mapsto n](l') =
  \begin{cases}
    n & \text{se } l = l' \\
    s(l') & \text{altrimenti}
    \end{cases}
\end{align*}
Il comportamento dei programmi dipende dallo stato della memoria. 
\section{Sistema di transizione}
Un sistema di transizione è composto da un insieme di
configurazioni (\textit{Config}) e una relazione binaria (\(\subseteq\)) su coppie
di configurazioni. La relazione rappresenta come una configurazione può effettuare
una transizione verso un'altra.
\[
  \textit{Relazione binaria} \rightarrow \subseteq \textit{Config} \times \textit{Config}
\]

In particolare, gli elementi di \textit{Config} sono spesso chiamati configurazioni
o stati. La relazione è chiamata relazione di transizione o di riduzione. Adottiamo
una notazione infix, quindi \(c \rightarrow c'\) dovrebbe essere letto come ``la
configurazione \(c\) può fare una transizione alla configurazione \(c'\)''.

L'esecuzione completa di un programma trasforma uno stato iniziale in uno
stato terminale. Un sistema di transizione è simile a un automa a stati
finiti non deterministico ($\textit{NFA}^\varepsilon$) con un alfabeto vuoto,
tranne che può
avere un numero infinito di stati. Non specifichiamo uno stato di partenza o stati
di accettazione.

\section{Semantica operazionale nel nostro linguaggio imperativo}
Le configurazioni sono coppie $\langle e, s\rangle$ di espressioni $e$ e memorie $s$.
Le relazioni di transizione sono definite come segue:
\[
  \langle e, s \rangle \rightarrow \langle e', s' \rangle
\]
dove $e'$ è l'espressione risultante dalla valutazione di $e$ nello stato $s$ e
$s'$ è lo stato risultante dalla valutazione di $e$ nello stato $s$.

Le transizioni rappresentano singoli passi di calcolo. Ad esempio, avremo:

\[
\begin{array}{ll}
\rightarrow & \langle l:=2+!l, \{l \mapsto 3\} \rangle \\
\rightarrow & \langle l:=2+3, \{l \mapsto 3\} \rangle \\
\rightarrow & \langle l:=5, \{l \mapsto 3\} \rangle \\
\rightarrow & \langle \text{skip}, \{l \mapsto 5\} \rangle \\
\not \rightarrow & 
\end{array}
\]

Dove \(\langle e, s \rangle\) rappresenta una configurazione, \(e\) è
un'espressione e \(s\) è uno stato. Le transizioni sono passi di calcolo
singoli che portano da una configurazione all'altra. La notazione
\(\langle e, s \rangle\) è ``bloccata" o in uno stato di ``deadlock"
se \(e\) non è un valore e \(\langle e, s \rangle\) non ha una transizione
seguente, ovvero \(\langle e, s \rangle \not\rightarrow \).

Ad esempio, \(3 + \text{false}\) è ``bloccato" o in uno stato di ``deadlock"
perché \(3 + \text{false}\) non è un valore e non può fare una transizione successiva.
\subsection{Operazioni di base}
\subsubsection{Somma}
\begin{prooftree}
  \AxiomC{$-$}
  \LeftLabel{(op $+$)}
  \UnaryInfC{$\langle n_1 + n_2, s \rangle \rightarrow \langle n_1 + n_2, s \rangle$}
\end{prooftree}
\subsubsection{Disuguaglianza}
\begin{prooftree}
  \AxiomC{$-$}
  \LeftLabel{(op $\geq$)}
  \UnaryInfC{$\langle n_1 \geq n_2, s \rangle \rightarrow \langle \text{b}, s \rangle$}
\end{prooftree}
\subsubsection{Operazione 1}
\begin{prooftree}
  \AxiomC{$\langle e_1, s \rangle \rightarrow \langle e_1', s' \rangle$}
  \LeftLabel{(op $1$)}
  \UnaryInfC{$\langle e_1 \,\texttt{op}\, e_2, s \rangle 
  \rightarrow \langle e_1' \,\texttt{op}\, e_2, s' \rangle$}
\end{prooftree}
\subsubsection{Operazione 2}
\begin{prooftree}
  \AxiomC{$\langle e_2, s \rangle \rightarrow \langle e_2', s' \rangle$}
  \LeftLabel{(op $1$)}
  \UnaryInfC{$\langle e_1 \,\texttt{op}\, e_2, s \rangle 
  \rightarrow \langle e_1 \,\texttt{op}\, e_2', s' \rangle$}
\end{prooftree}
Le regole di transizione introducono i cambiamenti nella memoria.
\subsubsection{Dereferenziazione}
\begin{prooftree}
  \AxiomC{$-$}
  \LeftLabel{(deref)}
  \RightLabel{$\quad\textit{se} \, l \in dom(s) \, \textit{e}\, s(l) = n$}
  \UnaryInfC{$\langle !l, s \rangle \rightarrow \langle s(l), s \rangle$}
\end{prooftree}
\subsubsection{Assegnamento}
\begin{prooftree}
  \AxiomC{$-$}
  \LeftLabel{(assign1)}
  \RightLabel{\, \textit{se} \, $l \in dom(s)$}
  \UnaryInfC{$\langle l:=n, s \rangle \rightarrow \langle \texttt{skip},
  s[l \mapsto n] \rangle$}
\end{prooftree}
\begin{prooftree}
  \AxiomC{$\langle e, s \rangle \rightarrow \langle e', s' \rangle$}
  \LeftLabel{(assign2)}
  \UnaryInfC{$\langle l:=e, s \rangle \, \rightarrow \langle l:=e', s' \rangle$}
\end{prooftree}
\subsubsection{Condizionale}
\begin{prooftree}
  \AxiomC{$-$}
  \LeftLabel{(if_tt)}
  \UnaryInfC{$\langle \texttt{if true then } e_1 \texttt{ else } e_2, s \rangle
  \rightarrow \langle e_1, s \rangle$}
\end{prooftree}
\begin{prooftree}
  \AxiomC{$-$}
  \LeftLabel{(if_ff)}
  \UnaryInfC{$\langle \texttt{if false then } e_1 \texttt{ else } e_2, s \rangle
  \rightarrow \langle e_2, s \rangle$}
\end{prooftree}
\begin{prooftree}
  \AxiomC{$\langle e_1, s \rangle \rightarrow \langle e_1', s' \rangle$}
  \LeftLabel{(if)}
  \UnaryInfC{$\langle \texttt{if } e_1 \texttt{ then } e_2 \texttt{ else } e_3, s \rangle
  \rightarrow \langle \texttt{if } e_1' \texttt{ then } e_2 \texttt{ else } e_3, s' \rangle$}
\end{prooftree}
\subsubsection{Sequenza}
\begin{prooftree}
  \AxiomC{$\langle e_1, s \rangle \rightarrow \langle e_1', s' \rangle$}
  \LeftLabel{(seq)}
  \UnaryInfC{$\langle e_1; e_2, s \rangle
  \rightarrow \langle e_1'; e_2, s' \rangle$}
\end{prooftree}
\begin{prooftree}
  \AxiomC{$-$}
  \LeftLabel{(seq.Skip)}
  \UnaryInfC{$\langle \texttt{skip; } e_2, s \rangle
  \rightarrow \langle e_2, s \rangle$}
\end{prooftree}
\subsubsection{While}
\begin{prooftree}
  \AxiomC{$-$}
  \LeftLabel{(while)}
  \UnaryInfC{$\langle \texttt{while } e_1 \texttt{ do } e_2, s \rangle
  \rightarrow \langle \texttt{if } e_1 \texttt{ then } e_2; \texttt{ while } e_1 \texttt{ do } e_2
  \texttt{ else skip}, s \rangle$}
\end{prooftree}
Questa è una regola di riscrittura chiamata anche \textit{unwinding}, che
consente di rivalutare la l'espressione $e_1$ ad ogni iterazione del ciclo.
\section{Esecuzione di un programma}
Per eseguire un programma $P$ a partire da uno stato $s$, è
possibile trovare uno stato $s'$ tale che 
\[\langle P, s \rangle \rightarrow_* \langle v,
s' \rangle
\]
per $v \in \mathbb{V} = \mathbb{B} \cup \mathbb{Z} \cup \{ \texttt{skip}\}$.

Le configurazioni della forma $\langle v, s \rangle$ sono considerate
terminali. Qui, $ \rightarrow_*$ denota la chiusura riflessiva e transitiva
della relazione di riduzione $\rightarrow$.
\section{Proprietà del linguaggio}
\begin{theorem}[Normalizzazione forte]
  Per ogni stato $s$ e per ogni programma $P$, esistono degli stati $s'$ tali
  che $\langle P, s \rangle \rightarrow_* \langle v, s' \rangle$, dove $\langle v,
  s \rangle$ è una configurazione terminale.
\end{theorem}
  
\begin{theorem}[Determinismo]
  Se $\langle e, s \rangle \rightarrow \langle e_1, s_1 \rangle$ e $\langle e,
  s \rangle \rightarrow \langle e_2, s_2 \rangle$, allora $\langle e_1, s_1 \rangle
  = \langle e_2, s_2 \rangle$.
\end{theorem}
\subsection{Funzione di interpretazione semantica}
Possiamo usare la semantica operazionale per fornire una semantica 
formale del seguente programma:
\begin{algorithm}
  \caption{Esempio}
  $l_1 \leftarrow 1$\;
  $l_2 \leftarrow 0$\;
  \While{$\lnot (!l_1 = !l_2)$}{
      $l_2 := !l_2 + 1$\;
      $l_3 := !l_3 + 1$\;
  }
  $l_1 := !3;$
\end{algorithm}
Quindi:
\[
  [\![ - ]\!]: Exp \rightarrow (Store \rightharpoonup Store)
\]
Dove forniamo una espressione arbitraria $e$, $[\![e]\!]$ è una 
funzione parziale che mappa uno stato $s$ in un nuovo stato $s'$.
\begin{tcolorbox}[title={Definizione}]
  \[
    [\![e]\!](s) = 
    \begin{cases}
      s' & \text{se } \langle e, s \rangle \rightarrow^* \langle v, s' \rangle \\
      \text{undefined} & \text{altrimenti}
    \end{cases}
  \]
\end{tcolorbox}
Il nostro programma d'esempio possiamo descriverlo come segue:
\[
  \llbracket P \rrbracket = 
  \begin{cases}
    s(l_1) - 1 & \text{se } l \in \{l_1, l_3\} \text{ e } s(l_1) > 0\\
    s(l_1) & \text{se } l = l_2 \text{ e } s(l_1) > 0 \\
    s(l) & \text{se } l \not \in \{l_1, l_2, l_3\} \text{ e } s(l_1) > 0\\
  \end{cases}
\]
\section{Espressività del linguaggio}
Un linguaggio si dice espressivo se è possibile esprimerci qualsiasi funzione
calcolabile. Per esempio, il linguaggio imperativo è Turing completo, quindi
esprime qualsiasi funzione calcolabile.

Il nostro linguaggio è però troppo espressivo perché è possibile esprimere
funzioni di questa tipologia $3 + true$, il modo per evitare 
questo problema è quello di introdurre il \textbf{type system}.
\section{Type system}
il \textit{type system} è un componente fondamentale nei linguaggi di programmazione, e le sue regole formali sono essenziali per garantire la correttezza e la sicurezza dei programmi. Ecco perché è importante definire queste regole in modo formale e strutturato:

\begin{itemize}
  \item Evitare errori di runtime: Il principale scopo di un "type system" è evitare errori di runtime.
  Ciò significa che il sistema è in grado di individuare potenziali errori nel tipo di dati o nell'uso
  di operatori prima che il programma venga eseguito. Ciò è particolarmente importante poiché gli errori
  di runtime possono comportare malfunzionamenti del programma, crash o risultati imprevisti.
  \item Soundness: Un "type system" è considerato "sound" quando garantisce che se un programma è
  ben tipato, allora non si verificheranno errori di tipo durante l'esecuzione. Questo è un aspetto
  fondamentale per garantire che i programmi siano corretti dal punto di vista del tipo.
  \item Incompletezza: Tuttavia, i ``type system" sono spesso incompleti, il che significa che ci
  sono programmi che sono corretti dal punto di vista del tipo ma che vengono respinti dal sistema.
  Questo può accadere quando il sistema non è in grado di dedurre in modo completo il tipo dell'espressione
  o quando le regole del ``type system" sono troppo conservative. L'incompletezza può portare alla
  rifiutazione di programmi validi, ma è un compromesso necessario per garantire la sicurezza del tipo.
  \item Proprietà di progress: L'obiettivo principale del "type system" è garantire che i
  programmi ben tipati siano in grado di fare progressi durante l'esecuzione, ovvero che non si blocchino
  o entrino in cicli infiniti. Questo aspetto è strettamente correlato all'obiettivo di non tipare programmi che
  vanno in regola, e rappresenta un'altra dimensione importante della correttezza dei programmi.
\end{itemize}
Definiremo la seguente espressione ternaria:
\[
  \Gamma \vdash e : T
\]
L'espressione si legge come: \textit{in un contesto $\Gamma$, l'espressione $e$
ha tipo $T$}. Il contesto $\Gamma$ è un insieme di assegnazioni di variabili
a tipi, per esempio:
\[
\begin{array}{lllll}
  \{\} & \vdash & \texttt{if true then 2 else 3 + 4} & : & \texttt{int} \\
  l_1:\texttt{intref} & \vdash & {\texttt{if }l_1 \geq 3\texttt{ then } l_1 \texttt{ else } 3} & : & \texttt{int} \\
  \{\} & \not \vdash & 3 + \,\texttt{false} & : & $T$ \text{ per ogni } $T$ \\
  \{\} & \not \vdash & \texttt{if true then }3 \texttt{ else false} & : & $T$ \text{ per ogni } $T$ \\
\end{array}
\]
Da notare che l'ultimo programma non è ben tipato, infatti in alcuni 
casi il type system dovrebbe assegnare un interno e in altri un booleano. Esso 
definisce un'approssimazione del comportamento del programma.
Vogliamo generalmente che fossero \textbf{decidibili}, in modo da garantire che la compilazione 
sia affidabile. 
\subsection{Tipi per il linguaggio \texttt{while}}
\begin{grammar}
  <T> ::= \texttt{int} | \texttt{bool} | \texttt{unit}
\end{grammar}
I tipi delle locazioni saranno:
\begin{grammar}
  <$T_{loc}$> ::=  \texttt{intref}
\end{grammar}
Dove \( \texttt{intref} \) rappresenta un tipo utilizzato per riferimenti a valori interi nel programma.

L'ambiente dei tipi, indicato come \( \Gamma \), è un insieme di funzioni parziali che associano le
localizzazioni ($L$) ai tipi di localizzazione ($T_{\text{loc}}$). Per una rappresentazione più chiara, possiamo esprimere $\Gamma$  nel seguente formato:
\[ \Gamma = \{l_1 : \texttt{intref}, \ldots, l_k : \texttt{intref}\} \]

Questo ambiente dei tipi associa le localizzazioni \( l_1, \ldots, l_k \) al tipo \( \texttt{intref} \).
In un contesto più avanzato, \( T_{\text{loc}} \) potrebbe contenere tipi più complessi, ma per ora,
consideriamo solo il tipo \( \texttt{intref} \).
\subsection{Definizione delle valutazioni dei tipi}
\begin{prooftree}
  \AxiomC{$-$}
  \LeftLabel{(int)}
  \RightLabel{$\texttt{for }n \in \mathbb{Z}$}
  \UnaryInfC{$\Gamma \vdash n:\texttt{int}$}
\end{prooftree}

\begin{prooftree}
  \AxiomC{$-$}
  \LeftLabel{(bool)}
  \RightLabel{$\texttt{for } b \in \{\texttt{true, false}\}$}
  \UnaryInfC{$\Gamma \vdash b:\texttt{bool}$}
\end{prooftree}

\begin{prooftree}
  \AxiomC{$\Gamma \vdash e_1 : \texttt{int}$}
  \AxiomC{$\Gamma \vdash e_2 : \texttt{int}$}
  \LeftLabel{(op $+$)}
  \BinaryInfC{$\Gamma \vdash e_1 + e_2:\texttt{int}$}
\end{prooftree}

\begin{prooftree}
  \AxiomC{$\Gamma \vdash e_1 : \texttt{int}$}
  \AxiomC{$\Gamma \vdash e_2 : \texttt{int}$}
  \LeftLabel{(op $\geq$)}
  \BinaryInfC{$\Gamma \vdash e_1 \geq e_2:\texttt{bool}$}
\end{prooftree}

\begin{prooftree}
  \AxiomC{$\Gamma \vdash e_1 : \texttt{bool}$}
  \AxiomC{$\Gamma \vdash e_2 : T$}
  \AxiomC{$\Gamma \vdash e_3 : T$}
  \LeftLabel{(if)}
  \TrinaryInfC{$\Gamma \vdash \texttt{if } e_1 \texttt{ then } e_2 \texttt{ else } e_3 : T$}
\end{prooftree}

Con le regole di tipaggio scartiamo quindi le espressioni che non hanno senso,
a tempo di compilazione.

Nel processo di tipizzazione, utilizzo $\Gamma$ per portare con me informazioni parziali sul programma. Questo è fondamentale per scoprire errori a tempo di compilazione.

Per ora, il tipo dell'assegnamento sarà semplicemente di tipo \texttt{unit}.

\begin{prooftree}
\AxiomC{$\Gamma \vdash e : \texttt{int}$}
\LeftLabel{(assign)}
\RightLabel{$\Gamma(l) = \texttt{intref}$}
\UnaryInfC{$\Gamma \vdash l := e : \texttt{unit}$}
\end{prooftree}

\begin{prooftree}
\AxiomC{$-$}
\LeftLabel{(deref)}
\RightLabel{$\Gamma(l) = \texttt{intref}$}
\UnaryInfC{$\Gamma \vdash !l : \texttt{int}$}
\end{prooftree}

Di seguito, riporto le condizioni e sopra le regole induttive, le quali dipendono dal tipaggio del sottoprogramma.

Si ricordi che il tipo delle locazioni è rappresentato da \texttt{intref}.

\begin{prooftree}
\AxiomC{$-$}
\LeftLabel{(skip)}
\UnaryInfC{$\Gamma \vdash \texttt{skip} : \texttt{unit}$}
\end{prooftree}

\begin{prooftree}
\AxiomC{$\Gamma \vdash e_1 : \texttt{unit}$}
\AxiomC{$\Gamma \vdash e_2 : T$}
\LeftLabel{(seq)}
\BinaryInfC{$\Gamma \vdash e_1;e_2 : T$}
\end{prooftree}

Nel caso della composizione sequenziale, il tipo del secondo termine sarà uguale al tipo del primo, che in questo caso è \texttt{unit}.

Nel caso del ciclo ``while":

\begin{prooftree}
\AxiomC{$\Gamma \vdash e_1 : \texttt{bool}$}
\AxiomC{$\Gamma \vdash e_2 : \texttt{unit}$}
\LeftLabel{(while)}
\BinaryInfC{$\Gamma \vdash \texttt{while } e_1 \texttt{ do } e_2 : \texttt{unit}$}
\end{prooftree}

Considerando la regola di tipizzazione, il tipo del corpo del ciclo "while" sarà anch'esso di tipo \texttt{unit}.
\section{Proprietà}

\subsection{Teorema della Progressione}

Il Teorema della Progressione afferma che: 
\begin{tcolorbox}
Se $\Gamma \vdash e : T$ e $dom(\Gamma) \subseteq dom(s)$, allora $e$ è un
valore o esiste $e', s'$ tali che $\langle e, s \rangle \rightarrow \langle e', s' \rangle$.
\end{tcolorbox}
Quindi, durante l'esecuzione di
un programma, siamo in grado di fare progressi. Se un'espressione ha un tipo
valido e il contesto è adeguato, o l'espressione è già un valore (\textit{cioè non può essere valutata ulteriormente}),
oppure possiamo effettuare una transizione a uno stato successivo.

\subsection{Teorema della Preservazione del Tipo}
\begin{tcolorbox}
Il Teorema della Preservazione del Tipo stabilisce che:
Se $\Gamma \vdash e : T$ e $dom(S) \subseteq dom(s)$,
e se $\langle e, s \rangle \rightarrow \langle e', s' \rangle$, allora $\Gamma \vdash e' : T$ e $dom( \Gamma)
\subseteq dom(s')$.
\end{tcolorbox}
Questo teorema ci assicura che, durante l'esecuzione di un programma, se un'espressione ha un tipo
valido e il contesto è adeguato, allora qualsiasi transizione non romperà la coerenza dei tipi. Il tipo
delle espressioni sarà preservato durante l'esecuzione.

Mettendo insieme i due teoremi, otteniamo un nuovo teorema, il \textbf{Teorema della Safety}.

\subsection{Teorema della Safety}
\begin{tcolorbox}
  Il Teorema della Safety ci dice che: 
Se $\Gamma \vdash e : T$ e $dom(\Gamma) \subseteq dom(s)$, e se $\langle e,
s \rangle \rightarrow^* \langle v', s' \rangle$, allora $e'$ sarà un valore o esisterà $e'', s''$ tali che
$\langle e', s' \rangle \rightarrow \langle e'', s'' \rangle$. 
\end{tcolorbox}


In parole povere, se prendiamo una configurazione
iniziale e la facciamo avanzare attraverso un numero qualsiasi di passi, avremo due possibilità: o raggiungeremo una
configurazione finale, oppure saremo in grado di effettuare un ulteriore passo.

Quando diciamo che $dom(\Gamma) \subseteq dom(s)$, intendiamo che il dominio dello store (\textit{lo spazio
in cui sono memorizzati i valori delle variabili}) deve essere contenuto almeno nel dominio del contesto
(\textit{l'insieme delle variabili dichiarate nel programma}).

Ecco un esempio che illustra questi concetti:

\begin{algorithm}[H]
  \While{true}
  {skip}
\end{algorithm}

In base alle regole di tipizzazione, questo programma avrà il tipo \texttt{unit},
ma non impedisce al programma di essere in uno stato di blocco infinito,
in cui continua a eseguire l'istruzione ``skip" all'infinito.


\subsection{Type Checking, Typeability e Type Inference}

\begin{tcolorbox}[title = {Problema del Controllo di Tipo}]
  Dato un type system, un tipo ambiente $\Gamma$, un'espressione $e$, e un tipo $T$, stabilire se $\Gamma \vdash e : T$ è derivabile?
\end{tcolorbox}

\begin{tcolorbox}[title = {Problema di Type Inference}]
  Dato un type system, un tipo ambiente $\Gamma$, e un'espressione $e$, trovare un tipo $T$ tale che $\Gamma \vdash e : T$ è
  derivabile, oppure mostrare che non esiste.
\end{tcolorbox}

Il secondo problema è più difficile del primo, in quanto devo trovare un tipo $T$ che soddisfi la proprietà, mentre nel primo
caso devo solo stabilire se esiste un tipo $T$ che soddisfi la proprietà.

\subsection{Type Checking}
\begin{tcolorbox}
  Dato $\Gamma$, $e$ e $T$, possiamo trovare un $T$ tale che $\Gamma \vdash e : T$ o mostrare che non esiste.
\end{tcolorbox}
Il Teorema del Type Checking afferma che, dati un ambiente di tipi $\Gamma$, un'espressione $e$ e un tipo $T$, siamo in
grado di determinare se $\Gamma \vdash e : T$ è derivabile. In altre parole, possiamo verificare se un'espressione $e$ è ben
tipata rispetto a un ambiente di tipi $\Gamma$, e se è così, possiamo anche calcolare il tipo $T$ che la verifica.

\subsection{Preservazione del Tipo}
\begin{tcolorbox}
  Dato $\Gamma$, $e$ e $T$, si può decidere se $\Gamma \vdash e : T$.
\end{tcolorbox}
Il Teorema della Preservazione del Tipo afferma che, dati un ambiente di tipi $\Gamma$, un'espressione $e$ e un tipo $T$,
possiamo determinare se $\Gamma \vdash e : T$ è derivabile. In altre parole, questo teorema ci assicura che possiamo verificare
se il tipo di un'espressione $e$ rimane coerente durante l'esecuzione del programma.

\subsection{Unicità del Tipo}
\begin{tcolorbox}
  Se $\Gamma \vdash e: T $ e $\Gamma \vdash e : T'$ allora $T = T'$.
\end{tcolorbox}
Il Teorema dell'Unicità del Tipo afferma che se un'espressione $e$ ha due tipi $T$ e $T'$ rispetto a un ambiente di tipi $\Gamma$,
allora $T$ e $T'$ devono essere identici. In altre parole, non può esistere una situazione in cui un'espressione abbia più di un
tipo ben formato.
\chapter{Induzione}
\section{Induzione come principio di dimostrazione}
L'induzione è un principio di dimostrazione che consente di dimostrare proprietà su insiemi infiniti di elementi.
Questo principio si basa su passi finiti di calcolo, che su basano su insiemi che hanno una struttura ben definita.
Questa struttura ci aiuta a ridurre il problema complesso in una serie di passaggi più semplici.

Esistono diversi tipi di induzione tra cui:

\begin{itemize}
  \item Induzione matematica: Utilizzata per dimostrare affermazioni sui numeri naturali.
  Si basa sulla dimostrazione di una proprietà per un valore iniziale e sulla dimostrazione che,
  se la proprietà è vera per un certo numero, lo è anche per il numero successivo.

  \item Induzione strutturale: Utilizzata per dimostrare affermazioni su strutture ricorsive come
  alberi. La dimostrazione inizia dimostrando la proprietà per il caso base (\textit{ad esempio, un albero vuoto})
  e successivamente dimostra che se la proprietà è vera per le componenti strutturali, lo è anche per la struttura complessiva.

  \item Induzione delle regole: Spesso utilizzata per dimostrare proprietà delle regole in un sistema formale.
  La dimostrazione inizia dimostrando la proprietà per ciascuna regola e successivamente dimostra che la proprietà
  si conserva quando si applicano le regole in sequenza.
\end{itemize}

\subsection{I numeri naturali $\mathbb{N}$}
I numeri naturali sono costruiti seguendo due regole fondamentali:

\begin{itemize}
  \item \textbf{Regola di base}: Il numero $0$ appartiene all'insieme dei
  numeri naturali, indicato come $0 \in \mathbb{N}$.

  \item \textbf{Passo induttivo}: Se un numero $k$ è un
  numero naturale, allora il successore di $k$, ovvero $k+1$,
  è anch'esso un numero naturale.
\end{itemize}

\subsubsection{Esempio}
Per definire una funzione $f: \mathbb{N} \rightarrow X$, è necessario seguire due passi:

\begin{itemize}
  \item \textbf{Regola di base}: Si descrive il risultato di $f$ per il valore iniziale, ovvero $0$.

  \item \textbf{Passo induttivo}: Si assume che $f$ sia definita per un valore $k$, e si
  descrive il risultato di $f$ per $k+1$ in termini di $f(k)$. Questo approccio di definizione
  ricorsiva è spesso utilizzato nella programmazione, in particolare nell'ambito del ``pattern matching".
\end{itemize}

\subsubsection{Esempio}
Nel contesto della semantica ``small step" nel caso deterministico,
possiamo definire una funzione $\texttt{red}$ come segue:

\[
\texttt{red}: \texttt{Exp} \times \mathbb{N} \rightarrow \text{Exp}
\]

\begin{itemize}
  \item \textbf{Regola di base}: $\texttt{red}(E,0) = E$ per ogni espressione $E$.

  \item \textbf{Passo induttivo}: $\texttt{red}(E,k+1) = E''$ se esiste un'espressione
  $E'$ tale che $\text{red}(E,k) = E'$ e $E' \rightarrow E''$. Questa definizione permette
  di rappresentare l'espressione ottenuta riducendo $E$ per $k$ passi.

\end{itemize}

La dimostrazione per induzione è un metodo formale per dimostrare affermazioni matematiche
o logicamente corrette su insiemi infiniti, come i numeri naturali. L'obiettivo è dimostrare
una proprietà $P$ per un numero arbitrario $k$ e dimostrare che la proprietà è vera anche per
il successivo $k+1$:

\[
\forall k \in \mathbb{N}. P(k) \Rightarrow P(k+1)
\]

Nel processo di dimostrazione per induzione, è essenziale dimostrare che la proprietà è vera per il caso base (spesso $k = 0$) e che il passaggio all'elemento successivo è valido. Questo garantisce che la proprietà sia valida per tutti i numeri naturali.
\section{Induzione matematica}
Il metodo più semplice di induzione è l'induzione matematica, che è un tipo di induzione basato sui
numeri naturali. Il principio può essere descritto come segue. Dato un'affermazione $P(-)$ sui numeri naturali,
vogliamo dimostrare che $P(n)$ è vera per ogni numero naturale $n$:

\begin{itemize}
\item \textbf{Caso base}: dimostrare che $P(0)$ è vera (\textit{utilizzando alcuni
fatti matematici noti}).

\item \textbf{Caso induttivo}: assumere l'ipotesi induttiva, cioè che $P(k)$ è vera.
A partire dall'ipotesi induttiva, dimostrare che segue $P(k + 1)$ (\textit{utilizzando
alcuni fatti matematici noti}).
\end{itemize}
Se $(a)$ e $(b)$ vengono stabiliti, allora $P(n)$ è vera per ogni numero naturale $n$.

L'induzione matematica è un principio valido perché ogni numero naturale può essere ``costruito"
a partire da $0$ come punto di partenza e usando l'operazione di aggiunta di uno per creare nuovi numeri.
\section{Induzione strutturale}
L'induzione strutturale è un principio di dimostrazione che consente di dimostrare proprietà su elementi
di un insieme costruito induttivamente. Questo tipo di induzione è particolarmente utile quando si tratta
di oggetti con una struttura ricorsiva.

Un concetto importante nell'ambito dell'induzione strutturale è l'isomorfismo. Due oggetti si dicono
isomorfi se esiste una funzione biunivoca tra di essi, cioè una funzione che stabilisce una corrispondenza
uno a uno tra gli elementi degli oggetti.

Un esempio comune di passaggio dall'induzione matematica a quella strutturale coinvolge la seguente grammatica:

\begin{grammar}
  <N> ::= \texttt{zero} | \texttt{SUCC}(N)
\end{grammar}

Nell'induzione strutturale, possiamo dimostrare proprietà sugli elementi di questa grammatica in questo modo:

\begin{itemize}
  \item Caso base: Dimostrare che la proprietà è vera per \texttt{zero}.
  \item Passo induttivo: Assumere che la proprietà sia vera per un generico elemento \texttt{N} e dimostrare che è
  vera anche per \texttt{SUCC}(N).
\end{itemize}

Questo principio ci consente di affrontare dimostrazioni relative a strutture ricorsive in modo sistematico e rigoroso.

\subsubsection{Somma (\textit{sum})}
Il principio di definizione delle funzioni per induzione può essere applicato a questa rappresentazione dei numeri naturali allo stesso modo di prima. Vediamo un esempi:

\begin{itemize}
   \item Regola di base: $sum(zero) = zero$
   \item Regola induttiva: $sum(succ(N)) = succ^{(n+1)}(sum(N))$, dove $N = succ(...succ(zero))$ per un certo numero naturale $n$.
\end{itemize}

Ciò significa che il caso base è definito per $zero$, e nel caso induttivo, applichiamo la funzione $sum$ in modo ricorsivo a $succ(N)$ aggiungendo $succ$ ripetutamente $n+1$ volte.
\subsection{Induzione strutturale per i numeri naturali}
Il principio di induzione afferma che per dimostrare $P(N)$ per tutti i numeri N, è sufficiente fare due cose:
\begin{itemize}
  \item \textbf{Caso base}: Dimostrare che $P(zero)$ è vero.
  \item \textbf{Caso induttivo}: Dimostrare che $P(succ(K)$) segue dall'assunzione che $P(K)$ sia vero per un certo numero $K$.
\end{itemize}
È importante notare che nel tentativo di dimostrare $P(succ(K))$, l'ipotesi induttiva ci dice che possiamo assumere che
la proprietà sia valida per la \textbf{sottostruttura} di $succ(K)$, ovvero possiamo supporre che $P(K)$ sia vero.

Questa prospettiva strutturale, e la forma associata di induzione, chiamata induzione strutturale, è ampiamente applicabile.

\subsection{Albero binario attraverso l'induzione strutturale}
\begin{figure}[H]
  \centering
  \begin{forest}
    for tree={circle, draw, minimum size=1.5em, inner sep=0, s sep=20pt}
    [
      [
        [, fill=orange]
        [, fill=orange]
      ]
      [
        [, fill=orange]
        [, fill=orange]
      ]
    ]
  \end{forest}
\end{figure}
\subsubsection{Definizione induttiva}
\begin{itemize}
  \item \textbf{Caso base}: \texttt{leaf} è un albero binario.
  
  \begin{figure}[H]
    \centering
    \begin{forest}
      for tree={circle, draw, minimum size=1.5em, inner sep=0, s sep=20pt}
        [, fill=orange]
    \end{forest}
  \end{figure}
  \item \textbf{Passo induttivo}: se $L$ e $R$ sono alberi binari allora \texttt{node}($L$, $R$) è un albero binario.
  
  \begin{figure}[H]
    \centering
    \begin{forest}
      for tree={circle, draw, minimum size=1.5em, inner sep=0, s sep=20pt}
      [
        [L]
        [R]
      ]
      \end{forest}
  \end{figure}
\end{itemize}
\subsection{Regole di costruzione}
\begin{grammar}
  <$T \in \texttt{bTree}$> ::= \texttt{leaf} | \texttt{node}(T, T)
\end{grammar}
\begin{itemize}
  \item \textbf{Caso base}: \texttt{leaf} è un albero binario.
  \item \textbf{Passo induttivo}: se $T_1$ e $T_2$ sono alberi binari, allora \texttt{node}($T_1$, $T_2$) è un albero binario.
\end{itemize}

\subsubsection{Definizione induttiva}
\[
f: \texttt{bTree} \rightarrow X  
\]

Per definire una funzione $f$ che mappa alberi binari a elementi di un insieme $X$, possiamo seguire il principio della definizione induttiva:

\begin{itemize}
  \item \textbf{Regola di base:} Descriviamo il risultato dell'applicazione di $f$ a una foglia terminale, ad esempio $f(\texttt{leaf})$.

  \item \textbf{Regola induttiva:} Supponiamo che $f(T1)$ e $f(T2)$ siano già definiti. Ora, descriviamo il risultato dell'applicazione di
  $f$ all'albero binario $\texttt{node}(T1, T2)$.
\end{itemize}

Con queste regole, possiamo definire la funzione $f$ per ogni possibile albero binario, indipendentemente dalla sua complessità. Ogni passo nella definizione
si basa sui passi precedenti, garantendo che la funzione sia ben definita per tutti gli alberi binari.

\subsection{Dimostrazioni strutturali}
\subsubsection{Normalizzazione della big step semantics}
\begin{tcolorbox}[title = {Normalizzazione della big step semantics}]
Per ogni espressione aritmetica $E$ esiste un numero naturale $k$
t.c. $E \downarrow k$.

\begin{grammar}
  <$E \in \texttt{expr}$> ::= $n$ | $E_1 + E_2$
\end{grammar}
Abbiamo a disposizione due regole di costruzione:

\begin{minipage}{0.40\textwidth}
  \centering
  \begin{prooftree}
    \AxiomC{$-$}
    \RightLabel{(B-num)}
    \UnaryInfC{$n \downarrow n$}
  \end{prooftree}
\end{minipage}
\hfill
\begin{minipage}{0.60\textwidth}
  \centering
  \begin{prooftree}
    \AxiomC{$E_1 \downarrow n_1$}
    \AxiomC{$E_2 \downarrow n_2$}
    \RightLabel{$n_3 = \texttt{add}(n_1,n_2)$}
    \BinaryInfC{$E_1 + E_2 \downarrow n_3$}
  \end{prooftree}
\end{minipage}
\[
\texttt{Propr}: \forall E \in \texttt{expr} \exists k \quad numerale \quad \text{ t.c. } E \downarrow k  
\]
\end{tcolorbox}
\begin{proof}
  Per induzione sulla struttura di $E$.
  \begin{itemize}
    \item \textbf{Caso base}: Sia $E$ un non terminale $n$, $E\equiv n$, allora il
    $k$ in questione è proprio $n$. Per la regola (\texttt{B-num}), $n \downarrow n$.
    \begin{prooftree}
      \AxiomC{$-$}
      \RightLabel{(B-num)}
      \UnaryInfC{$n \downarrow n=k$}
    \end{prooftree}
    \item \textbf{Passo induttivo}: $E$ è nella forma $E_1 + E_2$ per qualche $E_1$ e $E_2$ (\textit{sottostrutture di $E$}).
    Per ipotesi induttiva,
    essendo $E_1$ e $E_2$ sottostrutture di $E$, vale che:
    \begin{itemize}
      \item $\exists k_1 \quad numerale \quad \text{ t.c. } E_1 \downarrow k_1$
      \item $\exists k_2 \quad numerale \quad \text{ t.c. } E_2 \downarrow k_2$
    \end{itemize}
    Usando la regola (\texttt{B-add}) e scegliendo $k_3 = \texttt{add}(k_1, k_2)$, otteniamo che:
      \begin{prooftree}
        \AxiomC{$E_1 \downarrow k_1$}
        \AxiomC{$E_2 \downarrow k_2$}
        \RightLabel{$k_3 = \texttt{add}(k_1,k_2)$}
        \BinaryInfC{$E \downarrow k_3$}
      \end{prooftree}
      Ho quindi trovato il $k$ t.c. $E \downarrow k$ che cercavo.
  \end{itemize}
\end{proof}
\subsubsection{Small step semantics}
\begin{tcolorbox}[title = {$E \rightarrow F$ implica $E \rightarrow_{CH} F$ per ogni espressione aritmetica}]
  Se $E \downarrow n$ e $E \downarrow m$, allora $n \equiv m$.
\end{tcolorbox}
\begin{proof}
  Per induzione sulla struttura di $E$.
  \begin{itemize}
    \item \textbf{Caso base}: Sia $E \equiv k$ per qualche $k \in \mathbb{N}$, allora sia 
    $E \downarrow n$ che $E \downarrow m$ sono derivabili solo per la regola (\texttt{B-num}), dove $m=k=n$.

      \begin{prooftree}
        \AxiomC{$-$}
        \RightLabel{(B-num)}
        \UnaryInfC{$k \downarrow k$}
      \end{prooftree}

    
    Ma allora $n \equiv m$.

    \begin{minipage}{0.45\textwidth}
      \centering
      \begin{prooftree}
        \AxiomC{$-$}
        \RightLabel{(B-num)}
        \UnaryInfC{$m \downarrow k$}
      \end{prooftree}
    \end{minipage}
    \hfill
    \begin{minipage}{0.45\textwidth}
      \centering
      \begin{prooftree}
        \AxiomC{$-$}
        \RightLabel{(B-num)}
        \UnaryInfC{$n \downarrow k$}
      \end{prooftree}
    \end{minipage}
    \item \textbf{Passo induttivo}: $E$ è nella forma $E_1 + E_2$.
    \begin{itemize}
      \item Ne consegue che $E \downarrow m$ è stato derivato usando la regola (\texttt{B-add}).
      Esistendo $m_1 + m_2 = m$, per ipotesi induttiva, $E_1 \downarrow m_1$ e $E_2 \downarrow m_2$.

      \begin{prooftree}
        \AxiomC{$E_1 \downarrow m_1$}
        \AxiomC{$E_2 \downarrow m_2$}
        \RightLabel{$m = \texttt{add}(m_1,m_2)$}
        \BinaryInfC{$E \downarrow m$}
      \end{prooftree}
      \item Ne consegue che $E \downarrow n$ è stato derivato usando la regola (\texttt{B-add}).
      Esistendo $n_1 + n_2 = n$, per ipotesi induttiva, $E_1 \downarrow n_1$ e $E_2 \downarrow n_2$.

      \begin{prooftree}
        \AxiomC{$E_1 \downarrow n_1$}
        \AxiomC{$E_2 \downarrow n_2$}
        \RightLabel{$n = \texttt{add}(n_1,n_2)$}
        \BinaryInfC{$E \downarrow n$}
      \end{prooftree}
    \end{itemize}
  \end{itemize}
\end{proof}
\subsubsection{Determinismo forte per la small-step semantics}
\begin{tcolorbox}[title = {Determinismo forte per la small-step semantics}]
  Se $E \rightarrow F$ e $E \rightarrow G$, allora $F \equiv G$.
\end{tcolorbox}
\begin{proof}
  Per induzione sulla struttura di $E$.
  \begin{itemize}
    \item \textbf{Caso base}: $E \equiv n$ per qualche $n \in \mathbb{N}$. Non 
    esiste alcuna $E$ e $F$ t.c. $E \rightarrow F$ e $E \rightarrow G$. Quindi 
    la conclusione  è falsa. 
    \item \textbf{Passo induttivo}: $E$ è nella forma $E_1 + E_2$.
    Sia $E \rightarrow F$ per qualche $F$.
    \begin{itemize}
      \item Ho usato la regola (\texttt{S-left}), allora $\exists E_1'$ t.c. 
      $E_1 \rightarrow E_1'$ e $E = E_1 + E_2 \rightarrow E_1' + E_2$.
      Avendo considerato $E \rightarrow G$ e avendo utilizzato la regola 
      (\texttt{S-left}), dove $E_1 \rightarrow \text{\^E}_1$ e 
      $E = E_1 + E_2 \rightarrow \text{\^E}_1 + E_2$, perché $E_1 \rightarrow
      \text{\^E}_1$ per ipotesi induttiva su $E_1$, $\text{\^E}_1 \equiv E_1'$, 
      quindi $F=G$.
      \item Ho usato la regola (\texttt{S-right}), allora $E = m + E_2$ e
      $E_2'$ t.c $E_2 \rightarrow E_2'$ e $E = m + E_2 \rightarrow m + E_2'$.
      Avendo considerato $E \rightarrow G$ e avendo utilizzato la regola 
      (\texttt{S-right}), dove $E_2 \rightarrow \text{\^E}_2$ e 
      $E = m + E_2 \rightarrow m + \text{\^E}_2$, perché $E_2 \rightarrow
      \text{\^E}_2$ per ipotesi induttiva su $E_2$, $\text{\^E}_2 \equiv E_2'$,
      quindi $F=G$.
      \item Ho usato la regola (\texttt{S-add}), allora $E = E_1 + E_2 = m_1 + m_2$ 
      per qualche $m_1, m_2$ con $F=m_3 = \texttt{add}(m_1, m_2)$, 
      inoltre $E \rightarrow G$, ma avendo utilizzato la regola (\texttt{S-add}),
      ne consegue che $F=G=m_3$.
      
    \end{itemize}
  \end{itemize}
\end{proof}
\subsubsection{Determinismo debole per la small-step semantics}
\begin{tcolorbox}[title = {Determinismo forte per la small-step semantics}]
  Se $E \rightarrow^* m$ e $E \rightarrow^* n$, allora $F \equiv G$.
\end{tcolorbox}
Per dimostrare il determinismo debole, si utilizza la dimostrazione per induzione
strutturale basata sulla dimostrazione del determinismo forte.
\section{Induzione delle regole}
Il comportamento delle espressioni aritmetiche $E$ è completamente definito 
dalle regole dei suoi componenti. Per questa ragione l'induzione strutturale è sufficientemente 
potente per dimostrare proprietà per differenti semantiche di \texttt{Exp}.
Però, in linguaggi più complessi, con operatori di controllo ricorsivi o induttivi, 
abbiamo bisogno di strumenti più sofisticati. L'idea di base della \textbf{induzione delle regole}
è di ignorare la struttura degli oggetti e concentrarsi nella \textbf{dimensione delle derivazioni} 
\subsection{Dimensione delle derivazioni}
Per esempio, consideriamo le seguenti regole, che definiscono una relazione binaria sui 
numeri naturali:

\begin{minipage}{0.5\textwidth}
  \begin{prooftree}
    \AxiomC{$-$}
    \LeftLabel{(Ax)}
    \UnaryInfC{$n \texttt{ Div }0$}
  \end{prooftree}
\end{minipage}
\begin{minipage}{0.5\textwidth}
  \begin{prooftree}
    \AxiomC{$n \texttt{ Div } m$}
    \LeftLabel{(Plus)}
    \UnaryInfC{$n \texttt{ Div }(m + n)$}
  \end{prooftree}
\end{minipage}

Derivazioni:

\begin{minipage}{0.5\textwidth}
  \begin{prooftree}
    \AxiomC{$-$}
    \LeftLabel{(Ax)}
    \UnaryInfC{$7 \text{ Div } 0$}
    \LeftLabel{(Plus)}
    \UnaryInfC{$7 \text{ Div } 7$}
    \LeftLabel{(Plus)}
    \UnaryInfC{$7 \text{ Div } 14$}
    \LeftLabel{(Plus)}
    \UnaryInfC{$7 \text{ Div } 21$}
\end{prooftree}
\end{minipage}
\begin{minipage}{0.5\textwidth}
  \begin{prooftree}
    \AxiomC{$-$}
    \LeftLabel{(Ax)}
    \UnaryInfC{$2 \text{ Div } 0$}
    \LeftLabel{(Plus)}
    \UnaryInfC{$2 \text{ Div } 2$}
    \LeftLabel{(Plus)}
    \UnaryInfC{$2 \text{ Div } 4$}
\end{prooftree}
\end{minipage}

Notimo che la derivazione a sinistra è più lunga di quella a destra, infatti \texttt{2 Div 4} è più
piccolo di \texttt{7 Div 21}.

Supponiamo di voler dimostrare una dichiarazione della forma:
\[
  n \texttt{ Div } m \rightarrow P(n,m)
\]
Dove possiamo utilizzare l'induzione sulla dimensione della derivazione di $n \texttt{ Div } m$.
Supponiamo quindi che $n \texttt{ Div } m$, dove $m = n \cdot k$ per qualche $k \in \mathbb{N}$.
Ciò significa effettivamente che le regole \texttt{(Ax)} e \texttt{(Plus)} catturano correttamente
il concetto di divisione. 
Quindi, dimostriamo che $n \, \text{Div} \, m \rightarrow P(n, m)$ per

induzione matematica sulla dimensione della derivazione della valutazione $n \, \text{Div} \, m$
dalle regole \texttt{(Ax)} e \texttt{(Plus)}.

Ciò significa effettivamente che le regole  \texttt{(Ax)} e \texttt{(Plus)} catturano correttamente
il concetto di divisione. Quindi, supponiamo di avere una derivazione di $n \, \text{Div} \, m$.
Utilizzare l'induzione matematica significa avere un'ipotesi induttiva che afferma che $P(k_1, k_2)$
è vera per qualsiasi $k_1$, $k_2$ per i quali esiste una derivazione $k_1 \, \text{Div} \, k_2$ la cui
dimensione è inferiore alla dimensione della derivazione di $n \, \text{Div} \, m$.

Quindi $n \texttt{ Div } m$ può essere derivato solo da \texttt{(Ax)} o \texttt{(Plus)}.

Abbiamo due possibilità:
\begin{itemize}
  \item Abbiamo l'applicazione di un assioma \texttt{(Ax)}: 
  \begin{prooftree}
    \AxiomC{$-$}
    \LeftLabel{(Ax)}
    \UnaryInfC{$n \texttt{ Div }0$}
  \end{prooftree}
  Solo se $m = 0$, quindi $P(n, 0)$ è vera, ma $k$ deve essere $0$.
  \item Abbiamo l'applicazione di una regola \texttt{(Plus)}:
    \begin{prooftree}
      \AxiomC{...}
      \LeftLabel{(...)}
      \UnaryInfC{$n \texttt{ Div } m_1$}
      \LeftLabel{(Plus)}
      \UnaryInfC{$n \texttt{ Div }(m_1 + n)$}
    \end{prooftree}
    Dove $m = m_1 + n$.
\end{itemize}
Però questo significa che la valutazione di $n \texttt{ Div } m$ ha anche una derivazione mediante regole.
La dimensione della derivazione è minore della dimensione della derivazione di $n \texttt{ Div } m$.
Quindi per ipotesi induttiva, sappiamo che esiste un $k_1$ tale che $m_1 = n \cdot k_1$.

Ora, $P(n, m)$ è una immediata conseguenza di $m = n \cdot (K_1 + 1)$.
\section{Definizione formale di induzione delle regole}
Per dimostrare una proprietà $P(D)$ per ogni derivazione $D$, è sufficiente fare quanto segue:

\begin{enumerate}
    \item[(a)] \textbf{Caso base:} dimostra che $P(A)$ è vera per ogni assioma
    $A$ (utilizzando fatti matematici noti).
    \item[(b)] \textbf{Caso induttivo:} per ogni regola della forma 
    
    \begin{prooftree}
      \AxiomC{$h_1, \dots, h_n$}
      \LeftLabel{(Rule)}
      \UnaryInfC{$c$}
    \end{prooftree}

    dimostra che ogni derivazione che termina con
    l'uso di questa regola soddisfa la proprietà. Tale derivazione ha sottoderivazioni
    $D_1, \dots, D_n$ con conclusioni $h_1, \dots, h_n$. Per ipotesi induttiva,
    assumiamo che $P(D_i)$ valga per ogni sotto-derivazione $D_i$, $1 \leq i \leq n$.
\end{enumerate}
\begin{theorem}[Progress]
  Se $\Gamma \vdash e : T$ e $dom(\Gamma) \subseteq dom(s)$ allora $e$ è
  un valore o esistono $e'$, $s'$ tali che $\langle e, s \rangle \rightarrow \langle e', s' \rangle$.
\end{theorem}
%\begin{proof}
%  Supponiamo che $\phi$ sia una relazione ternaria definita come segue:
%  \[
%    \phi (\Gamma, e, T) \stackrel{\text{def}}{=} \forall s. dom(\Gamma) \subseteq dom(s) \implies 
%    valore(e) \lor (\exists e', s'. \langle e, s \rangle \rightarrow \langle e', s' \rangle)
%  \]
%  Lo dimostriamo per ogni:
%  \[
%    \Gamma, e, T, \texttt{ if } \Gamma \vdash e : T \texttt{ then } \phi(\Gamma, e, T)
%  \]
%  Per dimostrare per induzione sulle regole il perché $\Gamma \vdash e : T$,
%  ciò significa che facciamo un'analisi dei casi sulla ``ultima" regola di
%  tipizzazione applicata per derivare $\Gamma \vdash e : T$. Ci sono
%  \begin{itemize}
%    \item[(a)] 4 casi base: gli assiomi (int), (bool), (deref) e (skip);
%    \item[(b)] 6 casi induttivi: le regole $(\texttt{op}\, +)$ e $(\texttt{op}\, \geq)$
%    , (\texttt{if}), (\texttt{assign}), (\texttt{seq}) e (\texttt{while}).
%  \end{itemize}
%\end{proof}
\subsubsection{Example}
Come esempio, dimostriamo che se $\Gamma \supseteq \{I, \texttt{intref}\}$, allora 
$\Gamma \vdash (!l + 2) + 3 : \texttt{int} \implies \phi (\Gamma, (!l + 2) + 3, \texttt{int})$.
\begin{prooftree}
  \AxiomC{$-$}
  \LeftLabel{(\texttt{deref})}
  \UnaryInfC{$\Gamma \vdash !l : \texttt{int}$}
  \AxiomC{$-$}
  \LeftLabel{(\texttt{int})}
  \UnaryInfC{$\Gamma \vdash 2 : \texttt{int}$}
  \LeftLabel{(\texttt{op} +)}
  \BinaryInfC{$\Gamma \vdash (!l + 2) : \texttt{int}$}
  \AxiomC{$-$}
  \LeftLabel{(\texttt{int})}
  \UnaryInfC{$\Gamma \vdash 3 : \texttt{int}$}
  \LeftLabel{(\texttt{op} +)}
  \BinaryInfC{$\Gamma \vdash (!l + 2) + 3 : \texttt{int}$}
\end{prooftree}

\subsection{Preservazione del tipo}
\begin{lemma}
  Se $\langle e, s \rangle \rightarrow \langle e', s' \rangle$ allora $\text{dom}(s) = \text{dom}(s')$.
\end{lemma}
\begin{proof}
  Per induzione sulle regole su perché $\langle e, s \rangle \rightarrow \langle e', s' \rangle$. Sia $\Phi(e, s, e', s') = (\text{dom}(s) = \text{dom}(s'))$. Tutte le regole sono utilizzi immediati dell'ipotesi induttiva, tranne la regola (assign1), per la quale notiamo che se $l \in \text{dom}(s)$ allora $\text{dom}(s[l \mapsto n]) = \text{dom}(s)$. $\Box$
\end{proof}

\begin{theorem}
  [Conservazione del tipo] Se $\Gamma \vdash e : T$ e $\text{dom}(\Gamma) \subseteq \text{dom}(s)$ e
  $\langle e, s \rangle \rightarrow \langle e', s' \rangle$ allora $\Gamma \vdash e' : T$ e
  $\text{dom}(\Gamma) \subseteq \text{dom}(s')$.
\end{theorem}

\chapter{Funzioni}
Molti dei linguaggi di programmazione hanno delle 
notazione per indicare le funzioni, metodi o procedure.

Gli esempi possono essere di vari natura, vediamone alcuni:
\begin{itemize}
\item \texttt{fn x : int} $\implies$ \texttt{x + 1}
\item $(\texttt{fn x : int} \implies \texttt{x + 1})8$
\item $\texttt{fn y} \implies (\texttt{fn x : int} \implies x + y)$
\item \dots
\end{itemize}
Per semplicità le nostre funzioni saranno \textbf{anonime}, prenderanno 
un solo argomento e restituiranno un solo valore e saranno sempre tipate.
\section{Funzioni - Estensione della sintassi}

\begin{grammar}
<Variabili x> $\in \mathbb{X}$, per $\mathbb{X} = \{x, y, z, \dots\}$

<Espressioni e> ::= \texttt{fn x : T} $\implies$ $e$ | $e\,e$ | $x$

<Tipi T> ::= \texttt{int} | \texttt{bool} | \texttt{unit} | \texttt{T} $\rightarrow$ \texttt{T}

<Tipi locazioni$T_{loc}$> ::= \texttt{intref}
\end{grammar}
\subsubsection{Convenzioni}
L'applicazione di funzione si associa
a sinistra: $e_1e_2e_3=(e_1e_2)e_3$. Il tipo di funzione si associa
a destra: 
\[T_1\rightarrow T_2\rightarrow T_3=T_1\rightarrow(T_2\rightarrow T_3)\]

$f$ si estende verso destra quanto più possibile, quindi $\texttt{fn x : unit}
\Rightarrow x$; $x$ corrisponde a $\texttt{fn x : unit} \Rightarrow(x;x)$
e $\texttt{fn x : unit} \Rightarrow \texttt{fn y : unit}\Rightarrow x;y$
ha tipo $\texttt{unit}\rightarrow \texttt{int}\rightarrow \texttt{int}$.

Si osservi che le variabili non rappresentano posizioni ($\mathbb{X} \cap \mathbb{L} =
\emptyset$). Pertanto, l'assegnazione come $x := 3$ non è consentita. Inoltre, non
è possibile eseguire astrazioni sulle posizioni: $fnl:intref \Rightarrow !l + 5$ non
è conforme alla sintassi. Le variabili (\textit{non meta}) $x$, $y$, $z$ non sono
le stesse delle meta-variabili $x$, $y$, $z$, ... La grammatica dei tipi e la sintassi
delle espressioni suggeriscono che il linguaggio includa funzioni di ordine superiore:
è possibile astrarsi su una variabile di qualsiasi tipo.

\section{Shadowing delle variabili}
In un linguaggio con definizioni di funzioni annidate,
è auspicabile definire una funzione senza essere consapevoli delle variabili
nello spazio circostante.
\[
  \texttt{fn x : int} \implies (\texttt{fn x : int} \implies x + 1)
\]
Lo \textit{shadowing} delle variabili è una tecnica che permette di definire
una nuova variabile con lo stesso nome di una variabile già esistente, in modo
che la nuova variabile \textit{nasconda} la precedente. Questa tecnica però non 
è consentita nel linguaggio di programmazione \textit{Java}.
\section{Alpha conversion, variabile libera e vincolata}
Nelle espressioni della forma $\texttt{fn x : T} \Rightarrow e$, la variabile $x$ è
\textbf{vincolata}. Essa rappresenta il parametro formale della funzione:
ogni occorrenza di $x$ in $e$, che non si trovi all'interno di una definizione
di funzione annidata, ha lo stesso significato al di fuori del termine
$\texttt{fn x : T} \Rightarrow e$. Di conseguenza, la variabile $x$ non ha un
significato proprio! Questo implica che non importa quale variabile sia stata
scelta come parametro formale: le espressioni $\texttt{fn x : int} \Rightarrow x+2$ e $\texttt{fn y : int}
\Rightarrow y+2$ denotano esattamente la stessa funzione!

Diremo che un'occorrenza di una variabile $x$ all'interno di un'espressione
$e$ è libera se $x$ non è all'interno di alcun termine della forma $\texttt{fn x : T}\Rightarrow \dots$.

Ad esempio, la variabile $x$ è libera nelle seguenti espressioni: 
\begin{itemize}
    \item $21$
    \item $x+y$
    \item $\texttt{fn z : T}\Rightarrow x+z$
\end{itemize}

Si noti che, nell'ultimo esempio, la variabile $x$ è libera,
ma la variabile $z$ è vincolata dalla definizione di funzione
più interna $\texttt{fn z : T}\Rightarrow...$. 

Si noti inoltre che, nell'espressione 
\[
    \texttt{fn x : T'}\Rightarrow \texttt{fn z : T}\Rightarrow x+z
\]
la variabile $x$ non è più libera, ma è vincolata dalla definizione di funzione
più interna $\texttt{fn x : T'}\Rightarrow...$.

\subsubsection{Convenzione}
Ci consentiremo sempre, in qualsiasi espressione 
\[...(\texttt{fn x : T}\Rightarrow e)...\]
di sostituire il vincolo $x$ e tutte le occorrenze di $x$ in $e$
che sono vincolate a quel legante, con qualsiasi altra variabile
\textit{fresh} che non appare altrove. 

Ad esempio:
\begin{itemize}
    \item $\texttt{fn x : T}\Rightarrow x+z=\texttt{fn y : T}\Rightarrow y+z$
    \item $\texttt{fn x : T}\Rightarrow x+y \not= \texttt{fn y : T}\Rightarrow y+y$
\end{itemize}

Questo processo è chiamato ``Alpha conversion''.

L'intuizione è che le variabili libere non sono ancora vincolate a
nessuna espressione. Definiamo la seguente funzione per induzione:
\[
    \texttt{fv() : Exp}\rightarrow 2^\mathbb{X} \\  
\]
\[
\begin{array}{lcl}
    \texttt{fv}(x) & \defeq & \{x\} \\
    \texttt{fv}(\texttt{fn x : T}\Rightarrow e) & \defeq &  \texttt{fv}(e) \setminus \{x\} \\
    \texttt{fv}(e_1;e_2) = \texttt{fv}(e_1e_2) & \defeq &  \texttt{fv}(e_1) \cup \texttt{fv}(e_2) \\
    \texttt{fv}(n) = \texttt{fv}(b) = \texttt{fv}(!l) = \texttt{fv}(\texttt{skip}) &
    \defeq & \varnothing \\
    \texttt{fv}(e_1 \oplus e_2) \texttt{fv}(\texttt{while } e_1 \texttt{ do } e_2) & 
    \defeq & = \texttt{fv}(e_1) \cup \texttt{fv}(e_2) \\
    \texttt{fv}(l := e) & \defeq & \texttt{fv}(e) \\
    \texttt{fv}(\texttt{if } e_1 \texttt{ then } e_2 \texttt{ else } e_3)
    & \defeq & \texttt{fv}(e_1) \cup \texttt{fv}(e_2) \cup \texttt{fv}(e_3)
\end{array}
\]
Un'espressione $e$ si dice \textbf{chiusa} se $fv(e) = \varnothing$.
\subsection{Sostituzione}
La semantica delle funzioni coinvolge la sostituzione del parametro effettivo per
i parametri formali. Scriviamo $e_2\{e_1/x\}$ per il risultato della sostituzione
di $e_1$ per tutte le occorrenze libere di $x$ in $e_2$.

Ad esempio:
\begin{itemize}
    \item $(x \geq x)\{3/x\} = (3 \geq 3)$
    \item $(\texttt{fn } x:\texttt{int} \Rightarrow x+y)x\{3/x\} = 
    (\texttt{fn } x:\texttt{int} \Rightarrow x+y)3$
    \item $(\texttt{fn } y:\texttt{int} \Rightarrow x+y)\{y+2/x\} = 
    \texttt{fn } z:\texttt{int} \Rightarrow (y+2)+z$
\end{itemize}

Si noti che nell'ultima sostituzione, lavoriamo con l'ausilio dell'alfa conversion
per evitare la cattura di nomi!

Definiamo la sostituzione $\hat{e}\{e/x\}$ come la sostituzione dell'espressione
$e$ per ogni occorrenza libera di $x$ in $\hat{e}$. 

\[
\begin{array}{lcl}
    n\{e/x\} & \defeq & n \\
    b\{e/x\} & \defeq & b \\
    \texttt{skip}\{e/x\} & \defeq & \texttt{skip} \\
    x\{e/x\} & \defeq & e \\
    y\{e/x\} & \defeq & y \\
    (\texttt{fn } x:T\Rightarrow e_1)\{e/z\} & \defeq & \texttt{fn } x:T\Rightarrow e_1\{e/z\} \text{ se } x \not\in \texttt{fv}(e) \\
    (\texttt{fn } x:T\Rightarrow e_1)\{e/z\} & \defeq & (\texttt{fn } y:T\Rightarrow (e_1\{y/x\})\{e/z\}) \text{ se }
    x \in \texttt{fv}(e) \text{ e } y \texttt{ fresh} \\
    (\texttt{fn } x:T\Rightarrow e_1)\{e/x\} & \defeq & \texttt{fn } x:T\Rightarrow e_1 \\
    (e_1e_2)\{e/x\} & \defeq & (e_1\{e/x\}e_2\{e/x\}) \\
\end{array}
\]
D'altra parte, la sostituzione è un omomorfismo per le altre espressioni.
\subsection{Sostituzioni simultanee}
Le sostituzioni possono essere facilmente generalizzate per sostituire
contemporaneamente più variabili. Più precisamente, una sostituzione
simultanea $\sigma$ è una funzione parziale:
\[\sigma: X \rightharpoonup \text{Exp}\]
Data un'espressione $e$, scriviamo $e\sigma$ per indicare l'espressione
risultante dalla sostituzione simultanea di ciascun $x \in \texttt{dom}(\sigma)$
con l'espressione corrispondente $\sigma(x)$. 

Notazione: scriviamo $\sigma$ come $\{e_1/x_1, \ldots, e_k/x_k\}$ invece di
$\{x_1 \mapsto e_1, \ldots, x_k \mapsto e_k\}$. Useremo $e\sigma$ per indicare
l'espressione $e$ che è stata influenzata dalla sostituzione $\sigma$.
\section{Lambda calcolo}

Il $\lambda$ calcolo può essere arricchito in una varietà di modi.
È spesso conveniente aggiungere costrutti speciali per caratteristiche
come numeri, booleani, tuple, record, ecc. Tuttavia, tutte queste
funzionalità possono essere codificate nel $\lambda$ calcolo, quindi rappresentano
solo ``zucchero sintattico''. Tali estensioni portano infine a linguaggi
di programmazione come \texttt{Lisp} (\textit{McCarthy, 1960}),
\texttt{ML} (\textit{Milner et al., 1990}), \texttt{Haskell} (\textit{Hudak et al., 1992})
o \texttt{Scheme} (\textit{Sussman and Steele, 1975}).

Nel $\lambda$ calcolo, abbiamo fondamentalmente esteso il linguaggio con i seguenti
costrutti primitivi:

\begin{grammar}
    <M> $\in Lambda$ ::= $x$ | $\lambda x.M$ | $MM$
\end{grammar}

dove:

\begin{itemize}
    \item $x$ è una variabile, utilizzata per definire parametri formali
    \item $\lambda x.M$ è chiamato $\lambda$-abstraction e definisce funzioni anonime;
    questo costrutto lega la variabile $x$ nel corpo della funzione $M$
    \item $MM$ è l'applicazione della funzione $M$ all'argomento $M$
\end{itemize}

Così, $(\lambda x.M)N$ evolve in $M\{N/x\}$, dove l'argomento $N$ sostituisce
ogni occorrenza (\textit{libera}) di $x$ in $M$. Nel $\lambda$ calcolo puro,
le funzioni sono gli unici valori. Gli interi, i booleani e altri valori di
base possono essere facilmente codificati e non sono primitivi nel $\lambda$ calcolo.
\section{Applicazione di funzioni}
Per valutare $M_1 M_2$,
Prima valutiamo $M_1$ come una funzione $\lambda x.M$.
Quindi, la strategia di valutazione dipende dalla strategia di valutazione adottata.
\begin{tcolorbox}[title = Call by value]
    Valutiamo prima $M_2$ ad un valore $v$ e poi valutiamo $M\{M_2/x\}$. Quindi 
    $M_2$ viene valutato prima di essere passato alla funzione.
\end{tcolorbox}
\begin{tcolorbox}[title = Call by name]
    Valutiamo $M\{M_2/x\}$, perciò $M_2$ verrà valutato se e solo se è necessario.
\end{tcolorbox}
\subsection{Semantica formale}
La regola dell'applicazione è la seguente:
\begin{prooftree}
    \AxiomC{$M_1 \rightarrow M_1'$}
    \LeftLabel{(App)}
    \UnaryInfC{$M_1 M_2 \rightarrow M_1' M_2$}
\end{prooftree}
\subsubsection{Call by value}
\begin{minipage}{0.5\textwidth}
    \begin{prooftree}
        \AxiomC{$M_2 \rightarrow M_2'$}
        \LeftLabel{(CBV.A)}
        \UnaryInfC{$(\lambda x.M) M_2 \rightarrow (\lambda x.M) M_2'$}
    \end{prooftree}
\end{minipage}
\begin{minipage}{0.5\textwidth}
    \begin{prooftree}
        \AxiomC{-}
        \LeftLabel{(CBV.B)}
        \UnaryInfC{$(\lambda x.M) v \rightarrow M\{v/x\}$}
    \end{prooftree}
\end{minipage}

Dove $v \in Val ::= \lambda x.M$.
\subsubsection{Call by name}

    \begin{prooftree}
        \AxiomC{-}
        \LeftLabel{(CBN)}
        \UnaryInfC{$(\lambda x.M) M_2 \rightarrow M\{M_2/x\}$}
    \end{prooftree}

Dove $M_2$ è un termine chiuso.
\subsection{Applicazione ricorsiva}
È possibile esprimere programmi non terminanti nel lambda calcolo? Sì,
naturalmente! Ad esempio, il combinatore di divergenza $\omega$
definito come 
\[\Omega = (\lambda x.xx)(\lambda x.xx)\]
contiene un solo \textit{redex} e la riduzione di questo \textit{redex} produce
nuovamente $\Omega$.
\[\Omega \rightarrow \Omega \rightarrow \Omega \rightarrow \dots \Omega \dots\]
La non terminazione è intrinseca nel lambda calcolo!
\subsection{Differenze tra call by value e call by name}
Notiamo che, a differenza del linguaggio precedente, avere definizioni
di funzioni tra i costrutti può portare a risultati diversi. 
\begin{tcolorbox}
    Danno risultati diversi:
    \begin{itemize}
        \item $(\lambda x.0)(\Omega)\rightarrow_{cbn} = 0 $
        \item $(\lambda x.0)(\Omega)\rightarrow_{cbv} = (\lambda x.0)(\Omega)
        \rightarrow_{cbv}\dots \rightarrow_{cbv}\dots$
    \end{itemize}
\end{tcolorbox}
\begin{tcolorbox}
Il risultato è ancora più sorprendentemente:
\begin{itemize}
    \item $(\lambda x.\lambda y.x)(Id 0)\rightarrow^{*}_{ cbn} = \lambda y.(Id 0) $
    \item $(\lambda x.\lambda y.x)(Id 0)\rightarrow^{*}_{ cbv} = \lambda y.0$
\end{itemize}
\end{tcolorbox}
\section{Comportamento delle funzioni}

Consideriamo l'espressione:

\[ e \defeq = \texttt{fn } x : \texttt{unit} \Rightarrow (l := 1);x(l := 2) \]

Poi consideriamo una esecuzione nello store dove $l$ è associato a $0$:

\[ \langle e, \{l \mapsto 0\} \rangle_* \langle \texttt{skip}, \{l \mapsto ???\} \rangle \]

Il risultato dello store potrebbe variare a seconda della valutazione dell'espressione $e$, poiché
potrebbero essere presenti dichiarazioni o assegnamenti che possono influire sul risultato della 
locazione $l$, quindi avere \textit{side effects}.

Come valutiamo una chiamata di funzione $e_1 \ e_2$?

\begin{tcolorbox}
[title = Call-by-value (\textit{detta anche valutazione eager})]
\begin{enumerate}
    \item Valutiamo $e_1$ a una funzione $\text{fn } x:T \Rightarrow e$
    \item Valutiamo $e_2$ a un valore $v$
    \item Sostituiamo il \textbf{parametro attuale} $v$ per il \textbf{parametro
    formale} $x$ nel corpo della funzione $e$
    \item Valutiamo $e\{v/x\}$
\end{enumerate}
\end{tcolorbox}

Questo metodo è utilizzato in molti linguaggi come \texttt{C}, \texttt{Scheme}, \texttt{ML},
\texttt{OCaml}, \texttt{Java}, ecc. (\textit{ci sono diverse varianti di call-by-value}).

C'è un altro metodo per valutare $e_1 \ e_2$
\begin{tcolorbox}[title = Call-by-name (\textit{detta anche valutazione lazy})]
    \begin{enumerate}
        \item Valutiamo $e_1$ a una funzione $\text{fn } x:T \Rightarrow e$
        \item Sostituiamo il \textbf{parametro attuale} $e_2$ per il \textbf{parametro
        formale} $x$ nel corpo della funzione $e$
        \item Valutiamo $e\{e_2/x\}$
    \end{enumerate}
\end{tcolorbox}
Varianti del call-by-name sono state utilizzate in alcuni linguaggi
di programmazione ben noti, in particolare \texttt{Algol-60} (\textit{Naur et al., 1963})
e \texttt{Haskell} (\textit{Hudak et al., 1992}). Haskell utilizza effettivamente una versione
ottimizzata nota come call-by-need (\textit{Wadsworth, 1971}) che, anziché rivalutare
un argomento ogni volta che viene utilizzato, sovrascrive tutte le occorrenze
dell'argomento con il suo valore la prima volta che viene valutato.

Valutiamo il nostro esempio precedente in una strategia call-by-value:

\[ e = (\text{fn } x : \text{unit} \Rightarrow (l:=1);x)(l:=2) \]

Poi:

\[
\begin{array}{lll}
    \langle e, \{l \mapsto 0\}\rangle & \rightarrow &
    \langle (\texttt{fn } x : \texttt{unit} \Rightarrow (l:= 1); x)\texttt{skip}, \{l \mapsto 2\} \rangle\\
    & \rightarrow & \langle (l:=1;\texttt{skip}), \{l \mapsto 2\} \rangle \\
    & \rightarrow & \langle \texttt{skip};\texttt{skip}, \{l \mapsto 1\} \rangle \\
    & \rightarrow & \langle \texttt{skip}, \{l \mapsto 1\} \rangle
\end{array}
\]
Alla fine della valutazione, la locazione $l$ è associata a $1$.

Procediamo ora con la strategia call-by-name:

\[
\begin{array}{lll}
    \langle e, \{l \mapsto 0\}\rangle &\rightarrow& \langle (l:=1);l := 2, \{l \mapsto 0\} \rangle  \\
    & \rightarrow & \langle \texttt{skip};l := 2, \{l \mapsto 1\} \rangle \\
    & \rightarrow & \langle l := 2, \{l \mapsto 1\} \rangle \\
    & \rightarrow & \langle \texttt{skip}, \{l \mapsto 2\} \rangle
\end{array}
\]

Notiamo quindi che le due strategie di valutazione hanno prodotto risultati diversi.
\subsection{Call-by-value: small-step semantics}
\begin{grammar}
    <Valori v> ::= $b$ | $n$ | $\texttt{skip}$ | $\texttt{fn } x:T \Rightarrow e$
\end{grammar}
\begin{prooftree}
    \AxiomC{$\langle e_1, s \rangle \rightarrow \langle e_1', s'\rangle$ }
    \LeftLabel{(CBV-app1)}
    \UnaryInfC{$\langle e_1 \ e_2, s \rangle \rightarrow \langle e_1' \ e_2, s'\rangle$}
\end{prooftree}
\begin{prooftree}
    \AxiomC{$\langle e_2, s \rangle \rightarrow \langle e_2', s'\rangle$ }
    \LeftLabel{(CBV-app2)}
    \UnaryInfC{$\langle v_1 \ e_2, s \rangle \rightarrow \langle v_1 \ e_2', s'\rangle$}
\end{prooftree}
\begin{prooftree}
    \AxiomC{$-$ }
    \LeftLabel{(CBV-fn)}
    \UnaryInfC{$\langle (\texttt{fn } x:T \Rightarrow e) v, s 
    \rangle \rightarrow \langle e\{v/x\}, s\rangle$}
\end{prooftree}
La valutazione della funzione non tocca lo store, infatti in un linguaggio funzionale
puro non avremmo bisogno di uno store. In $e\{v/x\}$ il valore $v$ verrebbe
copiato tante volte quante sono le occorrenze libere di $x$ in $e$. Le implementazioni
reali non fanno questo.
\subsection{Call-by-name: small-step semantics}
\begin{prooftree}
    \AxiomC{$\langle e_1, s \rangle \rightarrow \langle e_1', s'\rangle$ }
    \LeftLabel{(CBN-app)}
    \UnaryInfC{$\langle e_1 \ e_2, s \rangle \rightarrow \langle e_1' \ e_2, s'\rangle$}
\end{prooftree}
\begin{prooftree}
    \AxiomC{$-$ }
    \LeftLabel{(CBN-fn)}
    \UnaryInfC{$\langle (\texttt{fn } x:T \Rightarrow e) e_2, s 
    \rangle \rightarrow \langle e\{e_2/x\}, s\rangle$}
\end{prooftree}
Non valutiamo l'argomento della funzione, ma lo sostituiamo direttamente nel corpo, in questo 
modo se tale argomento non verrà utilizzato non verrà mai valutato, ma se verrà utilizzato
più volte verrà valutato ogni volta che viene utilizzato.
\section{Tipizzazione delle funzioni}
Fino ad ora, un ambiente di tipi $\Gamma$ fornisce il tipo delle locazioni dello store.
Da adesso in poi, deve anche fornire assunzioni sul tipo delle variabili usate nelle
funzioni, ad esempio 
\[
    \Gamma = \{l_1 : \text{int ref}, x : \text{int}, y : \text{bool} \rightarrow \text{int}\}  
\]
Quindi, estendiamo il set \texttt{TypeEnv} degli ambienti dei tipi come segue:
\[
  \texttt{TypeEnv} \defeq \mathbb{L} \cup \mathbb{X} \rightharpoonup 
  T_{loc} \cup T
\]
Tale che:
\begin{itemize}
    \item $\forall l \in \texttt{dom}(\Gamma). \Gamma(l) \in T_{loc}$
    \item $\forall x \in \texttt{dom}(\Gamma). \Gamma(x) \in T$
\end{itemize}
\textbf{Notazioni}: se $x \notin \text{dom}(\Gamma)$, scriviamo $\Gamma, x : T$ per
la funzione parziale che mappa $x$ su $T$ ma altrimenti è come $\Gamma$.

Si noti che con l'introduzione delle funzioni ci sono più configurazioni
bloccate (ad esempio $2$, $\texttt{true}$, $\texttt{true} \texttt{ fn }x:T \Rightarrow e$, ecc.).
Il nostro sistema di tipi respingerà queste configurazioni tramite le seguenti regole:

\begin{prooftree}
    \AxiomC{$-$}
    \LeftLabel{(var)}
    \RightLabel{\textit{se }$\Gamma(x) = T$}
    \UnaryInfC{$\Gamma \vdash x : T$}
\end{prooftree}



\begin{prooftree}
    \AxiomC{$\Gamma, x : T \vdash e : T'$}
    \LeftLabel{(fn)}
    \UnaryInfC{$\Gamma \vdash \texttt{fn }x:T \Rightarrow e : T \rightarrow T'$}
\end{prooftree}



\begin{prooftree}
    \AxiomC{$\Gamma \vdash e_1 : T \rightarrow T'$}
    \AxiomC{$\Gamma \vdash e_2 : T$}
    \LeftLabel{(app)}
    \BinaryInfC{$\Gamma \vdash e_1 \ e_2 : T'$}
\end{prooftree}
\subsection{Proprietà del sistema di tipi}
Consideriamo solo esecuzioni di programmi chiusi, senza variabili libere.

\begin{theorem}[Progress]
Se $e$ è chiuso e $\Gamma \vdash e : T$ e $\text{dom}(\Gamma)
\subseteq \text{dom}(s)$, allora o c'è un valore o esiste $e'$, $s'$
tale che $\langle e,s \rangle \rightarrow \langle e',s' \rangle$.
\end{theorem}

\begin{theorem}[Conservazione del tipo]
Se $e$ è chiuso e $\Gamma \vdash e : T$ e $\text{dom}(\Gamma)
\subseteq \text{dom}(s)$ e $\langle e,s \rangle \rightarrow
\langle e',s' \rangle$ allora $\Gamma \vdash e' : T$ e $e'$ è
chiuso e $\text{dom}(\Gamma) \subseteq \text{dom}(s')$.
\end{theorem}

Questo richiede il seguente lemma:

\begin{lemma}[Sostituzione]
Se $\Gamma \vdash e : T$ e $\Gamma,x : T \vdash e' : T'$ con
$x \notin \text{dom}(\Gamma)$ allora $\Gamma \vdash e'\{e/x\} : T'$.
\end{lemma}

\begin{theorem}[Normalizzazione]
Nei sotto-linguaggi senza cicli while o operazioni di store,
se $\Gamma \vdash e : T$ ed $e$ è chiuso, allora c'è un valore $v$
tale che, per ogni store, $\langle e,s \rangle \rightarrow^* \langle
v,s \rangle$. In altre parole, se consideriamo un linguaggio funzionale
puro, come il calcolo lambda, la sua versione tipizzata non è più Turing-completa.
\end{theorem}
\section{Dichiarazioni locali}
Per la leggibilità, vogliamo essere in grado di nominare le
espressioni e di limitarne la loro portata.

Consideriamo infatti: 
\[((1+2) \geq (1+2)+4)\]

Questa è una caratteristica molto comune di molti linguaggi di programmazione.
Nuova costruzione: 
\[\texttt{let y:int} = 1+2 \texttt{ in } y \geq (y+4)\]

\textbf{Intuizione}: prima valutiamo $1+2$ per ottenere $3$.
Poi valutiamo $y \geq (y+4)$ con $y$ sostituito da $3$.
Qui, $y$ è un \textit{binder}, lega ogni occorrenza libera di $y$ in $y \geq (y+4)$.
\subsection{Sintassi e tipi}

Estendiamo la sintassi delle nostre espressioni: 
\begin{grammar}
    <e> ::= ... | $\texttt{ let  }x : T = e_1 \texttt{ in } e_2$
\end{grammar}

Forniamo la regola di tipizzazione del nuovo costrutto:

\begin{prooftree}
    \AxiomC{$\Gamma \vdash e_1 : T$}
    \LeftLabel{(let)}
    \AxiomC{$\Gamma, x:T \vdash e_2 : T'$}
    \BinaryInfC{$\Gamma \vdash \texttt{let } x:T=e_1 \texttt{ in } e_2 : T'$}
\end{prooftree}


Si noti che, poiché $x$ è una variabile locale, $\Gamma$ non contiene una voce per la variabile
$x$. Ciò significa che, come per \texttt{fn}, la tipizzazione di un costrutto \texttt{let} può
richiedere un'alpha conversione .

\subsection{Intuizione}
In un costrutto \texttt{let}, le variabili sono segnaposto per quantità sconosciute.

\textbf{Problema 1:} Molte espressioni non hanno senso
\[
\begin{aligned}
    \texttt{let } y:T = 2+3 \texttt{ in } y+\textcolor{red}{z} \\
    \texttt{let } z:T = 2+\textcolor{red}{x} \texttt{ in } z \times z
\end{aligned}
\]

\textbf{Problema 2:} Le variabili possono essere usate in più ruoli
\[ \text{let } z:T = 2+\textcolor{red}{z} \texttt{ in } \textcolor{red}{z} \times y \]

\textbf{Problema 3:} Dichiarazioni multiple per lo stesso valore
\[ \texttt{let } \textcolor{red}{y}:T = 1 \texttt{ in let }
\textcolor{red}{y}:T' = (1+2) \texttt{ in } y \times (y+4) \]

\subsection{Variabili legate e libere}
Intuizione: Nel costrutto \texttt{let}, $\textcolor{blue}{y}:T = 2+3$ \texttt{in} $\textcolor{blue}{y}+
\textcolor{red}{z}$
\begin{itemize}
    \item $\textcolor{blue}{y}$ è legato, rappresenta l'espressione $2+3$.
    \item $\textcolor{red}{z}$ è libero, non rappresenta alcuna espressione.
\end{itemize}

Nel costrutto \texttt{let}, $\textcolor{blue}{z}:T = 2+\textcolor{red}{x}$ \texttt{in} $\textcolor{blue}{z} \times
(\textcolor{blue}{z}+\textcolor{red}{y})$
\begin{itemize}
    \item $\textcolor{blue}{z}$ è legato, rappresenta l'espressione $2+x$
    \item $\textcolor{red}{x}$ e $\textcolor{red}{y}$ sono libere, non rappresentano alcuna espressione.
\end{itemize}

Nel costrutto \texttt{let}, $\textcolor{blue}{z}:T = 2+\textcolor{red}{z}$ \texttt{in} $\textcolor{blue}{z} \times
(\textcolor{blue}{z}+\textcolor{red}{y})$
\begin{itemize}
    \item $\textcolor{blue}{z}$ è legato, rappresenta l'espressione $2+\textcolor{red}{z}$
    \item $\textcolor{red}{z}$ e $\textcolor{red}{y}$ sono libere, non rappresentano alcuna espressione.
\end{itemize}
\subsection{Alpha conversion}
Come il costrutto \texttt{let} è un binder per la variabile locale, possiamo utilizzare
l'alfa conversion quando necessario. 

Convenzione: ci permetteremo ogni volta, in qualsiasi espressione
\texttt{let} $x:T=e_1 \texttt{ in } e_2$ di sostituire il legame $x$ e tutte le
occorrenze di $x$ in $e_2$ che sono legate a tale binder, con qualsiasi altra variabile
fresh che non compare altrove: 

\[
    \texttt{let } x:T=e1 \texttt{ in } e_2 \equiv_\alpha \text{let } y:T=e_1 \texttt{ in } e_2\{y/x\}
\]

dove $y$ è una variabile fresh, ovvero non compare né in $e_1$ né in $e_2$.

La definizione delle variabili libere per il costrutto \texttt{let} è come ci si potrebbe aspettare:

\[
    \text{fv}(\text{let } x:T=e1 \text{ in } e2) \equiv \text{fv}(e1) \cup (\text{fv}(e2) \setminus \{x\})
\]

Per quanto riguarda la sostituzione, dobbiamo fare attenzione in quanto potremmo
aver bisogno di lavorare fino all'alfa conversion:
\[
\begin{array}{lll}
(\texttt{let } x:T=e_1 \texttt{ in } e_2)\{e/z\} & \defeq & 
(\texttt{let } x:T=e_1\{e/z\} \texttt{ in } e_2\{e/z\}) \\
& & \text{se }x \notin \texttt{fv}(e)\\
(\texttt{let } x:T=e_1 \texttt{ in } e_2)\{e/z\} & \defeq & 
(\texttt{let } y:T=e_1\{e/z\} \texttt{ in } (e_2\{y/x\}\{e/z\}))  \\
& &  \text{se }x \in \texttt{fv}(e) \text{ and } y \texttt{ fresh}\\
(\texttt{let } x:T=e_1 \texttt{ in } e_2)\{e/z\} & \defeq & 
(\texttt{let } x:T=e_1\{e/z\} \texttt{ in } e_2)\\
\end{array}
\]
Le nostre definizioni utilizzano variabili e non meta-variabili: quindi $x\not=z$!
\subsection{Small-step semantics}
Per quanto riguarda l'applicazione di funzione, possiamo avere almeno un paio di
semantica differenti per il costrutto \texttt{let}.
\subsubsection{Call-by-value}
\begin{prooftree}
    \AxiomC{$\langle e_1, s \rangle \rightarrow \langle e_1', s' \rangle$}
    \LeftLabel{(CBV-let1)}
    \UnaryInfC{$\langle \texttt{let } x:T=e_1 \texttt{ in } e_2, s \rangle
    \rightarrow \langle \texttt{let } x:T=e_1' \texttt{ in } e_2, s' \rangle$}
\end{prooftree}
\begin{prooftree}
    \AxiomC{$-$}
    \LeftLabel{(CBV-let2)}
    \UnaryInfC{$\langle \texttt{let } x:T=v \texttt{ in } e_2, s \rangle
    \rightarrow \langle \texttt{let } x:T=e_2\{v / x\}, s \rangle$}
\end{prooftree}
\subsubsection{Call-by-name}
\begin{prooftree}
    \AxiomC{$-$}
    \LeftLabel{(CBN-let)}
    \UnaryInfC{$\langle \texttt{let } x:T=e_1 \texttt{ in } e_2, s \rangle
    \rightarrow \langle e_2\{e_1 / x\}, s \rangle$}
\end{prooftree}
\section{Ricorsione}
Nel contesto del nostro linguaggio, la modellazione delle funzioni ricorsive solleva
importanti questioni sulle loro proprietà e sulla loro presenza libera. È innegabile che,
grazie al teorema di normalizzazione, abbiamo acquisito la consapevolezza che senza istruzioni
di ciclo o memorizzazione non possiamo sostenere calcoli infiniti, rendendo così impossibile
la presenza diretta di meccanismi di ricorsione.

Tuttavia, possiamo notare che mediante l'uso combinato di costrutti come \texttt{if}, \texttt{while}
e la memorizzazione, siamo in grado di implementare la ricorsione all'interno del nostro linguaggio.
È importante sottolineare che, pur offrendo questa possibilità, l'utilizzo di tali costrutti può
diventare complesso e richiedere una pianificazione attenta.

D'altra parte, è interessante osservare che abbiamo già intrapreso questa strada. Pertanto, per garantire
un'implementazione efficiente e strutturata delle funzioni ricorsive, è auspicabile prendere 
come ispirazione il $\lambda$-calcolo.

\subsection{Punto fisso}
In matematica, un punto fisso $p$
(\textit{noto anche come punto invariante}) di una funzione $f$ è un punto che viene
mappato su se stesso, cioè $f(p) = p$. I punti fissi rappresentano il cuore di
ciò che è noto come teoria della ricorsione.

\begin{tcolorbox}[title = Teoremi di ricorsione di Kleene]
I teoremi di ricorsione di Kleene
sono un paio di risultati fondamentali sull'applicazione di funzioni calcolabili
alle proprie descrizioni. I due teoremi di ricorsione possono essere applicati per
costruire punti fissi di determinate operazioni su funzioni calcolabili, per generare
quines e per costruire funzioni definite tramite definizioni ricorsive.
\end{tcolorbox}

\begin{theorem}[Rice]
per ogni proprietà non
banale delle funzioni parziali, non esiste un metodo generale ed efficace per decidere
se un algoritmo calcoli una funzione parziale con tale proprietà.
\end{theorem}
Quindi, è molto importante dimostrare che il $\lambda$-calcolo può esprimere i punti fissi. Utilizziamo il combinatorio di Turing per derivare i punti fissi:

\[A \defeq \lambda x. \lambda y. y(xxy)\]

\[\texttt{fix} \defeq AA\]

$\texttt{fix}$ è una funzione ricorsiva che, dato un termine $M$,
restituisce un punto fisso di $M$, indicato con $\text{fix}M$.
Infatti, per qualsiasi termine $M$, utilizzando una valutazione call-by-name, abbiamo:

\[
  \begin{array}{lll}
    \textcolor{red}{\texttt{fix }M} & \rightarrow & (\lambda y.y(AAy))M \\
    & \rightarrow & M(\textcolor{red}{\texttt{fix }M}) \\
  \end{array}                
\]
Ora, se scegliamo $M$ nella forma $\lambda f.\lambda x.B$, per qualche
corpo $B$, allora, in una semantica call-by-name, abbiamo:
\[
  \begin{array}{lll}
    \textcolor{red}{\texttt{fix}} (\lambda f.\lambda x.B) & \rightarrow^* &
    (\lambda f.\lambda y(B))(\textcolor{red}{\texttt{fix}} (\lambda f. \lambda x.B)) \\
    & \rightarrow & \lambda y.B\{\textcolor{red}{\texttt{fix}} (\lambda f. \lambda x.B) / f\}\\
  \end{array}  
\]
\subsubsection{Definizioni ricorsive}
Quindi, se definiamo:

\[\textcolor{red}{\texttt{rec} f.B} \textit{ come abbreviazione per }
\textcolor{red}{\texttt{fix}(\lambda f.\lambda x.B)}\]

allora possiamo riscrivere la riduzione precedente come:

\[
    \textcolor{red}{\texttt{rec} f.B} \rightarrow^* \lambda x.B\{\textcolor{red}{\texttt{rec} f.B}/f\}
\]
\subsubsection{Punto fisso nella semantica call-by-name}
Abbiamo quindi trovato un modo per esprimere funzioni ricorsive nel $\lambda$-calcolo,
secondo la semantica call-by-name. Questo meccanismo funziona anche con la semantica
call-by-value? Proviamo a vedere:

\[
\begin{array}{lcl}
    \texttt{rec} f.B & \defeq & \texttt{fix}(\lambda f.\lambda x.B) \\
    & \rightarrow^* & (\lambda f.\lambda x.B)(\texttt{fix}(\lambda f.\lambda x.B)) \\
    & \rightarrow^* & (\lambda f.\lambda x.B)(\lambda f.\lambda x.B)(\texttt{fix}(\lambda f.\lambda x.B)) \\
    & \rightarrow^* & (\lambda f.\lambda x.B)(\lambda f.\lambda x.B)(\lambda f.\lambda x.B)(\texttt{fix}(\lambda f.\lambda x.B)) \\
    & \rightarrow^* & \text{... all'infinito!} \\
\end{array}
\]
Dobbiamo quindi fermare l'espansione indefinita di $\texttt{fix}(\lambda f.\lambda x.B)$.

Redifiniamo quindi il combinatore di punto fisso come:
\begin{tcolorbox}[title = Call-by-name fixed-point combinator]
    \[
      \begin{array}{lll}
        A \defeq \lambda x. \lambda y. y(xxy) & \texttt{fix} \defeq AA 
        & \textcolor{red}{\texttt{fix }M \rightarrow^* M(\texttt{fix }M)}\\
      \end{array}  
    \]
\end{tcolorbox}
\begin{tcolorbox}[title = Call-by-value fixed-point combinator]
    \[
      \begin{array}{lll}
        A_v \defeq \lambda x. \lambda y. y(\lambda z .(xxy)z) & \texttt{fix}_v \defeq A_vA_v 
        & \textcolor{red}{\texttt{fix}_vM \rightarrow^* M(\lambda z.(\texttt{fix}_vM)z)}\\
      \end{array}  
    \]
\end{tcolorbox}
\begin{tcolorbox}[title = Call-by-value nelle funzioni ricorsive]
Sia $\texttt{rec}_v f.B$ un'abbreviazione per $\texttt{fix}_v(\lambda f.\lambda x.B)$, allora
\[
  \textcolor{red}{\texttt{rec}_v f.B} \rightarrow^* \lambda x.B\{
    \textcolor{red}{\lambda z .(\texttt{rec}_v f.B)z}/f\}  
\]
Notiamo che non diverge più.
\end{tcolorbox}
\section{Punto fisso nel linguaggio funzionale}
Dobbiamo adottare uno dei combinatori di punto fisso precedenti? No, nessuno dei combinatori
precedenti è ben tipizzato per il teorema di Normalizzazione.
\subsubsection{Sintassi}
\begin{grammar}
    <e> ::= \dots | \texttt{fix} e
\end{grammar}
\subsubsection{Tipizzazione}
\begin{prooftree}
    \AxiomC{$\Gamma \vdash e : (T_1 \rightarrow T_2) \rightarrow (T_1 \rightarrow T_2)$}
    \LeftLabel{(T-Fix)}
    \UnaryInfC{$\Gamma \vdash \texttt{fix }e : T_1 \rightarrow T_2$}
\end{prooftree}
\subsubsection{Semantica}
\begin{minipage}{0.45\textwidth}
    \begin{prooftree}
        \AxiomC{$-$}
        \LeftLabel{(Fix-cbn)}
        \UnaryInfC{$\texttt{fix}.e \rightarrow e(\texttt{fix}.e)$}
    \end{prooftree}
\end{minipage}
\begin{minipage}{0.55\textwidth}
    \begin{prooftree}
        \AxiomC{$e \equiv \texttt{fn } f: T_1 \rightarrow T_2 \implies e_2$}
        \LeftLabel{(Fix-cbv)}
        \UnaryInfC{$\texttt{fix }e \rightarrow e(\texttt{fn }x: T_1 \implies (\texttt{fix}.e)x)$}
    \end{prooftree}
\end{minipage}
\subsection{La regola (\textit{Fix-cbv}) non funziona per la semantica call-by-value}
Se utilizziamo la regola (\textit{Fix-cbv}) per la semantica call-by-value, otteniamo:
\[
  \begin{array}{lll}
    \texttt{fact }1 & \rightarrow & \texttt{Fact}(\texttt{fact})1 \\
    & \rightarrow & \texttt{Fact}(\texttt{Fact}(\texttt{fact}))1 \\
    & \rightarrow & \dots \\
    & \rightarrow & \texttt{Fact}(\texttt{Fact}(\texttt{Fact}(\dots\texttt{fact})))1 \\
  \end{array}  
\]
La ricorsione call-by-value ha bisogno di un meccanismo per interrompere
la valutazione della iterazione successiva. Ecco perché abbiamo definito una
regola diversa (\texttt{Fix-cbv}) da utilizzare nella semantica call-by-value.
\subsection{Decodifica del \texttt{while}}
Avendo definito la ricorsione come primitiva, potremmo rimuovere il costrutto \texttt{while}
se lo volessimo:
\[
    \begin{array}{lll}
        W & \defeq & \texttt{fn w : unit} \rightarrow 
        \texttt{unit}.\texttt{fn }y:\texttt{unit} \text{ se } e_1 \text{ allora }
        (e_2;(w \texttt{ skip})) \text{ altrimenti } \texttt{skip} \\
    \end{array}
    \]
Per $w$ e $y$ non presenti in $\texttt{fv}(e_1) \cup \texttt{fv}(e_2)$.

Allora
\[
  \texttt{while }e_1 \texttt{ do } e_2 \rightarrow \texttt{fix }W \texttt{ skip}
\]
e l'operatore \texttt{while} non è più primitivo.

\end{document}
