\chapter{Funzioni}
Molti dei linguaggi di programmazione hanno delle 
notazione per indicare le funzioni, metodi o procedure.

Gli esempi possono essere di vari natura, vediamone alcuni:
\begin{itemize}
\item \texttt{fn x : int} $\implies$ \texttt{x + 1}
\item $(\texttt{fn x : int} \implies \texttt{x + 1})8$
\item $\texttt{fn y} \implies (\texttt{fn x : int} \implies x + y)$
\item \dots
\end{itemize}
Per semplicità le nostre funzioni saranno \textbf{anonime}, prenderanno 
un solo argomento e restituiranno un solo valore e saranno sempre tipate.
\section{Funzioni - Estensione della sintassi}

\begin{grammar}
<Variabili x> $\in \mathbb{X}$, per $\mathbb{X} = \{x, y, z, \dots\}$

<Espressioni e> ::= \texttt{fn x : T} $\implies$ $e$ | $e\,e$ | $x$

<Tipi T> ::= \texttt{int} | \texttt{bool} | \texttt{unit} | \texttt{T} $\rightarrow$ \texttt{T}

<Tipi locazioni$T_{loc}$> ::= \texttt{intref}
\end{grammar}
\subsubsection{Convenzioni}
L'applicazione di funzione si associa
a sinistra: $e_1e_2e_3=(e_1e_2)e_3$. Il tipo di funzione si associa
a destra: 
\[T_1\rightarrow T_2\rightarrow T_3=T_1\rightarrow(T_2\rightarrow T_3)\]

$f$ si estende verso destra quanto più possibile, quindi $\texttt{fn x : unit}
\Rightarrow x$; $x$ corrisponde a $\texttt{fn x : unit} \Rightarrow(x;x)$
e $\texttt{fn x : unit} \Rightarrow \texttt{fn y : unit}\Rightarrow x;y$
ha tipo $\texttt{unit}\rightarrow \texttt{int}\rightarrow \texttt{int}$.

Si osservi che le variabili non rappresentano posizioni ($\mathbb{X} \cap \mathbb{L} =
\emptyset$). Pertanto, l'assegnazione come $x := 3$ non è consentita. Inoltre, non
è possibile eseguire astrazioni sulle posizioni: $fnl:intref \Rightarrow !l + 5$ non
è conforme alla sintassi. Le variabili (\textit{non meta}) $x$, $y$, $z$ non sono
le stesse delle meta-variabili $x$, $y$, $z$, ... La grammatica dei tipi e la sintassi
delle espressioni suggeriscono che il linguaggio includa funzioni di ordine superiore:
è possibile astrarsi su una variabile di qualsiasi tipo.

\section{Shadowing delle variabili}
In un linguaggio con definizioni di funzioni annidate,
è auspicabile definire una funzione senza essere consapevoli delle variabili
nello spazio circostante.
\[
  \texttt{fn x : int} \implies (\texttt{fn x : int} \implies x + 1)
\]
Lo \textit{shadowing} delle variabili è una tecnica che permette di definire
una nuova variabile con lo stesso nome di una variabile già esistente, in modo
che la nuova variabile \textit{nasconda} la precedente. Questa tecnica però non 
è consentita nel linguaggio di programmazione \textit{Java}.
\section{Alpha conversion, variabile libera e vincolata}
Nelle espressioni della forma $\texttt{fn x : T} \Rightarrow e$, la variabile $x$ è
\textbf{vincolata}. Essa rappresenta il parametro formale della funzione:
ogni occorrenza di $x$ in $e$, che non si trovi all'interno di una definizione
di funzione annidata, ha lo stesso significato al di fuori del termine
$\texttt{fn x : T} \Rightarrow e$. Di conseguenza, la variabile $x$ non ha un
significato proprio! Questo implica che non importa quale variabile sia stata
scelta come parametro formale: le espressioni $\texttt{fn x : int} \Rightarrow x+2$ e $\texttt{fn y : int}
\Rightarrow y+2$ denotano esattamente la stessa funzione!

Diremo che un'occorrenza di una variabile $x$ all'interno di un'espressione
$e$ è libera se $x$ non è all'interno di alcun termine della forma $\texttt{fn x : T}\Rightarrow \dots$.

Ad esempio, la variabile $x$ è libera nelle seguenti espressioni: 
\begin{itemize}
    \item $21$
    \item $x+y$
    \item $\texttt{fn z : T}\Rightarrow x+z$
\end{itemize}

Si noti che, nell'ultimo esempio, la variabile $x$ è libera,
ma la variabile $z$ è vincolata dalla definizione di funzione
più interna $\texttt{fn z : T}\Rightarrow...$. 

Si noti inoltre che, nell'espressione 
\[
    \texttt{fn x : T'}\Rightarrow \texttt{fn z : T}\Rightarrow x+z
\]
la variabile $x$ non è più libera, ma è vincolata dalla definizione di funzione
più interna $\texttt{fn x : T'}\Rightarrow...$.

\subsubsection{Convenzione}
Ci consentiremo sempre, in qualsiasi espressione 
\[...(\texttt{fn x : T}\Rightarrow e)...\]
di sostituire il vincolo $x$ e tutte le occorrenze di $x$ in $e$
che sono vincolate a quel legante, con qualsiasi altra variabile
\textit{fresh} che non appare altrove. 

Ad esempio:
\begin{itemize}
    \item $\texttt{fn x : T}\Rightarrow x+z=\texttt{fn y : T}\Rightarrow y+z$
    \item $\texttt{fn x : T}\Rightarrow x+y \not= \texttt{fn y : T}\Rightarrow y+y$
\end{itemize}

Questo processo è chiamato ``Alpha conversion''.

L'intuizione è che le variabili libere non sono ancora vincolate a
nessuna espressione. Definiamo la seguente funzione per induzione:
\[
    \texttt{fv() : Exp}\rightarrow 2^\mathbb{X} \\  
\]
\[
\begin{array}{lcl}
    \texttt{fv}(x) & \defeq & \{x\} \\
    \texttt{fv}(\texttt{fn x : T}\Rightarrow e) & \defeq &  \texttt{fv}(e) \setminus \{x\} \\
    \texttt{fv}(e_1;e_2) = \texttt{fv}(e_1e_2) & \defeq &  \texttt{fv}(e_1) \cup \texttt{fv}(e_2) \\
    \texttt{fv}(n) = \texttt{fv}(b) = \texttt{fv}(!l) = \texttt{fv}(\texttt{skip}) &
    \defeq & \varnothing \\
    \texttt{fv}(e_1 \oplus e_2) \texttt{fv}(\texttt{while } e_1 \texttt{ do } e_2) & 
    \defeq & = \texttt{fv}(e_1) \cup \texttt{fv}(e_2) \\
    \texttt{fv}(l := e) & \defeq & \texttt{fv}(e) \\
    \texttt{fv}(\texttt{if } e_1 \texttt{ then } e_2 \texttt{ else } e_3)
    & \defeq & \texttt{fv}(e_1) \cup \texttt{fv}(e_2) \cup \texttt{fv}(e_3)
\end{array}
\]
Un'espressione $e$ si dice \textbf{chiusa} se $fv(e) = \varnothing$.
\subsection{Sostituzione}
La semantica delle funzioni coinvolge la sostituzione del parametro effettivo per
i parametri formali. Scriviamo $e_2\{e_1/x\}$ per il risultato della sostituzione
di $e_1$ per tutte le occorrenze libere di $x$ in $e_2$.

Ad esempio:
\begin{itemize}
    \item $(x \geq x)\{3/x\} = (3 \geq 3)$
    \item $(\texttt{fn } x:\texttt{int} \Rightarrow x+y)x\{3/x\} = 
    (\texttt{fn } x:\texttt{int} \Rightarrow x+y)3$
    \item $(\texttt{fn } y:\texttt{int} \Rightarrow x+y)\{y+2/x\} = 
    \texttt{fn } z:\texttt{int} \Rightarrow (y+2)+z$
\end{itemize}

Si noti che nell'ultima sostituzione, lavoriamo con l'ausilio dell'alfa conversion
per evitare la cattura di nomi!

Definiamo la sostituzione $\hat{e}\{e/x\}$ come la sostituzione dell'espressione
$e$ per ogni occorrenza libera di $x$ in $\hat{e}$. 

\[
\begin{array}{lcl}
    n\{e/x\} & \defeq & n \\
    b\{e/x\} & \defeq & b \\
    \texttt{skip}\{e/x\} & \defeq & \texttt{skip} \\
    x\{e/x\} & \defeq & e \\
    y\{e/x\} & \defeq & y \\
    (\texttt{fn } x:T\Rightarrow e_1)\{e/z\} & \defeq & \texttt{fn } x:T\Rightarrow e_1\{e/z\} \text{ se } x \not\in \texttt{fv}(e) \\
    (\texttt{fn } x:T\Rightarrow e_1)\{e/z\} & \defeq & (\texttt{fn } y:T\Rightarrow (e_1\{y/x\})\{e/z\}) \text{ se }
    x \in \texttt{fv}(e) \text{ e } y \texttt{ fresh} \\
    (\texttt{fn } x:T\Rightarrow e_1)\{e/x\} & \defeq & \texttt{fn } x:T\Rightarrow e_1 \\
    (e_1e_2)\{e/x\} & \defeq & (e_1\{e/x\}e_2\{e/x\}) \\
\end{array}
\]
D'altra parte, la sostituzione è un omomorfismo per le altre espressioni.
\subsection{Sostituzioni simultanee}
Le sostituzioni possono essere facilmente generalizzate per sostituire
contemporaneamente più variabili. Più precisamente, una sostituzione
simultanea $\sigma$ è una funzione parziale:
\[\sigma: X \rightharpoonup \text{Exp}\]
Data un'espressione $e$, scriviamo $e\sigma$ per indicare l'espressione
risultante dalla sostituzione simultanea di ciascun $x \in \texttt{dom}(\sigma)$
con l'espressione corrispondente $\sigma(x)$. 

Notazione: scriviamo $\sigma$ come $\{e_1/x_1, \ldots, e_k/x_k\}$ invece di
$\{x_1 \mapsto e_1, \ldots, x_k \mapsto e_k\}$. Useremo $e\sigma$ per indicare
l'espressione $e$ che è stata influenzata dalla sostituzione $\sigma$.
\section{Lambda calcolo}

Il $\lambda$ calcolo può essere arricchito in una varietà di modi.
È spesso conveniente aggiungere costrutti speciali per caratteristiche
come numeri, booleani, tuple, record, ecc. Tuttavia, tutte queste
funzionalità possono essere codificate nel $\lambda$ calcolo, quindi rappresentano
solo ``zucchero sintattico''. Tali estensioni portano infine a linguaggi
di programmazione come \texttt{Lisp} (\textit{McCarthy, 1960}),
\texttt{ML} (\textit{Milner et al., 1990}), \texttt{Haskell} (\textit{Hudak et al., 1992})
o \texttt{Scheme} (\textit{Sussman and Steele, 1975}).

Nel $\lambda$ calcolo, abbiamo fondamentalmente esteso il linguaggio con i seguenti
costrutti primitivi:

\begin{grammar}
    <M> $\in Lambda$ ::= $x$ | $\lambda x.M$ | $MM$
\end{grammar}

dove:

\begin{itemize}
    \item $x$ è una variabile, utilizzata per definire parametri formali
    \item $\lambda x.M$ è chiamato $\lambda$-abstraction e definisce funzioni anonime;
    questo costrutto lega la variabile $x$ nel corpo della funzione $M$
    \item $MM$ è l'applicazione della funzione $M$ all'argomento $M$
\end{itemize}

Così, $(\lambda x.M)N$ evolve in $M\{N/x\}$, dove l'argomento $N$ sostituisce
ogni occorrenza (\textit{libera}) di $x$ in $M$. Nel $\lambda$ calcolo puro,
le funzioni sono gli unici valori. Gli interi, i booleani e altri valori di
base possono essere facilmente codificati e non sono primitivi nel $\lambda$ calcolo.
\section{Applicazione di funzioni}
Per valutare $M_1 M_2$,
Prima valutiamo $M_1$ come una funzione $\lambda x.M$.
Quindi, la strategia di valutazione dipende dalla strategia di valutazione adottata.
\begin{tcolorbox}[title = Call by value]
    Valutiamo prima $M_2$ ad un valore $v$ e poi valutiamo $M\{M_2/x\}$. Quindi 
    $M_2$ viene valutato prima di essere passato alla funzione.
\end{tcolorbox}
\begin{tcolorbox}[title = Call by name]
    Valutiamo $M\{M_2/x\}$, perciò $M_2$ verrà valutato se e solo se è necessario.
\end{tcolorbox}
\subsection{Semantica formale}
La regola dell'applicazione è la seguente:
\begin{prooftree}
    \AxiomC{$M_1 \rightarrow M_1'$}
    \LeftLabel{(App)}
    \UnaryInfC{$M_1 M_2 \rightarrow M_1' M_2$}
\end{prooftree}
\subsubsection{Call by value}
\begin{minipage}{0.5\textwidth}
    \begin{prooftree}
        \AxiomC{$M_2 \rightarrow M_2'$}
        \LeftLabel{(CBV.A)}
        \UnaryInfC{$(\lambda x.M) M_2 \rightarrow (\lambda x.M) M_2'$}
    \end{prooftree}
\end{minipage}
\begin{minipage}{0.5\textwidth}
    \begin{prooftree}
        \AxiomC{-}
        \LeftLabel{(CBV.B)}
        \UnaryInfC{$(\lambda x.M) v \rightarrow M\{v/x\}$}
    \end{prooftree}
\end{minipage}

Dove $v \in Val ::= \lambda x.M$.
\subsubsection{Call by name}

    \begin{prooftree}
        \AxiomC{-}
        \LeftLabel{(CBN)}
        \UnaryInfC{$(\lambda x.M) M_2 \rightarrow M\{M_2/x\}$}
    \end{prooftree}

Dove $M_2$ è un termine chiuso.
\subsection{Applicazione ricorsiva}
È possibile esprimere programmi non terminanti nel lambda calcolo? Sì,
naturalmente! Ad esempio, il combinatore di divergenza $\omega$
definito come 
\[\Omega = (\lambda x.xx)(\lambda x.xx)\]
contiene un solo \textit{redex} e la riduzione di questo \textit{redex} produce
nuovamente $\Omega$.
\[\Omega \rightarrow \Omega \rightarrow \Omega \rightarrow \dots \Omega \dots\]
La non terminazione è intrinseca nel lambda calcolo!
\subsection{Differenze tra call by value e call by name}
Notiamo che, a differenza del linguaggio precedente, avere definizioni
di funzioni tra i costrutti può portare a risultati diversi. 
\begin{tcolorbox}
    Danno risultati diversi:
    \begin{itemize}
        \item $(\lambda x.0)(\Omega)\rightarrow_{cbn} = 0 $
        \item $(\lambda x.0)(\Omega)\rightarrow_{cbv} = (\lambda x.0)(\Omega)
        \rightarrow_{cbv}\dots \rightarrow_{cbv}\dots$
    \end{itemize}
\end{tcolorbox}
\begin{tcolorbox}
Il risultato è ancora più sorprendentemente:
\begin{itemize}
    \item $(\lambda x.\lambda y.x)(Id 0)\rightarrow^{*}_{ cbn} = \lambda y.(Id 0) $
    \item $(\lambda x.\lambda y.x)(Id 0)\rightarrow^{*}_{ cbv} = \lambda y.0$
\end{itemize}
\end{tcolorbox}
\section{Comportamento delle funzioni}

Consideriamo l'espressione:

\[ e \defeq = \texttt{fn } x : \texttt{unit} \Rightarrow (l := 1);x(l := 2) \]

Poi consideriamo una esecuzione nello store dove $l$ è associato a $0$:

\[ \langle e, \{l \mapsto 0\} \rangle_* \langle \texttt{skip}, \{l \mapsto ???\} \rangle \]

Il risultato dello store potrebbe variare a seconda della valutazione dell'espressione $e$, poiché
potrebbero essere presenti dichiarazioni o assegnamenti che possono influire sul risultato della 
locazione $l$, quindi avere \textit{side effects}.

Come valutiamo una chiamata di funzione $e_1 \ e_2$?

\begin{tcolorbox}
[title = Call-by-value (\textit{detta anche valutazione eager})]
\begin{enumerate}
    \item Valutiamo $e_1$ a una funzione $\text{fn } x:T \Rightarrow e$
    \item Valutiamo $e_2$ a un valore $v$
    \item Sostituiamo il \textbf{parametro attuale} $v$ per il \textbf{parametro
    formale} $x$ nel corpo della funzione $e$
    \item Valutiamo $e\{v/x\}$
\end{enumerate}
\end{tcolorbox}

Questo metodo è utilizzato in molti linguaggi come \texttt{C}, \texttt{Scheme}, \texttt{ML},
\texttt{OCaml}, \texttt{Java}, ecc. (\textit{ci sono diverse varianti di call-by-value}).

C'è un altro metodo per valutare $e_1 \ e_2$
\begin{tcolorbox}[title = Call-by-name (\textit{detta anche valutazione lazy})]
    \begin{enumerate}
        \item Valutiamo $e_1$ a una funzione $\text{fn } x:T \Rightarrow e$
        \item Sostituiamo il \textbf{parametro attuale} $e_2$ per il \textbf{parametro
        formale} $x$ nel corpo della funzione $e$
        \item Valutiamo $e\{e_2/x\}$
    \end{enumerate}
\end{tcolorbox}
Varianti del call-by-name sono state utilizzate in alcuni linguaggi
di programmazione ben noti, in particolare \texttt{Algol-60} (\textit{Naur et al., 1963})
e \texttt{Haskell} (\textit{Hudak et al., 1992}). Haskell utilizza effettivamente una versione
ottimizzata nota come call-by-need (\textit{Wadsworth, 1971}) che, anziché rivalutare
un argomento ogni volta che viene utilizzato, sovrascrive tutte le occorrenze
dell'argomento con il suo valore la prima volta che viene valutato.

Valutiamo il nostro esempio precedente in una strategia call-by-value:

\[ e = (\text{fn } x : \text{unit} \Rightarrow (l:=1);x)(l:=2) \]

Poi:

\[
\begin{array}{lll}
    \langle e, \{l \mapsto 0\}\rangle & \rightarrow &
    \langle (\texttt{fn } x : \texttt{unit} \Rightarrow (l:= 1); x)\texttt{skip}, \{l \mapsto 2\} \rangle\\
    & \rightarrow & \langle (l:=1;\texttt{skip}), \{l \mapsto 2\} \rangle \\
    & \rightarrow & \langle \texttt{skip};\texttt{skip}, \{l \mapsto 1\} \rangle \\
    & \rightarrow & \langle \texttt{skip}, \{l \mapsto 1\} \rangle
\end{array}
\]
Alla fine della valutazione, la locazione $l$ è associata a $1$.

Procediamo ora con la strategia call-by-name:

\[
\begin{array}{lll}
    \langle e, \{l \mapsto 0\}\rangle &\rightarrow& \langle (l:=1);l := 2, \{l \mapsto 0\} \rangle  \\
    & \rightarrow & \langle \texttt{skip};l := 2, \{l \mapsto 1\} \rangle \\
    & \rightarrow & \langle l := 2, \{l \mapsto 1\} \rangle \\
    & \rightarrow & \langle \texttt{skip}, \{l \mapsto 2\} \rangle
\end{array}
\]

Notiamo quindi che le due strategie di valutazione hanno prodotto risultati diversi.
\subsection{Call-by-value: small-step semantics}
\begin{grammar}
    <Valori v> ::= $b$ | $n$ | $\texttt{skip}$ | $\texttt{fn } x:T \Rightarrow e$
\end{grammar}
\begin{prooftree}
    \AxiomC{$\langle e_1, s \rangle \rightarrow \langle e_1', s'\rangle$ }
    \LeftLabel{(CBV-app1)}
    \UnaryInfC{$\langle e_1 \ e_2, s \rangle \rightarrow \langle e_1' \ e_2, s'\rangle$}
\end{prooftree}
\begin{prooftree}
    \AxiomC{$\langle e_2, s \rangle \rightarrow \langle e_2', s'\rangle$ }
    \LeftLabel{(CBV-app2)}
    \UnaryInfC{$\langle v_1 \ e_2, s \rangle \rightarrow \langle v_1 \ e_2', s'\rangle$}
\end{prooftree}
\begin{prooftree}
    \AxiomC{$-$ }
    \LeftLabel{(CBV-fn)}
    \UnaryInfC{$\langle (\texttt{fn } x:T \Rightarrow e) v, s 
    \rangle \rightarrow \langle e\{v/x\}, s\rangle$}
\end{prooftree}
La valutazione della funzione non tocca lo store, infatti in un linguaggio funzionale
puro non avremmo bisogno di uno store. In $e\{v/x\}$ il valore $v$ verrebbe
copiato tante volte quante sono le occorrenze libere di $x$ in $e$. Le implementazioni
reali non fanno questo.
\subsection{Call-by-name: small-step semantics}
\begin{prooftree}
    \AxiomC{$\langle e_1, s \rangle \rightarrow \langle e_1', s'\rangle$ }
    \LeftLabel{(CBN-app)}
    \UnaryInfC{$\langle e_1 \ e_2, s \rangle \rightarrow \langle e_1' \ e_2, s'\rangle$}
\end{prooftree}
\begin{prooftree}
    \AxiomC{$-$ }
    \LeftLabel{(CBN-fn)}
    \UnaryInfC{$\langle (\texttt{fn } x:T \Rightarrow e) e_2, s 
    \rangle \rightarrow \langle e\{e_2/x\}, s\rangle$}
\end{prooftree}
Non valutiamo l'argomento della funzione, ma lo sostituiamo direttamente nel corpo, in questo 
modo se tale argomento non verrà utilizzato non verrà mai valutato, ma se verrà utilizzato
più volte verrà valutato ogni volta che viene utilizzato.
\section{Tipizzazione delle funzioni}
Fino ad ora, un ambiente di tipi $\Gamma$ fornisce il tipo delle locazioni dello store.
Da adesso in poi, deve anche fornire assunzioni sul tipo delle variabili usate nelle
funzioni, ad esempio 
\[
    \Gamma = \{l_1 : \text{int ref}, x : \text{int}, y : \text{bool} \rightarrow \text{int}\}  
\]
Quindi, estendiamo il set \texttt{TypeEnv} degli ambienti dei tipi come segue:
\[
  \texttt{TypeEnv} \defeq \mathbb{L} \cup \mathbb{X} \rightharpoonup 
  T_{loc} \cup T
\]
Tale che:
\begin{itemize}
    \item $\forall l \in \texttt{dom}(\Gamma). \Gamma(l) \in T_{loc}$
    \item $\forall x \in \texttt{dom}(\Gamma). \Gamma(x) \in T$
\end{itemize}
\textbf{Notazioni}: se $x \notin \text{dom}(\Gamma)$, scriviamo $\Gamma, x : T$ per
la funzione parziale che mappa $x$ su $T$ ma altrimenti è come $\Gamma$.

Si noti che con l'introduzione delle funzioni ci sono più configurazioni
bloccate (ad esempio $2$, $\texttt{true}$, $\texttt{true} \texttt{ fn }x:T \Rightarrow e$, ecc.).
Il nostro sistema di tipi respingerà queste configurazioni tramite le seguenti regole:

\begin{prooftree}
    \AxiomC{$-$}
    \LeftLabel{(var)}
    \RightLabel{\textit{se }$\Gamma(x) = T$}
    \UnaryInfC{$\Gamma \vdash x : T$}
\end{prooftree}



\begin{prooftree}
    \AxiomC{$\Gamma, x : T \vdash e : T'$}
    \LeftLabel{(fn)}
    \UnaryInfC{$\Gamma \vdash \texttt{fn }x:T \Rightarrow e : T \rightarrow T'$}
\end{prooftree}



\begin{prooftree}
    \AxiomC{$\Gamma \vdash e_1 : T \rightarrow T'$}
    \AxiomC{$\Gamma \vdash e_2 : T$}
    \LeftLabel{(app)}
    \BinaryInfC{$\Gamma \vdash e_1 \ e_2 : T'$}
\end{prooftree}
\subsection{Proprietà del sistema di tipi}
Consideriamo solo esecuzioni di programmi chiusi, senza variabili libere.

\begin{theorem}[Progress]
Se $e$ è chiuso e $\Gamma \vdash e : T$ e $\text{dom}(\Gamma)
\subseteq \text{dom}(s)$, allora o c'è un valore o esiste $e'$, $s'$
tale che $\langle e,s \rangle \rightarrow \langle e',s' \rangle$.
\end{theorem}

\begin{theorem}[Conservazione del tipo]
Se $e$ è chiuso e $\Gamma \vdash e : T$ e $\text{dom}(\Gamma)
\subseteq \text{dom}(s)$ e $\langle e,s \rangle \rightarrow
\langle e',s' \rangle$ allora $\Gamma \vdash e' : T$ e $e'$ è
chiuso e $\text{dom}(\Gamma) \subseteq \text{dom}(s')$.
\end{theorem}

Questo richiede il seguente lemma:

\begin{lemma}[Sostituzione]
Se $\Gamma \vdash e : T$ e $\Gamma,x : T \vdash e' : T'$ con
$x \notin \text{dom}(\Gamma)$ allora $\Gamma \vdash e'\{e/x\} : T'$.
\end{lemma}

\begin{theorem}[Normalizzazione]
Nei sotto-linguaggi senza cicli while o operazioni di store,
se $\Gamma \vdash e : T$ e $e$ è chiuso, allora c'è un valore $v$
tale che, per ogni store, $\langle e,s \rangle \rightarrow^* \langle
v,s \rangle$. In altre parole, se consideriamo un linguaggio funzionale
puro, come il calcolo lambda, la sua versione tipizzata non è più Turing-completa.
\end{theorem}
\section{Dichiarazioni locali}
\subsection{Sintassi e tipi}
\subsection{Intuizione}
\subsection{Variabili legate e libere}
\subsection{Alpha conversion}
\subsection{Small-step semantics}
\section{Ricorsione}
\subsection{Punto fisso}
\section{Punto fisso nel linguaggio funzionale}