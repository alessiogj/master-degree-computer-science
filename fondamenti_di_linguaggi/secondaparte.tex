\chapter{Semantiche}
Vicina alle operazioni, in qualche modo specifica il comportamento in base 
ad una macchia astratta. Effetto sugli stati della macchina, utile 
in caso di implementazione di interpreti, utile per dimostrare 
equivalenza di programmi, saranno equivalenti se la loro semantica 
operazionale produce gli stessi effetti. (\textit{sequenza di passi})

C'è una forma più astratta che è la semantica denotazionale, che
che modella il comportamento di un programma attraverso una funzione, 
non guardando la storia del calcolo. Si basa sui calcoli di punto fisso.

Semantica assiomatica, assomiglia alle deduzioni logiche, 
quindi si basa su proprietà logiche sugli stati.
Insiemi di input che rappresentano le precondizioni,
insiemi di output che rappresentano le postcondizioni.

Definiamo un semplice linguaggio \texttt{imp}, che ha una serie di numeri 
$N$ interi e rappresentiamo delle metavariabili $m,n$ che rappresentano
numeri interi, $T=\{\texttt{int}, \texttt{bool}\}$,con la metavariabile $t$, 
locazioni che sono rappresentate da $\texttt{Loc}$, rappresentate dalle
dalle metavariabili $x,y$
e le espressioni $e$.