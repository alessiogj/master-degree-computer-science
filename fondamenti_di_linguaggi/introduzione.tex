\chapter{Introduzione}
\section{Linguaggi di programmazione}
Un linguaggio di programmazione è un linguaggio formale che specifica un
insieme di istruzioni che possono essere usate per produrre un insieme di
output.
Esso è definito da:
\begin{itemize}
    \item \textbf{Sintassi}: specifica la forma delle istruzioni. Ci permette di
    capire quali stringhe sono ammissibili e quali no mediante diversi strumenti come 
    grammatiche, analizzatori lessicali e sintattici, teoria degli automi.
    \item \textbf{Pragmatica}: specifica l'effetto delle istruzioni. Ci permette
    di capire le ragioni per introdurre un nuovo linguaggio e di programmazione 
    invece di utilizzarne uno già esistente.
    \item \textbf{Semantica}: specifica il significato dei programmi scritti nel linguaggio, ovvero il loro 
    comportamento a tempo di esecuzione. Ci permette di capire se due programmi 
    apparentemente diversi sono equivalenti.
\end{itemize}
\subsection{Benefici di una semantica formale}
I benefici dei linguaggi di programmazione diversi, tra cui:
\begin{itemize}
    \item \textbf{Implementazione}: Consente di fornire la specifica (\textit{del comportamento}) 
    dei programmi indipendentemente dalla macchina o dal compilatore utilizzato.
    \item \textbf{Verifica}: una semantica formale consente di ragionare 
    sui programmi e sulle loro proprietà di correttezza.
    \item \textbf{Progettazione di Linguaggio}: spesso una semantica formale consente di 
    scoprire ambiguità all'interno di linguaggi già esistenti. Questo aiuta a progettare 
    nuovi linguaggi in maniera più accurata.
\end{itemize}
\section{Un linguaggio per le espressioni aritmetiche: sintassi}
Definiamo il seguente linguaggio:
\[
    \mathcal{E}\quad ::= \quad n \quad | \quad \mathcal{E} + \mathcal{E} \quad | 
    \quad \mathcal{E} * \mathcal{E} \quad | \quad \dots
\]
dove:
\begin{itemize}
    \item $n$ è lo spazio del dominio dei numerali.
    \item $\mathcal{E}$ è il range del dominio delle espressioni aritmetiche.
    \item $+, x, \dots$ sono simboli del linguaggio.
\end{itemize}
I numerali sono parte della sintassi del nostro linguaggio e non vanno confusi con i numeri
che sono oggetti matematici.
Ciò potrebbe significare che nel nostro linguaggio al posto di $0, 1, \dots$ avremmo 
potuto usare $zero, uno, \dots$ e sarebbero potuti essere uguali.

Nel nostro caso assumiamo che esista una corrispondenza ovvia tra il simbolo ``numerale" (n)
e il numero naturale n. Questo è fatto solo per semplificare la spiegazione. In un
altro contesto, il simbolo ``numeral" 3 potrebbe essere associato al numero 42!

\section{Semantica Operazionale}

La semantica operazionale ha l'obiettivo di valutare un'espressione aritmetica
del linguaggio per ottenere il suo valore numerico associato. Questo può essere
fatto in due modi differenti:

\begin{itemize}
  \item \textbf{Semantica Small-Step (\textit{o strutturale})}: Fornisce un metodo per
  valutare un'espressione passo dopo passo, considerando le azioni intermedie.
  Questo approccio fornisce una valutazione dettagliata dell'espressione.

  \item \textbf{Semantica Big-Step (\textit{o naturale})}: Ignora i passaggi intermedi
  e fornisce direttamente il risultato finale della valutazione dell'espressione. Questo
  approccio semplifica la valutazione, concentrando l'attenzione sul risultato finale.

\end{itemize}
\subsection{Big-Step Semantics}

\begin{tcolorbox}[title = {Valutazione}]  
    $E \Downarrow n$
\end{tcolorbox}
\textbf{Significato}: La valutazione dell'espressione $\mathcal{E}$ produce il numerale $n$.

\begin{tcolorbox}[title = {Assiomi e regole di inferenza}]  
\begin{figure}[H]
    \begin{subfigure}{0.3\textwidth}
    \begin{prooftree}
        \AxiomC{$-$}
        \LeftLabel{(B-Num)}
        \UnaryInfC{$n \Downarrow n$}
    \end{prooftree}
    \end{subfigure}%
    \begin{subfigure}{0.7\textwidth}
    \begin{prooftree}
        \AxiomC{$\mathcal{E}_1 \Downarrow n_1$}
        \AxiomC{$\mathcal{E}_2 \Downarrow n_2$}
        \LeftLabel{(B-Add)}
        \RightLabel{$n_3 = add(n_1, n_2)$}
        \BinaryInfC{$\mathcal{E}_1 + \mathcal{E}_2 \Downarrow n_3$}
    \end{prooftree}
    \end{subfigure}
\end{figure}
\end{tcolorbox}
\textbf{Significato}: 
\begin{itemize}
\item (B-Num): Questo è un assioma che afferma che quando valutiamo un singolo
numero $n$, otteniamo lo stesso numero $n$ come risultato. Questo è il caso
base della valutazione.

\item (B-Add): Questa regola di inferenza afferma che date due espressioni
$\mathcal{E}_1$ e $\mathcal{E}_2$:
\begin{itemize}
  \item Se è il caso che $\mathcal{E}_1 \Downarrow n_1$ (cioè $\mathcal{E}_1$ si valuta a $n_1$) e
  \item È anche il caso che $\mathcal{E}_2 \Downarrow n_2$ (cioè $\mathcal{E}_2$ si valuta a $n_2$),
  allora segue che $\mathcal{E}_1 + \mathcal{E}_2 \Downarrow n_3$, dove $n_3$ è il numerale associato
  al numero $n_3$ tale che $n_3 = add(n_1, n_2)$.
  Si noti che in questa regola, $E1$, $E2$, $n1$, $n2$, $n3$ sono meta-variabili.
\end{itemize}
\end{itemize}
Questa regola (B-Add) ci dice come valutare un'addizione tra due espressioni
$\mathcal{E}_1$ e $\mathcal{E}_2$ nel contesto della semantica big-step. La
regola stabilisce che se possiamo valutare entrambe le espressioni operandi
($\mathcal{E}_1$ e $\mathcal{E}_2$) e otteniamo i numeri $n_1$ e $n_2$ rispettivamente,
allora possiamo calcolare la somma di $\mathcal{E}_1$ e $\mathcal{E}_2$ come $n_3$,
dove $n_3$ è il risultato della somma dei numeri $n_1$ e $n_2$.
Si noti che la funzione di addizione $add$ opera sui numeri, non sui numerali.
\subsection{Small-Step Semantics}

\begin{tcolorbox}[title = {Valutazione}]  
$\mathcal{E}_1 \rightarrow \mathcal{E}_2$

\end{tcolorbox}
\textbf{Significato:} 
Dopo aver eseguito un passo di valutazione su $\mathcal{E}_1$, l'espressione $\mathcal{E}_2$ rimane da valutare.
\begin{tcolorbox}[title = {Assiomi e regole di inferenza}]  
\begin{prooftree}
    \AxiomC{$\mathcal{E}_1 \rightarrow \mathcal{E}_1'$}
    \LeftLabel{(S-Left)}
    \UnaryInfC{$\mathcal{E}_1 + \mathcal{E}_2 \rightarrow \mathcal{E}_1' + \mathcal{E}_2$}
    \end{prooftree}
    
    \begin{prooftree}
    \AxiomC{$\mathcal{E}_2 \rightarrow \mathcal{E}_2'$}
    \LeftLabel{(S-N.Right)}
    \UnaryInfC{$n_1 + \mathcal{E}_2 \rightarrow n_1 + \mathcal{E}_2'$}
    \end{prooftree}
    
    \begin{prooftree}
    \AxiomC{-}
    \LeftLabel{(S-Add)}
    \RightLabel{$n_3 = add(n_1, n_2)$}
    \UnaryInfC{$n_1 + n_2 \rightarrow n_3$}
    \RightLabel{(S-Add)}
\end{prooftree}
\end{tcolorbox}
Fissiamo l'ordine di valutazione da sinistra a destra. Qualcosa di 
simile non è possibile nella big-step semantics, dove le espressioni sono 
valutate in un solo passo.
\subsubsection{La scelta dell'ordine di valutazione}
\begin{tcolorbox}[title = {Assiomi e regole di inferenza}]  
    \begin{prooftree}
        \AxiomC{$\mathcal{E}_1 \rightarrow_{ch} \mathcal{E}_1'$}
        \LeftLabel{(S-Left)}
        \UnaryInfC{$\mathcal{E}_1 + \mathcal{E}_2 \rightarrow_{ch} \mathcal{E}_1' + \mathcal{E}_2$}
        \end{prooftree}
        
        \begin{prooftree}
        \AxiomC{$\mathcal{E}_2 \rightarrow_{ch} \mathcal{E}_2'$}
        \LeftLabel{(S-Right)}
        \UnaryInfC{$\mathcal{E}_1 + \mathcal{E}_2 \rightarrow_{ch} \mathcal{E}_1 + \mathcal{E}_2'$}
        \end{prooftree}
        
        \begin{prooftree}
        \AxiomC{-}
        \LeftLabel{(S-Add)}
        \RightLabel{$n_3 = add(n_1, n_2)$}
        \UnaryInfC{$n_1 + n_2 \rightarrow_{ch} n_3$}
        \RightLabel{(S-Add)}
    \end{prooftree}
\end{tcolorbox}
In questo caso non abbiamo precedenza stabilita per la valutazione delle espressioni.
Regole simili possono essere applicate anche con gli altri operatori.
\subsubsection{Esecuzione della small-step semantics}
La relazione $\rightarrow^k$, per $k \in \mathbb{N}$ è definita per un numero di passi 
di valutazione definito da $k$.
Mentre la relazione $\rightarrow^*$ è definita per un numero non definito di passi di valutazione.