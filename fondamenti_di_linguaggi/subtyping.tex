\chapter{Subtyping}
Il subtyping è un concetto che rende facile la vita del programmatore, 
facendo si che possa riutilizzare il codice. Prende una particolare valenza 
nella programmazione ad oggetti, rendendo meno rigido il type system, consentendo 
di compilare programmi che a runtime non darebbero problemi nonostante non siano 
tipabili.
\section{Polimorfismo}
\begin{itemize}
    \item \textbf{Polimorfismo ad-hoc}:
    si tratta del classico overloading, data per esempio 
    dalla possibilità di sommare due interi o due float.
    \item \textbf{Polimorfismo parametrico}: per esempio, una funzione 
    \item che prende in input un argo,mento di tipo $\texttt{list }\alpha$ 
    e la computa per una parametrica lunghezza. 
    \item \textbf{Polimorfismo di subtyping}: un tipo può essere
    usato al posto di un altro, e noi ci focalizzeremo su questo.
\end{itemize}
Ridefiniamo le regole di subtyping per il nostro linguaggio funzionale:
\begin{prooftree}
    \AxiomC{$\Gamma ,x, T \vdash e : T'$}
    \LeftLabel{(fun)}
    \UnaryInfC{$\Gamma \vdash \lambda x.e : T \rightarrow T'$}
\end{prooftree}
\begin{prooftree}
    \AxiomC{$\Gamma \vdash e_1 : T \rightarrow T'$}
    \AxiomC{$\Gamma \vdash e_2 : T$}
    \LeftLabel{(app)}
    \BinaryInfC{$\Gamma \vdash e_1 e_2 : T'$}
\end{prooftree}
Il type system presentato introduce una funzione rappresentata come:

\[
\Gamma \vdash (\texttt{fn }x : \texttt{left : int} \implies \#
\texttt{left }x): \{ \texttt{left : int} \} \rightarrow \texttt{int}
\]

Questa funzione accetta un argomento \(x\) con un campo \texttt{left}
di tipo \texttt{int} e restituisce il valore associato a \texttt{left},
dichiarato come \(\# \texttt{left }x\), producendo un risultato di tipo
\texttt{int}. Tuttavia, è evidente che non è possibile applicare la
stessa funzione a un argomento \(x\) che ha un tipo diverso da
\(\{ \texttt{left : int} \}\).

Nel secondo esempio, si cerca di applicare la stessa funzione a un
argomento di tipo 
\[\{ \texttt{left = 3$, $right = 5} \}\]
che è
chiaramente incompatibile con il tipo richiesto dalla funzione.
Questo viene espresso come:

\[
\Gamma \not \vdash (\texttt{fn }x : \texttt{left : int} \implies
\# \texttt{left }x): \{ \texttt{left = 3, right = 5} \}
\]

Per gestire casi in cui si desidera consentire l'utilizzo di
tipi più specifici in contesti che richiedono tipi più generali,
viene introdotto il concetto di subtyping. La relazione di subtyping
\(<:\) viene definita in modo che un tipo possa essere considerato
come un sottotipo di un altro. Nell'esempio fornito, il
tipo \(\{ \texttt{left : int, right : int} \}\) è un sottotipo di
\(\{ \texttt{left : int} \}\), che a sua volta è un sottotipo di
\(\{\}\).

\[
\{ \texttt{left : int, right : int} \} <: \{ \texttt{left : int} \}
<: \{\}
\]

Tuttavia, l'introduzione di subtyping può portare alla perdita
dell'unicità dei tipi. La regola di subtyping (\texttt{sub}) è
aggiunta al sistema di tipi per consentire la sostituzione di
un tipo con uno dei suoi sottotipi, ma ciò può portare a ambiguità
nelle assegnazioni di tipo, poiché un'espressione potrebbe avere più
tipi validi.

In breve, mentre il subtyping offre maggiore flessibilità nell'uso
dei tipi, è importante considerare l'effetto sulla chiarezza e
sull'unicità dei tipi nel contesto del sistema specifico che si sta
definendo.
\begin{prooftree}
    \AxiomC{$\Gamma \vdash e : T$}
    \AxiomC{$T<:T'$}
    \LeftLabel{(sub)}
    \BinaryInfC{$\Gamma \vdash e : T$}
\end{prooftree}

\subsection{Il sottotipo relazione $T <: T'$}
Si tratta di una relazione riflessiva e transitiva.
\begin{prooftree}
    \AxiomC{$-$}
    \LeftLabel{(s-refl)}
    \UnaryInfC{$T <: T$}
\end{prooftree}
\begin{prooftree}
    \AxiomC{$T<:T'$}
    \AxiomC{$T'<:T''$}
    \LeftLabel{(s-trans)}
    \BinaryInfC{$T<:T''$}
\end{prooftree}
In accordo con il riordino dei campi:
\begin{prooftree}
    \AxiomC{$\pi$ una permutazione di $1,2,\dots,k$}
    \LeftLabel{(rec-perm)}
    \UnaryInfC{$\{p_1 : T_1', \dots, p_n : T_k\}
    <: \{p_{\pi(1)}: T_{\pi(1)},\dots, p_{\pi(k)}: T_{\pi(k)}\}$}
\end{prooftree}

La relazione di subtyping sui record è antisimmetrica: un preordine
e non un ordine parziale. 
Dimenticandoci dei campi a destra otteniamo:
\begin{prooftree}
    \AxiomC{$-$}
    \LeftLabel{(rec-width)}
    \UnaryInfC{$\{p_1 : T_1, \dots, p_k : T_k, p_{k+1}: T_{k+1}, \dots, p_z : T_z\}
    <: \{p_1 : T_1, \dots, p_k : T_k\}$}
\end{prooftree}

Quindi se volessimo riordinare prima, possiamo dimenticarci dei
campi e in accordo con il riordino dei campi otteniamo:
\begin{prooftree}
    \AxiomC{$T_1 <: T_1' \dots T_k <: T_k'$}
    \LeftLabel{(rec-perm)}
    \UnaryInfC{$\{p_1 : T_1, \dots, p_k : T_k \} <: \{p_1 : T'_1, \dots, p_k : T'_k\}$}
\end{prooftree}

Si dice infatti che il subtyping sui record è \textbf{covariante}.

Per quanto riguarda il sistema di tipi precedentemente discusso,
l'introduzione del sottotipaggio sui record aggiunge ulteriore
flessibilità nel trattamento dei tipi di dati strutturati.
\section{Subtyping e funzioni}
Contesto in cui si aspetta un certo algoritmo. Potrei avere una funzione 
che si presta a funzionare in quel contesto, come tipo argomento vuole di meno 
rispetto a quello che gli passo. Il subtyping è un modo per dire che una funzione
che si aspetta un tipo, può essere usata con un tipo più generale.

La regola di sottotipo è la seguente:
\begin{prooftree}
    \AxiomC{$T_1 :> T'_1$}
    \AxiomC{$T_2 <: T'_2$}
    \LeftLabel{(fun-sub)}
    \BinaryInfC{$T_1 \rightarrow T_2 <: T'_1 \rightarrow T'_2$}
\end{prooftree}
Diremo che il sottotipaggio sulle funzioni è \textbf{controvariante} a 
sinistra della freccia e \textbf{covariante} a destra della freccia.

Così, se $f: T_1 \rightarrow T_2$ allora possiamo usare $f$ in ogni contesto 
dove forniamo ad $f$ un qualsiasi argomento di tipo $T'_1$
dove $T_1 <: T'_1$, e usiamo il risultato di $f$ in ogni contesto dove
ci aspettiamo un risultato di tipo $T'_2$ dove $T_2 <: T'_2$.

Per istanza, definiamo $f$ in un costruttore \texttt{let}:
\[
    \texttt{let } f : T = \texttt{fn } x :
    \{ p : \texttt{int} \} \rightarrow 
    \{ a = \# p\, x , b = 28\} \texttt{ in } e
\]
allora quando tipiamo $e$ dobbiamo usare il tipe environment $\Gamma$ 
tale che:
\[
    \Gamma (f) = T = \{ p : \texttt{int} \} \rightarrow
    \{ a : \texttt{int}, b : \texttt{int} \}
\]
Dalle regole di sottotipaggio usiamo $f$ in $e$ tale che abbia uno 
dei seguenti tipi:
\[
\begin{array}{lll}
    \{ p : \texttt{int} \} & \rightarrow & \{ a : \texttt{int} \}\\
    \{ p : \texttt{int}, q: \texttt{int} \} & \rightarrow & \{ a : \texttt{int}, b : \texttt{int} \}\\
    \{ p : \texttt{int}, q: \texttt{int}\} & \rightarrow & \{ a : \texttt{int}\}\\
\end{array}
\]
perché:
\[
  \begin{array}{lll}
    \{ p : \texttt{int} \} & <: & \{ p : \texttt{int}, q: \texttt{int} \}\\
    \{ a : \texttt{int}, b : \texttt{int} \} & <: & \{ a : \texttt{int} \}\\
  \end{array}  
\]

D'altro canto, nel programma:
\[
    \texttt{let } f : \hat{T} = \texttt{fn } x :
    \{ p : \texttt{int}, q: \texttt{int} \} \rightarrow 
    \{ a = (\# p\,x) + (\# q\,x)\} \texttt{ in } e
\]
quando tipiamo $e$ dobbiamo usare il tipe environment $\Gamma$ tale che:
\[
  \Gamma(f) = \hat{T} = \{ p : \texttt{int}, q: \texttt{int} \}
  \rightarrow \{ a : \texttt{int} \}  
\]
Dalle regole di sottotipaggio usiamo $f$ in $e$ tale che abbiamo come 
tipo:
\[
    \{ p : \texttt{int}, q: \texttt{int} \} \rightarrow
    \{ a : \texttt{int} \}
\]
Tuttavia, non possiamo usare $f$ in $e$ con tipi:
\[
  \begin{array}{lll}
    \{ p : \texttt{int} \} & \rightarrow & T \quad \textit{per ogni } \Gamma \textit{ e } T\\
    T & \rightarrow & \{ a : \texttt{int} \} \quad \textit{per ogni } \Gamma \textit{ e } T\\
  \end{array}  
\]
