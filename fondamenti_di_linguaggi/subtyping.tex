\chapter{Subtyping}
Il subtyping è un concetto che rende facile la vita del programmatore, 
facendo si che possa riutilizzare il codice. Prende una particolare valenza 
nella programmazione ad oggetti. Rende meno rigido il type system, consentendo 
di compilare programmi che a runtime non darebbero problemi nonostante non siano 
tipabili.
\section{Polimorfismo}
\begin{itemize}
    \item \textbf{Polimorfismo ad-hoc}:
    opratori opverloaded, che possono lavorare con tipi diversi.
    \item \textbf{Polimorfismo parametrico}: tipi polimorfi 
    che possono essere istanziati con tipi diversi.
    \item \textbf{Polimorfismo di subtyping}: un tipo può essere
    usato al posto di un altro. CI focalizzeremo su questo.
\end{itemize}
Motivazioni:
\begin{prooftree}
    \AxiomC{$\Gamma ,x, T \vdash e : T'$}
    \LeftLabel{(fun)}
    \UnaryInfC{$\Gamma \vdash \lambda x.e : T \rightarrow T'$}
\end{prooftree}
\begin{prooftree}
    \AxiomC{$\Gamma \vdash e_1 : T \rightarrow T'$}
    \AxiomC{$\Gamma \vdash e_2 : T$}
    \LeftLabel{(app)}
    \BinaryInfC{$\Gamma \vdash e_1 e_2 : T'$}
\end{prooftree}
Prendendo in considerazione le nostre regole:
\[
\Gamma \vdash (\texttt{fn }x : \texttt{left : int} \implies 
\# \texttt{left }x): \textcolor{red}{\texttt{\{left : int\} }}
\textcolor{red}{\rightarrow \texttt{int}}
\]
In qualche modo introduciamo una relazione d'ordine tra i tipi,
in modo che un tipo possa essere usato al posto di un altro.
Quindi:
\[
  \{\texttt{left : int, right : int}\} <: \{left : int\} <: \{\}  
\]
Il sottotipo relazione $<:$ è introdotto per introdurre per introdurre la regola 
di subtyping:
\begin{prooftree}
    \AxiomC{$\Gamma \vdash e : T$}
    \AxiomC{$T <: T'$}
    \LeftLabel{(sub)}
    \BinaryInfC{$\Gamma \vdash e : T'$}
\end{prooftree}
In questo modo però non abbiamo più \textbf{l'unicità dei tipi}
\subsection{Relazione $T <: T'$}
Si tratta di una relazione riflessiva e transitiva.
\begin{prooftree}
    \AxiomC{}
    \LeftLabel{(s-refl)}
    \UnaryInfC{$T <: T$}
\end{prooftree}
\begin{prooftree}
    \AxiomC{$T<:T'$}
    \AxiomC{$T'<:T''$}
    \LeftLabel{(s-trans)}
    \BinaryInfC{$T<:T''$}
\end{prooftree}
\subsection{Subsection sui record}
\begin{prooftree}
    \AxiomC{$\pi \textit{ una permutazione di }1,2,\dots,k$}
    \LeftLabel{(s-rec)}
    \UnaryInfC{$\Gamma \vdash e : \{l_1 : T_1', \dots, l_n : T_k\}
    <: \{p_{\pi(1)}: T_{\pi(1)},\dots, p_{\pi(k)}: T_{\pi(k)}\}$}
\end{prooftree}
La relazione di subtyping sui record non è simmetica e non è transitiva: un 
preordine e non un ordine parziale:

\dots
\section{Subtyping e funzioni}
Contesto in cui si aspetta un certo algoritmo. Potrei avere una funzione 
che si presta a funzionare in quel contesto, come tipo argomento vuole di meno 
rispetto a quello che gli passo. Il subtyping è un modo per dire che una funzione
che si aspetta un tipo, può essere usata con un tipo più generale.