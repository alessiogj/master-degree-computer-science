\chapter{Induzione}
\section{Induzione come principio di dimostrazione}
L'induzione è un principio di dimostrazione che consente di dimostrare proprietà su insiemi infiniti di elementi.
Questo principio si basa su passi finiti di calcolo, che su basano su insiemi che hanno una struttura ben definita.
Questa struttura ci aiuta a ridurre il problema complesso in una serie di passaggi più semplici.

Esistono diversi tipi di induzione tra cui:

\begin{itemize}
  \item Induzione matematica: Utilizzata per dimostrare affermazioni sui numeri naturali.
  Si basa sulla dimostrazione di una proprietà per un valore iniziale e sulla dimostrazione che,
  se la proprietà è vera per un certo numero, lo è anche per il numero successivo.

  \item Induzione strutturale: Utilizzata per dimostrare affermazioni su strutture ricorsive come
  alberi. La dimostrazione inizia dimostrando la proprietà per il caso base (\textit{ad esempio, un albero vuoto})
  e successivamente dimostra che se la proprietà è vera per le componenti strutturali, lo è anche per la struttura complessiva.

  \item Induzione delle regole: Spesso utilizzata per dimostrare proprietà delle regole in un sistema formale.
  La dimostrazione inizia dimostrando la proprietà per ciascuna regola e successivamente dimostra che la proprietà
  si conserva quando si applicano le regole in sequenza.
\end{itemize}

\subsection{I numeri naturali $\mathbb{N}$}
I numeri naturali sono costruiti seguendo due regole fondamentali:

\begin{itemize}
  \item \textbf{Regola di base}: Il numero $0$ appartiene all'insieme dei
  numeri naturali, indicato come $0 \in \mathbb{N}$.

  \item \textbf{Passo induttivo}: Se un numero $k$ è un
  numero naturale, allora il successore di $k$, ovvero $k+1$,
  è anch'esso un numero naturale.
\end{itemize}

\subsubsection{Esempio}
Per definire una funzione $f: \mathbb{N} \rightarrow X$, è necessario seguire due passi:

\begin{itemize}
  \item \textbf{Regola di base}: Si descrive il risultato di $f$ per il valore iniziale, ovvero $0$.

  \item \textbf{Passo induttivo}: Si assume che $f$ sia definita per un valore $k$, e si
  descrive il risultato di $f$ per $k+1$ in termini di $f(k)$. Questo approccio di definizione
  ricorsiva è spesso utilizzato nella programmazione, in particolare nell'ambito del ``pattern matching".
\end{itemize}

\subsubsection{Esempio}
Nel contesto della semantica ``small step" nel caso deterministico,
possiamo definire una funzione $\texttt{red}$ come segue:

\[
\texttt{red}: \texttt{Exp} \times \mathbb{N} \rightarrow \text{Exp}
\]

\begin{itemize}
  \item \textbf{Regola di base}: $\texttt{red}(E,0) = E$ per ogni espressione $E$.

  \item \textbf{Passo induttivo}: $\texttt{red}(E,k+1) = E''$ se esiste un'espressione
  $E'$ tale che $\text{red}(E,k) = E'$ e $E' \rightarrow E''$. Questa definizione permette
  di rappresentare l'espressione ottenuta riducendo $E$ per $k$ passi.

\end{itemize}

La dimostrazione per induzione è un metodo formale per dimostrare affermazioni matematiche
o logicamente corrette su insiemi infiniti, come i numeri naturali. L'obiettivo è dimostrare
una proprietà $P$ per un numero arbitrario $k$ e dimostrare che la proprietà è vera anche per
il successivo $k+1$:

\[
\forall k \in \mathbb{N}. P(k) \Rightarrow P(k+1)
\]

Nel processo di dimostrazione per induzione, è essenziale dimostrare che la proprietà è vera per il caso base (spesso $k = 0$) e che il passaggio all'elemento successivo è valido. Questo garantisce che la proprietà sia valida per tutti i numeri naturali.
\section{Induzione matematica}
Il metodo più semplice di induzione è l'induzione matematica, che è un tipo di induzione basato sui
numeri naturali. Il principio può essere descritto come segue. Dato un'affermazione $P(-)$ sui numeri naturali,
vogliamo dimostrare che $P(n)$ è vera per ogni numero naturale $n$:

\begin{itemize}
\item \textbf{Caso base}: dimostrare che $P(0)$ è vera (\textit{utilizzando alcuni
fatti matematici noti}).

\item \textbf{Caso induttivo}: assumere l'ipotesi induttiva, cioè che $P(k)$ è vera.
A partire dall'ipotesi induttiva, dimostrare che segue $P(k + 1)$ (\textit{utilizzando
alcuni fatti matematici noti}).
\end{itemize}
Se $(a)$ e $(b)$ vengono stabiliti, allora $P(n)$ è vera per ogni numero naturale $n$.

L'induzione matematica è un principio valido perché ogni numero naturale può essere ``costruito"
a partire da $0$ come punto di partenza e usando l'operazione di aggiunta di uno per creare nuovi numeri.
\section{Induzione strutturale}
L'induzione strutturale è un principio di dimostrazione che consente di dimostrare proprietà su elementi
di un insieme costruito induttivamente. Questo tipo di induzione è particolarmente utile quando si tratta
di oggetti con una struttura ricorsiva.

Un concetto importante nell'ambito dell'induzione strutturale è l'isomorfismo. Due oggetti si dicono
isomorfi se esiste una funzione biunivoca tra di essi, cioè una funzione che stabilisce una corrispondenza
uno a uno tra gli elementi degli oggetti.

Un esempio comune di passaggio dall'induzione matematica a quella strutturale coinvolge la seguente grammatica:

\begin{grammar}
  <N> ::= \texttt{zero} | \texttt{SUCC}(N)
\end{grammar}

Nell'induzione strutturale, possiamo dimostrare proprietà sugli elementi di questa grammatica in questo modo:

\begin{itemize}
  \item Caso base: Dimostrare che la proprietà è vera per \texttt{zero}.
  \item Passo induttivo: Assumere che la proprietà sia vera per un generico elemento \texttt{N} e dimostrare che è
  vera anche per \texttt{SUCC}(N).
\end{itemize}

Questo principio ci consente di affrontare dimostrazioni relative a strutture ricorsive in modo sistematico e rigoroso.

\subsubsection{Somma (\textit{sum})}
Il principio di definizione delle funzioni per induzione può essere applicato a questa rappresentazione dei numeri naturali allo stesso modo di prima. Vediamo un esempi:

\begin{itemize}
   \item Regola di base: $sum(zero) = zero$
   \item Regola induttiva: $sum(succ(N)) = succ^{(n+1)}(sum(N))$, dove $N = succ(...succ(zero))$ per un certo numero naturale $n$.
\end{itemize}

Ciò significa che il caso base è definito per $zero$, e nel caso induttivo, applichiamo la funzione $sum$ in modo ricorsivo a $succ(N)$ aggiungendo $succ$ ripetutamente $n+1$ volte.
\subsection{Induzione strutturale per i numeri naturali}
Il principio di induzione afferma che per dimostrare $P(N)$ per tutti i numeri N, è sufficiente fare due cose:
\begin{itemize}
  \item \textbf{Caso base}: Dimostrare che $P(zero)$ è vero.
  \item \textbf{Caso induttivo}: Dimostrare che $P(succ(K)$) segue dall'assunzione che $P(K)$ sia vero per un certo numero $K$.
\end{itemize}
È importante notare che nel tentativo di dimostrare $P(succ(K))$, l'ipotesi induttiva ci dice che possiamo assumere che
la proprietà sia valida per la \textbf{sottostruttura} di $succ(K)$, ovvero possiamo supporre che $P(K)$ sia vero.

Questa prospettiva strutturale, e la forma associata di induzione, chiamata induzione strutturale, è ampiamente applicabile.

\subsection{Albero binario attraverso l'induzione strutturale}
\begin{figure}[H]
  \centering
  \begin{forest}
    for tree={circle, draw, minimum size=1.5em, inner sep=0, s sep=20pt}
    [
      [
        [, fill=orange]
        [, fill=orange]
      ]
      [
        [, fill=orange]
        [, fill=orange]
      ]
    ]
  \end{forest}
\end{figure}
\subsubsection{Definizione induttiva}
\begin{itemize}
  \item \textbf{Caso base}: \texttt{leaf} è un albero binario.
  
  \begin{figure}[H]
    \centering
    \begin{forest}
      for tree={circle, draw, minimum size=1.5em, inner sep=0, s sep=20pt}
        [, fill=orange]
    \end{forest}
  \end{figure}
  \item \textbf{Passo induttivo}: se $L$ e $R$ sono alberi binari allora \texttt{node}($L$, $R$) è un albero binario.
  
  \begin{figure}[H]
    \centering
    \begin{forest}
      for tree={circle, draw, minimum size=1.5em, inner sep=0, s sep=20pt}
      [
        [L]
        [R]
      ]
      \end{forest}
  \end{figure}
\end{itemize}
\subsection{Regole di costruzione}
\begin{grammar}
  <$T \in \texttt{bTree}$> ::= \texttt{leaf} | \texttt{node}(T, T)
\end{grammar}
\begin{itemize}
  \item \textbf{Caso base}: \texttt{leaf} è un albero binario.
  \item \textbf{Passo induttivo}: se $T_1$ e $T_2$ sono alberi binari, allora \texttt{node}($T_1$, $T_2$) è un albero binario.
\end{itemize}

\subsubsection{Definizione induttiva}
\[
f: \texttt{bTree} \rightarrow X  
\]

Per definire una funzione $f$ che mappa alberi binari a elementi di un insieme $X$, possiamo seguire il principio della definizione induttiva:

\begin{itemize}
  \item \textbf{Regola di base:} Descriviamo il risultato dell'applicazione di $f$ a una foglia terminale, ad esempio $f(\texttt{leaf})$.

  \item \textbf{Regola induttiva:} Supponiamo che $f(T1)$ e $f(T2)$ siano già definiti. Ora, descriviamo il risultato dell'applicazione di
  $f$ all'albero binario $\texttt{node}(T1, T2)$.
\end{itemize}

Con queste regole, possiamo definire la funzione $f$ per ogni possibile albero binario, indipendentemente dalla sua complessità. Ogni passo nella definizione
si basa sui passi precedenti, garantendo che la funzione sia ben definita per tutti gli alberi binari.

\subsection{Dimostrazioni strutturali}
\subsubsection{Normalizzazione della big step semantics}
\begin{tcolorbox}[title = {Normalizzazione della big step semantics}]
Per ogni espressione aritmetica $E$ esiste un numero naturale $k$
t.c. $E \downarrow k$.

\begin{grammar}
  <$E \in \texttt{expr}$> ::= $n$ | $E_1 + E_2$
\end{grammar}
Abbiamo a disposizione due regole di costruzione:

\begin{minipage}{0.40\textwidth}
  \centering
  \begin{prooftree}
    \AxiomC{$-$}
    \RightLabel{(B-num)}
    \UnaryInfC{$n \downarrow n$}
  \end{prooftree}
\end{minipage}
\hfill
\begin{minipage}{0.60\textwidth}
  \centering
  \begin{prooftree}
    \AxiomC{$E_1 \downarrow n_1$}
    \AxiomC{$E_2 \downarrow n_2$}
    \RightLabel{$n_3 = \texttt{add}(n_1,n_2)$}
    \BinaryInfC{$E_1 + E_2 \downarrow n_3$}
  \end{prooftree}
\end{minipage}
\[
\texttt{Propr}: \forall E \in \texttt{expr} \exists k \quad numerale \quad \text{ t.c. } E \downarrow k  
\]
\end{tcolorbox}
\begin{proof}
  Per induzione sulla struttura di $E$.
  \begin{itemize}
    \item \textbf{Caso base}: Sia $E$ un non terminale $n$, $E\equiv n$, allora il
    $k$ in questione è proprio $n$. Per la regola (\texttt{B-num}), $n \downarrow n$.
    \begin{prooftree}
      \AxiomC{$-$}
      \RightLabel{(B-num)}
      \UnaryInfC{$n \downarrow n=k$}
    \end{prooftree}
    \item \textbf{Passo induttivo}: $E$ è nella forma $E_1 + E_2$ per qualche $E_1$ e $E_2$ (\textit{sottostrutture di $E$}).
    Per ipotesi induttiva,
    essendo $E_1$ e $E_2$ sottostrutture di $E$, vale che:
    \begin{itemize}
      \item $\exists k_1 \quad numerale \quad \text{ t.c. } E_1 \downarrow k_1$
      \item $\exists k_2 \quad numerale \quad \text{ t.c. } E_2 \downarrow k_2$
    \end{itemize}
    Usando la regola (\texttt{B-add}) e scegliendo $k_3 = \texttt{add}(k_1, k_2)$, otteniamo che:
      \begin{prooftree}
        \AxiomC{$E_1 \downarrow k_1$}
        \AxiomC{$E_2 \downarrow k_2$}
        \RightLabel{$k_3 = \texttt{add}(k_1,k_2)$}
        \BinaryInfC{$E \downarrow k_3$}
      \end{prooftree}
      Ho quindi trovato il $k$ t.c. $E \downarrow k$ che cercavo.
  \end{itemize}
\end{proof}
\subsubsection{Small step semantics}
\begin{tcolorbox}[title = {$E \rightarrow F$ implica $E \rightarrow_{CH} F$ per ogni espressione aritmetica}]
  Se $E \downarrow n$ e $E \downarrow m$, allora $n \equiv m$.
\end{tcolorbox}
\begin{proof}
  Per induzione sulla struttura di $E$.
  \begin{itemize}
    \item \textbf{Caso base}: Sia $E \equiv k$ per qualche $k \in \mathbb{N}$, allora sia 
    $E \downarrow n$ che $E \downarrow m$ sono derivabili solo per la regola (\texttt{B-num}), dove $m=k=n$.

      \begin{prooftree}
        \AxiomC{$-$}
        \RightLabel{(B-num)}
        \UnaryInfC{$k \downarrow k$}
      \end{prooftree}

    
    Ma allora $n \equiv m$.

    \begin{minipage}{0.45\textwidth}
      \centering
      \begin{prooftree}
        \AxiomC{$-$}
        \RightLabel{(B-num)}
        \UnaryInfC{$m \downarrow k$}
      \end{prooftree}
    \end{minipage}
    \hfill
    \begin{minipage}{0.45\textwidth}
      \centering
      \begin{prooftree}
        \AxiomC{$-$}
        \RightLabel{(B-num)}
        \UnaryInfC{$n \downarrow k$}
      \end{prooftree}
    \end{minipage}
    \item \textbf{Passo induttivo}: $E$ è nella forma $E_1 + E_2$.
    \begin{itemize}
      \item Ne consegue che $E \downarrow m$ è stato derivato usando la regola (\texttt{B-add}).
      Esistendo $m_1 + m_2 = m$, per ipotesi induttiva, $E_1 \downarrow m_1$ e $E_2 \downarrow m_2$.

      \begin{prooftree}
        \AxiomC{$E_1 \downarrow m_1$}
        \AxiomC{$E_2 \downarrow m_2$}
        \RightLabel{$m = \texttt{add}(m_1,m_2)$}
        \BinaryInfC{$E \downarrow m$}
      \end{prooftree}
      \item Ne consegue che $E \downarrow n$ è stato derivato usando la regola (\texttt{B-add}).
      Esistendo $n_1 + n_2 = n$, per ipotesi induttiva, $E_1 \downarrow n_1$ e $E_2 \downarrow n_2$.

      \begin{prooftree}
        \AxiomC{$E_1 \downarrow n_1$}
        \AxiomC{$E_2 \downarrow n_2$}
        \RightLabel{$n = \texttt{add}(n_1,n_2)$}
        \BinaryInfC{$E \downarrow n$}
      \end{prooftree}
    \end{itemize}
  \end{itemize}
\end{proof}
\subsubsection{Determinismo forte per la small-step semantics}
\begin{tcolorbox}[title = {Determinismo forte per la small-step semantics}]
  Se $E \rightarrow F$ e $E \rightarrow G$, allora $F \equiv G$.
\end{tcolorbox}
\begin{proof}
  Per induzione sulla struttura di $E$.
  \begin{itemize}
    \item \textbf{Caso base}: $E \equiv n$ per qualche $n \in \mathbb{N}$. Non 
    esiste alcuna $E$ e $F$ t.c. $E \rightarrow F$ e $E \rightarrow G$. Quindi 
    la conclusione  è falsa. 
    \item \textbf{Passo induttivo}: $E$ è nella forma $E_1 + E_2$.
    Sia $E \rightarrow F$ per qualche $F$.
    \begin{itemize}
      \item Ho usato la regola (\texttt{S-left}), allora $\exists E_1'$ t.c. 
      $E_1 \rightarrow E_1'$ e $E = E_1 + E_2 \rightarrow E_1' + E_2$.
      Avendo considerato $E \rightarrow G$ e avendo utilizzato la regola 
      (\texttt{S-left}), dove $E_1 \rightarrow \text{\^E}_1$ e 
      $E = E_1 + E_2 \rightarrow \text{\^E}_1 + E_2$, perché $E_1 \rightarrow
      \text{\^E}_1$ per ipotesi induttiva su $E_1$, $\text{\^E}_1 \equiv E_1'$, 
      quindi $F=G$.
      \item Ho usato la regola (\texttt{S-right}), allora $E = m + E_2$ e
      $E_2'$ t.c $E_2 \rightarrow E_2'$ e $E = m + E_2 \rightarrow m + E_2'$.
      Avendo considerato $E \rightarrow G$ e avendo utilizzato la regola 
      (\texttt{S-right}), dove $E_2 \rightarrow \text{\^E}_2$ e 
      $E = m + E_2 \rightarrow m + \text{\^E}_2$, perché $E_2 \rightarrow
      \text{\^E}_2$ per ipotesi induttiva su $E_2$, $\text{\^E}_2 \equiv E_2'$,
      quindi $F=G$.
      \item Ho usato la regola (\texttt{S-add}), allora $E = E_1 + E_2 = m_1 + m_2$ 
      per qualche $m_1, m_2$ con $F=m_3 = \texttt{add}(m_1, m_2)$, 
      inoltre $E \rightarrow G$, ma avendo utilizzato la regola (\texttt{S-add}),
      ne consegue che $F=G=m_3$.
      
    \end{itemize}
  \end{itemize}
\end{proof}
\subsubsection{Determinismo debole per la small-step semantics}
\begin{tcolorbox}[title = {Determinismo forte per la small-step semantics}]
  Se $E \rightarrow^* m$ e $E \rightarrow^* n$, allora $F \equiv G$.
\end{tcolorbox}
Per dimostrare il determinismo debole, si utilizza la dimostrazione per induzione
strutturale basata sulla dimostrazione del determinismo forte.
\section{Induzione delle regole}
Il comportamento delle espressioni aritmetiche $E$ è completamente definito 
dalle regole dei suoi componenti. Per questa ragione l'induzione strutturale è sufficientemente 
potente per dimostrare proprietà per differenti semantiche di \texttt{Exp}.
Però, in linguaggi più complessi, con operatori di controllo ricorsivi o induttivi, 
abbiamo bisogno di strumenti più sofisticati. L'idea di base della \textbf{induzione delle regole}
è di ignorare la struttura degli oggetti e concentrarsi nella \textbf{dimensione delle derivazioni} 
\subsection{Dimensione delle derivazioni}
Per esempio, consideriamo le seguenti regole, che definiscono una relazione binaria sui 
numeri naturali:

\begin{minipage}{0.5\textwidth}
  \begin{prooftree}
    \AxiomC{$-$}
    \LeftLabel{(Ax)}
    \UnaryInfC{$n \texttt{ Div }0$}
  \end{prooftree}
\end{minipage}
\begin{minipage}{0.5\textwidth}
  \begin{prooftree}
    \AxiomC{$n \texttt{ Div } m$}
    \LeftLabel{(Plus)}
    \UnaryInfC{$n \texttt{ Div }(m + n)$}
  \end{prooftree}
\end{minipage}

Derivazioni:

\begin{minipage}{0.5\textwidth}
  \begin{prooftree}
    \AxiomC{$-$}
    \LeftLabel{(Ax)}
    \UnaryInfC{$7 \text{ Div } 0$}
    \LeftLabel{(Plus)}
    \UnaryInfC{$7 \text{ Div } 7$}
    \LeftLabel{(Plus)}
    \UnaryInfC{$7 \text{ Div } 14$}
    \LeftLabel{(Plus)}
    \UnaryInfC{$7 \text{ Div } 21$}
\end{prooftree}
\end{minipage}
\begin{minipage}{0.5\textwidth}
  \begin{prooftree}
    \AxiomC{$-$}
    \LeftLabel{(Ax)}
    \UnaryInfC{$2 \text{ Div } 0$}
    \LeftLabel{(Plus)}
    \UnaryInfC{$2 \text{ Div } 2$}
    \LeftLabel{(Plus)}
    \UnaryInfC{$2 \text{ Div } 4$}
\end{prooftree}
\end{minipage}

Notimo che la derivazione a sinistra è più lunga di quella a destra, infatti \texttt{2 Div 4} è più
piccolo di \texttt{7 Div 21}.

Supponiamo di voler dimostrare una dichiarazione della forma:
\[
  n \texttt{ Div } m \rightarrow P(n,m)
\]
Dove possiamo utilizzare l'induzione sulla dimensione della derivazione di $n \texttt{ Div } m$.
Supponiamo quindi che $n \texttt{ Div } m$, dove $m = n \cdot k$ per qualche $k \in \mathbb{N}$.
Ciò significa effettivamente che le regole \texttt{(Ax)} e \texttt{(Plus)} catturano correttamente
il concetto di divisione. 
Quindi, dimostriamo che $n \, \text{Div} \, m \rightarrow P(n, m)$ per

induzione matematica sulla dimensione della derivazione della valutazione $n \, \text{Div} \, m$
dalle regole \texttt{(Ax)} e \texttt{(Plus)}.

Ciò significa effettivamente che le regole  \texttt{(Ax)} e \texttt{(Plus)} catturano correttamente
il concetto di divisione. Quindi, supponiamo di avere una derivazione di $n \, \text{Div} \, m$.
Utilizzare l'induzione matematica significa avere un'ipotesi induttiva che afferma che $P(k_1, k_2)$
è vera per qualsiasi $k_1$, $k_2$ per i quali esiste una derivazione $k_1 \, \text{Div} \, k_2$ la cui
dimensione è inferiore alla dimensione della derivazione di $n \, \text{Div} \, m$.

Quindi $n \texttt{ Div } m$ può essere derivato solo da \texttt{(Ax)} o \texttt{(Plus)}.

Abbiamo due possibilità:
\begin{itemize}
  \item Abbiamo l'applicazione di un assioma \texttt{(Ax)}: 
  \begin{prooftree}
    \AxiomC{$-$}
    \LeftLabel{(Ax)}
    \UnaryInfC{$n \texttt{ Div }0$}
  \end{prooftree}
  Solo se $m = 0$, quindi $P(n, 0)$ è vera, ma $k$ deve essere $0$.
  \item Abbiamo l'applicazione di una regola \texttt{(Plus)}:
    \begin{prooftree}
      \AxiomC{...}
      \LeftLabel{(...)}
      \UnaryInfC{$n \texttt{ Div } m_1$}
      \LeftLabel{(Plus)}
      \UnaryInfC{$n \texttt{ Div }(m_1 + n)$}
    \end{prooftree}
    Dove $m = m_1 + n$.
\end{itemize}
Però questo significa che la valutazione di $n \texttt{ Div } m$ ha anche una derivazione mediante regole.
La dimensione della derivazione è minore della dimensione della derivazione di $n \texttt{ Div } m$.
Quindi per ipotesi induttiva, sappiamo che esiste un $k_1$ tale che $m_1 = n \cdot k_1$.

Ora, $P(n, m)$ è una immediata conseguenza di $m = n \cdot (K_1 + 1)$.
\section{Definizione formale di induzione delle regole}
Per dimostrare una proprietà $P(D)$ per ogni derivazione $D$, è sufficiente fare quanto segue:

\begin{enumerate}
    \item[(a)] \textbf{Caso base:} dimostra che $P(A)$ è vera per ogni assioma
    $A$ (utilizzando fatti matematici noti).
    \item[(b)] \textbf{Caso induttivo:} per ogni regola della forma 
    
    \begin{prooftree}
      \AxiomC{$h_1, \dots, h_n$}
      \LeftLabel{(Rule)}
      \UnaryInfC{$c$}
    \end{prooftree}

    dimostra che ogni derivazione che termina con
    l'uso di questa regola soddisfa la proprietà. Tale derivazione ha sottoderivazioni
    $D_1, \dots, D_n$ con conclusioni $h_1, \dots, h_n$. Per ipotesi induttiva,
    assumiamo che $P(D_i)$ valga per ogni sotto-derivazione $D_i$, $1 \leq i \leq n$.
\end{enumerate}
\begin{theorem}[Progress]
  Se $\Gamma \vdash e : T$ e $dom(\Gamma) \subseteq dom(s)$ allora $e$ è
  un valore o esistono $e'$, $s'$ tali che $\langle e, s \rangle \rightarrow \langle e', s' \rangle$.
\end{theorem}
%\begin{proof}
%  Supponiamo che $\phi$ sia una relazione ternaria definita come segue:
%  \[
%    \phi (\Gamma, e, T) \stackrel{\text{def}}{=} \forall s. dom(\Gamma) \subseteq dom(s) \implies 
%    valore(e) \lor (\exists e', s'. \langle e, s \rangle \rightarrow \langle e', s' \rangle)
%  \]
%  Lo dimostriamo per ogni:
%  \[
%    \Gamma, e, T, \texttt{ if } \Gamma \vdash e : T \texttt{ then } \phi(\Gamma, e, T)
%  \]
%  Per dimostrare per induzione sulle regole il perché $\Gamma \vdash e : T$,
%  ciò significa che facciamo un'analisi dei casi sulla ``ultima" regola di
%  tipizzazione applicata per derivare $\Gamma \vdash e : T$. Ci sono
%  \begin{itemize}
%    \item[(a)] 4 casi base: gli assiomi (int), (bool), (deref) e (skip);
%    \item[(b)] 6 casi induttivi: le regole $(\texttt{op}\, +)$ e $(\texttt{op}\, \geq)$
%    , (\texttt{if}), (\texttt{assign}), (\texttt{seq}) e (\texttt{while}).
%  \end{itemize}
%\end{proof}
\subsubsection{Example}
Come esempio, dimostriamo che se $\Gamma \supseteq \{I, \texttt{intref}\}$, allora 
$\Gamma \vdash (!l + 2) + 3 : \texttt{int} \implies \phi (\Gamma, (!l + 2) + 3, \texttt{int})$.
\begin{prooftree}
  \AxiomC{$-$}
  \LeftLabel{(\texttt{deref})}
  \UnaryInfC{$\Gamma \vdash !l : \texttt{int}$}
  \AxiomC{$-$}
  \LeftLabel{(\texttt{int})}
  \UnaryInfC{$\Gamma \vdash 2 : \texttt{int}$}
  \LeftLabel{(\texttt{op} +)}
  \BinaryInfC{$\Gamma \vdash (!l + 2) : \texttt{int}$}
  \AxiomC{$-$}
  \LeftLabel{(\texttt{int})}
  \UnaryInfC{$\Gamma \vdash 3 : \texttt{int}$}
  \LeftLabel{(\texttt{op} +)}
  \BinaryInfC{$\Gamma \vdash (!l + 2) + 3 : \texttt{int}$}
\end{prooftree}

\subsection{Preservazione del tipo}
\begin{lemma}
  Se $\langle e, s \rangle \rightarrow \langle e', s' \rangle$ allora $\text{dom}(s) = \text{dom}(s')$.
\end{lemma}
\begin{proof}
  Per induzione sulle regole su perché $\langle e, s \rangle \rightarrow \langle e', s' \rangle$. Sia $\Phi(e, s, e', s') = (\text{dom}(s) = \text{dom}(s'))$. Tutte le regole sono utilizzi immediati dell'ipotesi induttiva, tranne la regola (assign1), per la quale notiamo che se $l \in \text{dom}(s)$ allora $\text{dom}(s[l \mapsto n]) = \text{dom}(s)$. $\Box$
\end{proof}

\begin{theorem}
  [Conservazione del tipo] Se $\Gamma \vdash e : T$ e $\text{dom}(\Gamma) \subseteq \text{dom}(s)$ e
  $\langle e, s \rangle \rightarrow \langle e', s' \rangle$ allora $\Gamma \vdash e' : T$ e
  $\text{dom}(\Gamma) \subseteq \text{dom}(s')$.
\end{theorem}
