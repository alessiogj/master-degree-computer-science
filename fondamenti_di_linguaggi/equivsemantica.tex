\chapter{Equivalenza semantica}
La semantica formale di un linguaggio di programmazione ci autorizza 
a ragionare su proprietà di programmi, basati sul loro linguaggio di 
programmazione.
\begin{tcolorbox}[title= Intuizione]
    Due programmi $P_1$ e $P_2$ si dicono \textbf{semanticamente
    equivalenti}, $P_1 \simeq P_2$, se entrambi possono essere
    rimpiazzati l'uno con l'altro in qualsiasi contesto senza
    alterare il comportamento del programma.
\end{tcolorbox}
Con una buona equivalenza semantica possiamo:
\begin{itemize}
    \item si possa capire cosa fa un programma;
    \item dimostrare se una particolare espressione (\textit{ad esempio,
    un algoritmo efficiente}) è equivalente ad un'altra; questa operazione 
    è detta \textbf{verifica di programmi};
    \item dimostrare che le ottimizzazioni di un compilatore
    non alterano il comportamento del programma (\textit{soundsness});
    \item capire le differenze semantiche tra programmi.
\end{itemize}
\subsubsection{Esempio}
Possiamo dire che:
\[
  (l:=0;4) \simeq (l:=1; 3+!l) ?
\]
I due frammenti producono lo stesso risultato a partire da uno stato iniziale.
Il fatto è che non possiamo rimpiazzare arbitrariamente 
i due frammenti di codice in ogni contesto, ma solo in alcuni. 
Vediamo perché:
\[
    C[\cdot] \eqdef C[\cdot] + !l
\]
allora:
\[
    C(l:=0;4) \not\simeq C[l:=1; 3+!l]
\]
\[
    (l:=0;4) + !l \not\simeq (l:=1; 3+!l) + !l
\]
Di fatto, $C[l:=0;4]$ ritorna $4$ mentre $C[l:=1; 3+!l]$ ritorna $5$.
\section{Il significato di buona equivalenza semantica}
Cosa significa per $\simeq$ essere ``buona''? 
\begin{itemize}
    \item I programmi che risultano in valori osservabilmente diversi (\textit{partento 
    da un certo store iniziale}) non devono essere equivalenti.
    \[
        \exists s, s_1, s_2, v_1, v_2 \quad . \quad \langle e, s \rangle 
        \rightarrow^* \langle v_1, s_1 \rangle \quad \land \quad 
        \langle e_2, s \rangle \rightarrow^*
        \langle v_2, s_2 \rangle \quad \land \quad v_1 \neq v_2 \implies
        e_1 \not \simeq e_2
    \]
    \item I programmi che terminano non devono essere equivalenti a quelli che non
    terminano.
    \item La relazione $\simeq$ deve essere una relazione di equivalenza:
    \[
        e \simeq e \qquad e_1 \simeq e_2 \implies e_2 \simeq e_1 \qquad 
        e_1 \simeq e_2 \simeq e_3 \implies e_1 \simeq e_3
    \]
    \item La relazione $\simeq$ deve essere una congruenza, cioè preservata 
    \item dai contesti del programma: se $e_1 \simeq e_2$
    allora per ogni contesto $C[\cdot]$ si ha che $C[e_1] \simeq C[e_2]$.
    \item La relazione $\simeq$ deve essere la più grande possibile, cioè
    mettere in relazione il maggior numero di programmi possibile.
\end{itemize}
\section{Il contesto di un programma}
Il contesto di un programma, ovvero $C[\cdot]$, è un programma non 
completamente definito. In gergo, si dice che $C[\cdot]$ denota un 
programma con un ``buco'' $[\cdot]$ che ha bisogno di essere riempito 
con un qualche programma $P$.
Scrivendo $C[P]$ denotiamo un qualche programma ottenuto riempiendo il
codice mancate in $C[\cdot]$ con il codice di $P$.

Ad esempio, nel linguaggio While, il contesto di un programma è definito
dalla seguente grammatica:
\begin{grammar}
    <$C[\cdot] \in \texttt{Cxt}$> ::= $[\cdot]$ | $C[\cdot] \, \textit{op} \, e_2$ |
    $e_1 \, \textit{op} \, C[\cdot]$ | $l:= C[\cdot]$ \alt
    \texttt{if} $C[\cdot]$ \texttt{then} $e_2$ \texttt{else} $e_3$ |
    \texttt{if} $e_1$ \texttt{then} $C[\cdot]$ \texttt{else} $e_3$ \alt
    \texttt{if} $e_1$ \texttt{then} $e_2$ \texttt{else} $C[\cdot]$ |
    $C[\cdot]$;e_2 | e_1;$C[\cdot]$ \alt 
    \texttt{while} $C[\cdot]$ \texttt{do} e_2
    | \texttt{while} $e_1$ \texttt{do} $C[\cdot]$
\end{grammar}
Per esempio, se $C[\cdot]$ è nel contesto $\texttt{while} \, !l = 0 \,
\texttt{do}\, [\cdot]$, allora $C[l:=!l + 1]$ è 
$\texttt{while} \, !l = 0 \, \texttt{do}\, l:=!l + 1$.
\subsection{Congruenza rispetto ai contesti}
È fondamentale che l'equivalenza di programma sia una congruenza rispetto ai
contesti. Ciò significa che, se \( e_1 \simeq e_2 \), allora \( C[e_1] \simeq
C[e_2] \) per ogni contesto \( C[\cdot] \). Immaginiamo di avere un ampio programma,
\texttt{Sys}, che controlla un vasto sistema e include un sottoprogramma \( P \).
Potremmo esprimere \texttt{Sys} come \( \texttt{Sys} \defeq C[P] \), per un contesto
appropriato \( C[\cdot] \), e ipotizziamo che ci venga chiesto di sviluppare una
versione ottimizzata di \( P \), chiamata \texttt{Pfast}. Come possiamo assicurarci
che il comportamento dell'intero programma resti inalterato sostituendo il
sottoprogramma \( P \) con \texttt{Pfast}? Dovremmo verificare che \(
C[\texttt{Pfast}] \simeq C[P] \), ma i due sistemi potrebbero essere molto
estesi. Tuttavia, se l'uguaglianza è una \textbf{congruenza}, allora è
sufficiente dimostrare che i due sottoprogrammi sono equivalenti, ovvero che
\( \texttt{Pfast} \simeq P \). Di conseguenza, l'uguaglianza dei sistemi
interi, cioè \( C[\texttt{Pfast}] \simeq C[P] \), ne consegue automaticamente.
\subsection{Equivalenza semantica basata sulle tracce per il linguaggio While}
Consideriamo il nostro linguaggio While tipato, senza funzioni:
\begin{tcolorbox}[title = Equivalenza delle tracce $\simeq_\Gamma^T$]
    Definiamo \( e_1 \simeq^T_\Gamma e_2 \) se e solo se, per tutti gli store \( s \)
    tali che \( \texttt{dom}(\Gamma) \subseteq \texttt{dom}(s) \), abbiamo
    \( \Gamma \vdash e_1 : T \), \( \Gamma \vdash e_2 : T \), e
    \begin{itemize}
        \item $\langle e1, s \rangle_* \langle v, s' \rangle \implies \langle e2, s \rangle_* \langle v, s' \rangle$,
        \item $\langle e2, s \rangle_* \langle v, s' \rangle \implies \langle e1, s \rangle_* \langle v, s' \rangle$.
    \end{itemize}
\end{tcolorbox}
\begin{tcolorbox}[title = Proprietà di congruenza]
    La relazione di equivalenza \( \simeq_T^\Gamma \) gode della proprietà di
    congruenza perché, ogniqualvolta \( e_1 \simeq_T^\Gamma e_2 \), abbiamo che,
    per tutti i contesti \( C \) e i tipi \( T' \), se \( \Gamma \vdash C[e_1] :
    T' \) e \( \Gamma \vdash C[e_2] : T' \), allora \( C[e_1] \simeq_{T'}^\Gamma
    C[e_2] \).
\end{tcolorbox}
\subsection{Equivalenza delle tracce su $\simeq_\Gamma^T$}
Dato $e_1 \simeq_\Gamma^T e_2$ allora:
\begin{itemize}
    \item Se una delle due configurazioni diverge da uno store $s$ allora 
    l'altra configurazione deve divergere con lo stesso store $s$.
    \item Dato uno store $s$ se una delle due configurazioni converge, allora 
    deve essere nello stesso valore $v$ e nello stesso store $s$.
\end{itemize}
Supponiamo che dato uno store $s$, le due configurazioni $\langle e_1, s \rangle$ e
$\langle e_2, s \rangle$ convergano, rispettivamente a $\langle v_1, s_1 \rangle$ e
$\langle v_2, s_2 \rangle$, con $s_1(l) \neq s_2(l)$ per qualche
$l$ e $v$ di tipo $T$. Allora il contesto distintivo potrebbe essere il seguente:
\begin{itemize}
    \item Se \( T = \texttt{unit} \) allora definiamo \( C[\cdot] \defeq [\cdot];
    \text{!l} \)
    \item Se \( T = \texttt{bool} \) allora definiamo \( C[\cdot] \defeq \texttt{if}
    \; [\cdot] \; \text{\texttt{then } !l \texttt{ else } !l} \)
    \item Se \( T = \texttt{int} \) allora definiamo \( C[\cdot] \defeq {l_1}
    := [\cdot]; \texttt{!l} \)
\end{itemize}
dove \( \langle C[e_1], s \rangle \rightarrow_* \langle v_1, s'_1 \rangle \) e \( \langle
    C[e_2], s \rangle \rightarrow_* \langle v_2, s'_2 \rangle \), con \( v_1 \neq v_2 \).

In queste definizioni, \( C[\cdot] \) rappresenta un contesto distintivo che
varia in base al tipo di \( T \), e le espressioni \( e1 \) ed \( e2 \) sono
inserite in questo contesto per testare la loro equivalenza.

\subsubsection{Esempi}
\begin{itemize}
    \item $2 + 2 \simeq_\Gamma^{\text{int}} 4$ per ogni contesto $\Gamma$.
    \item $(l:=0;4) \not \simeq_\Gamma^{\text{int}} (l:=1;3+!l)$ per ogni contesto $\Gamma$.
    \item $(l:=!l+1);(l:=!l-1) \simeq_\Gamma^{\text{int}} (l:=!l)$ per ogni contesto $\Gamma
    \supseteq \{l : \texttt{intref}\}$.
    \item $(l:=!l+1);(k:=!j + 1) \simeq_\Gamma^{\text{int}} (k:=!j + 1);(l:=!l+1)$ per
    ogni contesto $\Gamma \supseteq \{l : \texttt{intref}, j : \texttt{intref}\}$.
\end{itemize}
\section{Regole generali}
\begin{tcolorbox}[title = Associatività del $;$]
    \[
        e_1; (e_2; e_3) \simeq (e_1; e_2); e_3    
    \]
    Per ogni $\Gamma, T, e_1, e_2$ e $e_3$ tale che $\Gamma \vdash e_1 : \texttt{unit}$,
    $\Gamma \vdash e_2 : \texttt{unit}$ e $\Gamma \vdash e_3 : T$.
\end{tcolorbox}
\begin{tcolorbox}[title = rimozione dello \texttt{skip}]
    \begin{itemize}
        \item $e_2 \simeq_{\Gamma_2}^{T} \texttt{skip}; e_1$
        \item $e_1; \texttt{skip} \simeq_{\Gamma_1}^{T} e_1$
    \end{itemize}
    Per ogni $\Gamma_1, \Gamma_2, T, e_1$ e $e_2$ tali che
    $\Gamma_2 \vdash e_2 : T$,
    $\Gamma_1 \vdash e_1 : \texttt{unit}$.
\end{tcolorbox}
\begin{tcolorbox}[title = \texttt{if true}]
    \[
      \texttt{if true then } e_1 \texttt{ else } e_2 \simeq_\Gamma^T e_1  
    \]
    Per ogni $\Gamma, T, e_1$ e $e_2$ tali che $\Gamma \vdash e_1 : T$ e
    $\Gamma \vdash e_2 : T$.
\end{tcolorbox}
\begin{tcolorbox}[title = \texttt{if false}]
    \[
      \texttt{if false then } e_1 \texttt{ else } e_2 \simeq_\Gamma^T e_2  
    \]
    Per ogni $\Gamma, T, e_1$ e $e_2$ tali che $\Gamma \vdash e_1 : T$ e
    $\Gamma \vdash e_2 : T$.
\end{tcolorbox}
\begin{tcolorbox}[title = Distributività dell'\texttt{if} rispetto all'$;$]
    \[
        (\texttt{if } e_1 \texttt{ then } e_2 \texttt{ else } e_3); e \simeq_\Gamma^T
        (\texttt{if } e_1 \texttt{ then } e_2; e \texttt{ else } e_3; e)
    \]
    per ogni $\Gamma, T, e_1, e_2$ e $e_3$ tali che $\Gamma \vdash e_1 : \texttt{bool}$,
    $\Gamma \vdash e_2 : \texttt{unit}$, $\Gamma \vdash e_3 : \texttt{unit}$ e
    $\Gamma \vdash e : T$.
\end{tcolorbox}
\begin{tcolorbox}[title = Distributività dell'$;$ rispetto all'\texttt{if}]
    \[
        e; (\texttt{if } e_1 \texttt{ then } e_2 \texttt{ else } e_3) \simeq_\Gamma^T
        (\texttt{if } e_1 \texttt{ then } e; e_2 \texttt{ else } e; e_3)
    \]
    per ogni $\Gamma, T, e_1, e_2$ e $e_3$ tali che $\Gamma \vdash e_1 : \texttt{bool}$,
    $\Gamma \vdash e_2 : \texttt{unit}$, $\Gamma \vdash e_3 : \texttt{unit}$ e
    $\Gamma \vdash e : T$.
\end{tcolorbox}
\subsection{Regole errate}
Sappiamo che:
\[
  (e;\texttt{if }e_1\texttt{ then }e_2\texttt{ else }e_3) \not \simeq_\Gamma^T
  (\texttt{if }e_1\texttt{ then }e;e_2\texttt{ else }e;e_3)  
\]
Per esempio, consideriamo che:
\begin{itemize}
    \item $e$ sia $l := 1$
    \item $e_1$ sia $!l = 0$
    \item $e_2$ sia $\texttt{skip}$
    \item $e_3$ sia $\texttt{while true do skip}$
\end{itemize}
Allora, in qualsiasi store $s$, dove la locazione $l$ è associata a $0$, 
l'espressione a sinistra diverge, mentre quella a destra converge.
\section{Approccio alla simulazione}
\begin{tcolorbox}[title = Simulazione]
    Diciamo che $e_1$ è simulato da $e_2$, ovvero $e_1 \sqsubseteq e_2$,
    se e solo se:
    \begin{itemize}
        \item $\Gamma \vdash e_1 : T$ e $\Gamma \vdash e_2 : T$
        per qualche $\Gamma$ e $T$.
        \item Per ogni store $s$ con $\texttt{dom}(\Gamma) \subseteq \texttt{dom}(s)$,
        se $\langle e_1, s \rangle \rightarrow^* \langle e_1', s_1' \rangle$ allora
        esiste $e_2'$ tale che $\langle e_2, s \rangle
        \rightarrow^* \langle e_2', s_2' \rangle$, con $e_1' \sqsubseteq_\Gamma^T e_2'$
        e $s_1' = s_2'$.
    \end{itemize}
\end{tcolorbox}
\begin{tcolorbox}[title = Bisimulazione]
    Diciamo che $e_1$ è bisimile da $e_2$, ovvero $e_1 \approx e_2$ se e solo se:
    \begin{itemize}
        \item $\Gamma \vdash e_1 : T$ e $\Gamma \vdash e_2 : T$
        per qualche $\Gamma$ e $T$.
        \item Per ogni store $s$ con $\texttt{dom}(\Gamma) \subseteq \texttt{dom}(s)$,
        se $\langle e_1, s \rangle \rightarrow^* \langle e_1', s_1' \rangle$ allora
        esiste $e_2'$ tale che $\langle e_2, s \rangle
        \rightarrow^* \langle e_2', s_2' \rangle$, con $e_1' \approx_\Gamma^T e_2'$
        e $s_1' = s_2'$.
        \item Per ogni store $s$ con $\texttt{dom}(\Gamma) \subseteq \texttt{dom}(s)$,
        se $\langle e_2, s \rangle \rightarrow^* \langle e_2', s_2' \rangle$ allora
        esiste $e_1'$ tale che $\langle e_1, s \rangle \rightarrow^* \langle e_1', s_1' \rangle$,
        con $e_1' \approx_\Gamma^T e_2'$ e $s_1' = s_2'$.
    \end{itemize}
\end{tcolorbox}