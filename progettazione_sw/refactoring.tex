\chapter{Refactoring}
\section{Introduzione}
\begin{tcolorbox}[title = Refactoring]
    Il codice sorgente, soprattutto con approcci incrementali, tende a 
    subire cambiamento nel tempo. Questi cambiamenti possono degradare 
    la struttura interna del software, rendendolo difficile da
    comprendere e mantenere. Il refactoring è un processo di
    trasformazione del codice sorgente che non altera il comportamento
    esterno del software ma migliora la sua struttura interna.
\end{tcolorbox}
Si tratta di una manutenzione preventiva, la qualità del codice è normale 
che degradi. Limitando la normale decadenza del codice, migliorare  
la leggibilità (\textit{non muore una volta scritto, bisogna 
continuare a modificarlo}), non migliorando è più facile inserire 
bug. 

Il refactoring è diverso dal reingegnering. Il reingegnering è un
processo che mira a riprogettare il software che è stato scritto
per architetture datate o tecnologie obsolete (\textit{soprattutto 
cambio di architetture, come per esempio da monolitiche e cloud}).
Il refactoring
invece è un processo che mira a migliorare la struttura interna
e va in parallelo con lo sviluppo del software.

\begin{tcolorbox}[title = Debito tecnico]
    I debiti tecnici sono tutti i problemi di comprensione del codice 
    che sono disposto ad avere per qualcosa che serve attualmente, 
    ma probabilmente causerà in futuro.
\end{tcolorbox}
Per ridurre il debito tecnico si può fare refactoring. Ci sono solamente 
in cui è opportuno fare refactoring, ad esempio quando si deve iniziare 
a sviluppare nuove funzionalità, in modo che sia facile nel codice recepire 
questa nuova funzionalità. Oppure quando si deve correggere un bug, o 
in caso di identificazione di \textit{code smell}. 
\subsection{Code smell}
Sensazione che qualcosa nel codice non va, e che gli sviluppatori
esperti sanno che è un problema. Gli sviluppatori esperti hanno 
una sensazione di disagio quando vedono un code smell, lo sviluppatore 
non avrebbe fatto così.
\subsubsection{Code smell comuni}
Duplicazione di codice, quando avviene si utilizza una funzione
per rimuovere la duplicazione. Preventivamente vengono evitati 
problemi, riducendo il costo di manutenzione futuro.

I metodi e le funzioni troppo lunghi, sono difficili da capire, 
e ci si accorge quando si tende a commentare il codice ad ogni step. In 
questo caso è utile spezzare il codice in più segmenti, già individuati 
dai commenti.

Spesso lo switch case rappresenta un code smell, soprattutto quando
si utilizza in un linguaggi ad oggetti. In questo caso è meglio
sfruttare il polimorfismo.
In termine di oggetti è più naturale.

Data clumping quando delle strutture dati sono sempre usate insieme,
è un code smell. Ha senso creare una classe che racchiude queste
strutture dati.

Generalità speculativa, quando si crea un'interfaccia 
per qualcosa che non è ancora necessario, ma questo crea difficoltà
nella complessità del codice. 

\subsection{Classi di smell}
\begin{itemize}
    \item Troppo codice:
    \begin{itemize}
        \item \textbf{Large class}: una classe che ha troppe responsabilità
        \item \textbf{Long method}: un metodo che è troppo lungo.
        \item \textbf{Long parameter list}: una lista di parametri troppo lunga.
        \item \textbf{Duplicated code}: codice duplicato.
        \item \textbf{Dead code}: codice che non viene mai eseguito.
    \end{itemize}
    \item Troppo poco codice:
    \begin{itemize}
        \item \textbf{Classi con poco codice}: una classe che non ha abbastanza
        codice.
        \item \textbf{Empty catch clause}: un blocco catch che non fa nulla.
        \item \textbf{Classi di dati}: classi con solo getter e setter.
    \end{itemize}
    \item Codice con troppi commenti: il codice non è autoesplicativo.
\end{itemize}
\section{Software clone}
\begin{tcolorbox}[title = Clone]
    Un clone è un frammento di codice che è simile ad un altro frammento
    di codice.
\end{tcolorbox}
Spesso in copia incolla e vengono fatte modifiche, ma non sempre. Tuttavia 
abbiamo problemi di manutenzione, perché se si modifica un frammento
di codice, bisogna ricordarsi di modificare anche il clone. Attorno 
al 5\%. Esistono tool per identificare i cloni.
\subsubsection{Tipi di cloni}
\begin{itemize}
    \item \textbf{Type 1}: cloni identici, copia incolla.
    \item \textbf{Type 2}: cloni simili, ma con modifiche, a meno di 
    ridenominazioni di identificatori.
    \item \textbf{Type 3}: cloni simili, ma con modifiche e aggiunte, 
    ad esempio quando i parametri sono di tipi diversi e anche il valore 
    di ritorno.
\end{itemize}
\section{Refactoring}
Il fatto di avere i test automatizzati è importante, perché
si può fare refactoring in modo sicuro. Test di non regressione,
per verificare che il comportamento non cambi. 
Quando tutti i test passano, si può fare refactoring, si identifica 
il code smell e si determina il refactoring appropriato. Una volta completata 
questa modifica si rieseguono i test, se passano si può continuare.

Il ritmo da seguire è quello su scala piccola, si fa refactoring
su una piccola parte di codice, si eseguono i test, si fa refactoring 
su un'altra piccola parte di codice, si eseguono i test, e così via.
In questo modo si riduce il rischio di introdurre bug e ci si accorge
subito se si è introdotto un bug.
\subsection{Applicazione del refactoring}
La ridenominazione è il refactoring più semplice, si cambia il nome
di una variabile, di una funzione, di una classe, di un metodo. Per farlo possiamo 
utilizzare tool automatici, che ci permettono di fare refactoring
in modo sicuro, e vengono gestite in automatico dipendenze.

I tool riescono anche a rilevare dipendenze anche in altri file,
che possono sfuggire all'occhio umano. Lo svantaggio è che automatizzano 
solamente i refactoring più semplici, quelli più complessi vanno fatti
manualmente. Ovviamente i casi di test devono esserci e devono coprire 
la funzionalità.

Di solito capita di dare nomi sbagliati alle variabili, soprattutto 
quando non è chiaro il dominio del problema. In questo caso è utile
fare refactoring per rendere più chiaro il codice. Poiché le nuove funzionalità 
tenderanno a seguire la vita del progetto.
\subsubsection{Estrazione di interfaccia}
Una classe dipendente non dalla struttura, ma gli si da un'interfaccia
che deve essere implementata. In questo modo si può cambiare la struttura
della classe, ma la classe dipendente non cambia, riducendo il numero di 
cambiamenti da fare sulla classe. Si ragiona meglio in termini di funzionalità 
e non di implementazione in termini di cambiamento.
\subsubsection{Pull up method}
Si spostano dei metodi da una classe a una superclasse, in modo che
le sottoclassi ereditino il metodo. In questo modo si evita di avere
codice duplicato. 
\subsubsection{Extract method}
Si estrae un metodo da una classe, in modo da avere un metodo che
fa una sola cosa. In questo modo si evita di avere metodi troppo lunghi.
\subsubsection{Move method}
Si sposta un metodo da una classe a un'altra, in modo che il metodo
sia più vicino ai dati che utilizza. In questo modo si evita di avere
metodi che utilizzano dati di altre classi.
Come ad esempio:
\[
    \textit{La persona partecipa al progetto}\qquad \texttt{p.partecipate(x)}
\]
\[
    \textit{Il progetto contiene la persona} \qquad \texttt{x.partecipate(p)}
\]
In questo modo il progetto viene controllato la classe progetto,
poiché si tratta di una proprietà del progetto.
\subsubsection{Replace temp}
Si sostituisce una variabile temporanea con un metodo, in modo da
rendere più chiaro il codice.
\begin{lstlisting}
    double basePrice = quantity * itemPrice;
    if(basePrice > 1000)
        return basePrice * 0.95;
    else
        return basePrice * 0.98;
\end{lstlisting}

Si crea il metodo \texttt{basePrice()} che restituisce il
prezzo base, in questo modo il codice diventa più chiaro.
\begin{lstlisting}
    if(basePrice() > 1000)
        return basePrice() * 0.95;
    else
        return basePrice() * 0.98;
\end{lstlisting}
\subsubsection{Replace method with method object}
Si sostituisce un metodo con un oggetto, in modo da avere un metodo
che fa una sola cosa. In questo modo si evita di avere metodi troppo lunghi.
\begin{lstlisting}
    int basePrice = quantity * itemPrice;
    double discountLvel = getDiscountLevel();
    double finalPrice = computeFinalPrice(basePrice, discountLevel);
\end{lstlisting}
Si elimina il parametro discountLevel, chiamando il metodo al suo interno 
per ridurre la complessità del metodo.
\subsection{Extract class}
Si estrae una classe da un'altra, in modo da avere una classe che
fa una sola cosa. In questo modo si evita di avere classi troppo grandi.
\subsection{Replace inheritance with delegation}
Si sostituisce l'ereditarietà con la delega, in modo da avere una classe che
fa una sola cosa. In questo modo si evita di avere classi troppo grandi.

Pensando all'esempio dello stack, potremmo utilizzare un design dove lo stack 
che al posto di ereditare informazioni del vettore come, utilizzerà funzionalità 
di delega. incapsulando il vettore all'interno della classe. Una eredità 
selettiva e parziale.
\begin{lstlisting}
    public void push(Object elemento)
    {
        _vector.insertElementAt(elemento, 0);
    }
\end{lstlisting}
\subsection{Replace conditional with polymorphism}
Si sostituisce un costrutto condizionale con il polimorfismo, in modo da
avere una classe che fa una sola cosa. In questo modo si evita di avere
classi troppo grandi.

Prendiamo il calcolo dell'area:
\begin{lstlisting}
    private double a, b, r;
    ...
    public double area()
    {
        switch(type)
        {
            case SQUARE:
                return a * a;
            case RECTANGLE:
                return a * b;
            case CIRCLE:
                return Math.PI * r * r;
        }
    }
\end{lstlisting}
A seconda della forma geometrica, si calcola l'area
in modo diverso.
Ha più senso creare una classe astratta \texttt{Shape} che
ha un metodo \texttt{area()} che viene implementato dalle classi
\texttt{Square}, \texttt{Rectangle}, \texttt{Circle}.
\begin{lstlisting}
interface class Shape
{
    public double area();
}

class Square implements Shape
{
    public double area()
    {
        return side * side;
    }
}
...
\end{lstlisting}
\subsection{Separate domain from presentation}
Separazione dal dominio della presentazione, in modo da avere una classe che
fa una sola cosa. In questo modo si evita di avere classi troppo grandi.
Dividendo quindi le responsabilità.
