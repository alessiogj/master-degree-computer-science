\chapter{Agile}
\section{Agile}
\subsection{Approccio Agile}
Costruire sistemi safety-critical, governativi o gestire molti team distribuiti sul territorio richiede una
coordinazione efficace. Verso la fine degli anni '90, il contesto di sviluppo software è cambiato radicalmente.
I piani rigidi non erano più sempre efficaci, specialmente considerando le richieste di consegne sempre più rapide.
Con team piccoli, l'overhead generato da processi pesanti sarebbe stato un costo eccessivo, portando così all'adozione
di un approccio Agile.

Il requisito fondamentale dell'approccio Agile è rispondere rapidamente ai cambiamenti anziché seguire pedissequamente
un piano preciso. Data l'incertezza sulle esigenze finali e la loro probabile evoluzione nel corso dello sviluppo,
l'Agile propone una strategia per gestire tali cambiamenti. In questo contesto, si evita di congelare i requisiti
iniziali, ma si cerca piuttosto di capire meglio cosa si vuole ottenere.

Nell'approccio Agile, la specifica, il design e l'implementazione procedono in parallelo e sono strettamente collegati.
Il software finale viene gestito attraverso una serie di versioni, con ciascuna versione successiva che rappresenta un
incremento del prodotto. È cruciale coinvolgere il cliente e gli stakeholder durante tutto il processo di sviluppo Agile
in modo da assicurare un feedback costante e un allineamento continuo con le aspettative.

Di solito, si stabilisce un intervallo di tempo limitato per lo sviluppo di ciascuna versione, che va da 2 a 4 settimane.
La documentazione è mantenuta al minimo per evitare discrepanze e inconsistenze tra la documentazione e il codice stesso.
La comunicazione all'interno del team è spesso informale ma frequente, promuovendo un flusso costante di informazioni.

Per automatizzare il processo di testing, è fondamentale adottare strumenti per il testing automatico, in quanto il testing
manuale risulterebbe troppo lento e inefficiente data la necessità di coprire numerosi scenari di test. Alcuni strumenti
chiave per l'approccio Agile includono:

\begin{itemize}
    \item Testing automatico
    \item Gestione della configurazione
    \item Integrazione continua
    \item Produzione automatica dell'interfaccia utente
\end{itemize}

L'obiettivo è rilevare e correggere gli errori il prima possibile, in modo da evitare che si propaghino attraverso il sistema.

\subsection{Manifesto Agile}
L'Agile si basa su alcuni principi chiave che sono riassunti nel Manifesto Agile:

\begin{itemize}
    \item L'importanza è spostata dal processo formale all'individuo e alle interazioni.
    \item È più importante avere un software funzionante rispetto a una documentazione esaustiva.
    \item Si promuove la collaborazione con il cliente, che dovrebbe essere parte integrante del team di sviluppo piuttosto che
    un soggetto esterno con cui contrattare.
    \item L'attenzione è posta sulla capacità di rispondere al cambiamento in modo rapido e flessibile, piuttosto che
    sull'aderenza rigorosa a un piano prestabilito.
\end{itemize}

\subsection{Implementazione dell'Approccio Agile}
Per realizzare appieno i principi dell'Agile, è fondamentale coinvolgere attivamente il cliente nel processo di sviluppo.
Il coinvolgimento del cliente deve andare oltre il semplice fornitore di specifiche, incoraggiando il cliente a partecipare
attivamente alle discussioni e al processo decisionale. Questo assicura che le funzionalità sviluppate siano allineate con
le aspettative del cliente e che i feedback siano integrati tempestivamente nel processo di sviluppo.

Un altro aspetto chiave è lo sviluppo incrementale, dove le funzionalità vengono fornite in maniera graduale e integrata
nel sistema in crescita. In questo contesto, è fondamentale avere team con competenze solide e diversificate, che si fidino
a vicenda per garantire un flusso di lavoro efficiente e un'efficace condivisione delle responsabilità.

Nell'approccio Agile, è importante prepararsi al cambiamento continuo dei requisiti nel corso del progetto. La capacità di
adattarsi rapidamente ai nuovi requisiti e di integrarli nel processo di sviluppo è un fattore cruciale per il successo.

Il principio cardine per lo sviluppo Agile è quello di mantenere la semplicità. Ciò implica resistere alla tentazione di
aggiungere funzionalità non essenziali che potrebbero introdurre una complessità eccessiva nel sistema. L'obiettivo è quello
di adottare strategie che semplifichino il processo e riducano al minimo il rischio di complicazioni impreviste durante lo
sviluppo.

\subsection{Applicabilità dell'Approccio Agile}
L'approccio Agile è particolarmente adatto in contesti in cui il cliente è disponibile e desideroso di partecipare attivamente
al processo di sviluppo del software. È efficace in ambienti in cui le normative sono meno restrittive e permettono una maggiore
flessibilità nel processo di sviluppo. I processi di sviluppo Agile sono spesso implementati in combinazione con metodologie di
gestione progetti agili, come ad esempio Scrum.

\section{Extreme Programming (\texttt{XP})}
Spesso, nell'ambito dell'approccio Agile, si fa riferimento a \textbf{Extreme Programming}, che spinge all'estremo
alcune delle caratteristiche fondamentali dell'Agile. Ad esempio, in Extreme Programming, sono comuni varie versioni del software
consegnate al cliente quotidianamente, con incrementi che vengono immediatamente messi in produzione. Un principio chiave è che i
casi di test devono superare con successo per ogni build, pertanto è necessario assicurarsi che tutti i casi di test siano funzionanti
prima di procedere con una nuova build.

Nel contesto di Extreme Programming, viene dato ampio risalto alla pratica di scrivere i test prima di scrivere il codice effettivo.
Il refactoring è una pratica costante con l'obiettivo di semplificare il codice sorgente per renderlo più leggibile e mantenibile nel
tempo. Altro aspetto fondamentale di \texttt{XP} include il pair programming, in cui due sviluppatori lavorano insieme su un singolo codice,
garantendo una maggiore qualità e condivisione delle conoscenze. Inoltre, si mira a mantenere un ritmo di sviluppo sostenibile nel tempo,
evitando eccessive accelerazioni che potrebbero compromettere la qualità del software.

\section{User Stories}
Nel contesto dell'Agile, i requisiti vengono spesso raccolti in maniera informale tramite user stories, che sono narrazioni in prosa
che descrivono le interazioni tra l'utente e il sistema. Questo metodo aiuta a mantenere i requisiti concreti e facilita l'elicitazione
dei requisiti stessi poiché le user stories sono facili da scrivere e da comprendere.

Le user stories vengono successivamente suddivise in parti più piccole, comunemente chiamate task cards. Per ogni task, è necessario
fornire una stima temporale per la sua realizzazione. L'obiettivo è quello di coinvolgere attivamente il cliente nella priorizzazione
delle varie attività, poiché, in un contesto di sviluppo incrementale, non è possibile affrontare tutto contemporaneamente.

Tuttavia, con questo tipo di approccio, non si ha la certezza di implementare tutte le funzionalità richieste, poiché alcune di esse
potrebbero non emergere in determinati scenari o iterazioni di sviluppo.

\section{Test-Driven Development (\texttt{TDD})}
Nel Test-Driven Development (\texttt{TDD}), i test vengono scritti prima del codice stesso, e tali test possono essere eseguiti durante la fase
di scrittura del codice. Questo approccio consente di individuare errori il prima possibile, valutando ogni singola microcomponente
prima di passare a quella successiva. Il codice risultante è di alta qualità in quanto viene costantemente testato e non presenta bug
evidenti.

I test, in questo contesto, rappresentano una documentazione degli scenari previsti e consentono di pensare al comportamento del sistema
prima ancora di implementarlo effettivamente. Ciò porta a una maggiore consapevolezza del sistema e del suo funzionamento da parte del
team di sviluppo.

Inoltre, è importante coinvolgere attivamente l'utente nella fase di verifica, in modo da sviluppare i cosiddetti acceptance test per le
varie user stories. Questo è possibile solo grazie all'utilizzo di framework di test automatizzati. Di conseguenza, il numero di test
generati è notevolmente elevato, garantendo la coerenza e la stabilità del sistema, specialmente in un contesto di sviluppo incrementale.

\subsection{Refactoring}
Il refactoring rappresenta il processo di riscrittura del codice al fine di renderlo più leggibile e mantenibile nel lungo periodo.
Spesso, ogni singolo incremento potrebbe richiedere alcuni compromessi che, per essere integrati nel sistema, richiedono una sorta di
``degradazione" del sistema. Il refactoring permette di affrontare tali problematiche e di mantenere l'integrità del sistema nel tempo,
assicurandone la stabilità e la longevità. Il refactoring risulta essere una pratica altrettanto valida anche nel contesto di sviluppo
pianificato (\textit{plan-driven}).

\subsection{Pair Programming}
Il pair programming è una pratica in cui due persone lavorano insieme su un singolo computer. Mentre una persona scrive il codice,
l'altra controlla attivamente il lavoro. I ruoli si alternano regolarmente, e le decisioni cruciali vengono prese in modo collaborativo.
I vantaggi del pair programming sono numerosi, inclusa la condivisione della responsabilità, il controllo continuo e un feedback costante.
Inoltre, il continuo scambio di conoscenze e la code review continua portano a un miglioramento delle abilità individuali e della qualità
del codice. Il refactoring continuo contribuisce ulteriormente a migliorare la comunicazione all'interno del team. Contrariamente a ciò
che si potrebbe pensare, il pair programming non causa una diminuzione significativa della produttività, ma piuttosto conduce a un miglioramento
generale delle prestazioni del team.
\section{CI/CD nel Sviluppo Software Moderno}

CI/CD è un acronimo che sta per "Continuous Integration" (Integrazione Continua)
e "Continuous Delivery" o "Continuous Deployment" (Consegna Continua o Distribuzione Continua). Questi sono concetti chiave nelle pratiche di
sviluppo software moderno, in particolare nell'ambito della metodologia Agile e DevOps.

\subsection{Continuous Integration (CI)}
La Continuous Integration si riferisce alla pratica di integrare automaticamente i cambiamenti del codice nel repository principale di un progetto
più volte al giorno. Il suo scopo è identificare e risolvere rapidamente i problemi di compatibilità o altri problemi introdotti da nuove modifiche.
Questo processo coinvolge tipicamente l'uso di strumenti automatizzati per compilare e testare il codice ogni volta che un cambiamento viene \textit{committato},
assicurando che il codice nuovo funzioni correttamente con il codice esistente.

\subsection{Continuous Delivery (CD)}
La Continuous Delivery si estende dalla CI e implica la consegna automatica di cambiamenti del codice a un ambiente di test o di produzione dopo il
processo di integrazione. L'obiettivo è rendere il processo di rilascio del software il più efficiente e prevedibile possibile, riducendo i tempi
e i costi di distribuzione.

\subsection{Continuous Deployment}
Il Continuous Deployment è simile al Continuous Delivery. Tuttavia, mentre nel Continuous Delivery ogni rilascio richiede ancora un'intervento
manuale per la distribuzione in produzione, nel Continuous Deployment ogni cambiamento che passa tutte le fasi di produzione viene rilasciato
automaticamente senza intervento umano.

\section{Project Management Agile}
Nel contesto dell'approccio Agile, il project manager ha la responsabilità di assicurarsi che il software venga consegnato entro i tempi
previsti, con tutte le funzionalità richieste e nel rispetto del budget stabilito. È responsabile del coordinamento del progetto e del
monitoraggio dei progressi.

\subsection{Scrum}
Scrum è un metodo di project management agile che si basa sul lavoro per iterazioni, richiamando così il processo di sviluppo software agile.
Le sue fasi principali includono:

\begin{enumerate}
    \item Definizione di obiettivi generali e architettura ad alto livello del sistema.
    \item Sprint planning, in cui si definiscono gli obiettivi per la prossima iterazione.
    \item Fase conclusiva, che include la documentazione finale, la code review e la riflessione sulle lezioni apprese.
\end{enumerate}

\subsection{Terminologia}
Alcuni termini chiave in Scrum includono:

\begin{itemize}
    \item Team piccolo, con un massimo di $7$ persone.
    \item Backlog, che rappresenta la lista di tutte le funzionalità richieste dal cliente.
    \item Product Owner, responsabile della priorizzazione delle funzionalità e delle decisioni su cosa è più necessario.
    \item Sprint, che rappresenta un periodo di lavoro di $2$-$4$ settimane.
    \item Scrum Master, responsabile del monitoraggio e del supporto al processo di sviluppo.
    \item Velocità, che rappresenta la produttività del team e viene calcolata e aggiornata durante lo sviluppo per la
    pianificazione dei prossimi sprint.
    \item Sprint Review, una sessione per riflettere sull'iterazione e fornire input per migliorare i prossimi sprint.
\end{itemize}

\subsection{Benefici}
I benefici di Scrum includono la gestione efficiente di funzionalità nel breve termine, una conoscenza approfondita
dei problemi e delle attività di ogni membro del team, nonché la puntualità delle consegne per il cliente. Scrum favorisce
un clima di fiducia tra il cliente e il team di sviluppo, poiché il cliente ha la possibilità di visualizzare il sistema software
funzionante e fornire un feedback continuo. Questo feedback aiuta a chiarire le esigenze del cliente e ad adattare il prodotto
in modo efficace.

\section{Scrum Distribuito}
Lo Scrum distribuito si riferisce a situazioni in cui il team di sviluppo è distribuito in più sedi geografiche. In questo contesto,
la comunicazione tra i membri del team avviene attraverso strumenti come chat in tempo reale, chiamate video e l'utilizzo di sistemi
di continuous integration per garantire che il software sia sempre funzionante. Un piano di sviluppo comune aiuta a facilitare la
comunicazione tra i vari team e sedi geografiche.

\subsection{Scalabilità}
La scalabilità di Scrum si riferisce alla sua capacità di adattarsi a progetti più grandi che coinvolgono più persone e team,
o che richiedono un'implementazione distribuita. Alcuni punti chiave da considerare includono la necessità di mantenere le pratiche
chiave come il Test Driven Development (TDD), la comunicazione e la consegna di incrementi funzionanti. La coordinazione tra lo
sviluppo Agile e la manutenzione del sistema è altrettanto importante, specialmente quando le persone coinvolte potrebbero cambiare
nel corso del progetto.

\subsection{Problemi e Resistenze}
Alcuni problemi e resistenze comuni nell'adozione di metodi Agile come Scrum includono la mancanza di esperienza del management nell'ambito
della metodologia Agile, la necessità di adattarsi a procedure di controllo della qualità preesistenti e le resistenze culturali all'interno
dell'organizzazione. Il successo della metodologia Agile può essere dimostrato attraverso casi di studio e una maggiore stabilità dei requisiti
nel tempo.


