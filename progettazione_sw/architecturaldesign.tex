\chapter{Design architetturale}
\section{Design architetturale}
È importante progettare l'architettura del sistema, prima ancora della progettazione 
dei singoli componenti e di come sono legati al loro interno, per avere una visione ad 
alto livello per capire organizzazione generale del sistema software.
L'biettivo è capire come il sistema software sarà organizzato e strutturato, 
capendo inoltre come questi macro-componenti interagiscono tra loro.

Di fatto la progettazione architetturale si colloca a metà strada tra l'ingegneria dei
requisiti e la progettazione dettagliata del sistema software. Siccome il software è 
essenzialmente astratto, è estremamente importante avere una rappresentazione chiara
dell'architettura del sistema.
\subsection{Importanza dell'architettura}
\begin{itemize}
    \item È importante come \textbf{mezzo di comunicazione con gli stakeholder}, perché da una visione
    ad alto livello del sistema, dando quindi un'idea di come può essere il sistema finale, che 
    può essere oggetto di negoziazione e confronto con gli stakeholder. 
    
    \item Si tratta di una \textbf{prima forma di decisione}, già scegliendo di quale architettura
    si vuole
    adottare, si fanno delle scelte che influenzeranno il sistema finale che potrebbero comportare 
    che alcuni requisiti non funzionali potranno o meno essere soddisfatti. I requisiti non funzionali
    sono quelli che hanno maggior impatto sui componenti del sistema.
    \item Discutere di architetture ricorrenti permette di avere una soluzione generica e condivisa 
    tra gli sviluppatori, avendo quindi della \textbf{terminologia comune}.
\end{itemize}
\subsection{Rappresentazione dell'architettura}
La \textbf{rappresentazione} dell'architettura è spesso realizzata tramite diagrammi a blocchi,
che mostrano le relazioni tra le varie entità. Questi diagrammi, sebbene utili, sono
talvolta criticati per la loro marcata astrazione e mancanza di significato dettagliato.

Gli \textbf{utilizzi} di tali rappresentazioni includono la semplificazione delle discussioni sulla
progettazione del sistema e la documentazione dell'architettura, evidenziando componenti,
connessioni e interfacce. Questo approccio fornisce una visione chiara e organizzata
dell'intero sistema, essenziale per una progettazione e implementazione efficace.

Le \textbf{viste} dell'architettura sono le rappresentazioni dell'architettura che mostrano
diversi aspetti del sistema. Queste viste sono utili per gli stakeholder che hanno
interessi diversi nel sistema. Ad esempio, un programmatore può essere interessato
alla vista dell'architettura che mostra i componenti del sistema e le loro interazioni,
mentre un amministratore di sistema può essere interessato alla vista che mostra
le risorse del sistema e le loro interazioni.
Si distinguono quindi le seguenti viste:
\begin{itemize}
  \item \textbf{Vista logica}: mostra le entità chiave del sistema e le loro interazioni.
  \item \textbf{Vista di processo}: mostra i processi del sistema e l'evoluzione durante l'esecuzione.
  \item \textbf{Vista di sviluppo}: mostra come il sistema è organizzato per il processo di sviluppo, 
  perché potenzialmente ognuno dei componenti potrebbe essere assegnato a sviluppatori differenti o 
  team di sviluppo differenti.
  \item \textbf{Vista fisica}: mostra l'hardware e il software che compongono il sistema.
\end{itemize}

\section{Pattern architetturale}

\subsection{Model View Controller}
Il \textbf{Model View Controller} è un pattern architetturale che separa i dati e la logica
dell'applicazione dalla loro rappresentazione e dal modulo che si occupa di gestire 
l'interazione con l'utente. Il pattern \texttt{MVC} prevede la suddivisione dell'applicazione
in tre principali responsabilità:
\begin{itemize}
  \item \textbf{Model}: rappresenta i dati dell'applicazione e le operazioni che possono essere
  eseguite su di essi. In questo componente non è presente alcuna logica di presentazione.
  \item \textbf{View}: rappresenta la visualizzazione dei dati contenuti nel model. In questo
  componente non è presente alcuna logica di business.
  \item \textbf{Controller}: riceve gli input dell'utente e li converte in comandi per il model
  o per la view.
\end{itemize}

L'utilizzo di \texttt{MVC} è consigliato quando si vuole separare la logica di business dalla
logica di presentazione. In questo modo è possibile modificare la rappresentazione
dei dati senza dover modificare la logica di business e viceversa. 
Quindi quando ci sono molti modi di vedere o interagire con gli stessi dati o quando il 
modo in cui le viste interagiscono con i dati è soggetto a evoluzione.
\subsubsection{Esempi di applicazione}
Un esempio tipico è la web application, dove il browser legge l'interfaccia 
grafica fornita dal componente con la responsabilità di view, e genera 
la pagina corrente con le informazioni di stato che provengono dal model. L'interfaccia 
ha dei widget grafici che permettono all'utente di interagire con il sistema, e gli 
eventi generati dal view vengono di fatto spediti al controller mediante richieste 
\texttt{HTTP}. Il model e il controller collaborano per generare la prossima 
pagina da visualizzare.

\subsubsection{Vantaggi e svantaggi}
Il pattern consigliato quando abbiamo diversi modi di visualizzare diversi dati, 
avendo così un solo punto in cui i componenti grafici leggono i dati. Inoltre, 
è opportuno dotarsi di tale pattern quando le viste che interagiscono con i dati
sono soggette a evoluzione, perché possiamo evolvere la view senza dover
modificare il model e viceversa.
Tra i vantaggi di questo pattern si possono citare:
\begin{itemize}
    \item \textbf{Separazione delle responsabilità}: il pattern \texttt{MVC} separa le responsabilità
    dell'applicazione in tre componenti. Questo rende più facile la manutenzione e l'evoluzione
    dell'applicazione.
    \item \textbf{Supporta la presentazione dei dati in maniere diverse}: il pattern \texttt{MVC}
    permette di presentare i dati in maniere diverse.
    \item \textbf{Cambiamenti nella logica}: il pattern \texttt{MVC} permette di cambiare la logica
    dell'applicazione senza dover modificare la presentazione dei dati.
\end{itemize}
Tra gli svantaggi principali abbiamo che la separazione dei componenti
può portare a un aumento
della quantità di codice necessario per implementare l'applicazione.
\subsection{Architettura stratificata}
Nell'architettura stratificata, si prevede la presenza di una serie di funzionalità \textbf{core}
di basso livello, sulla quale vengono costruite funzionalità di livello superiore, chiamate 
\textbf{basic utilities}. A sua volta c'è uno strato successivo che utilizza servizi messi 
a disposizione dallo stato immediatamente successivo. Ogni livello utilizza i servizi offerti
dal livello immediatamente sottostante e offre servizi al livello immediatamente sovrastante.

Tale architettura è stata definita per supportare alcuni vantaggi, tra cui il supporto
allo sviluppo incrementale. Infatti, dopo aver definito i servizi offerti da uno strato,
è possibile procedere con lo sviluppo dello strato successivo.

Permette inoltre di implementare una carta portabilità, in quanto è possibile rimpiazzare
un livello fintanto che l'interfaccia di comunicazione tra livelli adiacenti è rispettata.
\subsubsection{Esempi di applicazione}
Un esempio tipico di applicazione di questo pattern è il sistema operativo, dove
si ha un livello di basso livello che si occupa di gestire l'hardware, un livello
di sistema operativo che si occupa di gestire le risorse del sistema e un livello
di applicazione che si occupa di gestire le applicazioni.

Un altro esempio è il modello \texttt{ISO/OSI}, dove si ha un livello fisico, un livello
di collegamento, un livello di rete, un livello di trasporto, un livello di sessione, un
livello di presentazione e un livello di applicazione. Il livello applicativo può viaggiare 
su dispositivi fisici differenti, disinteressandosi del mezzo fisico.


\subsubsection{Vantaggi e svantaggi}
È opportuno dotarsi di tale pattern quando si vuole sviluppare un sistema software sopra 
a funzionalità esistenti. Inoltre, quando lo sviluppo di un sistema può essere assegnato  
a più team, perché di fatto ogni strato è abbastanza indipendente dagli altri. Potrebbe 
esserci inoltre un requisito multilivello.

Tra i vantaggi di questo pattern si possono citare:
\begin{itemize}
    \item Permette di rimpiazzare un livello fintanto che l'interfaccia di comunicazione 
    tra livelli adiacenti è rispettata.
    \item Si possono introdurre ridondanze tra i vari layer in maniera da incrementare 
    l'affidabilità del sistema.
\end{itemize}
Tra gli svantaggi principali abbiamo che:
\begin{itemize}
    \item Nella pratica una separazione dei livelli così netta e pulita non 
    è banale da raggiungere, e spesso si finisce per avere livelli che si
    interfacciano con livelli non adiacenti.
    \item L'aggiunta di un nuovo livello può portare a un aumento della latenza.
\end{itemize}
\subsection{Architettura Repository}
L'architettura Repository è un pattern architetturale che prevede la presenza di un
componente centralizzato che si occupa di gestire la persistenza dei dati.
Questo componente fornisce un'interfaccia per accedere ai dati e per modificarli, non 
è presente alcun collegamento diretto tra i componenti e i dati, ma tutti i componenti
devono interagire con la repository per accedere ai dati.

Viene solitamente utilizzata quando ci sono grandi quantità di dati da gestire o
quando devono essere gestiti dati che cambiano frequentemente, o in sistemi dove 
l'inclusione di dati nella repository fa si che venga azionata una serie di
eventi.

\subsubsection{Esempi di applicazione}
Un esempio tipico è un sistema di versionamento come \texttt{Git}, dove i dati sono
gestiti da una repository e i vari componenti interagiscono con la repository per
accedere ai dati.

Un altro esempio sono i software \texttt{CASE} (\textit{Computer Aided Software Engineering}),
dove i dati sono gestiti da una repository e i vari componenti interagiscono con la
repository per accedere ai dati.

\subsubsection{Vantaggi e svantaggi}
È opportuno dotarsi di tale pattern quando i volumi di dati sono molto grandi, che 
devono rimanere persistenti per un certo periodo e questa responsabilità può essere
assegnata a un singolo componente. I dati sono necessari a diverse componenti, in questo 
modo, avendo come riferimento un repository centrale, si evitano problematiche legate 
alla coerenza dei dati.
È opportuno dotarsi di tale architettura quando alcune condizioni dei dati
scatenano degli eventi.

Tra i vantaggi di questo pattern si possono citare:
\begin{itemize}
    \item Si realizza una grande \textbf{indipendenza} tra i vari componenti periferici,
    perché, in linea di principio, ogni componente non è a conoscenza della 
    esistenza degli altri componenti.
    \item L'aggiornamento dei dati apportato da un componente è immediatamente
    disponibile a tutti gli altri componenti.
    \item I dati vengono gestiti in maniera consistente essendo in un singolo 
    punto di accesso.
\end{itemize}
Gli svantaggi sono che:
\begin{itemize}
    \item C'è un \textit{single point of failure}. Se la repository fallisce,
    l'intero sistema fallisce.
    \item La repository può diventare un collo di bottiglia.
    \item Distribuire la repository su più server può essere difficile.
\end{itemize}
\subsection{Architettura Client-Server}
L'architettura Client-Server è un pattern architetturale che prevede la presenza
di due componenti: un server che fornisce un servizio e un client che fruisce
il servizio. Il client e il server comunicano attraverso una rete utilizzando
un protocollo ben definito.

\subsubsection{Vantaggi e svantaggi}
È opportuno dotarsi di tale pattern quando si vuole sviluppare un sistema software
in cui i servizi devono essere fruibili da una lista numerosa di client. Inoltre,
in contesti di \textit{load balancing} dove si vuole garantire l'affidabilità del
sistema, quando il numero di client è molto grande e si vuole garantire la
scalabilità del sistema.

Tra i vantaggi di questo pattern si possono citare:
\begin{itemize}
    \item I server possono essere distribuiti sulla rete.
    \item Funzionalità generali, possono essere fornite da un sottoinsieme di server senza che 
    vengano implementate da tutti i server.
\end{itemize}
Tra gli svantaggi principali abbiamo che:
\begin{itemize}
    \item Ogni server è un \textit{single point of failure}.
    \item La performance dipende dalla latenza della rete.
    \item Problemi di management quando i server sono posseduti da differenti 
    organizzazioni.
\end{itemize}

\subsection{Architettura Peer-to-Peer}
Nelle architetture \textbf{Peer-to-Peer}, ogni componente agisce sia
come client che come server. 
Di fatto ogni nodo della rete può fornire un servizio e fruirne servizi da altri
nodi. Il termine \textit{peer} indica che tutti i nodi sono equivalenti tra loro, 
indicando un \textit{dialogo tra pari}.
Per questo motivo la rete \texttt{P2P} è molto solida, in quanto essendoci problemi 
a livello di singolo nodo, la rete può continuare a funzionare.
\subsubsection{Esempi di applicazione}
Un esempio tipico è il software \texttt{BitTorrent}, dove i vari nodi della rete
forniscono e fruiscono servizi tra loro. Di fatto c'è un tracker che si occupa di
indirizzare i vari nodi verso i nodi che forniscono il servizio richiesto.
Scaricando i diversi \textit{chunks} di un file da diversi nodi, si ottiene un
download più veloce e si alleggerisce il carico sui singoli nodi. Questo 
senza che ogni peer possieda il file nella sua interezza.

Un'altra architettura \texttt{P2P} è \texttt{Bitcoin}, contribuiscono nel 
ledger distribuito, ovvero la blockchain, e possono essere ricompensati per
il loro contributo. Non essendoci un entre monetario centralizzato e non essendo
necessario un ente centrale per la validazione delle transazioni, il sistema
è molto solido e resistente a tentativi di attacco.

\subsection{Architettura Pipe and Filter}
L'architettura Pipe and Filter è un pattern architetturale che i dati 
vengano elaborati per step successivi, ad esempio prima generiamo i dati, poi li
filtriamo, poi li analizziamo e infine li visualizziamo.

È un'architettura tipicamente utilizzata in applicazioni di \textit{data processing} 
effettuate in batch. Non è adatta per applicazioni interattive che prevedono un'interazione
con l'utente.
Viene chiamata in questo modo per l'analogia con i sistemi \texttt{UNIX} in cui i
\textit{pipe} sono dei canali di comunicazione tra processi e i \textit{filter} sono
dei processi che trasformano i dati.

Si tratta di un'architettura molto appropriata per sistemi in cui i dati devono
essere elaborati in maniera sequenziale, in cui i dati sono disponibili in
quantità illimitate e in cui i dati possono essere elaborati in maniera
indipendente. Non sono adatte per sistemi interattivi.
La utilizziamo quindi in applicazioni che processano dei dati.

\subsubsection{Vantaggi e svantaggi}
Tra i vantaggi di questo pattern si possono citare:
\begin{itemize}
    \item Facile da capire e supporta il riuso.
    \item Il workflow è simile alla struttura di danti processi di business.
    \item L'evoluzione aggiungendo trasformazioni è semplice.
    \item Può essere implementato come un sistema sequenziale o concorrente.
\end{itemize}

Tra gli svantaggi principali abbiamo che:
\begin{itemize}
    \item Il formato dei dati deve essere comune a tutti gli step di trasformazione.
    \item Ogni trasformazione deve fare parsing dei dati come concordato, ciò incrementa 
    l'overhead.
\end{itemize}
\section{Architetture eterogenee}
Ovviamente si possono combinare tra di loro i vari pattern architetturali se vantaggioso.
Li si può organizzare in una composizione gerarchica, ossia ad esempio ho un sistema pipe
and filter che ha un sottosistema organizzato a livelli.
Nelle architetture eterogenee inoltre un componente può avere due ruoli diversi
in due pattern distinti, ad esempio un sottosistema accede ai dati tramite
una repository ma comunica con gli altri componenti tramite una pipe.