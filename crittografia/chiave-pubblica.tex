\chapter{Crittografia a chiave pubblica}
\section{Crittografia a chiave pubblica}
L'idea di crittografia a chiave pubblica è quella di avere due chiavi, una
pubblica $P_k$ e una privata $S_k$. La chiave pubblica è nota a tutti, mentre
quella privata è nota solo al proprietario. 
Con codifica avviene con la chiave pubblica, mentre la decodifica avviene con la
chiave privata.

Ovviamente l'idea di fondo è avendo in mano il testo cifrato non si riesce a
risalire al testo in chiaro senza la chiave privata. Chiunque può cifrare un
messaggio, ma solo il proprietario della chiave privata può decifrarlo.

In un sistema a chiave pubblica abbiamo i seguenti algoritmi:
\begin{itemize}
    \item Un algoritmo di generazione delle chiavi:
    \[
        \mathcal{G}: 1^k \to (P_k, S_k) 
    \]
    Dove $1^k$ è un parametro che indica la lunghezza della chiave, ovvero il \textit{security parameter}.
    \item Un algoritmo di encription:
      \[
        \mathcal{E}: m, P_k \to \mathcal{E}(m, P_k)
    \]
    \item Un algoritmo di decription:
    \[
          \mathcal{D}: C, S_k \to D(C, P_k)
    \]
\end{itemize}
Ovviamente vale la seguente relazione:
\begin{equation}
    \forall m \quad \mathcal{D}(\mathcal{E}(m, P_k), S_k) = m
\end{equation}
Oltre al fatto che decriptare un messaggio a partire dal testo cifrato senza 
la chiave privata è computazionalmente intrattabile. L'idea è che più la chiave
è lunga più è difficile rompere il sistema. L'idea è che il security parameter
$k$ è proporzionale alla lunghezza della chiave e al crescere di $k$ cresce
la sicurezza del sistema.

Gli algoritmi citati sono algoritmi probabilistici polinomiali. Se voglio 
affermare che tali algoritmi sono polinomiali l'input che rappresenta la 
lunghezza della chiave non può essere 
rappresentato in binario, perché la lunghezza della chiave sarebbe
esponenziale nel numero di bit utilizzati per rappresentare il numero. 
Sulle macchine di Turing la dimensione del problema è data dal numero di 
celle del nastro di input. Utilizzando la teoria della complessità 
basata su tali macchine, o in ogni caso su sistemi dove la dimensione 
dell'input è lo spazio che occupa nella nostra rappresentazione. Per 
voler dire che un algoritmo è polinomiale nel \textbf{valore del 
security parameter} e non nel modo in cui è rappresentato, imponiamo 
che il security parameter sia rappresentato in \textbf{unario}, ovvero 
tanti uni quanti la lunghezza della chiave.

\section{Diffie-Hellman}
Il problema di Diffie-Hellman è il seguente: Alice e Bob vogliono
scambiarsi un segreto senza che Eve lo possa intercettare. Lo strumento 
utilizzato per risolvere il problema è del logaritmo discreto.

Si fissa a priori un numero primo $p$ e un generatore $g$ di $\mathbb{Z}_p^*$.
Un agente A sceglie un numero $x \in_R \{1, \dots, p-1\}$ e calcola $g^x \mod p$. 
Un agente B sceglie un numero $y \in_R \{1, \dots, p-1\}$ e calcola $g^y \mod p$.

\begin{figure}[H]
    \centering
    \begin{tabular}[H]{l|l|l}
        & \textbf{Public} & \textbf{Private} \\
        \hline
        A & $g^x \mod p$ & $x$ \\
        \hline
        B & $g^y \mod p$ & $y$ \\
    \end{tabular}
\end{figure}

A questo punto A e B possono calcolare $g^{xy} \mod p$ e $g^{yx} \mod p$, che coincidono.

Il sistema è sicuro perché calcolare $g^{xy} \mod p$ è computazionalmente intrattabile. Avendo 
a disposizione $g^x$ e $g^y$ non è possibile calcolare $g^{xy}$. Se sappiamo rispondere al problema 
del logaritmo discreto, allora possiamo risolvere il problema di Diffie-Hellman, tale algoritmo 
potrebbe esistere, ma non è stato ancora trovato.

Sia $\mathcal{A} \in \texttt{PPT}$ che calcola $g^{xy} \mod p$ a partire da $g^x \mod p$ e $g^y \mod p$. 
Usiamo $\mathcal{A}$ per costruire un algoritmo $\mathcal{B}$ che risolve il problema del logaritmo discreto,
ma tale dimostrazione non è ancora stata data, perciò non l'algoritmo di Diffie-Hellman non è dimostrabilmente 
sicuro. non siamo in grado di dire che rompere Diffie-Hellman è almeno difficile quanto risolvere il problema
del logaritmo discreto.

\subsection{Ipotesi di Diffie-Hellman}
Qualcuno potrebbe costruire un algoritmo che si basa sull'ipotesi che l'algoritmo di Diffie-Hellman sia sicuro.
\begin{tcolorbox}[title = Ipotesi di Diffie-Hellman]
    Siano $x, y, z$ dei numeri causali scelti in $\{1, \dots, p-1\}$, allora è difficile 
    distinguere ($g^x, g^y, g^{xy}$) da ($g^x, g^y, g^z$).
\end{tcolorbox}

Il concetto di distinguibilità è un concetto probabilistico, ovvero che la possibilità di poter 
distinguere due insiemi di elementi è trascurabile. Ovvero che la probabilità di distinguere 
sia inferiore a $1/2 + \epsilon$, dove $\epsilon$ è trascurabile. Quindi l'attaccante non abbia 
alcun vantaggio rispetto ad un attaccante che non ha alcuna informazione.
Se un attaccante avesse anche un minimo vantaggio, allora potrebbe utilizzare tale vantaggio per
ottenere informazioni sul segreto. Tale vantaggio può essere utilizzato per ottenere l'informazione 
totale mediante esperimenti ripetuti.

\section{Rivest Shamir Adleman - RSA}
$g^x$ è una funzione che è facile da calcolare, ma è difficile da invertire, ovvero calcolare $x$
a partire da $g^x$. Una funzione con questa proprietà è detta \textbf{one-way function}. Il protocollo di 
Diffie-Hellman è sicuro se e solo se esiste una one-way function.
Vorremmo che la one-way function sia anche \textbf{trapdoor}, ovvero che esista un algoritmo 
efficiente che permetta di invertire la funzione. Tale algoritmo è detto \textbf{trapdoor algorithm}.
La trapdoor è una informazione aggiuntiva che permette di invertire la funzione.

Siano $p, q$ due numeri primi molto grandi, $n = pq$. Sia $e$ un numero casuale tale che 
\texttt{mcd}(e, $\varphi(n)$) = 1, dove $\varphi(n) = (p-1)(q-1)$, ovvero il numero di Eulero. Scegliamo 
une elemento $d$ co-primo con $\varphi(n)$ tale che $de \equiv 1 \mod \varphi(n)$.
La chiave pubblica è la coppia $(n, e)$, mentre la chiave privata è la coppia $(n, d)$.
\subsubsection{Algoritmo di cifratura}
\[
  \mathcal{E}: m, (n, e) \mapsto m^e \mod n  
\]
\subsubsection{Algoritmo di decifratura}
\[
  \mathcal{D}: c, (n, d) \mapsto c^d \mod n
\]
\subsection{Funzionamento}
Proviamo a prendere un messaggio $m$ e a cifrarlo con la chiave pubblica $(n, e)$,
e proviamo a decodificarlo con la chiave privata $(n, d)$.
\[
  (m^e)^d = m^{ed} = m^{k\varphi(n) + 1} = m \cdot (m^{\varphi(n)})^k \equiv m \mod n  
\]
Infatti $d\cdot e$ è congruo a 1 modulo $\varphi(n)$, quindi $d\cdot e = k\varphi(n) + 1$.
Per il teorema del resto cinese, $m^{k\varphi(n) + 1} \equiv m$, e non solo per gli elementi 
di $\mathbb{Z}_n^*$.

La funzione one way è la funzione di codifica, quindi $m^e$, l'inversa di 
$m^e$ è la radice $e$-esima, ovvero $m = c^{1/e}$, ma per calcolare la radice $e$-esima
ad oggi non esiste un algoritmo efficiente. L'unico modo per calcolare la radice $e$-esima
è calcolare $d$ e decifrare il messaggio, quindi $d$ è l'informazione trapdoor.

Ci piacerebbe dimostrare che se fossimo in grado di invertire la radice 
e-esima di un numero allora saremmo in grado di fattorizzare $n$, o di calcolare 
il residuo quadratico. Se fosse vero, allora RSA sarebbe sicuro, ma non è stato ancora
dimostrato ad oggi. L'unica sicurezza di RSA è l'esistenza dell'ipotesi di RSA.
\subsection{Attacchi a RSA}
Ci sono casi in cui RSA è stato attaccato, il motivo però era legato alla cattiva 
implementazione dell'algoritmo, e non all'algoritmo in sé.
Quando parliamo di un crittosistema in realtà parliamo di un insieme di algoritmi,
e non di un singolo algoritmo. L'algoritmo di generazione delle chiave impone 
la scelta \textbf{casuale} di $e$ in modo uniforme con gli elementi 
primi con $\varphi(n)$, ma se non fosse casuale, allora potremmo avere dei problemi.
\subsubsection{Attacco sulla base di messaggi piccoli}
Inoltre, in caso di messaggi piccoli, quindi se $m^e < n$, vuol dire che 
non applico nemmeno l'operazione di modulo, e vuol dire in particolare che 
la radice $e$-esima di $m^e$ è una normale radice $e$-esima nell'aritmetica 
dei numeri interi, e quindi è facile da calcolare.
Lavorando con messaggi che a livello numerico sono piccoli, allora 
invertiamo tutto facilmente, più è piccola $e$, più è facile avere 
messaggi che elevati a $e$ sono più piccoli di $n$. Bisogna quindi
star attenti a scenari in cui $m^e < n$.
\subsubsection{Attacco sulla base di messaggi sparsi}
Altri problemi che potrebbe avere RSA sono legati allo spargimento dei 
messaggi.
Immaginiamo di avere un messaggio:

\begin{center}
  \texttt{Buongiorno, il suo voto è 30L}\\
  \texttt{Buongiorno, il suo voto è 30}\\
  \texttt{Buongiorno, il suo voto è 29}\\
  \dots\\
  \texttt{Buongiorno, il suo voto è 0}\\
\end{center}
Uno che vuole decifrare il messaggio può prendere i $32$ messaggi e
cifrarli tutti, per poi distinguerli.
Se con RSA devo codificare messaggi che sono presi da un insieme piccolo,
devo far attenzione perché potrei venir attaccato da qualcuno che utilizza 
la stessa chiave pubblica.
\subsubsection{Attacco sulla informazione parziale}
Siamo sicuri che tutti i bit di questa inversa siano difficili da calcolare?
Non è che sulla radice $e$-esima di $m^e$ ci sono dei bit che sono più facili 
da calcolare? Ai fini di dire che il sistema è sicuro, non è 
sufficiente dire che il cypertext non si sappia ricavare il plaintext,
vorremmo dire che dal cypertext non si riesca a ricavare nessuna informazione 
binaria sul plaintext. Ma come possiamo definire tale proprietà?
\subsubsection{Attacco sulla base di messaggi ripetuti}
Per difendersi da tale problema, aggiungo un po' di rumore al messaggio,
ovvero aggiungo un po' di bit casuali al messaggio, in modo tale che con 
alta probabilità il messaggio non sia mai uguale. In questo modo, anche se
il messaggio è sempre lo stesso, il cypertext è sempre diverso, utilizzando 
quindi la probabilistic encryption.
\section{Sicurezza di un crittosistema}
RSA è sicuro perché non conosciamo un algoritmo probabilistico polinomiale 
per calcolare la radice $e$-esima di un numero. Ma cosa vuol dire?
Se esistesse un algoritmo probabilistico polinomiale per calcolare la radice 
$e$-esima di un numero con probabilità $\frac{1}{k}$, sarebbe un problema,
perché reiterando l'algoritmo $k$ volte, avrei una probabilità di successo.

Se fossimo nello scenario in cui con un $a\in_R \mathbb{Z}_n^*$, l'algoritmo 
mi dia risposta corretta con probabilità $\frac{1}{k}$, potremmo dichiararci 
tranquilli? No, perché potrebbero attaccare sempre.

Fissando $a$ e scegliendo $r \in_R \mathbb{Z}_n^*$, calcolo $(a\cdot r)^e \mod n$,
se prendo un elemento casuale di $\mathbb{Z}_n^*$ e lo elevo ad $e$, ottengo
l'oggetto che è distribuito uniformemente in $\mathbb{Z}_n^*$. Sappiamo che
l'elevamento di $r$ alla $e$-esima è distribuito uniformemente in $\mathbb{Z}_n^*$, 
perché $e$ è stata scelta in maniera tale che la radice $e$-esima dia esattamente 
$r$. Quindi $r^e$ è una funzione invertibile, di conseguenza la funzione che mappa $r$ in 
$r^e$ è una funzione biettiva, quindi un elemento scelto uniformemente in $\mathbb{Z}_n^*$
viene mappato da $r^e$ in un elemento scelto uniformemente scelto in $\mathbb{Z}_n^*$.
Ogni volta che prendiamo un elemento e creiamo una suriezione dello stesso insieme, 
se l'elemento di partenza è scelto uniformemente, il risultato della suriezione, che nella 
sostanza è una permutazione, è scelto uniformemente.

Se prendiamo $r^e$ e lo moltiplichiamo per $a$, otteniamo un elemento che è distribuito
uniformemente in $\mathbb{Z}_n^*$, perché $a$ è scelto uniformemente in $\mathbb{Z}_n^*$, 
poiché la moltiplicazione per $a$ è una funzione biettiva, perché $a$ ammette inverso.

Siamo partiti da un elemento fissato e abbiamo costruito un elemento distribuito uniformemente
e causale, se a quell'elemento applichiamo l'algoritmo della radice $e$-esima, otteniamo
$\frac{1}{k}$ di probabilità di successo, ma se ripetiamo l'algoritmo $k$ volte, abbiamo
una probabilità di successo di $1$, rendendo quindi l'algoritmo indipendente dalla $a$ di 
partenza.
\[
  ar^e \rightarrow \sqrt[e]{ar^e} = r \sqrt[e]{a}
\]
Quindi se prendo il risultato e lo divido per $r$, ottengo $\sqrt[e]{a}$,
che è distribuito uniformemente in $\mathbb{Z}_n^*$, perché $r$ è distribuito uniformemente.

Ed ecco che abbiamo un algoritmo che a partire da una blackbox che con $a$ causale 
calcola correttamente la radice $e$-esima di $a$ una volta su $k$, abbiamo una macchina 
che con $a$ fissato calcola la radice $e$-esima di $a$ con probabilità $1$.

La macchina che calcola la radice $e$-esima di $a$ darà una sequenza di bit, che
potrebbe essere la radice $e$-esima di $a$, oppure no. Bisognerebbe riconoscere la risposta
corretta, rielevando il risultato ad $e$, se ottengo l'input allora la risposta è 
corretta, altrimenti no.

Quindi abbiamo trasformato un algoritmo che funziona una volta su $k$ in un algoritmo
che funziona in un tempo medio di $k$.
\begin{tcolorbox}[title=Definizione di sicurezza]
  Diciamo che un sistema è attaccabile se il tempo medio per attaccarlo è polinomiale.
\end{tcolorbox}
Se esistesse un qualsiasi algoritmo in grado di attaccare la radice $e$-esima di $a$,
con una probabilità polinomiale in $k$, riusciamo a costruire un algoritmo che calcola
la stessa cosa con un tempo medio polinomiale in $k$.

Visto che partiamo  dall'idea che non esista un algoritmo probabilistico polinomiale 
in grado di calcolare la radice $e$-esima di $a$, allora non esiste un algoritmo
che sia in grado di calcolarlo con una probabilità che sia polinomialmente piccola.
Quindi la \textbf{probabilità di successo è più piccola di qualsiasi polinomio}, dove per polinomio 
si intende:
\[
  \probP[\texttt{attacco}] < \frac{1}{k}\quad \forall c
\]
Fissando un polinomio, con chiavi corte, però, la possibilità di trovare un polinomio esiste, 
perciò bisogna correggere tale definizione.
\begin{equation}
  \forall c \exists \bar{k} \forall k > \bar{k}\quad\probP[\texttt{attacco}] < k^{-c}
\end{equation}
Per attacco non intendiamo solo il fatto di non poter essere in grado di poter calcolare la 
radice $e$-esima di $a$, ma anche il fatto di non essere in grado di capire 
\textbf{informazioni binarie}.

L'algoritmo che calcola la radice $e$-esima di $a$ è l'algoritmo che calcola la
la fattorizzazione di $n$, la fattorizzazione di $n$ è l'informazione 
binaria che vogliamo proteggere, ovvero la \textbf{trapdoor} che permette di risolvere 
il problema.
\begin{tcolorbox}[title=Fattorizzazione di $n$]
  Si pensa che non esista un algoritmo \texttt{PPT} che dati $n$, $e$ e $a$,
  calcola $\sqrt[e]{a} \in \mathbb{Z}_n^*$ con probabilità polinomiale.
\end{tcolorbox}
\subsection{Utilizzo pratico di RSA}
Nel caso pratico il costo computazionale di RSA è molto alto, infatti codificare 
un blocco di $k$ bit con RSA richiede $k$ esponenziazioni modulari, ovvero $k^3$, 
un costo computazionale molto alto. Per questo motivo RSA viene utilizzato per
codificare una chiave di sessione, che viene utilizzata per codificare
il messaggio con un algoritmo simmetrico, che è molto più veloce di RSA.
Tipicamente l'algoritmo utilizzato è l'algoritmo simmetrico \texttt{AES} (\ref{section:des}).

Tra l'altro vi è una notevole differenza con Diffie-Hellman, infatti in Diffie-Hellman
riesco a scambiarmi un'unica chiave, a meno che non faccia un nuovo scambio di chiavi,
ogni volta.

Se si parte dall'idea che la crittografia simmetrica sia meno sicura della crittografia
asimmetrica, allora si può pensare di utilizzare una chiave di sessione diversa dopo 
un certo periodo. Con RSA è possibile fare questo, perché è possibile scambiarsi
chiavi diverse, mentre con Diffie-Hellman non è possibile, perché si dovrebbero scambiare
chiavi diverse ogni volta, e questo è molto costoso. La generazione della chiave di sessione 
per Diffie-Hellman è molto costosa, per via delle Certification Authority, che devono
essere coinvolte nel processo di generazione della chiave di sessione per certificare 
le chiavi pubbliche.
\section{Crittosistema di Micali per la codifica di un singolo bit}
\subsubsection{Algoritmo di generazione delle chiavi}
Si sceglie un numero primo $p_1$ e un numero $p_2$ tale che moltiplicati tra loro
diano un numero $n$ tale che $n = p_1p_2$. I due numeri devono essere scelti in modo
casuale con $\frac{k}{2}$ bit ciascuno, dove $k$ è la lunghezza della chiave. Sia 
$y \in_R$ ai non quadrati con simbolo di Jacobi $1$ modulo $n$. Ricordiamo che per 
costruire un numero non quadrato causale con simbolo di Jacobi $1$ basta scegliere un numero
casuale e verificare che il simbolo di Jacobi sia $1$, ovvero che appartenga 
a $\mathbb{Z}_n^*$, se non lo è si sceglie un altro
numero casuale e si ripete il procedimento. A questo punto verifico che il simbolo 
di Legendre rispetto a $p_1$ e $q_1$ sia $-1$. 

La chiave pubblica è:
\[
  P_k = (n, y)
\]
La chiave privata è:
\[
  S_k = (p_1, p_2)
\]
L'ipotesi di base è che sia difficile fattorizzare $n$, ma il problema di 
riferimento sarà il problema del residuo quadratico, ovvero il problema di
calcolare la radice quadrata di un numero modulo $n$.
\subsubsection{Algoritmo di codifica}
L'algoritmo di codifica prende un bit $b$, sia $x \in_R \mathbb{Z}_n^*$, se 
$b$ è $0$ allora $c = x^2 \mod n$, altrimenti $c = yx^2 \mod n$. 
$x^2$ è un quadrato casuale di $\mathbb{Z}_n^*$, mentre $yx^2$ è un 
non quadrato con simbolo di Jacobi $1$. Se prendo un quadrato con simbolo di 
Jacobi $1$ e lo moltiplico per un non quadrato con simbolo di Jacobi $1$ ottengo
un non quadrato con simbolo di Jacobi $1$. Se il quadrato è casuale, allora 
ottengo un non quadrato casuale con simbolo di Jacobi $1$ distribuito uniformemente
tra tutti i non quadrati con simbolo di Jacobi $1$.

Il risultato è che la codifica di $0$ è un quadrato a caso, mentre la codifica di $1$ è
un non quadrato a caso con simbolo di Jacobi $1$.
\[
  \mathcal{E}: \{0, 1\} \to x \in_R \mathbb{Z}_n^*
\]
\[
  f(x) = \begin{cases}
    x^2 \mod n & \text{se } b = 0\\
    yx^2 \mod n & \text{se } b = 1
  \end{cases}
\]
\subsubsection{Algoritmo di decodifica}
L'algoritmo di verifica prende in input $c$ e verifica se $c$ è un quadrato 
rispetto a $p_1$ e $p_2$, ovvero se $c$ è un residuo quadratico modulo $p_1$ e
modulo $p_2$. Se $c$ è un quadrato rispetto a $p_1$ e $p_2$ allora $b = 0$,
se entrambe le verifiche falliscono allora $b = 1$, in altri casi non siamo in presenza 
di un cypertext valido.
\[
  \left(\frac{c}{p_1}\right) = \left(\frac{c}{p_2}\right) = 1 \qquad \text{allora } b = 0
\]
\[
  \left(\frac{c}{p_1}\right) = \left(\frac{c}{p_2}\right) = -1 \qquad \text{allora } b = 1
\]
\subsection{Rompere il crittosistema di Micali}
Rompere tale protocollo significherebbe disporre di un algoritmo $\mathcal{A}$ che preso in 
input in cyphertext $c$ e la chiave pubblica $P_k$ restituisce $b$ con probabilità
diversa da $\frac{1}{2}$, poiché siamo in un contesto binario.

Un attaccante quindi dovrebbe essere in grado di ottenere un vantaggio rispetto 
a qualcuno che non conosce nulla, ovvero che indovini con probabilità $\frac{1}{2}$.
Il vantaggio consiste nell'allontanarsi da $\frac{1}{2}$, sia in positivo che in negativo, 
poiché se si allontana in negativo basta invertire il risultato per ottenere
un vantaggio positivo.

Il numero di esperimenti deve essere tale che la differenza delle probabilità sia
maggiore di $\frac{1}{2}$, ma il numero di esperimenti deve essere un numero 
polinomiale, in modo tale da poter osservare tali esperimenti in tempo polinomiale.
\begin{tcolorbox}[title = Sicurezza]
  \begin{equation}
    \forall c \, \exists \bar{k} \, \forall k \geq \bar{k} \quad \mid \probP[\texttt{Successo}] - \frac{1}{2} \mid  > k^{-c}
  \end{equation}
\end{tcolorbox}
Supponiamo che la probabilità di successo sia maggiore di $\frac{1}{2} + \epsilon$ e vorrei che la probabilità di successo sia quindi 
prossima a $1$. Per farlo eseguo due tipologie di esperimenti, il primo ripete l'esperimento $k$ volte e mediante l'algoritmo che 
ha a disposizione il vantaggio è $\frac{1}{2} + \epsilon$, mentre il secondo esperimento ripete l'esperimento $k$ volte e mediante
esperimenti casuali, ovvero senza l'algoritmo, ottiene un vantaggio di $\frac{1}{2}$. Ripetendo l'esperimento un numero abbastanza grande di 
volte si ottiene il risultato desiderato, poiché basterebbe visualizzare le due distribuzioni per vedere in cosa differiscono. L'algoritmo 
quindi indovina con probabilità $1$. Più $\epsilon$ è piccolo, più esperimenti sono necessari per ottenere il risultato desiderato. 
Servirebbe quindi stimare, dato un $\epsilon$ fissato, il numero di esperimenti necessari per ottenere il risultato desiderato.
\subsubsection{Limite di Chernoff} \label{limite_chernoff}
\begin{tcolorbox}[title = Limite di Chernoff]
  Siano $X_1, \dots, X_n$ variabili casuali e binarie indipendenti con probabilità 
  di successo $\probP[X_i = 1] > \frac{1}{2}$ e $\probP[X_i = 0] = 1 - \probP[X_i = 1]$.
  La probabilità che più della metà delle variabili casuali siano $1$ è:
  \[
    \mathcal{P} = \sum_{i = \frac{n}{2} + 1}^n \binom{n}{i} \probP[X_i = 1]^i \probP[X_i = 0]^{n - i}
  \]
  \[
    \mathcal{P} \geq 1 - e^{-2n\left(p - \frac{1}{2}\right)^2}
  \]
  Tale formula dice che la probabilità che più della metà degli eventi dia $1$ è 
  esponenzialmente vicina a $1$, dove l'eponenzialmente è in funzione di $n$.
  \[
    \probP[\texttt{errore}] = e^{-2 \epsilon n}
  \]
  Dove $\epsilon$ è il vantaggio.
\end{tcolorbox}
Supponiamo di volere $e^{-2 \epsilon n} < \frac{1}{2^k}$, quindi:
\[
  e^{-2c\epsilon^2 n} < 2^{-k}
\]
\[
  -2\epsilon^2 n < c' - k 
\]
\[
  n > \frac{c' - k}{2\epsilon^2}
\]
Se $\epsilon$ è polinomiale in $k$ allora $n$ è polinomiale in $k$.
Di conseguenza, se il vantaggio è polinomiale in qualche security parameter, allora
si riesce ad ottenere una quantità di errore nel security parameter che è
esponenzialmente piccola, scegliendo una quantità di esperimenti polinomiale in esso.

Un sistema è attaccato nel momento in cui esiste un algoritmo polinomiale in 
grado di romperlo.

Nel momento in cui $\epsilon$ è un $k^{-c}$, allora $n$ (\textit{dove $n$ è il numero di
esperimenti}) è polinomiale in $k$.

Visto che l'ipotesi di partenza è che non esistano algoritmi probabilistici polinomiali in grado
di rompere il sistema (\textit{vero}) e visto che abbiamo dimostrato che esiste tale algoritmo (\textit{falso}), allora
l'algoritmo di Micali è sicuro.
\subsubsection{Costruzione degli esperimenti indipendenti}

Una volta capito che la costruzione di $n$ esperimenti indipendenti funziona, bisogna 
capire come costruirli. Supponiamo che la probabilità di successo sia $\frac{1}{2} + \epsilon$ e
di disporre di un algoritmo $\mathcal{A}$ che prende in input un numero $z$ 
con $\left(\frac{z}{n}\right) = 1$, in output restituisce che $z$ è quadrato oppure no.

Per farlo si sceglie $r_1, \dots, r_n \in_R \mathbb{Z}_n^*$ e
si calcola $w_i = z \cdot r_i^2$. L'algoritmo quindi prende in input $w_i$ e restituisce
$b_i$.
\[
  b_i = \mathcal{A}(w_i)
\]

In sostanza si prende in input un numero (\textit{cyphertext}) e l'algoritmo lo moltiplica per una 
quadrato a caso, quindi l'algoritmo restituisce $1$ se il risultato è un quadrato casuale e $0$
altrimenti.

Tale algoritmo però ha un difetto, ovvero che se $z$ è un quadrato, allora $w_i$ è un quadrato
sempre, quindi l'algoritmo restituisce sempre $1$, se invece $z$ non è un quadrato, allora
$w_i$ è un quadrato con probabilità $\frac{1}{2} + \epsilon$.
\subsubsection{Costruzione di un controesempio}
Supponiamo che $\mathcal{A}$ dica correttamente che il $40\%$ dei numeri quadrati sono quadrati e 
che il $62\%$ dei non quadrati sono non quadrati. Sostanzialmente l'algoritmo $\mathcal{A}$
ha una probabilità di successo a seconda dell'input che gli viene dato. 
\[
  \probP[\mathcal{A}(\texttt{n})\texttt{ successo}] = \frac{1}{2} \cdot \frac{40}{100} + 
  \frac{1}{2} \cdot \frac{62}{100} = \frac{51}{100} = 51\%
\]
L'approccio di costruzione degli esperimenti indipendenti non è corretto, perché 
non fornisce all'algoritmo $\mathcal{A}$ un input secondo la misura di probabilità
che $\mathcal{A}$ si aspetta, quando diciamo che ha una certa probabilità di successo.
L'esperimento corretto avviene solamente quando ad $\mathcal{A}$ viene dato un input
un oggetto che sia distribuito uniformemente tra gli oggetti con simbolo di Jacobi $1$.
\subsubsection{Costruzione degli esperimenti indipendenti corretta}
L'idea di base è quella di lanciare una moneta per decidere se invertire o meno la 
quadraticità di $z$ in modo da ottenere un input che sia distribuito uniformemente.
Siano $r_1, \dots, r_n \in_R \mathbb{Z}_n^*$ e siano $s_1, \dots, s_n \in_R \{0,1\}$.
\[
  \forall i \quad w_i = 
  \begin{cases}
    z \cdot r_i^2 & \text{se } s_i = 0 \\
    y \cdot z \cdot r_i^2 & \text{se } s_i = 1
  \end{cases}
\]
Sia $b_i = \mathcal{A}(w_i)$. In questo modo lasciamo una moneta $s_i$ che decide se mantenere 
la quadraticità di $z$ oppure no. Quindi $z \cdot r_i^2$ è un oggetto a caso tra gli oggetti con stessa 
quadraticità di $z$, mentre $y \cdot z \cdot r_i^2$ è un oggetto a caso tra gli oggetti con quadraticità 
opposta di $z$, di conseguenza $w_i$ è un oggetto a caso distribuito
uniformemente tra gli oggetti con simbolo di Jacobi $1$.
In questo caso quindi $\mathcal{A}$ ha una probabilità di successo di $\frac{1}{2} + \epsilon$, ma 
$b_i$ è la risposta al problema trasformato, ma non è la stessa del problema originale.
\[
  \forall i \qquad b_i' = 
  \begin{cases}
    b_i & \text{se } s_i = 0 \\
    \bar{b_i} & \text{se } s_i = 1
  \end{cases}
\]
perché se $s_i = 1$ allora nell'input ho invertito la quadraticità di $z$, di conseguenza la risposta deve essere 
a sua volta invertita per avere una risposta corretta nei confronti di $z$.

Tale costruzione funziona, poiché è possibile passare dall'algoritmo $\mathcal{A}'$
all'algoritmo $\mathcal{A}$, semplicemente invertendo la risposta quando $s_i = 1$, ma non
sempre ciò è attuabile, perché non sempre è possibile invertire la risposta.

L'algoritmo che calcola il residuo quadratico prende in input $z$ e restituisce $1$ se $z$ è un quadrato
e $0$ altrimenti, ma a tale algoritmo abbiamo dato in input $y$, ovvero un non quadrato con simbolo di Jacobi
$1$. Ma siamo davvero capaci di costruire un algoritmo che calcola un non quadrato con simbolo di Jacobi $1$
senza usare la fattorizzazione di $n$? La risposta è no, perché se fosse possibile allora sarebbe possibile
fattorizzare $n$.
\begin{figure}[H]
  \centering
  \begin{tikzpicture}
    % rettangolo con due input e un output
    \node[draw, rectangle, minimum width=4cm, minimum height=3cm] (A) at (0,0) {$\mathcal{A}'$};
    \draw[->] (-5, 0.5) -- (-2, 0.5) node[midway, above] {$z$};
    \draw[->] (-3, -0.5) -- (-2, -0.5) node[midway, below] {$y$};
    \draw[->] (2, 0) -- (5, 0) node[midway, above] {$\{0,1\}$};
    \node[draw, rectangle, minimum width=8cm, minimum height=5cm] (B) at (0,0) {};
    \node (AS) at (-3.2, 2) {$\mathcal{A}''$};
  \end{tikzpicture}
\end{figure}
Sappiamo quindi che la macchina funziona correttamente dato $y$, ma non disponiamo di tale valore.
Perché non prendere un $y$ a caso e verificare se è un quadrato? 

Sia $x \in_R \mathbb{Z}_n^*$ e sia $s \in_R \{0,1\}$, allora 
\[
  w =
  \begin{cases}
    x^2 & \text{se } s = 0 \\
    y \cdot x^2 & \text{se } s = 1
  \end{cases}
\]
Sia $b$ la risposta da utilizzare per l'algoritmo. Se per puro caso però $y$ fosse un quadrato, non riuscirei 
a cambiare la quadraticità di $w$ nel caso $s = 1$. Quindi nel caso in cui $s$ fosse 
uguale a $1$ e $y$ fosse un quadrato, allora $w$ sarebbe un quadrato, quindi la risposta dell'algoritmo sarebbe
sempre sbagliata, poiché verrebbe complementata la risposta pensando che $w$ non sia un quadrato. In ogni caso 
però è possibile sorvolare tale problema, poiché basta vedere come si comporta statisticamente 
la macchina in presenza di non quadrati e di quadrati. Il comportamento della macchina è per forza diverso di fronte 
alle due situazioni, perché altrimenti non sarebbe capace di distinguere i due input.