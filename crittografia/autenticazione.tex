\chapter{Autenticazione}
\section{Autenticazione di messaggi}
I crittosistemi costruiti fino ad ora sono malleabili, quando riceviamo un messaggio vorremmo
essere sicuri che nessuno l'abbia modificato. Per farlo possiamo usare sistemi non malleabili
dove le modifiche non sono prevedibili, oppure aggiungere della ridondanza, aggiungendo
bit di parità o un codice ciclico di ridondanza.

Possedendo però un messaggio cifrato e la sua decifratura, o dalla chiave pubblica,
è possibile modificare un messaggio 
cifrato attraverso il non quadrato ricavato, alterando qualsiasi bit in maniera controllata, e 
quindi modificare in maniera controllata anche i codici di ridondanza.

Per garantire l'integrità e l'autenticità dei messaggi trasmessi, è fondamentale adottare sistemi di autenticazione avanzati. Uno di questi approcci consiste nell'utilizzare una funzione casuale condivisa tra mittente e destinatario.

Quando un mittente invia un messaggio, genera una sequenza di bit aggiuntiva chiamata
Message Authentication Code (\textit{MAC}). Questo \textit{MAC} viene ottenuto applicando una
funzione casuale precedentemente concordata al messaggio stesso. La casualità della
funzione è cruciale e deve essere condivisa solo tra le parti interessate, rimanendo
sconosciuta agli altri.

Il destinatario, una volta ricevuto il messaggio, applica la stessa funzione casuale
al messaggio ricevuto per calcolare un nuovo \textit{MAC}. Se il \textit{MAC} calcolato corrisponde a
quello ricevuto con il messaggio, ciò indica che il messaggio non è stato modificato
in modo controllato e l'autenticazione è riuscita.

L'uso di una funzione casuale impedisce a un potenziale aggressore di prevedere la
sequenza di bit aggiuntiva al messaggio. Inoltre, permette di verificare l'autenticità
del messaggio, poiché solo il mittente e il destinatario conoscono la funzione casuale
utilizzata.

Un possibile attaccante che dispone di un messaggio, non sarà in grado di produrre 
il codice di autenticazione corretto, poiché il risultato della applicazione di una funzione 
causale è una sequenza di bit casuale, e quindi non prevedibile e distinguibile.

\section{Funzioni pseudo-casuali}
Avendo imparato a generare bit pseudocasuali, siamo in grado di generare funzioni pseudo-casuali,
ovvero funzioni che non sono realmente casuali, ma sono generate da un seme, ma agli occhi 
di un osservatore esterno sono casuali.

Sia $\mathcal{U}_k$ l'insieme delle funzioni da $\{0,1\}^k$ a $\{0,1\}^k$, 
vogliamo scegliere $f$ appartenente a $\mathcal{U}_k$ in maniera casuale.

Possiamo generare una sequenza di bit pseudocasuali per generare l'indice di $f$ in 
una enumerazione di $\{0,1\}^k \to \{0,1\}^k$. Dalla teoria del calcolo combinatorio sappiamo
che la cardinalità di delle funzioni da $A$ a $B$ è $|B|^{|A|}$, quindi la cardinalità
di $\mathcal{U}_k$ è $\{0,1\}^{k\cdot \{0,1\}^k} = 2^{k\cdot 2^k}$.
Per denotare l'indice della funzione avremmo quindi bisogno di $\log_2(2^{k\cdot 2^k}) = k\cdot 2^k$ bit,
ovvero una quantità di bit esponenziale, cosa che non ci possiamo permettere.

Ci farebbe comodo denotare una funzione con $k$ bit, il problema è che con i nostri indici dovremmo 
denotare un insieme di funzioni minori dell'insieme delle funzioni in $\mathcal{U}_k$, dobbiamo 
quindi lavorare con un insieme $\mathcal{F}_k \subseteq \mathcal{U}_k$. Vogliamo comunque 
far in modo che chiunque non sia in grado di accorgersi che sia stata scelta una funzione
da un insieme più piccolo.

In un generatore di funzioni pseudocasuali, partiamo dall'idea che:
\begin{enumerate}
    \item Troviamo un insieme di funzioni
    $\mathcal{F}_k \subseteq \mathcal{U}_k$, tale che $|\mathcal{F}_k| = 2^k$; imponendo per definizione
    tale dimensione, siamo in grado di generare funzioni casuali avendo a disposizione $k$ bit;
    \item Esiste un algoritmo probabilistico polinomiale che calcola la funzione $(i,x) \mapsto f_i(x)$;
    \item Sia $\mathcal{D}\in \mathcal{PPT}$ un algoritmo che, su input $1^k$ (\textit{il 
    security parameter rappresentato in unario, nel valore, ovvero nel numero di bit che
    usiamo per rappresentare le nostre informazioni}), che interroga $f$ aribitrariamente,
    anche in maniera adattiva (\textit{scegliendo l'argomento su cui interrogare la funzione 
    $f$ anche sulla base dei risultati ottenuti dalle precedenti interrogazioni, ma sempre in 
    tempo polinomiale}). Siano $\mathcal{P}_k^{\mathcal{D}, \mathcal{U}}$ la probabilità che 
    $\mathcal{D}$ restituisca $1$ quando $f\in_R \mathcal{U}_k$ e $\mathcal{P}_k^{\mathcal{D}, \mathcal{F}}$
    la probabilità che $\mathcal{D}$ restituisca $1$ quando $f\in_R \mathcal{F}_k$, allora 
    \[
        \left| \mathcal{P}_k^{\mathcal{D}, \mathcal{U}} - \mathcal{P}_k^{\mathcal{D}, \mathcal{F}} \right| \leq k^{-\omega(1)}
    \]
\end{enumerate}
\subsection{Costruzione delle funzioni pseudo-casuali}
\subsubsection{Primo metodo}
Sia $\mathcal{G}$ un psudo random sequence generator (\texttt{PRSG}), ovvero un algoritmo che
genera una sequenza di bit pseudocasuali.
\begin{figure}[H]
    \centering
    \begin{tikzpicture}
      \node[draw, rectangle, minimum width=4cm, minimum height=3cm] (A) at (0,0) {$\mathcal{G}$};
      \draw[->] (-5, 0) -- (-2, 0) node[midway, above] {$i$};
      %freccia ondulata
      \draw[decorate, decoration={snake, amplitude=1mm}] (2, 0) -- (10, 0);
        \draw[-] (4, 0.2) -- (4, -0.2);
        \draw[-] (6, 0.2) -- (6, -0.2);
        \draw[-] (8, 0.2) -- (8, -0.2);
        \draw[-] (10, 0.2) -- (10, -0.2);

        %k bit
        \draw (3, 0.5) node {$k$};
        \draw (3, -0.5) node {$x$};
        \draw (5, 0.5) node {$k$};
        \draw (5, -0.5) node {$y$};
    \end{tikzpicture}
  \end{figure}
Su input $x$ generiamo $k$ bit e li restituiamo. Oltre a restituirli, ci ricordiamo che $x$ è 
associata ai primi $k$ bit generati dal \texttt{PRSG}. Sul successivo input $y$ se $y = x$
restituiamo il risultato precedente, altrimenti generiamo $k$ bit nuovi e li restituiamo.

Ogni volta che abbiamo un nuovo input, generiamo $k$ bit nuovi, altrimenti restituiamo il risultato
fornito in precedenza. 
Il problema di fondo abbastanza grosso è che una volta che abbiamo definito il seme $i$, 
la funzione è definita in maniera univoca da $i$? No, perché la funzione dipende dall'ordine 
in cui viene interrogata. Quindi passando prima $x$ e poi $y$ oppure prima $y$ e poi $x$
potremmo ottenere risultati diversi.

Ma agli occhi di chi osserva, siamo in presenza di una funzione casuale, se scegliamo un numero 
nuovo vediamo un risultato scelto uniformemente a caso, se scegliamo un numero già visto,
vediamo il risultato che abbiamo visto in precedenza.

Tuttavia questa costruzione non soddisfa la definizione che abbiamo dato, quindi possiamo 
far in modo che la definizione ammetta questa costruzione per la generazione di funzioni
pseudo-casuali. Ma per la costruzione che abbiamo in mente, ovvero per l'autenticazione di 
messaggi, questa costruzione non è adatta.
\[
  m, f(m)
\]
Chiunque non conosca $f$, e veda in $f(m)$ una sequenza casuale, 
e non sia in grado di calcolare $f(m')$ per $m' \neq m$. L'autenticazione di 
un messaggio $m'$ è possibile crearla solo con $f$.
\pseudocodeblock{
\begin{tikzpicture}
  \node[businessman, monitor, female, minimum size=2cm,label=below:{Alice}] (alice) {};
\end{tikzpicture}
\<\<
\begin{tikzpicture}
  \node[dave, monitor, mirrored, minimum size=2cm, label=below:{Bob}] (bob) {};
\end{tikzpicture}
\\[0.1\baselineskip][\hline] \\[-0.5\baselineskip]
\text{Dispone }i\<\<\text{Dispone }i \\
\<\sendmessageright*{x, f(x)} \\
\<\sendmessageleft*{y, f(y)} \\
}
Se le cose vanno esattamente come nel diagramma rappresentato, 
allora tutto va secondo i piani, poiché ogni agente riesce a verificare l'autenticità
di ogni messaggio.
\pseudocodeblock{
\begin{tikzpicture}
  \node[businessman, monitor, female, minimum size=2cm,label=below:{Alice}] (alice) {};
\end{tikzpicture}
\<\<
\begin{tikzpicture}
  \node[dave, monitor, mirrored, minimum size=2cm, label=below:{Bob}] (bob) {};
\end{tikzpicture}
\\[0.1\baselineskip][\hline] \\[-0.5\baselineskip]
\text{Dispone }i\<\<\text{Dispone }i \\
\<\sendmessagerightleft*{y, f(y)\qquad x, f(x)} \\
}
Scambiandosi i messaggi in maniera sincrona, poiché Alice utilizza il generatore 
e la prima volta che lo invoca, lo invoca su $x$, mentre Bob lo invoca su $y$,
Alice e Bob si trovano in una situazione in cui non sono in grado di verificare
poiché $f(x) = f(y)$.

Nel momento in cui la funzione $f$ dipende dall'ordine di interrogazione,
non è possibile distribuire la funzione $f$ a più agenti, e ci vorrebbe 
un oggetto centralizzato che distribuisce la funzione $f$ a tutti gli agenti.

Tale costruzione è utile in caso di generazione di funzioni pseudocasuali 
in cui l'utilizzo è locale, tuttavia la crescita di richieste fa si che 
la dimensione del database cresca e quindi allunghi i tempi di risposta e 
quindi ci si accorga della pseudocasualità nella generazione.
\subsubsection{Secondo metodo}
\begin{figure}[H]
  \centering
  \begin{tikzpicture}
    \node[draw, rectangle, minimum width=4cm, minimum height=3cm] (A) at (0,0) {$\mathcal{G}$};
    \draw[->] (-5, 0) -- (-2, 0) node[midway, above] {$i$};
    %freccia ondulata
    \draw[decorate, decoration={snake, amplitude=1mm}] (2, 0) -- (10, 0);
      \draw[-] (4, 0.2) -- (4, -0.2);
      \draw[-] (6, 0.2) -- (6, -0.2);
      \draw[-] (8, 0.2) -- (8, -0.2);
      \draw[-] (10, 0.2) -- (10, -0.2);

      %k bit
      \draw (3, 0.5) node {$n$};
      \draw (3, -0.5) node {$0$};
      \draw (5, 0.5) node {$n$};
      \draw (5, -0.5) node {$1$};
      \draw (7, 0.5) node {$n$};
      \draw (7, -0.5) node {$\dots$};
      \draw (9, 0.5) node {$n$};
      \draw (9, -0.5) node {$\dots$};
  \end{tikzpicture}
\end{figure}
Dove $f_i$ è definita come l'\textit{x-esimo gruppo di $n$ bit restituito da $\mathcal{G}$}.
In questo caso l'insieme $f_i$ è definito in maniera univoca. Per ogni $i$ c'è 
una funzione $f_i$ definita in maniera univoca, quindi l'insieme $\mathcal{F}_k$ è 
l'insieme delle funzioni $f_i$ definite nella maniera appena descritta.

Il problema sorge nella condizione che indica che esiste un algoritmo probabilistico
polinomiale tale che $(i,x) \mapsto f_i(x)$. Sia $x=1...1 = 2^{k-1} - 1$, per come è 
definito il generatore, per farlo dobbiamo generare tutti i numeri da $0$ a $2^{k-1} - 1$,
quindi i bit totali da generare sono $k \cdot 2^{k-1}$, quindi il tempo di esecuzione
non è polinomiale.

\subsubsection{Terzo metodo}

\begin{figure}[H]
  \centering
  \begin{tikzpicture}
    \node[draw, rectangle, minimum width=4cm, minimum height=3cm] (A) at (0,0) {$\mathcal{G}$};
    \draw[->] (-5, 0) -- (-2, 0) node[midway, above] {$i \oplus x$};
    %freccia ondulata
    \draw[decorate, decoration={snake, amplitude=1mm}] (2, 0) -- (4, 0);
      \draw[-] (4, 0.2) -- (4, -0.2);

      %k bit
      \draw (3, 0.5) node {$k$};
      
  \end{tikzpicture}
\end{figure}
Come seme usiamo $i \oplus x$, e prendiamo i primi $k$ bit del risultato. Quindi 
$f_i(x)$ è definita come i primi $k$ bit di $\mathcal{G}(i \oplus x)$.
Sappiamo che ogni 
indice $i$ determina una funzione $f_i$ definita in maniera univoca.
Inoltre l'algoritmo che genera $f_i$ è polinomiale, poiché per generare $f_i$
è sufficiente eseguire lo \texttt{XOR} tra $i$ e $x$ e, che è un'operazione polinomiale,
e poi generare con il generatore $\mathcal{G}$ i primi $k$ bit del risultato, e sappiamo che 
$\mathcal{G}$ è un algoritmo polinomiale.

Dobbiamo capire se esiste distinguisher in grado di distinguere $f_i$ da una funzione
casuale. Supponiamo che esista $\mathcal{D}$ tale per cui 
$\left| \mathcal{P}_k^{\mathcal{D}, \mathcal{U}} - \mathcal{P}_k^{\mathcal{D}, \mathcal{F}}
\right| > k^{-\omega(1)}$, vorremmo trasformare $\mathcal{D}$ in un algoritmo
che viola la correttezza del generatore $\mathcal{G}$.

Per farlo dobbiamo assicurarci che il distinguisher $\mathcal{D}$ stia facendo 
implicitamente su $\mathcal{G}$, tutti esperimenti che siano compatibili con quello 
che abbiamo detto per definire la correttezza di $\mathcal{G}$.

Nel caso specifico il distinguisher $\mathcal{D}$ può scegliere 
valori distinti di $x$, ma il generatore $\mathcal{G}$ conosce tali 
valori $x_1, x_2, \dots, x_n$, prendendo lo \texttt{XOR} di uno dei 
due valori in input a $\mathcal{G}$, otteniamo lo \texttt{XOR} 
del seme usato in $\mathcal{G}$, allora il distinguisher è in grado 
di osservare l'output di $\mathcal{G}$ su due semi di cui è nota 
la differenza.

Se interroghiamo $f(0) = a$ e $f(1^k)= b$, sappiamo che il seme
usato per generare $a$ è esattamente il complemento del seme usato 
per generare $b$. In generale se $f(x) = a'$ e $f(y)=a''$, allora
sappiamo che \texttt{XOR} dei semi usati per generare $a'$ e $a''$
è esattamente $x \oplus y$. Gli esperimenti che il distinguisher
riesce a fare implicitamente su $\mathcal{G}$ sono esperimenti 
che il distinguisher che abbiamo usato per definire la correttezza
su $\mathcal{G}$ non è in grado di fare.
Quindi il distinguisher che sta lavorando sulle funzioni pseudocasuali 
è più potente del distinguisher ammesso nella definizione di
correttezza di $\mathcal{G}$, quindi abbiamo una contraddizione.

In questo caso i casi possono essere due:
\begin{itemize}
  \item Abbiamo definito un generatore troppo debole, poiché 
  abbiamo bisogno di una definizione più forte, che ammetta 
  esperimenti di questo tipo;
  \item Possiamo far vedere che se abbiamo un generatore di bit 
  pseudocasuali, allora tale generatore resiste ad attacchi di questo
  tipo.
\end{itemize}
In questo caso è possibile costruire un generatore di bit pseudocasuali
$\mathcal{G}$ e che se usato in questo contesto permette di costruire un 
distinguisher $\mathcal{D}$.

Incapsuliamo il generatore in un generatore più grande $\mathcal{G}'$, che prende in 
input $i_0, i_1, \dots, i_{k-1}$. Se $i_0 = 0$ allora si inviano 
$i_1, i_2, \dots, i_{k-1}$ al generatore $\mathcal{G}$, altrimenti
si invia $\overline{i_1}, \overline{i_2}, \dots, \overline{i_{k-1}}$
al generatore $\mathcal{G}$.
Scegliendo un seme a caso usiamo $\mathcal{G}$ con un seme a caso,
poiché il complemento di una sequenza di bit casuali è sempre una 
sequenza casuale, quindi stiamo generando una sequenza di bit casuali con 
security parameter $k-1$.

Il nuovo generatore $\mathcal{G}'$ è un generatore di bit pseudocasuali
che soddisfa la definizione di correttezza. L'output di $\mathcal{G}'$
è effettivamente un seme distribuito 
uniformemente tra tutti i possibili semi, e quindi si comporta 
come $\mathcal{G}$. Tuttavia usando tale oggetto nel contesto 
riportato sopra, scopriamo che $f(0) = f(1^k)$. Il distinguisher 
$\mathcal{D}$ in questo caso riusciamo quindi a costruirlo facilmente,
poiché basta che controlli se $f(0) = f(1^k)$, e se questo è vero
allora $\mathcal{D}$ restituisce $1$, altrimenti restituisce $0$. Di fronte 
ad una funzione pseudocasuale risponderà sempre $1$, di fronte ad 
una funzione casuale risponderà $1$ solamente quando $f(0) = f(1^k)$,
ma tale uguaglianza avviene con probabilità $2^{-k}$, perché 
è la probabilità che due sequenze di $k$ bit siamo uguali, la differenza 
essendo enorme distingue con probabilità $1$.

La soluzione alla quale vogliamo ambire è una soluzione che cercherà 
di evitare il problema.
\subsubsection{Quarto metodo}
Torniamo al secondo metodo, dove $f_i(x)$ è l'$x-esimo$ gruppo di 
$\mathcal{G}$, eravamo rimasti al fatto che non riuscissimo a calcolare 
$f_i(x)$ in tempo polinomiale. Per poter calcolare $f_i(x)$ in tempo
polinomiale, usiamo un generatore di Blum-Blum-Shub, lavorando con il 
logaritmo discreto non siamo in grado di calcolare elementi molto distanti,
ma se dobbiamo calcolare $f_i(2^k - 1)$ servono solamente i bit dalla 
posizione $k \cdot (2^k - 1) - 1$ a $k \cdot k \cdot 2^k$.

\begin{figure}[H]
  \centering
  \begin{tikzpicture}
    \node[draw, rectangle, minimum width=4cm, minimum height=3cm] (A) at (0,0) {$\texttt{BBS}$};
    \draw[->] (-5, 0) -- (-2, 0) node[midway, above] {$i$};
    %freccia ondulata
    \draw[decorate, decoration={snake, amplitude=1mm}] (2, 0) -- (10, 0);
      \draw[-] (4, 0.2) -- (4, -0.2);
      \draw[-] (6, 0.2) -- (6, -0.2);
      \draw[-] (8, 0.2) -- (8, -0.2);
      \draw[-] (10, 0.2) -- (10, -0.2);

      %k bit
      \draw (3, 0.5) node {$n$};
      \draw (3, -0.5) node {$0$};
      \draw (5, 0.5) node {$n$};
      \draw (5, -0.5) node {$1$};
      \draw (7, 0.5) node {$n$};
      \draw (7, -0.5) node {$\dots$};
      \draw (9, 0.5) node {$n$};
      \draw (9, -0.5) node {$2^{k-1}$};
  \end{tikzpicture}
\end{figure}

Per calcolare tali bit, avendo a disposizione il seme $i$, per 
calcolare il bit $k \cdot (2^k - 1) - 1$ dobbiamo calcolare
$i^{k \cdot (2^k - 1) - 1}$, che è abbastanza complicato da calcolare.
Tuttavia gli esponenti lavorano in modulo $\phi(n)$ e una volta 
calcolato l'esponente possiamo lavorare in modulo $n$, quindi
possiamo calcolare $i^{(k \cdot (2^k - 1) - 1)\mod \phi(n)} \mod n$,
in tempo polinomiale $k^3$ dove $k$ è il numero di bit.
Di conseguenza l'oggetto è calcolabile in tempo polinomiale,
quindi abbiamo un algoritmo \texttt{PPT} che è in grado di calcolare 
$f_i(x)$.

Questa operazione non è possibile con tutti i generatori, il 
generatore di Blum-Blum-Shub lavora in un gruppo di cardinalità 
finita ed ha una struttura particolare che permette di fare
questo tipo di operazioni.

Dobbiamo vedere se tale oggetto soddisfa la terza proprietà di un 
oggetto pseudocasuale, ovvero che sia indistinguibile da una funzione
realmente casuale. Il distinguisher $\mathcal{D}$ interroga 
la funzione $f$ e nell'interrogare la funzione $f$ sta implicitamente 
facendo degli esperimenti su $\mathcal{G}$, ma gli esperimenti possono essere 
fatti in tempo polinomiale, quindi un numero esponenziale di bit non 
può essere vista dal distinguisher. Quindi con il generatore di funzioni 
pseudocasuali siamo in grado di vedere dei bit che il distinguisher non è 
in grado di vedere.

In questa situazione però non siamo riusciti a costruire un controesempio.

\subsubsection{Quinto metodo}
Usiamo l'indice della funzione $i$ di $k$ bit, come seme e di base usiamo un 
generatore pseudocasuale $\mathcal{G}$, per generare $2k$ bit,
$\mathcal{G}(s)[0,\dots,k-1]$ ovvero $s_0$ e lo stesso per $\mathcal{G}(s)[k,\dots,2k-1]$
ovvero $s_1$. e ripetiamo ricorsivamente il procedimento generando $s_{00}, s_{01}, s_{10}$
e $s_{11}$, che sono rispettivamente $s_0[0,\dots,k-1], s_0[k,\dots,2k-1], s_1[0,\dots,k-1]$.
\begin{figure}[H]
  \centering
  \begin{tikzpicture}[level distance=1.5cm,
  level 1/.style={sibling distance=3cm},
  level 2/.style={sibling distance=1.5cm}]
    \node {$s$}
      child { 
        node {$s_0$}
        child {
          node {$s_{00}$}
        }
        child {
          node {$s_{01}$}
        }
      }
      child {
        node {$s_1$}
        child {
          node {$s_{10}$}
        }
        child {
          node {$s_{11}$}
        }
      };
  \end{tikzpicture}
\end{figure}
Scendendo nell'albero, arriviamo al seme indicizzato da $x$, ovvero
$s_x$, che è un seme di $k$ bit, che è il risultato della funzione
$f_i(x)$. Ovviamente questo albero ha $2^k$ nodi e le foglie rappresentano i 
valori delle funzioni sui diversi argomenti a partire da $s$ in radice,
quindi la funzione $f_s(x)$ ovvero $f_i(x)$ è ben definita.

Per calcolare tale funzione in tempo polinomiale, dobbiamo calcolare
calcolare solamente un cammino dell'albero, quindi il tempo di calcolo
è polinomiale. Infatti per calcolare un nodo dell'albero nel caso pessimo 
calcoliamo $2k$ bit per $k$ volte a cui si moltiplica il tempo di generazione 
di $2k$ bit dal generatore di Blum-Blum-Shub, che esegue un elevamento al 
quadrato per ogni bit, quindi il tempo di calcolo è
$\theta(k^4)$.

Dobbiamo verificare che non esista un distinguisher $\mathcal{D}$ che
sia in grado di distinguere la funzione $f_s(x)$ da una funzione
realmente casuale. Creiamo un distinguisher $\mathcal{D}$ che 
interroga la funzione $f_s(x)$; implicitamente sta facendo degli esperimenti
sul generatore $\mathcal{G}$, in maniera nidificata.
Guardando l'albero a livelli, un generatore pseudocasuale trasforma una 
certa quantità di casualità in una quantità di casualità diversa.

Definiamo gli alberi:
\begin{align*}
  \begin{array}{r c lllllllll}
  \mathcal{T}_0 & = & s^1 &\dots &s^{i-1} & s^i &s^{i+1} & s^{i+2} &\dots &s^l \\
  \mathcal{T}_{l}& = & r^1 &\dots &r^{i-1} & r^i &r^{i+1} & r^{i+2} &\dots &r^l \\
  \end{array}
\end{align*}
Dove $\mathcal{T}_0$ è l'albero che rappresenta esattamente l'albero definito 
per la generazione di funzioni pseudocasuali, mentre $\mathcal{T}_l$ corrispondente ad una funzione 
casuale campionata uniformemente dell'insieme $\mathcal{U}_k$.

Il distinguisher secondo la nostra ipotesi vede la differenza tra un albero 
costruito secondo la prima tecnica e un albero costruito secondo la seconda.

\[
  \mathcal{P}_k^{\mathcal{D}, \mathcal{G}} - \mathcal{P}_k^{\mathcal{D}, \mathcal{U}} > k^{-c}
  \qquad\text{quindi}\qquad
  \mathcal{P}(0) - \mathcal{P}(k) > k^{-c}
\]

In questo caso costruiamo tutti gli alberi intermedi dove 
per ogni livello $i= 0, \dots, k$ definiamo l'albero $\mathcal{T}_i$ tale che:
\begin{itemize}
  \item I nodi dei livelli $0, \dots, i$ sono tutti casuali;
  \item I nodi dei livelli $i+1, \dots, k$ sono tutti pseudocasuali.
\end{itemize}
Con la solita tecnica di interpolazione, possiamo affermare che se il distinguisher è
in grado di rilevare la differenza tra un albero costruito secondo lo schema $\mathcal{T}_0$
e un albero costruito secondo lo schema $\mathcal{T}_k$, allora può individuare la differenza
tra livelli adiacenti mediante interpolazione. Poiché due livelli adiacenti differiscono solamente
per un livello specifico, otteniamo due alberi distinti che si differenziano solo per quel livello
particolare. Pertanto, calcolando $f_s(x)$ risulterebbe equivalente a retrocedere al nodo $i$-esimo
dell'albero $\mathcal{T}$ e osservarlo casualmente in un caso e pseudocasualmente nell'altro, con un
seme generato al livello precedente in maniera veramente casuale. L'esperimento di calcolare $f_s(x)$
è quindi il classico test su un generatore di bit pseudocasuali.
\[
  \exists i \in \{0, \dots, k-1\} \quad \big| \mathcal{P}(i) - \mathcal{P}(i+1) \big| > k^{-c'}
\]
Effettuando un esperimento su $f_s(y)$ che proviene dall'istanza dello stesso nodo $i$-esimo dell'albero
$\mathcal{T}$, possiamo considerarlo come un'istanza dello stesso esperimento, ossia un esperimento sui
bit di livello $i$ casuali o pseudocasuali. Se l'esperimento è condotto su $f_s(z)$ proveniente da una
nuova istanza di un nodo $i$-esimo dell'albero $\mathcal{T}$, allora può essere considerato un nuovo
esperimento indipendente, consentendo la realizzazione di più esperimenti indipendenti.

Se la probabilità di successo con un singolo esperimento è inferiore a qualsiasi polinomio,
la probabilità di successo
con un numero polinomiale di esperimenti indipendenti è comunque inferiore a qualsiasi polinomio.

Di conseguenza un esistenza di un attaccante al generatore di funzioni si traduce 
in un attaccante al generatore di bit pseudocasuali $\mathcal{G}$, quindi 
il distinguisher non esiste se assumiamo  che il generatore di partenza soddisfi 
la definizione.