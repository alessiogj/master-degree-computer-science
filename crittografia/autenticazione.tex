\chapter{Autenticazione}
\section{Autenticazione di messaggi}
I crittosistemi costruiti fino ad ora sono malleabili, quando riceviamo un messaggio vorremmo
essere sicuri che nessuno l'abbia modificato. Per farlo possiamo usare sistemi non malleabili
dove le modifiche non sono prevedibili, oppure aggiungere della ridondanza, aggiungendo
bit di parità o un codice ciclico di ridondanza.

Possedendo però un messaggio cifrato e la sua decifratura, o dalla chiave pubblica,
è possibile modificare un messaggio 
cifrato attraverso il non quadrato ricavato, alterando qualsiasi bit in maniera controllata, e 
quindi modificare in maniera controllata anche i codici di ridondanza.

Per garantire l'integrità e l'autenticità dei messaggi trasmessi, è fondamentale adottare sistemi di autenticazione avanzati. Uno di questi approcci consiste nell'utilizzare una funzione casuale condivisa tra mittente e destinatario.

Quando un mittente invia un messaggio, genera una sequenza di bit aggiuntiva chiamata
Message Authentication Code (\textit{MAC}). Questo \textit{MAC} viene ottenuto applicando una
funzione casuale precedentemente concordata al messaggio stesso. La casualità della
funzione è cruciale e deve essere condivisa solo tra le parti interessate, rimanendo
sconosciuta agli altri.

Il destinatario, una volta ricevuto il messaggio, applica la stessa funzione casuale
al messaggio ricevuto per calcolare un nuovo \textit{MAC}. Se il \textit{MAC} calcolato corrisponde a
quello ricevuto con il messaggio, ciò indica che il messaggio non è stato modificato
in modo controllato e l'autenticazione è riuscita.

L'uso di una funzione casuale impedisce a un potenziale aggressore di prevedere la
sequenza di bit aggiuntiva al messaggio. Inoltre, permette di verificare l'autenticità
del messaggio, poiché solo il mittente e il destinatario conoscono la funzione casuale
utilizzata.

Un possibile attaccante che dispone di un messaggio, non sarà in grado di produrre 
il codice di autenticazione corretto, poiché il risultato della applicazione di una funzione 
causale è una sequenza di bit casuale, e quindi non prevedibile e distinguibile.

\section{Funzioni pseudo-casuali}