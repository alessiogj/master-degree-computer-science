\chapter{Teoria dei numeri}
\section{Proprietà dei numeri}

La teoria dei numeri è una branca fondamentale della matematica che studia le proprietà degli interi e delle loro relazioni. Nel contesto della teoria dei numeri, diversi concetti chiave emergono dall'analisi delle operazioni e degli insiemi numerici.

\begin{tcolorbox}[title={Operazioni chiuse}]
Un'operazione si dice chiusa se l'operazione applicata a due numeri naturali restituisce un numero naturale.
\[
  \mathbb{N} \, \texttt{op} \, \mathbb{N} \to \mathbb{N}  
\]
Ad esempio, l'addizione e la moltiplicazione sono operazioni chiuse sugli interi.
\end{tcolorbox}

Quando un'operazione non è chiusa? La divisione, non produce sempre un numero naturale. Gli insiemi sono inoltre caratterizzati da proprietà che li rendono unici.

\begin{tcolorbox}[title={Proprietà commutativa}]
  \[
    a \, \texttt{op} \, b = b \, \texttt{op} \, a
  \]
L'addizione e la moltiplicazione sono esempi di operazioni che soddisfano la proprietà commutativa.
\end{tcolorbox}

\begin{tcolorbox}[title={Proprietà associativa}]
  \[
    \forall a, b, c \in A \qquad a \, \texttt{op} \, (b \, \texttt{op} \, c) = (a \, \texttt{op} \, b) \, \texttt{op} \, c
  \]
La proprietà associativa è verificata dall'addizione e dalla moltiplicazione su diversi insiemi numerici.
\end{tcolorbox}

\begin{tcolorbox}[title={Elemento neutro}]
  \[
    a \, \texttt{op} \, e = a
  \]
  quindi 
  \[
    \exists e \in A \texttt{ t.c. }\forall a \in A \, a \, \texttt{op} \, e = e \, \texttt{op} \, a = a
  \]
L'elemento neutro per l'addizione è lo zero, mentre per la moltiplicazione è l'unità.
\end{tcolorbox}

\begin{tcolorbox}[title={Elemento inverso}]
  \[
    \forall a \in A \, \exists a^{-1} \in A \texttt{ t.c. } a \, \texttt{op} \, a^{-1} = a^{-1} \, \texttt{op} \, a = e
  \]
Alcuni esempi di elementi inversi includono l'opposto di un numero per l'addizione e il reciproco di un numero non nullo per la moltiplicazione.
\end{tcolorbox}

Ogni volta che definiamo delle strutture matematiche che soddisfano tali proprietà, si parla di gruppi. Un \textbf{gruppo} è
una struttura algebrica che rispetta determinate regole, tra cui chiusura, associatività, presenza di un elemento neutro e di un elemento inverso.

Prendiamo in considerazione i numeri naturali rispetto all'operazione di somma. Hanno l'inverso? No, quindi non è un gruppo. Tuttavia,
se aggiungessimo altri elementi e arrivassimo ai numeri interi, allora avremmo un gruppo rispetto alla somma. Allo stesso modo, se
considerassimo i numeri interi rispetto alla moltiplicazione (\textit{senza lo zero}), non avremmo l'inverso. In questo caso, dovremmo aggiungere
gli inversi e arrivare ai numeri razionali, che costituiscono un gruppo rispetto alla moltiplicazione.

Capita spesso di avere strutture che non sono gruppi. In questi casi, si possono seguire due approcci: ridurre la struttura eliminando
gli elementi che non soddisfano le proprietà del gruppo o espandere la struttura aggiungendo elementi in modo da soddisfare tali proprietà.

Vorremmo lavorare su un'algebra diversa, possibilmente con gruppi finiti. In particolare, ci concentreremo su gruppi che costituiscono
l'algebra alla base dei nostri algoritmi crittografici. Utilizzeremo funzioni che sono facili da calcolare ma difficili da invertire,
comunemente conosciute come \textbf{one-way function}. Queste funzioni si basano sull'algebra moltiplicativa.

Successivamente, esamineremo anche la crittografia ellittica, che si basa su un'algebra diversa, operando su curve ellittiche. Tuttavia,
gli algoritmi fondamentali rimarranno gli stessi, utilizzando però un'algebra additiva.

\section{Classe di equivalenza $\mathbb{Z}_n$}
Possiamo dire che:
\[
  \mathbb{Z}_n \equiv \mathbb{Z}_n \qquad a \equiv b \pmod{n} \iff (a \mod n) = (b \mod n)
\]
