\chapter{Orientamento ai goal}
Il goal, nell'ingegneria dei requisiti, è un concetto intrinseco quando 
si parla di requisiti.
Di fatto abbiamo sempre avuto a che fare con gli obiettivi, e la metodologia
\textit{goal-oriented} definisce il concetto di goal come astrazione chiave che 
ci guida nell'ingegneria dei requisiti.
\section{Definizione di goal}
\subsection{Soddisfacibilità dei goal e la cooperazione degli agenti}
\begin{tcolorbox}[colback=blue!5!white,colframe=blue!75!black, title=Goal]
  Un goal è uno statement \textbf{prescrittivo} che esprime un desiderio, un obiettivo o
  che il sistema debba raggiungere attraverso la cooperazione dei suoi agenti.
\end{tcolorbox}
Lo statement prescrittivo perché può essere esprimibile con ``deve'', ``dovrebbe'', 
``può'', ``potrebbe'', e così via.
\begin{tcolorbox}[colback=cyan!5!white,colframe=cyan!75!black, title=Agente]
    Un agente è un componete \textbf{attivo} del sistema \textit{as-is} o \textit{to-be} che è
    \textbf{responsabile} del raggiungimento
    di un goal.
\end{tcolorbox}
Capiamo quindi che gli agenti possono essere umani, software o hardware. L'agente è quindi 
un \textbf{ruolo} che prende decisioni, rispetto che un individuo, in modo che il goal assegnato sia raggiunto.
Per farlo dovrà \textit{monitorare} e \textit{controllare} delle grandezze.
\subsubsection{Esempio di goal soddisfatto dalla cooperazione di più agenti}
Il raggiungimento del'obiettivo ``\textit{Restituzione di un libro in biblioteca}''
può essere soddisfatto dalla cooperazione di: colui che restituisce il libro, 
il bibliotecario e il software che gestisce il prestito.
\subsection{Goal rispetto alle proprietà di dominio}
\begin{tcolorbox}[colback=violet!5!white,colframe=violet!75!black, title=Proprietà di Dominio]
    Le proprietà di dominio sono degli \textit{statement} descrittivi rispetto 
    l'ambiente.
\end{tcolorbox}
Lo statement descrittivo si esprime con ``è'', ``sono'' e così via.

Se i goal possono essere negoziati, indeboliti o prioritizzati, le proprietà di
dominio sono \textbf{vincoli} che non possono essere negoziati, ma sono comunque 
necessari nel documento dei requisiti.

\begin{tcolorbox}
    I goal vanno \textbf{decisi}, mentre le proprietà di dominio vanno \textbf{identificate}.
\end{tcolorbox}
\subsubsection{Esempio di goal rispetto alle proprietà di dominio}
Consideriamo ``\textit{Se le porte del treno sono aperte, allora non sono 
chiuse}''. Sono quindi proprietà legate a vincoli fisici o dell'ambiente nel 
quale il sistema opera.
\section{Granularità dei goal e le relazioni con i requisiti e le assunzioni}
I goal possono essere espressi a diversi livelli di granularità:
\begin{itemize}
    \item \textbf{Goal di alto livello}: sono statement strategici, che hanno quindi 
    una granularità alta, come ad esempio ``\textit{Migliorare la qualità del servizio del 50\%}''.
    \item \textbf{Goal di basso livello}: sono statement tecnici, che hanno quindi
    una granularità bassa, come ad esempio ``\textit{Il sistema invia un reminder
    per il rinnovo del prestito}''.
\end{itemize}
Se ragioniamo in termini di agenti, per raggiungere un goal di alto livello
servirà la cooperazione di più agenti, mentre per raggiungere un goal di basso
livello servirà la cooperazione di un minor numero di agenti.

I goal prendono un nome distinto a seconda della relazione che hanno con gli agenti.
\begin{tcolorbox}[colback=lime!5!white,colframe=lime!75!black, title=Requisto]
    Il requisto è un goal che deve essere soddisfatto da un \textbf{unico} agente 
    nel \textit{software to-be} (\textit{come per esempio il meccanismo di frenata}).
\end{tcolorbox}
\begin{tcolorbox}[colback=orange!5!white,colframe=orange!75!black, title=Prospettiva]
    La prospettiva è un goal che deve essere soddisfatto da un \textbf{unico} agente 
    nel \textit{environment} (\textit{come per esempio le persone}).
\end{tcolorbox}
\section{Tipologie e categorie di goal}
I goal possono essere classificati in diverse tipologie: 
\begin{itemize}
    \item \textbf{Behavioral goal}: goal prescrittivi che esprimono il comportamento.
    Tali goal si classificano in goal che dobbiamo \textbf{raggiungere} o \textbf{mantenere}.
    \item \textbf{Soft goal}: esprimono una preferenza tra funzionamenti distinti.
\end{itemize}
\subsection{Behavioral goal (prescrittivi)}
Rappresentano dei vincoli che rappresentano un sottoinsieme dei funzionamenti che il nostro 
sistema deve accettare a cui è possibile attribuire una risposta univoca, o \textit{sì}
o \textit{no}.

Distinguiamo tra:
\begin{itemize}
    \item \texttt{Achieve[targetCondition]}: rappresentano una condizione target da raggiungere
    \begin{itemize}
        \item \texttt{[if CurrentCondition then] sooner-or-later targetCondition}
    \end{itemize}
    \item \texttt{Maintain[targetCondition]}: rappresentano una condizione target da mantenere
    \begin{itemize}
        \item \texttt{[if CurrentCondition then] always targetCondition}
    \end{itemize}
    \item \texttt{Avoid[targetCondition]}: rappresentano una condizione target da evitare
    \begin{itemize}
        \item \texttt{[if CurrentCondition then] never targetCondition}
    \end{itemize}
\end{itemize}
\subsection{Soft goal}
Rappresentano il concetto di preferenza del sistema, questi quindi non possono essere 
soddisfatti in maniera binaria come nei behavioral goal, ma possono essere più o meno
soddisfatti.

Tipicamente hanno la forma di massimizzazione o minimizzazione di una certa grandezza, 
come ad esempio ``\textit{La condizione di stress degli operatori deve essere minimizzata}''.

\subsection{Categorie di goal}
I goal possono essere classificati in diverse categorie, ma tale divisione non è
sempre netta, in quanto un goal può appartenere a più categorie.
\begin{itemize}
    \item \textbf{Funzionali}: sono goal che esprimono le funzionalità e i servizi che 
    il sistema dovrà offrire, usati per costruire un modello operativo del sistema.
    \item \textbf{Qualitativi}: sono goal che esprimono le qualità che il sistema dovrà
    avere, come ad esempio la sicurezza, l'usabilità, l'efficienza, ecc.
    \item \textbf{Di sviluppo}: sono goal che esprimono le attività che dovranno essere
    svolte per lo sviluppo del sistema, come ad esempio la documentazione, la formazione
    del personale, ecc.
\end{itemize}
\section{Il ruolo centrale dei goal nel processo di ingegneria dei requisiti}
Il ruolo dei goal è quello di supportare un meccanismo di \textbf{raffinamento e astrazione} 
dei goal, raggiungendo goal di più alto livello attraverso la cooperazione di goal in 
\texttt{AND} e \texttt{OR} di più basso livello.

Possiamo rappresentare formalmente il concetto di goal come:
\[
\{\texttt{REQ}, \texttt{Exp}, \texttt{Dom} \} \models \texttt{G} 
\]
Che possiamo leggere come ``\textit{Considerate le proprietà 
di dominio, i requisiti/sottogoal che assicurano che il goal G sia soddisfatto sotto 
l'ipotesi delle proprietà di dominio}''.

Tale rappresentazione formale ci permette di raggiungere due obiettivi:
\begin{itemize}
    \item \textbf{Completezza}: assicurare che tutti i goal siano raggiunti.
    \item \textbf{Rilevanza}: se un qualche requisito è utilizzato per ottenere 
    un goal, allora tale requisito è rilevante.
\end{itemize}

Ragionando in termini di evoluzione è importante decidere, quali sono i requisiti più stabili 
e quelli meno stabili in modo da osservarli con maggiore attenzione.
Tipicamente i goal di alto livello sono più stabili, mentre i goal di basso livello sono
più instabili perché potremmo avere alternative, anche data la presenza di \texttt{OR} decomposizione.

Bisogna evitare ti ragionare \textit{top-down}, tipicamente si utilizzano entrambe 
le strategie, \textit{top-down} e \textit{bottom-up}. La metodologia \textit{bottom-up}
viene chiamata \textbf{goal-abstraction}, dove si cerca di estrarre le motivazioni per cui un determinato 
goal è stato definito.

Tipicamente, però, si ragiona in termini di \textit{scenario-oriented} e \textit{agent-oriented}.
\begin{figure}[H]
    \centering
    \begin{tikzpicture}
        % Definizione degli stili dei nodi
        \tikzset{oval node/.style={draw, ellipse, minimum width=2.5cm, minimum height=1.5cm, align=center, fill=#1}}
        
        % Creazione dei nodi
        \node[oval node=blue!20] (A) at (0,3) {Goals};
        \node[oval node=green!20] (B) at (-3,-1) {Scenari};
        \node[oval node=red!20] (C) at (3,-1) {Agenti};

        % Creazione degli archi con etichette
        \draw (A) -- (B) node[midway, left] {Copertura};
        \draw (A) -- (C) node[midway, right] {Responsabilità};
        \draw (B) -- (C) node[midway, below] {Interazione};
    \end{tikzpicture}
\end{figure}
Gli scenari devono coprire tutti i goal, e abbiamo delle interazioni tra agenti e
scenari. Sono quindi un modo concreto in cui identifichiamo o validiamo i goal.