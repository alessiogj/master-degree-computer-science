\chapter{Validazione e verifica}
L'obiettivo di tale fare è quella di partire da un documento dei 
requisiti consolidato e di verificare che esso sia corretto e completo.

Alcuni errori siamo stati in grado di evitarli in partenza, grazie alle fasi 
precedenti. In questa fase ci concentriamo nell'individuazione di errori
che non sono stati individuati in precedenza. 

Il primo obiettivo è la \textbf{ricerca} che si divide in tre fasi.
La prima fase da seguire è la \textbf{validazione}, che consiste nel verificare 
che il documento dei requisiti soddisfi le esigenze del cliente.
La seconda fase è la \textbf{verifica}, che consiste nel verificare che il
documento sia corretto e completo.
La terza fase è la \textbf{controllo}, che riguarda il raggiungimento 
tei target di qualità.

Il secondo obiettivo è la \textbf{documentazione} dei difetti, che include 
anche l'\textbf{analisi} delle cause e la loro correzione.

\begin{tcolorbox}[colback=orange!5!white,colframe=orange!75!black,title=Output della fase]
    L'output di tale fase è la versione finale del documento dei requisiti.
    
    Oltre a tale documento possiamo considerare anche i test di accettazione consolidati, 
    eventualmente un prototipo, un piano di sviluppo e potenzialmente un contratto.
\end{tcolorbox}

\section{Come si affronta la revisione}
\subsection{Ispezione e revisione}
L'idea è quella di avere uno o più esperti che prendono il documento dei requisiti
che prendono il documento dei requisiti e lo analizzano oppure utilizzare dei meeting 
di gruppo per discutere il documento.

Principalmente si utilizzano due tecniche:
\begin{itemize}
    \item \textbf{Walkthrough}: si tratta di una lettura del documento,
    con l'obiettivo di trovare errori.
    \item \textbf{Più strutturato}: utilizzando degli ispettori esterni, dei 
    meeting ben strutturati, report o altro.
\end{itemize}

Un processo che mira ad assicurare la qualità del documento dei requisiti segue tali fasi:
\begin{figure}[H]
    \centering
    \begin{tikzpicture}
        % Styles
        \tikzstyle{process} = [rectangle, rounded corners, 
        minimum width=3cm, minimum height=1cm, text centered, draw=black, 
        fill=blue!20, text width=2.5cm]
      
        % Nodes
        \node (start) [circle, fill, inner sep=2pt] {};
        \node (identify) [process, right=0.5cm of start]
        {Pianificare \\ l'Ispezione del \\ Documento};
        \node (detect) [process, right=0.5cm of identify]
        {Revisione \\ individuale};
        \node (generate) [process, right=0.5cm of detect]
        {Valutazione \\ dei difetti};   
        \node (evaluate) [process, right=0.5cm of generate]
        {Consolidamento del \\ Documento dei \\ Requisiti};
      
        % Arrows
        \draw[->] (start) -- (identify);
        \draw[->] (identify) -- (detect);
        \draw[->] (detect) -- (generate);
        \draw[->] (generate) -- (evaluate);
        \draw[->]  (evaluate.south) -- ++(0,-0.5) -| (start.south);
    \end{tikzpicture}
\end{figure}
\begin{itemize}
    \item \textbf{Pianificare l'ispezione del documento}: si tratta di definire
    chi sono gli ispettori, come si svolgerà l'ispezione, quali sono gli obiettivi
    e quali sono i criteri di accettazione.
    \item \textbf{Revisione individuale}: ogni ispettore legge il documento e
    cerca di individuare errori.
    Tale fase può seguire diverse modalità:
    \begin{itemize}
        \item Liberamente: ogni ispettore legge il documento e cerca di individuare
        errori.
        \item Guidata mediante checklist: ogni ispettore segue una lista di controllo
        basata sul dominio.
        \item Basato sul processo.
    \end{itemize}
    \item \textbf{Valutazione dei difetti}: si tratta di valutare i difetti
    capendo come sono stati generati e come possono essere corretti.
    \item \textbf{Consolidamento del documento dei requisiti}: si tratta della 
    fase in cui viene consolidato il documento dei requisiti.
\end{itemize}
Tale processo può essere ripetuto più volte, fino a quando non si è soddisfatti
della qualità del documento.
\subsection{Linee guida per la revisione}
\begin{itemize}
    \item \textbf{Rapporto di Ispezione:}
    \begin{itemize}
        \item Deve essere \textit{informativo}, \textit{accurato} e \textit{costruttivo}.
        \item Non deve contenere opinioni personali o commenti che possano risultare 
        offensivi.
        \item Deve seguire una \textit{struttura standard} che permetta l'aggiunta 
        di commenti liberi e sia di facile lettura.
        \item Può servire da guida per la revisione individuale.
    \end{itemize}
    
    \item \textbf{Gli Ispettori:}
    \begin{itemize}
        \item Devono essere \textit{indipendenti} dagli autori del documento sotto 
        revisione.
        \item Devono rappresentare tutti gli \textit{stakeholder} e provenire da 
        \textit{background diversi}.
    \end{itemize}
    
    \item \textbf{Tempistica dell'Ispezione:}
    \begin{itemize}
        \item L'ispezione non deve avvenire né \textit{troppo presto} né 
        \textit{troppo tardi}.
        \item Incontri \textit{brevi} e \textit{ripetuti} risultano più efficaci.
    \end{itemize}
    
    \item \textbf{Focus Maggiore su Parti Critiche:}
    \begin{itemize}
        \item Maggiore è il numero di difetti in una parte, maggiore deve essere 
        lo \textit{scrutinio} su quella parte.
    \end{itemize}
\end{itemize}
\subsection{Checklist}
L'obiettivo principale delle liste di controllo per l'ispezione è di dirigere 
la ricerca dei difetti verso specifiche problematiche, migliorando così 
l'efficienza e l'efficacia del processo di revisione. Le liste di controllo 
si suddividono in diverse categorie, ognuna con un focus particolare:

\begin{itemize}
    \item \textbf{Liste basate sui difetti:} Queste liste comprendono domande 
    generiche che sono strutturate in base al tipo di difetto. L'intento è di 
    coprire un ampio spettro di possibili anomalie in modo organizzato.

    \item \textbf{Liste specifiche per qualità:} Tali liste affinano le domande 
    delle liste basate sui difetti per concentrarsi su specifiche categorie di 
    requisiti non funzionali (NFR), come la sicurezza, la performance e 
    l'usabilità. Queste liste possono anche essere basate su parole guida e 
    sono frequentemente orientate a identificare omissioni, migliorando così 
    la completezza del software o del prodotto analizzato.

    \item \textbf{Liste specifiche per dominio:} Queste liste rappresentano 
    un'ulteriore specializzazione e si concentrano sui concetti e le operazioni 
    specifici di un dominio. Sono particolarmente utili per offrire una guida 
    dettagliata nella ricerca di difetti, assicurando che le peculiarità del 
    dominio siano adeguatamente considerate.

    \item \textbf{Liste basate sul linguaggio:} Queste liste adattano le liste 
    basate sui difetti ai costrutti di linguaggi di specifica particolari, 
    supportando processi di automazione. La ricchezza dei linguaggi di specifica 
    consente implementazioni di controlli più sofisticati e mirati.
\end{itemize}
\begin{tcolorbox}[colback=green!5!white,colframe=green!75!black,title=Vantaggi 
    delle Checklist]
    \begin{itemize}
      \item \textbf{Maggiore efficacia rispetto all'ispezione del codice:} Utilizzo 
      di un modello basato su processi che combina checklist basate sui difetti, 
      specifiche per qualità e specifiche per linguaggio, risultando estremamente efficace.
      \item \textbf{Ampia applicabilità:} Adatte per ogni tipo di difetto e formato 
      di specifica, rendendole strumenti versatili.
    \end{itemize}
    \end{tcolorbox}
    
    \begin{tcolorbox}[colback=red!5!white,colframe=red!75!black,title=Limitazioni 
        delle Checklist]
    \begin{itemize}
      \item \textbf{Onere e costo del processo di ispezione:} La dimensione del 
      materiale di ispezione e il tempo/costo necessari per gli ispettori esterni 
      e le riunioni di revisione possono essere significativi.
      \item \textbf{Nessuna garanzia di rilevare tutti i difetti importanti:} Nonostante 
      l'efficacia, le checklist non possono garantire il rilevamento di tutti i 
      difetti critici.
    \end{itemize}
    \end{tcolorbox}
\section{Approccio guidato da Query}
La quality assurance può essere vista come delle particolari interrogazioni a 
delle strutture dati. Tipicamente le query possono essere formulate in presenza di una documentazione 
strutturata dei requisti. Per strutturata intendiamo una documentazione che
si basa su diagrammi.

Quando si formulano tali query utilizzando i \texttt{CASE} tools, si possono
ottenere delle risposte automatiche.
\section{Validazione dei requisiti mediante animazioni}
L'obiettivo principale di questa metodologia è verificare l'adeguatezza dei requisiti 
rispetto alle reali necessità degli utenti. Questo processo può essere attuato 
attraverso due approcci principali:

\subsubsection{Approccio 1: Visualizzazione di Scenari di Interazione}
Questo approccio consiste nel mostrare scenari di interazione concreti in azione, 
utilizzando strumenti di attuazione sui diagrammi di Evento Tempo (\texttt{ET}). Tuttavia, 
questo metodo presenta sfide legate alla copertura completa degli scenari, che 
può risultare incompleta.

\subsubsection{Approccio 2: Utilizzo di Strumenti di Animazione delle Specifiche}
L'animazione delle specifiche si svolge in quattro fasi principali:
\begin{enumerate}
    \item \textbf{Generazione del Modello Esecutivo:} Si estrae o si genera un 
    modello eseguibile direttamente dalle specifiche tecniche.
    \item \textbf{Simulazione del Comportamento:} Il sistema viene simulato 
    basandosi su questo modello, dove eventi stimolo vengono inviati per imitare 
    il comportamento dell'ambiente circostante e si osserva la risposta del modello.
    \item \textbf{Visualizzazione della Simulazione:} La simulazione viene 
    visualizzata mentre è in corso, permettendo una valutazione immediata e 
    dinamica del comportamento del sistema.
    \item \textbf{Feedback degli Utenti:} I feedback degli utenti vengono 
    raccolti e analizzati per valutare l'efficacia della simulazione e del 
    modello proposto.
\end{enumerate}
\begin{tcolorbox}[colback=green!5!white,colframe=green!75!black,title=Punti di Forza]
    \begin{itemize}
      \item \textbf{Verifica dell'Adeguazione:} Metodo efficace per controllare 
      l'adeguatezza dei requisiti rispetto alle necessità reali e all'ambiente attuale.
      \item \textbf{Coinvolgimento degli Stakeholder:} Il principio ``WYSIWYC" 
      (What You See Is What You Check) permette ai stakeholder di visualizzare e 
      verificare i requisiti direttamente.
      \item \textbf{Estensione a Controesempi:} Possibilità di animare controesempi 
      generati da altri strumenti come i verificatori di modelli, migliorando così la robustezza del sistema.
      \item \textbf{Riutilizzo degli Scenari:} Gli scenari di animazione possono 
      essere conservati per replay futuri e usati come dati di test di accettazione.
    \end{itemize}
    \end{tcolorbox}
    
    \begin{tcolorbox}[colback=red!5!white,colframe=red!75!black,title=Limitazioni]
    \begin{itemize}
      \item \textbf{Problema della Copertura:} La progettazione degli scenari deve 
      essere attentamente gestita per assicurare che non vengano tralasciati problemi significativi. Non vi è garanzia di identificare tutti i difetti importanti.
      \item È necessaria una specifica formale 
      dettagliata per eseguire un'animazione efficace dei requisiti, richiedendo 
      un investimento considerevole in termini di tempo e risorse.
    \end{itemize}
\end{tcolorbox}