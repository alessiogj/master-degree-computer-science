\documentclass[oneside,a4paper,11pt]{book}
\usepackage[utf8]{inputenc}
\usepackage{svg}
\usepackage[italian]{babel}
\usepackage{float}
\usepackage{fancyvrb}
\usepackage{titling}
\usepackage[margin=1in,footskip=0.25in]{geometry}
\usepackage{listings}
\usepackage[DIV=12,BCOR=2mm,headinclude=true,footinclude=false]{typearea}
\usepackage{color, colortbl,xcolor}
\usepackage[hidelinks]{hyperref}
\usepackage{tcolorbox}
\usepackage{chngcntr}
\usepackage{diagbox}
\usepackage{calc}
\usepackage{amssymb}
\usepackage{subcaption}
\usepackage{amsthm}
\usepackage{amsfonts}
\usepackage{mathtools}
\usepackage{parskip}
\usepackage{cancel}
\usepackage{forest}
\usepackage{listings}
\usepackage{mathrsfs}
\usepackage{enumitem}
\usepackage{makecell}
\usepackage{tikz}
\usepackage{pgfplots}
\pgfplotsset{compat=1.18}
\usepackage{fancyhdr}
\fancypagestyle{plain}{\fancyhf{}\renewcommand{\headrulewidth}{0pt}}
\pagestyle{fancy}
\fancyhf{}% Clear header/footer
\fancyhead[L]{\nouppercase\leftmark}
\fancyhead[R]{\thepage}
\usetikzlibrary{positioning,shapes.geometric,arrows.meta,matrix,automata,decorations.pathmorphing,patterns,decorations.pathreplacing,shapes.multipart,calc,snakes}
\usetikzlibrary{arrows.meta, backgrounds, chains, positioning, shapes.geometric, shapes.multipart}
\tcbuselibrary{skins}
\counterwithin{figure}{section}
%Nuovi comandi
\newcommand\myeq{\stackrel{\mathclap{\normalfont\mbox{def}}}{=}}
\newcommand\prodG{\stackrel{\mathclap{\normalfont\mbox{\tiny{G}}}}{\Longrightarrow}}
%asmthm
\newlength{\marginlabelsep}\setlength{\marginlabelsep}{0.5em}
\newtheoremstyle{italicstyle} %% Name
  {} %% <- Space above (empty = default = \topsep = 8.0pt plus 2.0pt minus 4.0pt)
  {} %% <- Space below (empty = default = \topsep = 8.0pt plus 2.0pt minus 4.0pt)
  {\itshape} %% <- Body font
  {} %% <- Indent amount (empty = no indent, \parindent = just that)
  {\bfseries} %% <- Thm head font
  {} %% <- Punctuation after thm head
  {1pt} %% <- Space after thm head (or " " or \newline) (default: 5pt plus 1pt minus 1pt)
  {\vtop to 0pt{\llap{\thmname{#1}\hskip\marginlabelsep}
                \llap{\thmnumber{#2}\hskip\marginlabelsep}}\thmnote{#3\\}%
  }
\newtheoremstyle{normStyle} %% Name
  {} %% <- Space above (empty = default = \topsep = 8.0pt plus 2.0pt minus 4.0pt)
  {} %% <- Space below (empty = default = \topsep = 8.0pt plus 2.0pt minus 4.0pt)
  {\normalfont} %% <- Body font
  {} %% <- Indent amount (empty = no indent, \parindent = just that)
  {\bfseries} %% <- Thm head font
  {} %% <- Punctuation after thm head
  {1pt} %% <- Space after thm head (or " " or \newline) (default: 5pt plus 1pt minus 1pt)
  {\vtop to 0pt{\llap{\thmname{#1}\hskip\marginlabelsep}
                \llap{\thmnumber{#2}\hskip\marginlabelsep}}\thmnote{#3\\}%
  }
\theoremstyle{italicstyle}
\newtheorem{corollary}{Corollario}[section]
\newtheorem{notazione}{Notazione}[section]
\newtheorem{lemma}{Lemma}[section]
\newtheorem{definizione}{Definizione}[section]
\newtheorem{nota}{Nota}[section]
\newtheorem{exercise}{Esercizio}[section]
\theoremstyle{normStyle}
\newtheorem{exmp}{Esempio}[section]
\newtheorem{theorem}{Teorema}[section]
\newtheorem{proposizione}{Proposizione}[section]
\tcbuselibrary{listings,skins}
\newtcblisting{mylisting}[2][]{
    arc=0pt, outer arc=0pt,
    listing only, 
    title=#2,
    #1,
    listing options= {escapechar=|}
}
\newcommand{\myboxedtext}[2][rectangle,draw]{%
    \tikz[baseline=-0.6ex] \node [#1]{#2};}%
%%======================================================================
\title{Ingegneria dei Requisiti}
\author{\textit{Alessio Gjergji}}
\date{}
\begin{document}
\begin{titlingpage}
  \centering
  \vspace*{\stretch{1}}
  \huge
  \textbf{\thetitle}\\[0.5cm]
  \normalsize
  Corso tenuto dal Professor Mariano Ceccato\\[0.5cm]
  Università degli Studi di Verona\\[1cm]
  \large
  \theauthor\\[0.5cm]
  \vspace{\stretch{2}}
\end{titlingpage}
\tableofcontents
\chapter{Introduzione}
Ci sono problematiche legate al mondo circostante, quindi anche 
sistemi legacy. Anche la natura fa parte dei componenti fisici su 
cui interagire (pensiamo a dover pilotare dispositivi con una grossa massa).
Dobbiamo tener contro di tutto, tenere in considerazione anche il contesto 
del mondo reale all'esterno dei software. Proprietà, parliamo di tutto l'insieme.

I requisiti descrivono solo i problemi del mondo reale, come il software deve 
interagire con il mondo reale. Ragioniamo su proprietà del mondo reale,
la progettazione del software è solo una conseguenza.

Non troviamo soluzioni, ma raccogliamo problemi, responsabilità, ma 
non di soluzioni. Il software è solo una conseguenza.

Abbiamo due versioni del mondo:
\begin{itemize}
    \item \textbf{System as is}: il sistema come è prima dello sviluppo
    del software. Il mondo senza il software.
    \item \textbf{System to be}: il sistema come sarà dopo lo sviluppo
    del software.
\end{itemize}
Dobbiamo catturare i vincoli e le opportunità del mondo reale,
per mapparli in requisiti software.
I requisiti si dividono in:
L'unione del mondo e del software è il sistema.
Dobbiamo ragionare in tre prospettive:
\begin{itemize}
    \item \textbf{Quali sono gli obiettivi del sistema?} perché ci serve un nuovo 
    sistema?
    \item \textbf{Quali sono le funzionalità del sistema?} cosa deve
    fare il sistema? Iniziamo a parlare di requisiti. Che dovranno dare una 
    risposta a cosa deve fare il sistema.
    \item \textbf{Quali sono i responsabili del sistema?} chi deve fare cosa? Anche 
    oggetti necressari.
\end{itemize}
\subsection{Perché}
Si tratta di identificare quali sono gli obiettivi del sistema.
Identificare ma che vanno analizzati e rifiniti. Tipicamente quello 
che si raccoglie dagli stakeholders è ad alto livello, è anche opportuno 
sapere quando smettere, criteri di completezza.

Ci sono difficoltà intrinsiche. Complesso anche il dominio, bsiogna avere 
un background, capire il contesto, la complessità del dominio.

Loro danno tutto per assunto. Ci sono diverse alternative da valotare, è opportuno 
limitare le scelte per far scelte consapevoli dopo.

Potrebbero esserci obiettivi in contrasto, che vanno in direzioni opposte.
Confidenzialità e accessibilità, ad esempio. si scontra con la facilità di utilizzo 
che vanno in direzioni opposte.

Opportuno a questo livello identificare obiettivi contrastanti.

\subsection{Cosa}
Mira a identificare le funzionalità del software. Parliamo di requisiti,
l'obiettivo è dare risposta al perché. Cosa deve fare il sistema? rispondendo in 
maniera adeguata, ci saranno metriche di qualità.
Risposta in maniera realistica, soluzione compatibile con i vincoli ambientali.

Esempio calcolo dell'accelerazione del treno, adeguata e sicura per 
i passeggeri. Non solo calcolo, ma anche sicurezza. Informazioni utili 
ai passeggeri all'interno del treno.

Difficoltà: identificare le necessità degli stakeholders, opportuno 
il linguaggio comprensibile a tutti. Garantire una tracciabilità degli obiettivi 
rispetto alle funzionalità del sistema. Deve esserci una motivazione 
alla funzionalità, quindi un link da mantenere. Se ci accorgiamo che
il perché è cambiato, probabilmente anche il cosa cambierà.

\subsection{Chi}
Fornire le responsabilità, assegnate a device, hardware, software,
utenti, collaboratori, \dots
Nel caso di accelerazione del treno potrebbe essere il macchinista,
o un software di controllo. Il macchinista potrebbe avere anche un software
di supporto.

La difficoltà sta nelle varie alternative che danno diverso grado di 
autonomia. è opportuno adottare un approccio iterativo dove le 
alternative vengono raffinate gradualmente. 

Registro descrittivo: registro indicativo, raccontiamo le proprietà del 
sistema indipendentemente da come dovrebbe funzionare. Regole universali 
vincoli intrinseci fisici. 
"Se il treno è in accelerazione, allora la velocità non è nulla"

Prescrittive: le porte dovrebbero rmanere chiuse durante
la marcia del treno. Proprietà desiderabile. Il secondo tipo otative possono 
essere manovrate, identificare il margine di manovra. Quelli descrittivi non sono 
negoziabili.

Le frasi possono essere collocate in contesti diversi, potrebbero riferirsi 
all'ambiente o sono condivisi tra ambiente e software.
Non ci sono requisiti sulla parte software.

Requisiti di sistema e software. Tipicamente nei requisiti di sistema abbiamo 
frasi prescrittive che si riferiscono ai fenomenti ambientali e devono 
essere soddisfatti dal software. Formulate con un vocabolario comprensibile 
da tutti gli stakeholders.
Nei requisiti software abbiamo frasi prescrittive che si riferiscono
ai fenomeni condivisi che sono supportate dal software che deve 
essere soddisfatto. Grandezze modellate all'interno del software.

Tipo di frasi all'interno dei requisiti:
Abbiamo proprietà di dominio, assunzioni e definizioni.
Una proprietà di dominio è una frase descrittivia su un fenomeno del mondo
reale, indipendente dall'esistenza del sistema software.
\[
  \texttt{velocitaMisurata} \geq 0 \rightarrow \texttt{accelerazione} \not = 0
\] 
Frasi soddisfatte dall'ambiente 
\[
\texttt{velocitaMisurata} \not = 0 \Leftrightarrow \texttt{accelerazione} \not = 0
\]
Definizione: una frase che definisce un concetto del dominio.
La definizione di velocitaMisurata è la velocità del treno misurata
dal tachimentro del treno.

Requisiti funzionali e non funzionali:
I requisiti funzionali raccontano del funzionalmento del software. 
Responsabilità del software, funzionalità da esporre.
I requisiti non funzionali raccontano della qualità del software.
Come per esempio la velocità di risposta, la sicurezza,
la confidenzialità.
"il comando di apertura delle porte deve essere eseguito entro 2 secondi"
Rappresentano vincoli che di fatto rescriveranno quelli che sono i vincoli 
architetturali che ci guidano nella scelta della architettura.
è opportuno adottare una tassonomia dei requisiti non funzionali. In modo 
da avere una visione chiara dei requisiti non funzionali.

Ci sono alcuni punti in cui i requisiti non funzionali si 
trasformano in requisiti funzionali. Ad esempio la sicurezza.
Nel contesto del firewall, la sicurezza è un requisito diventa funzionale.

Mandare i comandi troppo spesso ad esempio potrebbero causare overhead.

Tipicamente molto utili.

\section{Ciclo di vita dei requisiti}
Il processo di acquisizione dei requisiti è un processo iterativo.
Prima di tutto è necessario comprendere il dominio, è importante 
comprendere il contesto in cui il sistema dovrà operare. 

Principalmente si studia il mondo prima che il mondo as is. Si studia la organizzazione,
ruoli, prassi, i punti di forza e deboli del sistema attuale. Si identificano 
gli stakeholders, definendo anche le tipologie di utenti. Chi prende decisioni.

Ci permette di formare un primo glossario di terminologia. Gli oggetti possono avere gestioni 
diverse.

Dopodicchè si passa alla elicitation, ovvero l'identificazione dei requisiti.
Possiamo raccogliere i primi requisiti. Ci sarà una parte di negoziazione, 
per identificare i requisiti che sono in contrasto. Come gestire i conflitti.
Vanno identificati i rischi e le opportunità. Diverse alternative, le prioritizzazioni.
Il prodotto finale è il documento dei requisti, che verrà popolato man mano.

Il passo successivo è quindi la documentazione dei requisiti,
ha come obiettivo fissare in maniera precisa i requisiti.
Deve essere comprensibile e formulato in maniera che tutti possano comprenderlo.

La fase successiva è la validazione e verifica, per capire se è completo corretto, 
se ci sono lacune, se ci sono errori, se ci sono ambiguità.
Ci sono varie metodologie per fare la validazione e verifica. Se danno risposta coretta 
e completa alla necessità degli stakeholders. Si controllano le contraddizioni per 
essere risolte. Dovrà essere verificato rispetto a target di qualità. e tutto andrà 
discusso.
L'output è un documento di requisiti validato e verificato.

Può fornire il vantaggio per il piano di test di accettazione, scenari ad alto 
livello che hanno senso per un utente del sistema. Non sono eseguibili non appena 
il sistema è stato implementato. Serve copertura esaustiva, test per ogni requisito.

Posso abbozzare un piano di sviluppo e questo documento può essere 
usato come contratto di fornitura del software. Una sorta di preventivo.

Quando ho la prima versione itero nel ciclo. L'attività può essere continua.
Soprattutto nel caso di modelli agili. Insegue quindi i requisiti mutevoli.

\section{Obiettivi di qualità dei requisiti}
\begin{itemize}
  \item Completezza: tutti i requisiti sono stati catturati?
  \item Consistenza: non ci sono contraddizioni.
  \item Adeguatezza
  \item Non ambiguità: non devo lasciar spazio a interpretazioni implicite.
  \item Misurabilità: devo poter misurare se il requisito è stato soddisfatto.
  \item Pertinenti: devono essere rilevanti per gli stakeholders.
  \item Realizzabilità: devono essere realizzabili.
  \item Flessibilità: devono essere flessibili.
  \item Comprensibilità: devono essere comprensibili.
  \item Buona struttura: devono essere ben strutturati.
  \item Modificabilità: devono essere modificabili.
  \item Traceability: devono essere tracciabili. Link tra requisiti
  e sorgenti. Obiettivi di alto livello che rendono necessario
  un requisito. Analogamente anche per decisioni alternativi.
\end{itemize}
Gli errori nei requisiti sono molto costosi. 
Non bisogna mai omettere qualcosa, contraddire qualcosa, avere ambiguità.
Altrimenti potrei avere soluzioni sub-ottimali. Soluzioni di ripiego quindi,
errori di progettazione e sviluppo in caso di contraddizioni. Se i requisiti sono 
inadeguati potrei avere un sistema che non soddisfa le aspettative.
\chapter{Elicitazione dei requisiti}
Arrivare a una comprensione del dominio e capire informazioni. Perchè 
è la sorgente di informazione.
Prima bisogna raggiungere informazione sul dominio, studiando il sistema prima 
che il software arrivi, in modo da capire i flussi di lavoro, le prassi,
le regole, le procedure, le politiche, le normative, le leggi, i regolamenti.
Analizzare i problemi prima del software, terminologia di base.
Individuare le opportunità e identificare gli stakeholders, suddividendoli 
per ruoli e responsabilità. Obiettivi vincoli e tutto questo è acquisizione di 
conoscenza che prende il nome di elicitazione di requisiti.

Ci sono due classi principali:
\begin{itemize}
  \item Artefact-driven: si basa su documenti, modelli, specifiche,
  \item Stakeholder-driven: si basa su interviste, focus group, brainstorming.
\end{itemize}
Coinvolgere un campione rappresentativo di stakeholders, quindi almeno un 
individuo per ogni ruolo. Ci possono essere conflitti di interesse, quindi
bisogna mediare.
Acquisire questa conoscenza dagli stakeholders è complesso, perché le sorgenti 
di informazioni possono essere diverse, questo va gestito.
Un alto problema è che spesso è difficile agganciare le persone chiave, 
perché possono essere impegnate in altre attività legate all'azienda.
Diverse persone possono avere un background diverso, quindi possono
avere una visione diversa del problema.
Spesso danno per scontato concetti impliciti. La differenza tra una fattura e 
ordine di acquisto per esempio. bisogna quindi fare domande per capire
questi concetti impliciti.
Il problema duale è che a volte ci sono dettagli irrilevanti, che possono non 
essere rilevanti per il sistema. 
Ci potrebbe essere resistenza al cambiamento, non da sottovalutare, ci può 
essere a vari livelli, dai dipendenti ai manager. Ci può essere un 
turnover di personale, tipico di aziende di consulenza.

Sono necessarie soft-skill, buon comunicatore, ascoltare, fare domande.
Validare quello che si impara precedentemente, ristrotturare quando 
ci sono conoscenze nuove.

\section{Studio di background}
Capire il dominio, raccogliendo documentazione e riassumendo.
Ci sono i diagrammi organizzazione, che tipicamente ricalca le principali 
aree aziendali e ci fa capire molto sull'azienda. Leggere i business plan 
ci fa capire le strategie dell'azienda. Leggere i report finanziari ci fa capire la salute dell'azienda.

Dopo aver capito l'organizzazione si fa uno studio più letterario, ci informiamo sul 
dominio, leggendo libri, articoli, riviste, blog, forum, social network.

Importante analizzare il sistema as is, prima dell'avvento del software, come 
sono documentati i workflow, le regole di business, report di lamentele\dots

SOno fonti interessanti per raccogliere i requisiti i problemi degli utenti. Molte 
aziende mantengono le lamentele dei clienti. Pone le basi per quando 
andremo a incontrare gli stakeholders. Sapremo la terminologia, i problemi,
le opportunità, i vincoli, le regole di business. In modo da arrivare preparati.

L'attività può essere dispendiosa, alcune informazioni possono essere irrilevanti.
Una soluzione è la meta-conoscenza per tagliare fonti.
Studio solo quella che mi serve, capire quello che mi serve e andare a cercare
solo quello.
\subsection{Collezione di dati}
Consiste nel raccogliere dati aziendali per capire numericamente di cosa sto 
parlando.Si tratta di informazioni quantitative, dati del mercato, statistiche,
dati finanziari, dati di produzione, dati di vendita, dati di marketing.
Come si colloca nel mercato di riferimento.
Sono utili per i requisiti non funzionali, come performance, scalabilità,
usabilità, affidabilità, sicurezza.
Le difficoltà sono di raggiungere fonti di informazioni affidabili.
\subsection{Questionario}
Il questionario è un modo efficace per raccogliere informazioni dagli stakeholders.
Tipicamente sono domande con una breve spiegazione, domande a risposta multipla,
domande aperte, domande a scala. Efficace per informazione veloce, anche per 
persone che non sono in presenza. Si tratta di informazione soggettiva, utile per 
sgrezzare la nostra conoscenza.

è importante preparare correttamente questi questionari, è facile sbagliare. 
Potremmo far errore di suggerire la risposta e devono essere formulate 
in maniera neutrale senza inserire bias.
Stare attenti a ambiguità, ci sono linea guida. Per avere anche un campione 
significativo.
Aggiungere ridondanza nel questionario, chiedendo le cose in maniera diversa, 
in modo da avere conferme e verificare inconsistenze.
Prima di somministrare i questionari si fanno validare da altri membri del team.

\subsection{Card sort}
Ci sono altre tecniche per informazione.
Acquisire informazioni in più rispetto alle informazioni in più rispetto
ai questionari. Si scrivono le info sulle carte. CHe vengono messe insieme 
e si chiede il motivo per cui sono state messe insieme. In questo modo 
nascono dei requisti.

\subsection{Repertory grid}
Si tratta di una tecnica per acquisire informazioni sulle relazioni tra
concetti. Si chiede di mettere insieme concetti simili e diversi, in modo
da capire le relazioni tra i concetti. 
Gli utenti suggeriscono valori per i concetti, in modo da capire le relazioni
tra i concetti.

Conceptual labeling: i concetti simili vengono raggruppati insieme, in modo
da capire le relazioni tra i concetti dando un nome al gruppo. Rappresenta un 
livello di astrazione più elevato.
Sono metodi semplici ed economici che permettono di acquisire informazioni
che ci siamo persi precedentemente. Lo svantaggio è che potrebbero essere 
largamente soggettivi.

\subsection{Scenari e storyboard}
L'obiettivo è quello di acquisire informazione attraverso esempi concreti.
Tipicamente sotto forma narrativa.
Le storyboard sono una sequenza di immagini che rappresentano una storia, 
che racconta lo step nella vita di un utente. Possono essere figure o 
in forma scritta. Possono essere passive o attive. Passive sono per 
la validazione, attive sono per elaborare la storyboard o completarne.

Gli scenari sono rappresentazioni tipicamente testuali che rappresentano 
interazioni con il sistema che possono essere utilizzate per capire 
come il sistema è o per esplorare soluzioni. Sono usate per elicitare 
informazioni, chiedendo il perché. Sono necessari per l'acceptance test.

Gli scenari possono essere:
\begin{itemize}
  \item Positivi.
  \item Negativi.
  \item Normali.
  \item Anomali: situazione eccezionale.
\end{itemize}
Pro e contro:
Sono esempi concreti e controesempi del sistema. Sono di tipo narrativo e 
sono ben recepiti dagli stakeholders. Possono essere tradotti in test di
accettazione. 

Sono intrinsecamente parziali potrei dimenticare degli scenari. Potremmo avere 
una esplosione combinatoria in caso di copertura completa. Rischiamo 
si specificare troppo il sistema. Prendere decisioni premature dal software 
e l'environment.
Decisioni irrilevanti per il sistema.

Gli scenari concreti prima o poi arrivano opportuno utilizzarli.

\subsection{Prototipazione e mockup}
L'obiettivo è di fatto di raccogliere il feedback immediato con lo schizzo 
del software i nazione. Il prototipo potrebbe avere feature incomplete 
potrebbe essere funzionale per validare funzionalità oppure per interfacce grafiche 

In molti casi la produzione di prototipi va a pari passo con i requisiti.
Un mock up è tipicamente un prodotto che realizzo e poi butto, un 
prototipo posso raffinarlo.
Può formulare feedback immediato.

Come pro abbiamo che il feedback è simile al software finito, per verificare 
requisiti e vincoli. Sono utili per la validazione e la verifica. Anche per 
sorgere mancanze e errori. Sono utili per la comunicazione con gli stakeholders.
Sono utili anche per la formazione agli utenti, in modo da capire come
usare il software finale, o usati come stub per integration testing.

Gli svantaggi sono che potrebbero essere che non coprono tutti i requisiti.
Mancano funzionalità e potrebbero non catturare i requisiti non funzionali.
legati alle performance. Possono essere forvianti, aspettative troppo alte.
Potrebbero essere  non del tutto implementabili e potrebbero esserci delle 
inconsistenze con il software finale.

\subsection{Knowledge reuse}
Prendiamo la conoscenza su altre esperienze di cose simile per avere 
una elicitazione di requisiti più spedita.
Tipicamente si individua la conoscenza simile, la si traspone e poi 
si valida il risultato e lo si integra.
La trasposizione avviene in 3 modi:
\begin{itemize}
  \item Istanziazione: 
  \item Specializzazione:
  \item Riformulazione:
\end{itemize}
Le tassonomie sono utili per trovare la conoscenza simile. Indipendenti dal 
dominio.
Si usano anche i meta-modelli per trovare la conoscenza simile, istanziando ad 
esempio oggetto, goal, agenti e operazioni. Posso usare Informazioni specifiche 
su un dominio astratto e associarlo al dominio concreto specializzando entità.

In che modo posso personalizzare la conoscenza?
Lo stesso dominio astratto può avere specializzazioni multiple.
Alcuni domini concreti possono specializzare alcuni domini astratti.

Pro e contro:
L'analista esperto può usare in maniera efficiente la conoscenza e 
valociza la conoscenza.
Si eredita la struttura e la qualità del dominio, è abbastanza efficace nel 
caso si voglia completare il documento dei requisiti.

Negativo è che possiamo utilizzare conoscenza stratta solo se è abbastanza vicino.
Preparare i domini astratti è complesso, lavoro preparatorio è complesso e tipicamente 
si usa quello in letteratura. Effort non banale da fare.

\section{Coinvolgimento degli stakeholders}
Attraverso interviste. Organizzare meeting, registrare il contenuto di 
interviste, attraverso i trascritti. Devono essere elaborati, 
cosa abbiamo capito va sempre linkato e questo report va condiviso con
l'intervistato in modo da verificare la corretta interpretazione.

Se coinvolgiamo i gruppi alcuni potrebbero essere più influenti di altri.
Quindi perdiamo alcune informazione, si va più valoce. A volte si preferisce coinvolgere 
una a due persone per volta.

L'efficacia dell'intervista è misurabile.

Le interviste possono essere strutturate, alcune domande chiuse 
o aperte. è opportuno avere la lista di domande per avere un traccia per 
rimanere su binari opportuni. Oppure potrebbero essere non strutturate,
abbiamo quindi solo una lista di argomenti da trattare, raggiungiamo ogni 
argomento di volta in volta.
I punti di forza sono che: le info che raccogliamo potrebbero essere difficili 
da raccogliere in altri modi, possiamo avere un feedback immediato, possiamo 
raffinare la metodologia al volo, grazie alla formulazione possiamo 
modificare le domande successive, personalizzare le informazioni successive.

Gli svantaggi sono che potrebbero essere specifiche sulla persona, ci pososno 
essere inconsistenze. L'efficacia delle interviste dipende dalla competenza dell'intervistatore.
Per migliorare abbiamo linea guida:
Identificare le persone da coinvolgere, arrivare preparati e focalizzarci del 
giusto problema e del giusto momento. Focalizzare sul lavoro che  L'intervistato 
fa. Non lasciare l'intervistato solo, mantenendo il controllo dell'intervista.
L'intervistato in una situazione di comfort. Formulare all'inizio domande semplici,
considerare anche aspetti personali essendo anche empatici, apparire affidabili e amichevoli.
Essere focalizzati è importante, essere flessibili con risposte che ci 
sorprendono. Formulare domande sul perché e cercare di non essere offensivi.
Domande con bias vanno evitate. Domande ovvie, o difficili da rispondere.
Elaborare gli sbobinamenti appena dopo annotando con reazioni personali. 
che possono essere utili per la comprensione. Fargli avere gli elaborati per 
verificare la correttezza.

\subsection{Etnografia}
Osservare persone quando lavorano, system as is senza intrgire direttamente 
con loro. Passiva da parte dell'osservatore. Oppure coinvolto diventando anche 
membro del team.

I pro e i contro:
Questo tipo di indagine danno conoscenza implicita difficile da cogliere.
Rivelare modi articolati e complessi e permette di avere aspetti specifici della cultura aziendale 
quando raccogliamo i requisiti e riusciamo a contestualizzare i requisiti.

Lo svantaggio è che può essere lungo da fare, costa molto. Potrebbe essere 
inaccurato, sapendo di essere osservati potrebbero comportarsi in modo diverso.
Sarebbe interessante vedere errori. Ci si concentra sul sistema attuale.

\subsection{Sessioni di gruppo}
Sessioni in cui si discute in gruppo dei requisiti in maniera strutturata.
Opportuno che sia gestito con sessioni strutturate, con il moderatore.
Focus grop con temi specifici

Le sessioni possono essere meno strutturate, con brainstorming.
Si discute a ruota libera e si raccoglie tutto. Tipicamente con 
due fasi, nella prima ognuno lancia idee che vengono in mente 
a ruota libera, offrendo anche spunto dalle idee date.
Nella seconda fare si converge o di prioritizzare le idee.

Pro e contro:
SOno meno formali delle interviste quindi. Possono nascere delle sinnergie tra utenti 
per risolvere problemi. La composizione del gruppo è critico, a volte molti sono 
impegnati. è importante avere un moderatore con alte skill.
A volte emergono persone che dominano la discussione, divergendo quindi 
la discussione su loro idee. Il rischio è che si perda il focus, in questo caso si rischia
di perdere tempo. Aspetti superficiali non rilevanti.
\subsection{Combinare le tecniche}
Di solito di combinano tali tecniche con un mix compensando gli svantaggi 
di alcune tecniche con i vantaggi di altre.
\chapter{Specifica dei requisiti e documentazione}
Potrei avere errori con il linguaggio naturale, 
perché è intrinsescamente ambiguo.
Ci sono linea guida per scrivere requisiti,
riducendo l'ambiguità per raggiungere una certa completezza.
Abbiamo delle guide stilistiche per scrivere i requisiti.
\begin{itemize}
    \item Identificare a chi è rivolto il requisito.
    \item Brevi abstract per ogni requisito, per capire subito di cosa si tratta.
    \item Le cose vanno motivate prima di spiegare il requisito.
    \item Prima di usare un concetto è bene definirlo.
    \item Chiedersi se è sufficiente, comprensibile, rilevante.
    \item Non cercare di spiegare troppo in una frase, meglio 
    scrivere più frasi per spiegare meglio.
    \item Usare "deve" e "non deve" per esprimere i requisiti o 
    "dovrebbe" e "non dovrebbe" per esprimere i desiderata.
    \item Evitare gergo o acronimi.
    \item Usare esempi per spiegare meglio.
    \item Usare diagrammi per spiegare meglio relazioni 
    complesse.
\end{itemize}
\section{Regole locali}
Con condizioni complesse è bene usare 
una tabella di verità per spiegare meglio il concetto.
In questo caso l'interpretazione di \texttt{AND} 
e \texttt{OR} è ben definita. Il problema è che le combinazioni 
possono essere molte e quindi la tabella può diventare molto grande.
In questo caso sarà possibile collassare degli elementi.
Con una tabella di questo tipo è possibile avere i test 
di accettazione da far girare sul sistema non appena sarà
disponibile.
\section{Formare le frasi}
\begin{itemize}
    \item Requisiti enumerati e gerarchici.
    \item Requisiti divisi per categorie.
    \item Seguire regole stilistiche.
    \item Regole di fit.
    \item Tracciare le sorgenti dei requisiti.
    \item Motivo.
    \item Interazioni di conflitto o dipendenza.
    \item priorità.
\end{itemize}
Importante avere un criterio di fit per capire se la specifica 
è soddisfatta o meno dal sistema.
Requisiti quantitativi attraverso la soglia percentuale.
Anche attraverso interviste successive.

È opportuno ordinare i requisiti per sezioni, 
in modo da avere requisiti comuni vicini.

\section{Diagrammi}
Un altro modo per raccogliere i requisiti è attraverso i diagrammi.
Che sono però semi-formali.

\subsection{Context diagram}
Si tratta di un grafo che contiene i componenti del sistema (\textit{vertici})
e le relazioni tra i componenti (\textit{archi}).
Ad esempio il guidatore e il sistema di frenata sono due vertici,
mentre l'arco tra i due rappresenta il fatto che il guidatore
può azionare il sistema di frenata.
\subsection{Problem diagram}
È simile al context diagram, ma è più dettagliato, perché
contiene chi fa cosa.
Contiene anche notazioni in linguaggio naturale.
Ragiono i ntermini del problema e quali requisiti fanno 
riferimento a quali compinenti del sistema.
\subsection{Frame diagram}
Questo diagramma è più informativo perché permette 
di tipare i componenti e i fenomeni.
I componenti possono essere: Causali, Evento, Simbolico.
I fenomeni possono essere: Causale, Biddable, Lessicale.
Si va nella direzione del riutilizzo della conoscenza.

Ci sono altri diagrammi per ragionare in termini di 
strutturazione del sistema.
\subsection{Entity-Relationship diagram}
Diagramma visto anche in basi di dati.
è possibile specializzare le classi per aggiungere più 
proprietà.

Ci sono diagrammi che permettono di parlare del sistema
\subsection{SADT}


\end{document}
