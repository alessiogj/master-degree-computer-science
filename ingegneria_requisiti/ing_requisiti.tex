\documentclass[oneside,a4paper,11pt]{book}
\usepackage[utf8]{inputenc}
\usepackage{svg}
\usepackage[italian]{babel}
\usepackage{float}
\usepackage{fancyvrb}
\usepackage{titling}
\usepackage[margin=1in,footskip=0.25in]{geometry}
\usepackage{listings}
\usepackage[DIV=12,BCOR=2mm,headinclude=true,footinclude=false]{typearea}
\usepackage{color, colortbl,xcolor}
\usepackage[hidelinks]{hyperref}
\usepackage{tcolorbox}
\usepackage{chngcntr}
\usepackage{diagbox}
\usepackage{calc}
\usepackage{amssymb}
\usepackage{subcaption}
\usepackage{amsthm}
\usepackage{amsfonts}
\usepackage{mathtools}
\usepackage{parskip}
\usepackage{cancel}
\usepackage{forest}
\usepackage{listings}
\usepackage{mathrsfs}
\usepackage{enumitem}
\usepackage{makecell}
\usepackage{tikz}
\usepackage{pgfplots}
\pgfplotsset{compat=1.18}
\usepackage{fancyhdr}
\fancypagestyle{plain}{\fancyhf{}\renewcommand{\headrulewidth}{0pt}}
\pagestyle{fancy}
\fancyhf{}% Clear header/footer
\fancyhead[L]{\nouppercase\leftmark}
\fancyhead[R]{\thepage}
\usetikzlibrary{positioning,shapes.geometric,arrows.meta,matrix,automata,decorations.pathmorphing,patterns,decorations.pathreplacing,shapes.multipart,calc,snakes}
\usetikzlibrary{arrows.meta, backgrounds, chains, positioning, shapes.geometric, shapes.multipart, shadows}
\usetikzlibrary{shapes, arrows, positioning}
\tcbuselibrary{skins}
\counterwithin{figure}{section}
%Nuovi comandi
\newcommand\myeq{\stackrel{\mathclap{\normalfont\mbox{def}}}{=}}
\newcommand\prodG{\stackrel{\mathclap{\normalfont\mbox{\tiny{G}}}}{\Longrightarrow}}
%asmthm
\newlength{\marginlabelsep}\setlength{\marginlabelsep}{0.5em}
\newtheoremstyle{italicstyle} %% Name
  {} %% <- Space above (empty = default = \topsep = 8.0pt plus 2.0pt minus 4.0pt)
  {} %% <- Space below (empty = default = \topsep = 8.0pt plus 2.0pt minus 4.0pt)
  {\itshape} %% <- Body font
  {} %% <- Indent amount (empty = no indent, \parindent = just that)
  {\bfseries} %% <- Thm head font
  {} %% <- Punctuation after thm head
  {1pt} %% <- Space after thm head (or " " or \newline) (default: 5pt plus 1pt minus 1pt)
  {\vtop to 0pt{\llap{\thmname{#1}\hskip\marginlabelsep}
                \llap{\thmnumber{#2}\hskip\marginlabelsep}}\thmnote{#3\\}%
  }
\newtheoremstyle{normStyle} %% Name
  {} %% <- Space above (empty = default = \topsep = 8.0pt plus 2.0pt minus 4.0pt)
  {} %% <- Space below (empty = default = \topsep = 8.0pt plus 2.0pt minus 4.0pt)
  {\normalfont} %% <- Body font
  {} %% <- Indent amount (empty = no indent, \parindent = just that)
  {\bfseries} %% <- Thm head font
  {} %% <- Punctuation after thm head
  {1pt} %% <- Space after thm head (or " " or \newline) (default: 5pt plus 1pt minus 1pt)
  {\vtop to 0pt{\llap{\thmname{#1}\hskip\marginlabelsep}
                \llap{\thmnumber{#2}\hskip\marginlabelsep}}\thmnote{#3\\}%
  }
\theoremstyle{italicstyle}
\newtheorem{corollary}{Corollario}[section]
\newtheorem{notazione}{Notazione}[section]
\newtheorem{lemma}{Lemma}[section]
\newtheorem{definizione}{Definizione}[section]
\newtheorem{nota}{Nota}[section]
\newtheorem{exercise}{Esercizio}[section]
\theoremstyle{normStyle}
\newtheorem{exmp}{Esempio}[section]
\newtheorem{theorem}{Teorema}[section]
\newtheorem{proposizione}{Proposizione}[section]
\tcbuselibrary{listings,skins}
\newtcblisting{mylisting}[2][]{
    arc=0pt, outer arc=0pt,
    listing only, 
    title=#2,
    #1,
    listing options= {escapechar=|}
}

\usepackage{color}

\definecolor{pblue}{rgb}{0.13,0.13,1}
\definecolor{pgreen}{rgb}{0,0.5,0}
\definecolor{pred}{rgb}{0.9,0,0}
\definecolor{pgrey}{rgb}{0.46,0.45,0.48}

\usepackage{listings}
\lstset{language=Java,
  showspaces=false,
  showtabs=false,
  breaklines=true,
  showstringspaces=false,
  breakatwhitespace=true,
  commentstyle=\color{pgreen},
  keywordstyle=\color{pblue},
  stringstyle=\color{pred},
  basicstyle=\ttfamily,
  moredelim=[il][\textcolor{pgrey}]{$$},
  moredelim=[is][\textcolor{pgrey}]{\%\%}{\%\%}
}
\newcommand{\myboxedtext}[2][rectangle,draw]{%
    \tikz[baseline=-0.6ex] \node [#1]{#2};}%
%%======================================================================
\title{Ingegneria dei Requisiti}
\author{\textit{Alessio Gjergji}}
\date{}
\begin{document}
\begin{titlingpage}
  \centering
  \vspace*{\stretch{1}}
  \huge
  \textbf{\thetitle}\\[0.5cm]
  \normalsize
  Corso tenuto dal Professor Mariano Ceccato\\[0.5cm]
  Università degli Studi di Verona\\[1cm]
  \large
  \theauthor\\[0.5cm]
  \vspace{\stretch{2}}
\end{titlingpage}
\tableofcontents
\chapter{Introduzione}
\section{Linguaggi di programmazione}
Un linguaggio di programmazione è un linguaggio formale che specifica un
insieme di istruzioni che possono essere usate per produrre un insieme di
output.
Esso è definito da:
\begin{itemize}
    \item \textbf{Sintassi}: specifica la forma delle istruzioni. Ci permette di
    capire quali stringhe sono ammissibili e quali no mediante diversi strumenti come 
    grammatiche, analizzatori lessicali e sintattici, teoria degli automi.
    \item \textbf{Pragmatica}: specifica l'effetto delle istruzioni. Ci permette
    di capire le ragioni per introdurre un nuovo linguaggio e di programmazione 
    invece di utilizzarne uno già esistente.
    \item \textbf{Semantica}: specifica il significato dei programmi scritti nel linguaggio, ovvero il loro 
    comportamento a tempo di esecuzione. Ci permette di capire se due programmi 
    apparentemente diversi sono equivalenti.
\end{itemize}
\subsection{Benefici di una semantica formale}
I benefici dei linguaggi di programmazione diversi, tra cui:
\begin{itemize}
    \item \textbf{Implementazione}: Consente di fornire la specifica (\textit{del comportamento}) 
    dei programmi indipendentemente dalla macchina o dal compilatore utilizzato.
    \item \textbf{Verifica}: una semantica formale consente di ragionare 
    sui programmi e sulle loro proprietà di correttezza.
    \item \textbf{Progettazione di Linguaggio}: spesso una semantica formale consente di 
    scoprire ambiguità all'interno di linguaggi già esistenti. Questo aiuta a progettare 
    nuovi linguaggi in maniera più accurata.
\end{itemize}
\section{Un linguaggio per le espressioni aritmetiche: sintassi}
Definiamo il seguente linguaggio:
\[
    \mathcal{E}\quad ::= \quad n \quad | \quad \mathcal{E} + \mathcal{E} \quad | 
    \quad \mathcal{E} * \mathcal{E} \quad | \quad \dots
\]
dove:
\begin{itemize}
    \item $n$ è lo spazio del dominio dei numerali.
    \item $\mathcal{E}$ è il range del dominio delle espressioni aritmetiche.
    \item $+, x, \dots$ sono simboli del linguaggio.
\end{itemize}
I numerali sono parte della sintassi del nostro linguaggio e non vanno confusi con i numeri
che sono oggetti matematici.
Ciò potrebbe significare che nel nostro linguaggio al posto di $0, 1, \dots$ avremmo 
potuto usare $zero, uno, \dots$ e sarebbero potuti essere uguali.

Nel nostro caso assumiamo che esista una corrispondenza ovvia tra il simbolo ``numerale" (n)
e il numero naturale n. Questo è fatto solo per semplificare la spiegazione. In un
altro contesto, il simbolo ``numeral" 3 potrebbe essere associato al numero 42!

\section{Semantica Operazionale}

La semantica operazionale ha l'obiettivo di valutare un'espressione aritmetica
del linguaggio per ottenere il suo valore numerico associato. Questo può essere
fatto in due modi differenti:

\begin{itemize}
  \item \textbf{Semantica Small-Step (\textit{o strutturale})}: Fornisce un metodo per
  valutare un'espressione passo dopo passo, considerando le azioni intermedie.
  Questo approccio fornisce una valutazione dettagliata dell'espressione.

  \item \textbf{Semantica Big-Step (\textit{o naturale})}: Ignora i passaggi intermedi
  e fornisce direttamente il risultato finale della valutazione dell'espressione. Questo
  approccio semplifica la valutazione, concentrando l'attenzione sul risultato finale.

\end{itemize}
\subsection{Big-Step Semantics}

\begin{tcolorbox}[title = {Valutazione}]  
    $E \Downarrow n$
\end{tcolorbox}
\textbf{Significato}: La valutazione dell'espressione $\mathcal{E}$ produce il numerale $n$.

\begin{tcolorbox}[title = {Assiomi e regole di inferenza}]  
\begin{figure}[H]
    \begin{subfigure}{0.3\textwidth}
    \begin{prooftree}
        \AxiomC{$-$}
        \LeftLabel{(B-Num)}
        \UnaryInfC{$n \Downarrow n$}
    \end{prooftree}
    \end{subfigure}%
    \begin{subfigure}{0.7\textwidth}
    \begin{prooftree}
        \AxiomC{$\mathcal{E}_1 \Downarrow n_1$}
        \AxiomC{$\mathcal{E}_2 \Downarrow n_2$}
        \LeftLabel{(B-Add)}
        \RightLabel{$n_3 = add(n_1, n_2)$}
        \BinaryInfC{$\mathcal{E}_1 + \mathcal{E}_2 \Downarrow n_3$}
    \end{prooftree}
    \end{subfigure}
\end{figure}
\end{tcolorbox}
\textbf{Significato}: 
\begin{itemize}
\item (B-Num): Questo è un assioma che afferma che quando valutiamo un singolo
numero $n$, otteniamo lo stesso numero $n$ come risultato. Questo è il caso
base della valutazione.

\item (B-Add): Questa regola di inferenza afferma che date due espressioni
$\mathcal{E}_1$ e $\mathcal{E}_2$:
\begin{itemize}
  \item Se è il caso che $\mathcal{E}_1 \Downarrow n_1$ (cioè $\mathcal{E}_1$ si valuta a $n_1$) e
  \item È anche il caso che $\mathcal{E}_2 \Downarrow n_2$ (cioè $\mathcal{E}_2$ si valuta a $n_2$),
  allora segue che $\mathcal{E}_1 + \mathcal{E}_2 \Downarrow n_3$, dove $n_3$ è il numerale associato
  al numero $n_3$ tale che $n_3 = add(n_1, n_2)$.
  Si noti che in questa regola, $E1$, $E2$, $n1$, $n2$, $n3$ sono meta-variabili.
\end{itemize}
\end{itemize}
Questa regola (B-Add) ci dice come valutare un'addizione tra due espressioni
$\mathcal{E}_1$ e $\mathcal{E}_2$ nel contesto della semantica big-step. La
regola stabilisce che se possiamo valutare entrambe le espressioni operandi
($\mathcal{E}_1$ e $\mathcal{E}_2$) e otteniamo i numeri $n_1$ e $n_2$ rispettivamente,
allora possiamo calcolare la somma di $\mathcal{E}_1$ e $\mathcal{E}_2$ come $n_3$,
dove $n_3$ è il risultato della somma dei numeri $n_1$ e $n_2$.
Si noti che la funzione di addizione $add$ opera sui numeri, non sui numerali.
\subsection{Small-Step Semantics}

\begin{tcolorbox}[title = {Valutazione}]  
$\mathcal{E}_1 \rightarrow \mathcal{E}_2$

\end{tcolorbox}
\textbf{Significato:} 
Dopo aver eseguito un passo di valutazione su $\mathcal{E}_1$, l'espressione $\mathcal{E}_2$ rimane da valutare.
\begin{tcolorbox}[title = {Assiomi e regole di inferenza}]  
\begin{prooftree}
    \AxiomC{$\mathcal{E}_1 \rightarrow \mathcal{E}_1'$}
    \LeftLabel{(S-Left)}
    \UnaryInfC{$\mathcal{E}_1 + \mathcal{E}_2 \rightarrow \mathcal{E}_1' + \mathcal{E}_2$}
    \end{prooftree}
    
    \begin{prooftree}
    \AxiomC{$\mathcal{E}_2 \rightarrow \mathcal{E}_2'$}
    \LeftLabel{(S-N.Right)}
    \UnaryInfC{$n_1 + \mathcal{E}_2 \rightarrow n_1 + \mathcal{E}_2'$}
    \end{prooftree}
    
    \begin{prooftree}
    \AxiomC{-}
    \LeftLabel{(S-Add)}
    \RightLabel{$n_3 = add(n_1, n_2)$}
    \UnaryInfC{$n_1 + n_2 \rightarrow n_3$}
    \RightLabel{(S-Add)}
\end{prooftree}
\end{tcolorbox}
Fissiamo l'ordine di valutazione da sinistra a destra. Qualcosa di 
simile non è possibile nella big-step semantics, dove le espressioni sono 
valutate in un solo passo.
\subsubsection{La scelta dell'ordine di valutazione}
\begin{tcolorbox}[title = {Assiomi e regole di inferenza}]  
    \begin{prooftree}
        \AxiomC{$\mathcal{E}_1 \rightarrow_{ch} \mathcal{E}_1'$}
        \LeftLabel{(S-Left)}
        \UnaryInfC{$\mathcal{E}_1 + \mathcal{E}_2 \rightarrow_{ch} \mathcal{E}_1' + \mathcal{E}_2$}
        \end{prooftree}
        
        \begin{prooftree}
        \AxiomC{$\mathcal{E}_2 \rightarrow_{ch} \mathcal{E}_2'$}
        \LeftLabel{(S-Right)}
        \UnaryInfC{$\mathcal{E}_1 + \mathcal{E}_2 \rightarrow_{ch} \mathcal{E}_1 + \mathcal{E}_2'$}
        \end{prooftree}
        
        \begin{prooftree}
        \AxiomC{-}
        \LeftLabel{(S-Add)}
        \RightLabel{$n_3 = add(n_1, n_2)$}
        \UnaryInfC{$n_1 + n_2 \rightarrow_{ch} n_3$}
        \RightLabel{(S-Add)}
    \end{prooftree}
\end{tcolorbox}
In questo caso non abbiamo precedenza stabilita per la valutazione delle espressioni.
Regole simili possono essere applicate anche con gli altri operatori.
\subsubsection{Esecuzione della small-step semantics}
La relazione $\rightarrow^k$, per $k \in \mathbb{N}$ è definita per un numero di passi 
di valutazione definito da $k$.
Mentre la relazione $\rightarrow^*$ è definita per un numero non definito di passi di valutazione.
\chapter{Elicitazione dei requisiti}
L'obiettivo comprendere il dominio in cui stiamo lavorando e 
raccogliere e capire i requisiti del software che dovrà essere
implementato. 
\section{Acquisizione della conoscenza}

L'acquisizione della conoscenza è una fase cruciale nel processo di
ingegneria dei requisiti e include diverse attività importanti.

In primo luogo, è essenziale studiare il sistema attuale, noto come sistema
\textit{as-is}. Questo studio comprende la comprensione dell'organizzazione
aziendale (\textit{struttura, dipendenze, obiettivi strategici, politiche, flussi di
lavoro e procedure operative}) e del dominio applicativo (\textit{concetti, obiettivi,
compiti, vincoli e regolamenti}). Inoltre, è necessario analizzare i problemi
esistenti nel sistema attuale, identificandone sintomi, cause e conseguenze.

Un'altra attività chiave è l'analisi delle opportunità tecnologiche e delle
nuove condizioni di mercato, per valutare le possibilità offerte dalle nuove
tecnologie e comprendere come i cambiamenti nelle condizioni di mercato possano
influenzare il sistema.

Identificare gli stakeholder del sistema è fondamentale. Gli stakeholder sono
tutte le parti interessate che hanno un'influenza o sono influenzate dal sistema.
Comprendere le loro esigenze e aspettative è vitale per il successo del progetto.

L'identificazione degli obiettivi di miglioramento per il sistema futuro, noto
come sistema \textit{to-be}, è un'altra attività importante. Questo include
l'analisi dei vincoli organizzativi e tecnici, l'esplorazione di opzioni
alternative, la definizione delle responsabilità e lo sviluppo di scenari
ipotetici di interazione tra software e ambiente. Infine, è cruciale stabilire
i requisiti specifici per il software e fare assunzioni sull'ambiente operativo.

Queste attività forniscono una base solida per il processo di ingegneria
dei requisiti, assicurando che il sistema sviluppato soddisfi le esigenze
degli stakeholder e funzioni efficacemente nel suo ambiente operativo.

Ci sono sostanzialmente due macro-approcci per l'acquisizione della conoscenza:
\begin{itemize}
    \item \textbf{Artefact-driven}: Questo approccio si basa sull'analisi
    di documenti, modelli e altri artefatti esistenti per comprendere il
    sistema attuale e identificare i requisiti del sistema futuro. È
    utile quando il sistema attuale è ben documentato e i requisiti
    del sistema futuro sono chiari.
    \item \textbf{Stakeholder-driven}: Questo approccio si basa sull'interazione
    diretta con gli stakeholder per comprendere le loro esigenze e aspettative
    e identificare i requisiti del sistema futuro. È utile quando il sistema
    attuale è poco documentato o i requisiti del sistema futuro sono incerti.
\end{itemize}
\subsection{Analisi degli stakeholder}

\begin{tcolorbox}[colback=blue!5!white,colframe=blue!75!black]
    Gli stakeholder sono tutte le parti interessate che hanno un'influenza o sono
    influenzate dal sistema.
\end{tcolorbox}

La cooperazione degli stakeholder è essenziale per il
successo dell'ingegneria dei requisiti. Il processo di elicitazione de
 requisiti può essere visto come un apprendimento cooperativo, dove la
 collaborazione tra tutte le parti coinvolte porta a una migliore comprensione
 dei requisiti del sistema.

Per garantire una copertura adeguata e comprensiva del mondo dei problemi,
è necessario selezionare un campione rappresentativo degli stakeholder.
Questa selezione deve essere dinamica, adattandosi man mano che si acquisiscono
nuove conoscenze.

La selezione degli stakeholder si basa su diversi criteri, tra cui:

\begin{itemize}
    \item La posizione rilevante nell'organizzazione.
    \item Il ruolo nella presa di decisioni e nel raggiungimento degli accordi.
    \item Il tipo di conoscenza contribuita e il livello di competenza nel dominio.
    \item L'esposizione ai problemi percepiti.
    \item Gli interessi personali e i potenziali conflitti.
    \item L'influenza nell'accettazione del sistema.
\end{itemize}

Selezionare gli stakeholder giusti è cruciale per assicurare che tutte le
prospettive rilevanti siano considerate e che il sistema finale soddisfi le
esigenze di tutti gli interessati.

\subsection{Difficoltà nell'acquisizione della conoscenza dagli stakeholder}

L'acquisizione della conoscenza dagli stakeholder presenta diverse difficoltà,
tra cui fonti di informazione distribuite, punti di vista conflittuali e
difficoltà di accesso alle persone chiave e ai dati. Gli stakeholder possono
avere background, terminologie e culture differenti, e la conoscenza tacita o
le esigenze nascoste possono complicare ulteriormente il processo. Inoltre,
i dettagli irrilevanti, la politica interna, la competizione e la resistenza
al cambiamento possono influire negativamente.

Il turnover del personale e i cambiamenti organizzativi e delle priorità
possono creare ulteriori sfide. 

Per superare queste difficoltà, sono essenziali abilità di comunicazione,
costruzione di relazioni di fiducia e riformulazione continua della conoscenza
attraverso riunioni di revisione.

\subsection{Studio del background}

Il processo di acquisizione della conoscenza inizia con la raccolta, la
lettura e la sintesi dei documenti pertinenti. Questi documenti riguardano:

\begin{itemize}
    \item \textbf{L'organizzazione}: include organigrammi, piani aziendali, 
    rapporti finanziari, verbali di riunioni, ecc.
    \item \textbf{Il dominio}: comprende libri, indagini, articoli, regolamenti
    e rapporti su sistemi simili nello stesso dominio.
    \item \textbf{Il sistema attuale (as-is)}: include flussi di lavoro
    documentati, procedure, regole aziendali, documenti scambiati, rapporti
    di difetti/reclami, richieste di modifica, ecc.
\end{itemize}

Questo studio fornisce le basi necessarie per prepararsi prima di incontrare
gli stakeholder e rappresenta un prerequisito per altre tecniche. Tuttavia,
un problema comune è la gestione di una grande quantità di documentazione,
dettagli irrilevanti e informazioni obsolete. La soluzione è utilizzare la
meta-conoscenza per selezionare le informazioni rilevanti, sapendo cosa è
necessario conoscere e cosa non lo è.

\subsection{Raccolta dei dati}

La raccolta dei dati è un'attività importante che si concentra sulla raccolta
di fatti e cifre non documentati. Questi dati possono includere informazioni
di marketing, statistiche di utilizzo, dati sulle prestazioni e costi. La
raccolta può avvenire tramite esperimenti progettati o tramite la selezione
di set di dati rappresentativi da fonti disponibili, utilizzando tecniche di
campionamento statistico.

Questa attività può integrare lo studio del background, fornendo ulteriori
informazioni utili per l'elicitazione dei requisiti non funzionali relativi
a prestazioni, usabilità e costi.

Tuttavia, ci sono alcune difficoltà nella raccolta dei dati. Ottenere dati
affidabili può richiedere tempo e i dati devono essere interpretati
correttamente per essere utili.
\subsection{Questionari}

L'utilizzo dei questionari è un metodo efficace per raccogliere informazioni
dagli stakeholder in modo rapido, economico e a distanza. I questionari
consistono in una lista di domande inviate agli stakeholder selezionati,
ognuna con una lista di possibili risposte. Possono includere:

\begin{itemize}
    \item \textit{Domande a scelta multipla}: dove si seleziona una risposta
    da una lista di opzioni.
    \item \textit{Domande con pesatura}: dove si chiede di attribuire un peso
    a una lista di affermazioni, qualitativamente (ad esempio, ``alto'' o ``basso'')
    o quantitativamente (percentuali), per esprimere l'importanza percepita,
    le preferenze, i rischi, ecc.
\end{itemize}

I questionari sono utili per ottenere rapidamente informazioni soggettive da
molte persone e possono essere d'aiuto nella preparazione di interviste più
focalizzate.

Tuttavia, i questionari devono essere preparati con attenzione per evitare
bias multipli (\textit{dei destinatari, dei rispondenti, delle domande, delle risposte})
e informazioni inaffidabili dovute a fraintendimenti o risposte incoerenti.

Ecco alcune linee guida per la progettazione e la validazione dei questionari:

\begin{itemize}
    \item Selezionare un campione rappresentativo e statisticamente significativo
    di persone, fornendo motivazioni per rispondere.
    \item Verificare la copertura delle domande e delle possibili risposte.
    \item Assicurarsi che le domande e le formulazioni siano imparziali e non
    ambigue.
    \item Aggiungere domande ridondanti implicitamente per rilevare risposte
    incoerenti.
    \item Far controllare il questionario da una terza parte.
\end{itemize}

\begin{tcolorbox}[colback=green!5!white,colframe=green!75!black, title=Pro dei questionari]
    \begin{itemize}
        \item Raccolta rapida di informazioni
        \item Economici e facili da distribuire
        \item Utili per preparare interviste focalizzate
    \end{itemize}
\end{tcolorbox}

\begin{tcolorbox}[colback=red!5!white,colframe=red!75!black, title=Contro dei questionari]
    \begin{itemize}
        \item Possibili bias dei destinatari e rispondenti
        \item Rischio di fraintendimenti nelle domande e risposte
        \item Difficoltà nell'assicurare risposte coerenti
    \end{itemize}
\end{tcolorbox}

\subsection{Card sorting e repertory grids}

L'obiettivo del card sorting e delle repertory grids è acquisire ulteriori
informazioni sui concetti già elicitati. Nel card sorting, si chiede agli
stakeholder di dividere un set di carte, ognuna rappresentante un concetto,
in sottogruppi basati sui loro criteri. Per ogni sottogruppo, si indaga sulle 
proprietà condivise utilizzate per la classificazione. Questo processo è 
iterativo e può essere ripetuto per nuovi raggruppamenti e proprietà.

Ad esempio, nel contesto di un sistema di pianificazione delle riunioni, 
le carte ``Riunione'' e ``Partecipante'' potrebbero essere raggruppate insieme, 
suggerendo che \textit{i partecipanti devono essere invitati alla riunione}. Nella 
successiva iterazione, lo stesso raggruppamento potrebbe indicare che
\textit{i vincoli dei partecipanti per la riunione devono essere conosciuti}.

Nel repertory grid, si chiede agli stakeholder di caratterizzare un concetto
attraverso attributi e intervalli di valori, formando una griglia
concetto-attributo. Ad esempio, per il concetto di ``Riunione'', gli attributi
potrebbero essere \textit{Data} (Lun-Ven) e \textit{Luogo} (Europa).

Il conceptual laddering, invece, richiede agli stakeholder di classificare
i concetti target lungo collegamenti di classe-sottoclasse, come ad esempio
\texttt{RiunioneRegolare} e \texttt{RiunioneOccasionale} come sottoclassi
di \texttt{Riunione}.

Questi metodi sono semplici, economici e facili da usare per l'elicitazione
rapida di informazioni mancanti, ma i risultati possono essere soggettivi,
irrilevanti o inaccurati.

\begin{tcolorbox}[colback=green!5!white,colframe=green!75!black, title=Pro del card sorting e repertory grids]
    \begin{itemize}
        \item Semplici ed economici
        \item Facili da usare
        \item Rapida elicitazione di informazioni
    \end{itemize}
\end{tcolorbox}

\begin{tcolorbox}[colback=red!5!white,colframe=red!75!black, title=Contro del card sorting e repertory grids]
    \begin{itemize}
        \item Risultati possono essere soggettivi
        \item Possibilità di informazioni irrilevanti
        \item Possibili inaccuratezze nei risultati
    \end{itemize}
\end{tcolorbox}

\subsection{Scenari e storyboard}

Gli scenari e gli storyboard aiutano ad acquisire o validare informazioni
attraverso esempi concreti e narrazioni. Gli scenari illustrano sequenze
tipiche di interazione tra i componenti del sistema per raggiungere un
obiettivo implicito, utilizzati sia per spiegare il sistema attuale
(\textit{as-is}) che per esplorare il sistema futuro (\textit{to-be}).

Gli storyboard raccontano una storia attraverso una sequenza di istantanee,
che possono essere frasi, schizzi, diapositive o immagini, e possono includere
annotazioni che spiegano chi sono i partecipanti, cosa accade loro, perché
accade e cosa succede in caso di eventi alternativi.

Gli scenari possono essere:

\begin{itemize}
    \item \textit{Scenario positivo}: un comportamento che il sistema dovrebbe
    coprire.
    \item \textit{Scenario negativo}: un comportamento che il sistema dovrebbe
    escludere.
    \item \textit{Scenario normale}: tutto procede come previsto.
    \item \textit{Scenario anomalo}: una sequenza di interazione desiderata in
    situazioni di eccezione.
\end{itemize}

\begin{tcolorbox}[colback=green!5!white,colframe=green!75!black, title=Pro degli
    scenari e storyboard]
    \begin{itemize}
        \item Esempi concreti e contro-esempi
        \item Stile narrativo (\textit{appealing to stakeholders})
        \item Producono sequenze di animazione, casi di test di accettazione
    \end{itemize}
\end{tcolorbox}

\begin{tcolorbox}[colback=red!5!white,colframe=red!75!black, title=Contro degli
    scenari e storyboard]
    \begin{itemize}
        \item Inerentemente parziali (\textit{problema di copertura del test})
        \item Esplosione combinatoria (\textit{cf. tracce di programma})
        \item Sovraspecificazione potenziale: sequenziamento non necessario,
        confini prematuri software-ambiente
        \item Possono contenere dettagli irrilevanti, granularità incompatibili
        tra diversi stakeholder
        \item Mantengono i requisiti impliciti
    \end{itemize}
\end{tcolorbox}

Nonostante ciò, sono preziosi come veicoli iniziali per l'elicitazione dei
requisiti.

\subsection{Prototipi e mock-up}

L'obiettivo dei prototipi e dei mock-up è verificare l'adeguatezza dei requisiti
attraverso il feedback diretto degli utenti, mostrando uno schizzo ridotto del
software futuro in azione. Questo metodo si concentra su requisiti poco chiari
e difficili da formulare per elicitare ulteriori dettagli.

Un prototipo è un'implementazione rapida di alcuni aspetti del sistema:
\begin{itemize}
    \item \textit{Prototipo funzionale}: si focalizza su requisiti funzionali
    specifici, come l'avvio di una riunione o la raccolta dei vincoli dei
    partecipanti.
    \item \textit{Prototipo dell'interfaccia utente}: si concentra sulla
    usabilità, mostrando moduli di input-output e pattern di dialogo.
\end{itemize}

I prototipi possono essere implementati rapidamente utilizzando linguaggi
di programmazione di alto livello, linguaggi di specifica eseguibile, e
servizi generici.

\textit{Prototipazione dei requisiti} include due approcci principali:
\begin{itemize}
    \item \textit{Mock-up}: il prototipo viene scartato una volta che ha
    soddisfatto il suo scopo di chiarire e validare i requisiti.
    \item \textit{Prototipo evolutivo}: il prototipo viene trasformato
    e perfezionato fino a diventare parte del prodotto finale.
\end{itemize}

\begin{tcolorbox}[colback=green!5!white,colframe=green!75!black, title=Pro
    dei prototipi e mock-up]
    \begin{itemize}
        \item Forniscono un'idea concreta di come sarà il software
        \item Chiariscono i requisiti, elicitano quelli nascosti, migliorano
        l'adeguatezza, e aiutano a comprendere le implicazioni
        \item Utili anche per la formazione degli utenti e come stub per test
        di integrazione
    \end{itemize}
\end{tcolorbox}

\begin{tcolorbox}[colback=red!5!white,colframe=red!75!black, title=Contro
    dei prototipi e mock-up]
    \begin{itemize}
        \item Non coprono tutti gli aspetti del sistema
        \item Possono mancare funzionalità importanti
        \item Ignorano requisiti non funzionali rilevanti
        (\textit{prestazioni, costi, ecc.})
        \item Possono creare aspettative troppo alte e fuorvianti
        \item Codice ``quick-and-dirty" difficile da riutilizzare per lo sviluppo
        del software
        \item Potenziali incongruenze tra codice modificato e requisiti documentati
    \end{itemize}
\end{tcolorbox}

\chapter{Validazione}
L'obiettivo di questa fase è quello di prendere delle decisioni in caso di inconsistenze o 
quali decisioni prendere in maniera negoziata.
\section{Decisione Basata sulla Negoziazione}

La decisione basata sulla negoziazione nel processo di ingegneria dei sistemi
comporta diversi passaggi per affrontare e risolvere varie problematiche che emergono
durante il processo di sviluppo. Di seguito, discutiamo i principali componenti
evidenziati nel processo decisionale:

\begin{itemize}
    \item \textbf{Identificazione e Risoluzione delle inconsistenze}: Questo passaggio
    implica la comprensione e la risoluzione dei punti di vista degli stakeholder in
    conflitto e delle richieste non funzionali per raggiungere un consenso.
    \item \textbf{Identificazione, valutazione e risoluzione dei Rischi di sistema}:
    Questo passaggio è fondamentale per garantire che il sistema soddisfi tutti gli
    obiettivi critici di sicurezza e protezione. Include la revisione dei requisiti
    per sviluppare un sistema più robusto.
    \item \textbf{Confronto delle opzioni alternative}: Si considerano varie opzioni
    per raggiungere gli obiettivi, assegnare responsabilità e risolvere conflitti e
    rischi, il che aiuta nella selezione delle soluzioni più appropriate.
    \item \textbf{Prioritizzazione dei requisiti}: La prioritizzazione dei requisiti
    è essenziale per risolvere i conflitti, aderire ai vincoli di budget e di programma,
    e supportare lo sviluppo incrementale.
\end{itemize}
\section{Gestione delle Inconsistenze}
\subsection{Tipologie di Inconsistenze}
\subsection{Come affrontare le Inconsistenze}
\subsection{Come risolvere le Inconsistenze}

\section{Analisi dei rischi}
\subsection{Tipologie di rischi}
\subsection{Come affrontare i rischi}
\subsection{Come documentare i rischi}
\subsection{Come risolvere i rischi}

\section{Confronto delle opzioni alternative}

\section{Prioritizzazione dei requisiti}
\end{document}
