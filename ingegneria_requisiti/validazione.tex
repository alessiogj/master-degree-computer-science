\chapter{Validazione}
L'obiettivo di questa fase è quello di prendere delle decisioni in caso di inconsistenze o 
quali decisioni prendere in maniera negoziata.
\section{Decisione Basata sulla Negoziazione}

La decisione basata sulla negoziazione nel processo di ingegneria dei sistemi
comporta diversi passaggi per affrontare e risolvere varie problematiche che emergono
durante il processo di sviluppo. Di seguito, discutiamo i principali componenti
evidenziati nel processo decisionale:

\begin{itemize}
    \item \textbf{Identificazione e Risoluzione delle inconsistenze}: Questo passaggio
    implica la comprensione e la risoluzione dei punti di vista degli stakeholder in
    conflitto e delle richieste non funzionali per raggiungere un consenso.
    \item \textbf{Identificazione, valutazione e risoluzione dei Rischi di sistema}:
    Questo passaggio è fondamentale per garantire che il sistema soddisfi tutti gli
    obiettivi critici di sicurezza e protezione. Include la revisione dei requisiti
    per sviluppare un sistema più robusto.
    \item \textbf{Confronto delle opzioni alternative}: Si considerano varie opzioni
    per raggiungere gli obiettivi, assegnare responsabilità e risolvere conflitti e
    rischi, il che aiuta nella selezione delle soluzioni più appropriate.
    \item \textbf{Prioritizzazione dei requisiti}: La prioritizzazione dei requisiti
    è essenziale per risolvere i conflitti, aderire ai vincoli di budget e di programma,
    e supportare lo sviluppo incrementale.
\end{itemize}
\section{Gestione delle Inconsistenze}
L'inconsistenza rappresenta una violazione delle regole di coerenza tra vari
elementi e è un fenomeno altamente frequente nell'ingegneria dei requisiti.
Queste incongruenze possono sorgere da diverse prospettive:

\begin{itemize}
    \item \textbf{Inter-punti di vista}: Ogni stakeholder ha le proprie priorità
    e preoccupazioni, che possono variare significativamente. Per esempio, gli
    esperti di dominio possono avere requisiti che contrastano con quelli del
    reparto marketing.
    \item \textbf{Intra-punto di vista}: Possono emergere conflitti anche all'interno
    dello stesso punto di vista, come tra requisiti di qualità che si contraddicono
    (\textit{ad esempio, sicurezza vs. usabilità}).
\end{itemize}

l'inconsistenza può rappresentare anche un valore,
permettendo di avere due punti di vista differenti circa uno stesso requisito. Facendo 
quindi scoprire nuovi requisiti. 

Le inconsistenze devono essere rilevate e risolte in modo tempestivo:
\begin{itemize}
    \item \textbf{Non troppo presto}: perché potremmo avere una visione parziale 
    dei requisiti.
    \item \textbf{Non troppo tardi}: perché va risolta prima della progettazione e 
    la realizzazione del sistema.
\end{itemize}

\subsection{Tipi di Inconsistenza nell'Ingegneria dei Requisiti}

Nell'ambito dell'ingegneria dei requisiti, si possono identificare diversi tipi di
inconsistenze, ciascuno con specifiche sfide da gestire:

\subsubsection{Clash di Terminologia}
Questo tipo di inconsistenza si verifica quando lo stesso concetto viene nominato
in modi diversi nelle diverse dichiarazioni. Ad esempio, nella gestione di una
biblioteca, i termini \textit{borrower} e \textit{patron} possono essere
utilizzati per indicare la stessa figura.

\subsubsection{Clash di Designazione}
Si verifica un clash di designazione quando lo stesso nome viene utilizzato
per concetti differenti. Ad esempio, il termine \textit{user} potrebbe riferirsi
sia a un \textit{library user} che a un \textit{library software user},
creando confusione.

\subsubsection{Clash Strutturale}
Questo tipo di inconsistenza si verifica quando lo stesso concetto è strutturato
in modo diverso in dichiarazioni differenti. Un esempio è la \textit{latest return date},
che può essere interpretata sia come un punto temporale specifico (\textit{es. Venerdì alle 17:00})
che come un intervallo di tempo (\textit{es. il Venerdì}).

\subsubsection{Conflitto Forte}
Un conflitto forte si verifica quando le dichiarazioni sono logicamente inconsistenti
e non possono essere soddisfatte contemporaneamente, ad esempio, ``le restrizioni dei
partecipanti non possono essere divulgate a nessuno" contro ``l'iniziatore dell'incontro
dovrebbe conoscere le restrizioni dei partecipanti".

\subsubsection{Conflitto Debole (\textit{Divergenza})}
I conflitti deboli avvengono quando le dichiarazioni non possono essere soddisfatte
insieme sotto certe condizioni limite. Questi sono molto più frequenti nell'ingegneria 
dei requisiti. Ad esempio,
da un punto di vista del personale, ``i lettori devono restituire i libri presi
in prestito entro due settimane" può confliggere con la prospettiva del lettore che
``i lettori possono tenere i libri presi in prestito finché necessario",
se il lettore ha bisogno del libro per più di due settimane.

Questi tipi di inconsistenze richiedono attenzione e strategie specifiche per la loro
risoluzione, per garantire che il processo di sviluppo dei requisiti sia efficace e
che il prodotto finale soddisfi tutte le parti interessate.

\subsection{Gestione delle Inconsistenze}

La gestione efficace delle inconsistenze è fondamentale nell'ingegneria dei requisiti.
Ecco alcune strategie per affrontare questi problemi:

\subsubsection{Gestione dei Conflitti di Terminologia, Designazione e Struttura}
Per mitigare i conflitti di terminologia, designazione e struttura, è utile sviluppare
un \textbf{glossario} condiviso di termini durante la fase di raccolta dei requisiti. Questo
glossario dovrebbe includere:
\begin{itemize}
    \item Definizioni chiare per ogni termine rilevante.
    \item Sinonimi accettati, se necessario, per garantire coerenza nella
    comunicazione tra le parti interessate.
\end{itemize}

\subsubsection{Conflitti Deboli e Forti}
I conflitti, sia deboli che forti, possono essere radicati in obiettivi personali
contrastanti degli stakeholder:
\begin{itemize}
    \item \textbf{Radice dei Conflitti}: Spesso, questi conflitti sono profondamente
    radicati negli obiettivi personali degli stakeholder e devono essere affrontati
    a questo livello per poi essere propagati al livello dei requisiti.
    \item \textbf{Conflitti Funzionali e Non-Funzionali}: Sono frequenti in questioni
    non-funzionali come il bilanciamento tra prestazione e sicurezza, o tra riservatezza
    e consapevolezza. L'esplorazione di compromessi preferenziali è essenziale.
\end{itemize}

\subsection{Come risolvere le Inconsistenze}
\begin{figure}[H]
    \centering
    \begin{tikzpicture}
        % Styles
        \tikzstyle{process} = [rectangle, rounded corners, minimum width=3cm, minimum height=1cm, text centered, draw=black, fill=blue!20, text width=2.5cm]
      
        % Nodes
        \node (start) [circle, fill, inner sep=2pt] {};
        \node (identify) [process, right=0.5cm of start]
        {Identificare \\ statement con \\ overlapping};
        \node (detect) [process, right=0.5cm of identify]
        {Rilevare i conflitti};
        \node (generate) [process, right=0.5cm of detect]
        {Generare \\ strategie di \\ risoluzione};
        \node (evaluate) [process, right=0.5cm of generate]
        {Valutare \\ strategie di \\ risoluzione};
      
        % Arrows
        \draw[->] (start) -- (identify);
        \draw[->] (identify) -- (detect);
        \draw[->] (detect) -- (generate);
        \draw[->] (generate) -- (evaluate);
        \draw[->]  (evaluate.south) -- ++(0,-0.5) -| (start.south);
    \end{tikzpicture}
\end{figure}
L'\textbf{Overlap} si riferisce al riferimento a termini o fenomeni
comuni, che sono una precondizione per l'identificazione di
dichiarazioni conflittuali. Questo include attività come la
raccolta dei vincoli degli incontri e la determinazione dei programmi.

La \textbf{Rilevazione dei Conflitti} può avvenire attraverso vari
metodi:
\begin{itemize}
  \item \textbf{Informalmente}: Utilizzando conoscenze generali ed
  esperienza per identificare potenziali conflitti senza metodi
  formali.
  \item \textbf{Usando euristiche su categorie di requisiti in
  conflitto}, come la verifica dei conflitti tra ``requisiti di
  informazione e requisiti di riservatezza su oggetti correlati''
  e la valutazione dei requisiti su ``quantità in aumento e
  diminuzione correlate''.
  \item \textbf{Usando modelli di conflitto}: Identificazione di
  scenari o strutture ripetuti che tipicamente portano a conflitti
  nei requisiti.
  \item \textbf{Formalmente (tecniche di dimostrazione teoremi)}:
  Applicazione di metodi matematici o logici per provare se certe
  combinazioni di requisiti possono coesistere senza conflitti.
\end{itemize}
\subsubsection{Importanza della Documentazione dei Conflitti}
\begin{figure}[H]
    \centering
    \begin{tikzpicture}
        % Styles
        \tikzstyle{process} = [rectangle, rounded corners, minimum width=3cm, minimum height=1cm, text centered, draw=black, fill=blue!20, text width=2.5cm]
      
        % Nodes
        \node (start) [circle, fill, inner sep=2pt] {};
        \node (identify) [process, right=0.5cm of start] {Identificare \\ statement con \\ overlapping};
        \node (detect) [process, right=0.5cm of identify, line width=1mm] {Rilevare i conflitti};
        \node (generate) [process, right=0.5cm of detect] {Generare \\ strategie di \\ risoluzione};
        \node (evaluate) [process, right=0.5cm of generate] {Valutare \\ strategie di \\ risoluzione};
      
        % Arrows
        \draw[->] (start) -- (identify);
        \draw[->] (identify) -- (detect);
        \draw[->] (detect) -- (generate);
        \draw[->] (generate) -- (evaluate);
        \draw[->]  (evaluate.south) -- ++(0,-0.5) -| (start.south);
    \end{tikzpicture}
\end{figure}
Per una successiva risoluzione e analisi dell'impatto, è importante documentare i
conflitti rilevati. Utilizzando strumenti di documentazione e query che sfruttano
i collegamenti \textit{Conflict} registrati nel database dei requisiti, è possibile
tracciare e gestire efficacemente le dichiarazioni che presentano molteplici conflitti.

\subsubsection{Matrice di Interazione}
Una matrice di interazione è uno strumento utile per visualizzare le relazioni
tra i requisiti e identificare i conflitti. Questa struttura consente di vedere
facilmente quali requisiti si sovrappongono o entrano in conflitto, come illustrato
nella matrice seguente:

\begin{center}
\begin{tabular}{|c|c|c|c|c|c|}
\hline
\textbf{Statement} & \textbf{S1} & \textbf{S2} & \textbf{S3} & \textbf{S4} &
\textbf{Totale} \\
\hline
S1 & 0 & 1000 & 1 & 1 & 1002 \\
\hline
S2 & 1000 & 0 & 0 & 0 & 1000 \\
\hline
S3 & 1 & 0 & 0 & 1 & 2 \\
\hline
S4 & 1 & 0 & 1 & 0 & 2 \\
\hline
\textbf{Totale} & 1002 & 1000 & 2 & 2 & 2006 \\
\hline
\end{tabular}
\end{center}

Nella matrice:
\begin{itemize}
    \item \textbf{0} indica nessuna sovrapposizione,
    \item \textbf{1} indica un conflitto,
    \item \textbf{1000} indica l'assenza di conflitti.
\end{itemize}

La formula per calcolare il numero di conflitti per \textit{S1} è data dal resto
della divisione di 1002 per 1000, mentre il numero di sovrapposizioni non conflittuali
è il quoziente di 1002 diviso 1000.
\subsubsection{Generazione di Strategie di Risoluzione}
\begin{figure}[H]
    \centering
    \begin{tikzpicture}
        % Styles
        \tikzstyle{process} = [rectangle, rounded corners, minimum width=3cm, minimum height=1cm, text centered, draw=black, fill=blue!20, text width=2.5cm]
      
        % Nodes
        \node (start) [circle, fill, inner sep=2pt] {};
        \node (identify) [process, right=0.5cm of start] {Identificare \\ statement con \\ overlapping};
        \node (detect) [process, right=0.5cm of identify] {Rilevare i conflitti};
        \node (generate) [process, line width=1mm, right=0.5cm of detect] {Generare \\ strategie di \\ risoluzione};
        \node (evaluate) [process, right=0.5cm of generate] {Valutare \\ strategie di \\ risoluzione};
      
        % Arrows
        \draw[->] (start) -- (identify);
        \draw[->] (identify) -- (detect);
        \draw[->] (detect) -- (generate);
        \draw[->] (generate) -- (evaluate);
        \draw[->]  (evaluate.south) -- ++(0,-0.5) -| (start.south);
    \end{tikzpicture}
\end{figure}
Bisogna per prima cosa identificare le possibili soluzioni che si possono avere a 
disposizione. Nella fase successiva cerco di confrontare le soluzioni e valutare 
quale sia la migliore.
Per cercare dei candidati per la risoluzione dei conflitti, si possono utilizzare
altre tecniche di elicitazione dei requisiti, come interviste, questionari, o altro; 
oppure utilizzo delle strategie di risoluzione dei conflitti tra cui:

\begin{itemize}
  \item \textbf{Evitare le condizioni al limite:} ad esempio, ``\textit{Mantenere le copie
  dei libri molto richiesti non prenotabili}''.
  \item \textbf{Ripristinare le affermazioni in conflitto:} ad esempio, ``\textit{Copia restituita
  entro 2 settimane e poi presa in prestito di nuovo}''.
  \item \textbf{Indebolire le affermazioni in conflitto:} ad esempio, ``\textit{Copia restituita
  entro 2 settimane a meno di un permesso esplicito}''.
  \item \textbf{Omettere le affermazioni di minore priorità:} questo implica scartare le
  affermazioni meno critiche per risolvere il conflitto.
  \item \textbf{Specializzare la fonte o il target del conflitto:} ad esempio, ``\textit{Stato
  del prestito del libro noto solo agli utenti dello staff}''.
\end{itemize}

Queste strategie aiutano a trasformare le affermazioni in conflitto o gli oggetti coinvolti,
oppure a introdurre nuovi requisiti per facilitare la risoluzione dei conflitti.

\subsubsection{Valutazione delle Strategie di Risoluzione}
\begin{figure}[H]
    \centering
    \begin{tikzpicture}
        % Styles
        \tikzstyle{process} = [rectangle, rounded corners, minimum width=3cm,
        minimum height=1cm, text centered, draw=black, fill=blue!20, text width=2.5cm]
      
        % Nodes
        \node (start) [circle, fill, inner sep=2pt] {};
        \node (identify) [process, right=0.5cm of start] {Identificare \\ statement con \\
        overlapping};
        \node (detect) [process, right=0.5cm of identify] {Rilevare i conflitti};
        \node (generate) [process, right=0.5cm of detect] {Generare \\ strategie di \\
        risoluzione};
        \node (evaluate) [process, line width=1mm, right=0.5cm of generate] {Valutare \\
        strategie di \\ risoluzione};
      
        % Arrows
        \draw[->] (start) -- (identify);
        \draw[->] (identify) -- (detect);
        \draw[->] (detect) -- (generate);
        \draw[->] (generate) -- (evaluate);
        \draw[->]  (evaluate.south) -- ++(0,-0.5) -| (start.south);
    \end{tikzpicture}
\end{figure}
La valutazione delle strategie per risolvere i conflitti si basa su criteri ben definiti
che mirano a ottimizzare il processo decisionale e garantire la coerenza con gli obiettivi
del progetto:

\begin{itemize}
    \item \textbf{Contributo agli obiettivi non funzionali critici:} Le strategie di
    risoluzione vengono valutate in base al loro impatto sui requisiti non funzionali
    considerati critici per il successo del sistema.
    \item \textbf{Contributo alla risoluzione di altri conflitti e rischi:} Oltre a
    risolvere il conflitto specifico, una strategia efficace dovrebbe facilitare la
    gestione di altri conflitti potenziali e mitigare i rischi associati.
\end{itemize}

Questi criteri sono fondamentali per scegliere la risoluzione che non solo risolve
il conflitto immediato ma promuove anche la stabilità e l'integrità del sistema nel
lungo termine. Ulteriori dettagli sulla selezione e l'implementazione di queste
strategie possono essere trovati nella sezione ``Valutazione delle opzioni alternative".

\section{Analisi dei rischi}
\begin{tcolorbox}[colback=blue!5!white,colframe=blue!75!black,title=Rischio]
    Un \textbf{rischio} è un fattore incerto il cui verificarsi può comportare la perdita
di soddisfazione di un obiettivo corrispondente.
\end{tcolorbox}

Esempi includono:
\begin{itemize}
    \item Un passeggero che forza l'apertura delle porte mentre il treno è in movimento.
    \item Un partecipante a una riunione che non controlla regolarmente la posta elettronica.
\end{itemize}
Un rischio è caratterizzato da:
\begin{itemize}
    \item Una probabilità di occorrenza.
    \item Una o più conseguenze indesiderate, come i passeggeri che cadono dal treno in movimento con le porte aperte.
\end{itemize}
Ogni conseguenza del rischio ha:
\begin{itemize}
    \item Una probabilità di occorrenza se il rischio si verifica.
    \item Una gravità, che è il grado di perdita di soddisfazione dell'obiettivo.
\end{itemize}

\subsection{Tipologie di rischi}
I rischi possono essere categorizzati in:
\begin{itemize}
    \item \textbf{Rischi legati al prodotto:} Impattano sugli obiettivi funzionali o non funzionali del sistema. Esempi includono minacce alla sicurezza e pericoli per la sicurezza.
    \item \textbf{Rischi legati al processo:} Hanno un impatto sugli obiettivi di sviluppo, come ritardi nella consegna o superamenti di costo. Un esempio comune è il turnover del personale.
\end{itemize}
\subsection{Come affrontare i rischi}
\begin{figure}[H]
    \centering
    \begin{tikzpicture}
        % Styles
        \tikzstyle{process} = [rectangle, rounded corners, minimum width=3cm, minimum height=1cm, text centered, draw=black, fill=blue!20, text width=2.5cm]
      
        % Nodes
        \node (start) [circle, fill, inner sep=2pt] {};
        \node (identify) [process, right=0.5cm of start]
        {Identificare \\ i rischi};
        \node (detect) [process, right=0.5cm of identify]
        {Valutazione \\ dei rischi};
        \node (evaluate) [process, right=0.5cm of detect]
        {Controllo dei \\ rischi};
      
        % Arrows
        \draw[->] (start) -- (identify);
        \draw[->] (identify) -- (detect);
        \draw[->] (detect) -- (evaluate);
        \draw[->]  (evaluate.south) -- ++(0,-0.5) -| (start.south);
    \end{tikzpicture}
\end{figure}
L'identificazione del rischio è un processo iterativo, infatti
le azioni che prendiamo per controllare i rischi potrebbero dettare nuovi requisiti, 
ponendo nuovi rischi.

La gestione superficiale è una delle principali cause di fallimento di progetti software.

\subsubsection{Identificazione dei Rischi}
\begin{figure}[H]
    \centering
    \begin{tikzpicture}
        % Styles
        \tikzstyle{process} = [rectangle, rounded corners, minimum width=3cm,
        minimum height=1cm, text centered, draw=black, fill=blue!20, text width=2.5cm]
      
        % Nodes
        \node (start) [circle, fill, inner sep=2pt] {};
        \node (identify) [process, right=0.5cm of start, line width=1mm]
        {Identificare \\ i rischi};
        \node (detect) [process, right=0.5cm of identify]
        {Valutazione \\ dei rischi};
        \node (evaluate) [process, right=0.5cm of detect]
        {Controllo dei \\ rischi};
      
        % Arrows
        \draw[->] (start) -- (identify);
        \draw[->] (identify) -- (detect);
        \draw[->] (detect) -- (evaluate);
        \draw[->]  (evaluate.south) -- ++(0,-0.5) -| (start.south);
    \end{tikzpicture}
\end{figure}
L'identificazione dei rischi è un processo essenziale che si basa sull'uso di
liste di controllo specifiche per ciascuna categoria di progetto, al fine di
personalizzare l'approccio al contesto specifico del progetto. Queste liste
sono importanti per identificare i rischi potenziali associati alle diverse categorie
di requisiti.

\begin{itemize}
    \item \textbf{Rischi Legati al Prodotto}:
    derivano dalla possibile insoddisfazione nei
    requisiti funzionali o di qualità:
    \begin{itemize}
        \item \textbf{Inesattezze informative}, come stime errate della velocità o della
        posizione dei treni.
        \item \textbf{Indisponibilità o inutilizzabilità} delle informazioni.
        \item \textbf{Tempi di risposta scarsi} o \textbf{carenze durante picchi di carico},
        che possono compromettere l'efficienza del servizio.
    \end{itemize}
    \item \textbf{Rischi Legati al Processo}:
   includono quelli che possono influenzare negativamente il raggiungimento
   degli obiettivi di sviluppo:
    \begin{itemize}
        \item \textbf{Volatilità dei requisiti}, che può portare a frequenti cambiamenti
        e riallineamenti.
        \item \textbf{Carenze di personale} o \textbf{dipendenze da fonti esterne}, che
        possono ritardare il progresso del progetto.
        \item \textbf{Programmazioni o budget non realistici}, aumentando il rischio di
        superamenti di costi.
        \item \textbf{Gestione inefficace dei rischi}, come può avvenire con un team di
        sviluppatori inesperto.
    \end{itemize}
\end{itemize}

Questi rischi devono essere sistematicamente identificati e gestiti attraverso l'uso
di strategie di mitigazione adeguate, affinché il progetto possa procedere senza
intoppi e con il minor numero di sorprese possibili.

Per i rischi legati al prodotto, tipicamente analizziamo il \textit{system to-be} e
ci focalizziamo sulle sotto-componenti per capire in quanti modi ogni singolo componente
potrebbe fallire, comprendendo inoltre come ogni componente possa fallire.
Possiamo procedere spezzando ulteriormente i componenti in sotto-componenti, fino a
quando non siamo in grado di identificare i rischi.

Per identificare i rischi è possibile inoltre utilizzare il \textbf{risk tree}, che è
un albero che parte da un nodo radice e si dirama in nodi figli, che rappresentano
i sotto-rischi. Questo albero può essere utilizzato per identificare i rischi e
per capire come questi si propagano.

Quando ho un risk tree, posso analizzare il \textit{cut set}, ovvero la minima decomposizione 
delle foglie dell'albero che mi permette di analizzare il rischio ed essi stessi possono 
essere organizzati in un albero.

Per identificare i rischi, oltre a queste analisi, posso dotarmi di scenari e chiedendo 
agli stakeholder, sessioni di gruppo e il riuso della conoscenza sulla base di esperienze 
passate.
\subsubsection{Valutazione dei Rischi}
\begin{figure}[H]
    \centering
    \begin{tikzpicture}
        % Styles
        \tikzstyle{process} = [rectangle, rounded corners, minimum width=3cm,
        minimum height=1cm, text centered, draw=black, fill=blue!20, text width=2.5cm]
      
        % Nodes
        \node (start) [circle, fill, inner sep=2pt] {};
        \node (identify) [process, right=0.5cm of start]
        {Identificare \\ i rischi};
        \node (detect) [process, right=0.5cm of identify, line width=1mm]
        {Valutazione \\ dei rischi};
        \node (evaluate) [process, right=0.5cm of detect]
        {Controllo dei \\ rischi};
      
        % Arrows
        \draw[->] (start) -- (identify);
        \draw[->] (identify) -- (detect);
        \draw[->] (detect) -- (evaluate);
        \draw[->]  (evaluate.south) -- ++(0,-0.5) -| (start.south);
    \end{tikzpicture}
\end{figure}
La valutazione dei rischi è importante per determinare la probabilità e la gravità
delle conseguenze dei rischi identificati. Questo processo è guidato da:

\begin{itemize}
    \item \textbf{Stima Qualitativa}: Utilizzo di stime qualitative per la probabilità
    e la gravità:
    \begin{itemize}
        \item \textit{Probabilità}: ad esempio, molto probabile, probabile, possibile,
        improbabile.
        \item \textit{Gravità}: ad esempio, catastrofica, grave, alta, moderata.
    \end{itemize}
    \item \textbf{Tavola di Consequenzialità dei Rischi}: Ogni rischio viene valutato
    in base a una tabella che incrocia la probabilità del suo verificarsi con la gravità
    delle sue conseguenze.
    \item \textbf{Confronto e Prioritizzazione dei Rischi}: I rischi vengono confrontati
    e prioritizzati in base ai loro livelli di severità per focalizzarsi sul controllo
    dei rischi ad alta priorità.
\end{itemize}

Questi passaggi sono essenziali per sviluppare strategie efficaci di gestione e mitigazione
dei rischi, assicurando così che i rischi ad alta priorità siano gestiti con le risorse
adeguate e nei tempi appropriati.

Per la valutazione dei rischi possiamo dotarci di una tabella di valutazione dei rischi:
\begin{center}
    \begin{tabular}{|c|c|c|c|}
    \hline
    \textbf{Conseguenze} & \textbf{Probabile} & \textbf{Possibile} & \textbf{Improbabile} \\ \hline
    Perdita di vite      & Catastrofico       & Catastrofico       & Grave                \\ \hline
    Ferite gravi         & Catastrofico       & Grave              & Alta                 \\ \hline
    Danni al treno       & Alto               & Moderato           & Basso                \\ \hline
    Riduzione passeggeri & Alto               & Alto               & Basso                \\ \hline
    Danno reputazionale  & Moderato           & Basso              & Basso                \\ \hline
    \end{tabular}
\end{center}
\begin{tcolorbox}[colback=green!5!white,colframe=green!75!black,title=Pro]
    Questo approccio è \textbf{facile da utilizzare} e fornisce stime rapide.
\end{tcolorbox}
\begin{tcolorbox}[colback=red!5!white,colframe=red!75!black,title=Contro]
    Le conclusioni sono \textbf{limitate}: le stime sono grossolane e soggettive,
    e non considerano la probabilità complessiva delle conseguenze.
\end{tcolorbox}
Una valutazione del rischio più quantitativo può essere fatto mediante 
l'ausilio della probabilità e per la scala di severità sempre con un range di valori.
Questo permette di associare dei valori metrici quantitativi per calcolare delle metriche 
derivate.
\begin{tcolorbox}[colback=green!5!white,colframe=green!75!black,title=Pro]
    Questo approccio è \textbf{più preciso} e \textbf{più accurato}.
\end{tcolorbox}
\begin{tcolorbox}[colback=red!5!white,colframe=red!75!black,title=Contro]
    Tale approccio rimane soggettivo, di fatto questi valori dovrebbero essere 
    negoziati con gli stakeholder, che conoscono meglio il contesto.
\end{tcolorbox}
\subsubsection{Controllo dei Rischi}
\begin{figure}[H]
    \centering
    \begin{tikzpicture}
        % Styles
        \tikzstyle{process} = [rectangle, rounded corners, minimum width=3cm,
        minimum height=1cm, text centered, draw=black, fill=blue!20, text width=2.5cm]
      
        % Nodes
        \node (start) [circle, fill, inner sep=2pt] {};
        \node (identify) [process, right=0.5cm of start]
        {Identificare \\ i rischi};
        \node (detect) [process, right=0.5cm of identify]
        {Valutazione \\ dei rischi};
        \node (evaluate) [process, right=0.5cm of detect, line width=1mm]
        {Controllo dei \\ rischi};
      
        % Arrows
        \draw[->] (start) -- (identify);
        \draw[->] (identify) -- (detect);
        \draw[->] (detect) -- (evaluate);
        \draw[->]  (evaluate.south) -- ++(0,-0.5) -| (start.south);
    \end{tikzpicture}
\end{figure}
L'obiettivo principale nella gestione dei rischi è ridurre l'esposizione ai
rischi più critici attraverso l'implementazione di contromisure efficaci. Queste
contromisure non solo mitigano i rischi esistenti ma possono anche portare all'emergere
di nuovi requisiti per affrontare condizioni o sfide non precedentemente identificate.

L'adozione di tali misure richiede un'analisi accurata e una valutazione continua per
assicurare che i rischi vengano gestiti in modo proattivo e che le soluzioni implementate
migliorino effettivamente la robustezza del sistema o del processo.
\subsection{Documentazione dei Rischi}
La documentazione accurata dei rischi è fondamentale per supportare l'evoluzione del
sistema e garantire che le contromisure siano adeguatamente pianificate e implementate.
Per ogni rischio identificato, è essenziale documentare:

\begin{itemize}
    \item \textbf{Condizioni o Eventi di Occorrenza:} Specificare le circostanze sotto le
    quali il rischio può manifestarsi.
    \item \textbf{Probabilità Stimata:} Valutare la frequenza con cui il rischio può
    verificarsi.
    \item \textbf{Cause e Conseguenze Possibili:} Analizzare e registrare le potenziali
    cause dei rischi e le loro possibili conseguenze.
    \item \textbf{Probabilità e Gravità di Ogni Conseguenza:} Stabilire quanto spesso le
    conseguenze possono verificarsi e quanto gravi possono essere
    \item \textbf{Contromisure Identificate e Leverage di Riduzione del Rischio:} Elencare
    le contromisure proposte e valutare la loro efficacia nel ridurre la probabilità o
    l'impatto del rischio.
    \item \textbf{Contromisure Selezionate:} Indicare quali contromisure sono state
    effettivamente scelte per l'implementazione.
\end{itemize}

Questa documentazione dettagliata forma la base per un \textbf{albero dei rischi annotato},
che serve come strumento visivo e funzionale per comprendere e gestire i rischi durante
tutto il ciclo di vita del progetto.

\subsection{Defect Detection Prevention}
Il \textbf{Defect Detection Prevention (\texttt{DDP})} è una tecnica e uno strumento
sviluppati presso la \texttt{NASA} per fornire supporto quantitativo ai cicli di
\textit{Identificazione-Valutazione-Controlla} dei rischi. Questo approccio si
articola in tre fasi principali:

Il \texttt{DDP} è un metodo per identificare i rischi e le opportunità,
valutarli e controllarli. Questo metodo si basa su tre fasi principali:

\begin{enumerate}
    \item \textbf{Elaborazione della Matrice di Impatto del Rischio:} Questa
    fase prevede la creazione di una matrice che valuta l'impatto di ciascun
    rischio identificato, facilitando una comprensione quantitativa delle
    potenziali conseguenze.
    
    \item \textbf{Elaborazione della Matrice di Efficacia delle Contromisure:}
    Successivamente, viene sviluppata una matrice che valuta l'efficacia delle
    varie contromisure proposte contro i rischi identificati.
    
    \item \textbf{Determinazione del Bilanciamento Ottimale tra Riduzione del Rischio
    e Costo delle Contromisure:} L'ultima fase del processo mira a trovare un equilibrio
    ottimale tra il grado di riduzione del rischio ottenuto e il costo delle contromisure
    implementate.
\end{enumerate}

Questo metodo permette di approcciare la gestione dei rischi in maniera sistematica
e quantificabile, assicurando che le decisioni prese siano basate su dati solidi e
analisi approfondite.

\section{Confronto delle opzioni alternative}
l processo di Ingegneria dei Requisiti solleva diverse opzioni alternative per
soddisfare gli obiettivi di un sistema, risolvere i conflitti e ridurre i rischi. 
La selezione delle alternative preferite richiede una negoziazione basata su criteri 
ben definiti.

\subsection{Valutazione Qualitativa e Quantitativa}
\begin{itemize}
    \item \textbf{Valutazione Qualitativa:} Si determina il contributo di ciascuna 
    opzione ai requisiti non funzionali (NFRs), usando etichette qualitative come 
    molto positivo (++), positivo (+), negativo (-) e molto negativo (--). Ad esempio, 
    nella programmazione di riunioni, le opzioni possono variare nella loro capacità 
    di fornire risposte rapide o minimizzare gli inconvenienti.
    \item \textbf{Valutazione Quantitativa:} Si costruisce una matrice ponderata 
    per valutare ogni opzione secondo diversi criteri, basati sull'importanza 
    relativa di ciascun criterio. Il punteggio complessivo di ogni opzione è 
    calcolato come:
    \[
    \text{Punteggio Totale} (opt) = \sum_{\text{crit}} (\text{Punteggio} 
    (opt, \text{crit}) \times \text{Peso} (\text{crit}))
    \]
    dove i punteggi vanno da 0 a 1, indicando la percentuale di soddisfacimento 
    del criterio.
\end{itemize}

\subsection{Criteri di Valutazione}
Per selezionare le alternative preferenziali, si concorda sui seguenti criteri:
\begin{itemize}
    \item Contributo ai requisiti non funzionali critici.
    \item Contributo alla risoluzione di altri rischi.
    \item Costo-efficacia, misurato attraverso il leverage di riduzione del rischio.
\end{itemize}

Questo approccio consente di prendere decisioni informate durante il processo 
ingegneria dei requisiti, assicurando che le soluzioni scelte offrano il miglior 
equilibrio tra 
benefici e costi.

\section{Prioritizzazione dei requisiti}
La prioritizzazione dei requisiti è essenziale per affrontare conflitti di risorse,
sviluppo incrementale e per adattarsi a problemi imprevisti durante lo sviluppo
del progetto. Qui sono illustrati i principi e le tecniche per una prioritizzazione
efficace dei requisiti.

\subsection{Principi di Prioritizzazione dei Requisiti}
\begin{itemize}
    \item Prioritizzazione ordinata per livelli di priorità.
    \item Uso di livelli qualitativi e relativi (es. ``superiore a").
    \item Requisiti di granularità comparabile e stesso livello di astrazione.
    \item Requisiti non mutuamente dipendenti per permettere eliminazioni selettive.
    \item Convergenza sui requisiti accettati dai principali stakeholder.
\end{itemize}

\subsection{Tecnica di Valutazione Valore-Costo}
\begin{itemize}
    \item \textbf{Stima del Contributo Relativo}: Ogni requisito è valutato per il
    suo contributo al valore e al costo del progetto.
    \item \textbf{Diagramma Valore-Costo}: Viene utilizzato per visualizzare la
    trade-off tra valore e costo, categorizzando i requisiti in alta, media e bassa priorità.
    \item \textbf{Uso della Tecnica AHP}: L'Analytic Hierarchy Process aiuta a
    determinare quanto ogni requisito contribuisca al valore o al costo del
    progetto relativamente agli altri requisiti.
\end{itemize}

\subsection{Implementazione della Prioritizzazione}
I requisiti vengono quindi mappati in una matrice valore-costo per determinare
le priorità basate su una valutazione sistematica e quantificata. Questo metodo
fornisce una base oggettiva per decisioni di sviluppo e riallocazione delle risorse.
