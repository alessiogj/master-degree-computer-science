\chapter{Validazione}
L'obiettivo di questa fase è quello di prendere delle decisioni in caso di inconsistenze o 
quali decisioni prendere in maniera negoziata.
\section{Decisione Basata sulla Negoziazione}

La decisione basata sulla negoziazione nel processo di ingegneria dei sistemi
comporta diversi passaggi per affrontare e risolvere varie problematiche che emergono
durante il processo di sviluppo. Di seguito, discutiamo i principali componenti
evidenziati nel processo decisionale:

\begin{itemize}
    \item \textbf{Identificazione e Risoluzione delle inconsistenze}: Questo passaggio
    implica la comprensione e la risoluzione dei punti di vista degli stakeholder in
    conflitto e delle richieste non funzionali per raggiungere un consenso.
    \item \textbf{Identificazione, valutazione e risoluzione dei Rischi di sistema}:
    Questo passaggio è fondamentale per garantire che il sistema soddisfi tutti gli
    obiettivi critici di sicurezza e protezione. Include la revisione dei requisiti
    per sviluppare un sistema più robusto.
    \item \textbf{Confronto delle opzioni alternative}: Si considerano varie opzioni
    per raggiungere gli obiettivi, assegnare responsabilità e risolvere conflitti e
    rischi, il che aiuta nella selezione delle soluzioni più appropriate.
    \item \textbf{Prioritizzazione dei requisiti}: La prioritizzazione dei requisiti
    è essenziale per risolvere i conflitti, aderire ai vincoli di budget e di programma,
    e supportare lo sviluppo incrementale.
\end{itemize}
\section{Gestione delle Inconsistenze}
\subsection{Tipologie di Inconsistenze}
\subsection{Come affrontare le Inconsistenze}
\subsection{Come risolvere le Inconsistenze}

\section{Analisi dei rischi}
\subsection{Tipologie di rischi}
\subsection{Come affrontare i rischi}
\subsection{Come documentare i rischi}
\subsection{Come risolvere i rischi}

\section{Confronto delle opzioni alternative}

\section{Prioritizzazione dei requisiti}