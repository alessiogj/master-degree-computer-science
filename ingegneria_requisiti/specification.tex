\chapter{Specifica dei requisiti e documentazione}
Potrei avere errori con il linguaggio naturale, 
perché è intrinsescamente ambiguo.
Ci sono linea guida per scrivere requisiti,
riducendo l'ambiguità per raggiungere una certa completezza.
Abbiamo delle guide stilistiche per scrivere i requisiti.
\begin{itemize}
    \item Identificare a chi è rivolto il requisito.
    \item Brevi abstract per ogni requisito, per capire subito di cosa si tratta.
    \item Le cose vanno motivate prima di spiegare il requisito.
    \item Prima di usare un concetto è bene definirlo.
    \item Chiedersi se è sufficiente, comprensibile, rilevante.
    \item Non cercare di spiegare troppo in una frase, meglio 
    scrivere più frasi per spiegare meglio.
    \item Usare "deve" e "non deve" per esprimere i requisiti o 
    "dovrebbe" e "non dovrebbe" per esprimere i desiderata.
    \item Evitare gergo o acronimi.
    \item Usare esempi per spiegare meglio.
    \item Usare diagrammi per spiegare meglio relazioni 
    complesse.
\end{itemize}
\section{Regole locali}
Con condizioni complesse è bene usare 
una tabella di verità per spiegare meglio il concetto.
In questo caso l'interpretazione di \texttt{AND} 
e \texttt{OR} è ben definita. Il problema è che le combinazioni 
possono essere molte e quindi la tabella può diventare molto grande.
In questo caso sarà possibile collassare degli elementi.
Con una tabella di questo tipo è possibile avere i test 
di accettazione da far girare sul sistema non appena sarà
disponibile.
\section{Formare le frasi}
\begin{itemize}
    \item Requisiti enumerati e gerarchici.
    \item Requisiti divisi per categorie.
    \item Seguire regole stilistiche.
    \item Regole di fit.
    \item Tracciare le sorgenti dei requisiti.
    \item Motivo.
    \item Interazioni di conflitto o dipendenza.
    \item priorità.
\end{itemize}
Importante avere un criterio di fit per capire se la specifica 
è soddisfatta o meno dal sistema.
Requisiti quantitativi attraverso la soglia percentuale.
Anche attraverso interviste successive.

È opportuno ordinare i requisiti per sezioni, 
in modo da avere requisiti comuni vicini.

\section{Diagrammi}
Un altro modo per raccogliere i requisiti è attraverso i diagrammi.
Che sono però semi-formali.

\subsection{Context diagram}
Si tratta di un grafo che contiene i componenti del sistema (\textit{vertici})
e le relazioni tra i componenti (\textit{archi}).
Ad esempio il guidatore e il sistema di frenata sono due vertici,
mentre l'arco tra i due rappresenta il fatto che il guidatore
può azionare il sistema di frenata.
\subsection{Problem diagram}
È simile al context diagram, ma è più dettagliato, perché
contiene chi fa cosa.
Contiene anche notazioni in linguaggio naturale.
Ragiono i ntermini del problema e quali requisiti fanno 
riferimento a quali compinenti del sistema.
\subsection{Frame diagram}
Questo diagramma è più informativo perché permette 
di tipare i componenti e i fenomeni.
I componenti possono essere: Causali, Evento, Simbolico.
I fenomeni possono essere: Causale, Biddable, Lessicale.
Si va nella direzione del riutilizzo della conoscenza.

Ci sono altri diagrammi per ragionare in termini di 
strutturazione del sistema.
\subsection{Entity-Relationship diagram}
Diagramma visto anche in basi di dati.
è possibile specializzare le classi per aggiungere più 
proprietà.

Ci sono diagrammi che permettono di parlare del sistema
\subsection{SADT}

